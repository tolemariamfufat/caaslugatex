\documentclass[11pt,b5paper]{book}
\renewcommand{\chaptername}{Boqonnaa}
\renewcommand{\tablename}{Gabatee}
\usepackage[linguistics]{forest}
\usepackage{graphicx}
\usepackage{amssymb}
\usepackage{ragged2e}
\usepackage[T1]{fontenc}
\usepackage[utf8]{inputenc}
\usepackage[mathletters]{ucs}
\usepackage{tipa}
\usepackage{tipx}
\usepackage{blindtext}
\usepackage[]{acronym}
\usepackage{fancyhdr}
\fancyhf{}
\fancyhead[LE]{\leftmark}
\fancyhead[RO]{\nouppercase{\rightmark}}
\fancyfoot[LE,RO]{\thepage}
\pagestyle{fancy}
\renewcommand{\headrulewidth}{3.6pt}
\usepackage{apacite,pslatex}
\begin{document}
\renewcommand\refname{Kitaabilee Wabii} 	
\begin{titlepage}
\centering
\thispagestyle{empty}
{\Huge Caasluga Afaan Oromoo}\\
\vspace{3.0\baselineskip}
\vspace{3.0\baselineskip}
\vspace{3.0\baselineskip}
{\Large Tolemariam Fufa Teso}\\
\vspace{3.0\baselineskip}
\vspace{3.0\baselineskip}
\vspace{3.0\baselineskip}
\vspace{3.0\baselineskip}
\vspace{3.0\baselineskip}
\vspace{3.0\baselineskip}
{\small Finfinnee/Addis Ababa 2015}\\
\vspace{3.0\baselineskip}
\date{}
\end{titlepage}
\newpage

\thispagestyle{empty}
  Copyright: \copyright 2015 Tolemariam Fufa Teso\\
  % Email: \email tolemariam.fufa@aau.edu.et\\
  % Mobile Phone: \phone +2519607492\\

\newpage
\thispagestyle{empty}
For my friends

\newpage
\thispagestyle{empty}
\tableofcontents
\newpage
\section*{Gabaajeewwan}
\begin{description}
  \item[1]Ramaddii tokkoffaa
  \item[2]Ramaddii lammaffaa
  \item[3]Ramaddii sadaffaa
  \item[AO]Afaan Oromoo
  \item[BBO]Biiroo Barnoota Oromiyaa
  \item[C]Dubbifamtuu (Consonant)
  \item[CV]Dubbifamtuuf Dubbachiiftuu
  \item[D]Durduubee
  \item[Du.]Dubartii
  \item[G]Gochima
  \item[g.dho]Gad dho'aa
  \item[G.D.]Gamduubee
  \item[Hed.]Hedduu
  \item[MI]Maqibsa
  \item[KATO]Komishinii Aadaafi Turzimii Oromiyaa
  \item[m.x]Miti xiixaa
  \item[M]Maqaa
  \item[MU]Murteessittuu
  \item[Qen.]Qeentee
  \item[V]Dubbachiiftuu
  \item[x.]xiixaa
\end{description}

\chapter{Seensa}
\setlength{\parindent}{3em}

Kitaabni kun yeroo dheeraa keessatti qophaa’e. Qophii isaaf dhiibbaa kan taasisaan keessaa inni guddaan fedhii barattoota kooti. Kaayyoon kitaaba kanaa caasluga AO gadifageenyaan
addeessuudha. Faayidaan isaas barnoota AO digrii lammaffaaf sadaffaaf akka kitaaba barnootaatti tajaajiluudha. Qabiyyeen isaas gadfageenya qaba; bu’uurrii isaas hojiiwwan hayyoota
biyya keessaafi alaati.

Qophiin kitaaba knaa waggoota dheeraa fudhateera. Wixineen kitaaba kana yeroo jalqabaaf kan qophaa’ee bara 1990ti. Yeroo sana digrii lammaffaatiin eebbifameen sirna
barnootaa keessatti akka eksipartii ta’ee osoon hojjechaa jiruu hanqinni caasluga AO jiraachuu isaa hubadheen wixinee kitaaba caasluga AO qopheesse. Matadureen kitaabichaas,
“Seera Afaan Oromoo” kan jedhuu ture. Wixinichis sagaleewwan, dhamjechootaafi caashima AO qaacceessa ture. Yeroo muraasa booda Univarsitii Finfinnee keessatti AO
barsiisuuf carraan argadhe. Wixinichis ittiin barsiisuuf na gargaare. Adeemsa keessas duubdeebii barattootaafi kitaabilee dabalataa biyya alaafi waraqaa qorannoo biyyaa
keessaa dubbisuun wixinicha fooyyeessuun itti fufe. Itti aansee, bara 1996tti qorannoo Ph.D. gaggeessuuf biyya Awuroopaa deemuuf carraan argadhe. Qorannoon koos dhalatoo
gochimaa Afaanota Itoophiyaa keessatti argaman (Verbal Derivation in Ethiopian Afro-Asiatic Languages) irratti xiyyeeffata ture. Dhimma qorannoo kanaaf afaanota handhuura ta’an keessaa AO isa jalqabaa ture. Carraan qorannoo kunis caashimaafi hiika AO gadfageenyaan akkan hubadhu baayyee na gargaare. Akkan Ph.D. koo xumuree biyyatti deebi’eenis wixineen qopheessaa ture keessaa hamma tokko gara kitaabatti jijjiiree ittiin barsiisuun jalqabe. Haaluma kanaan kitaaba “Seera Afaan Oromoo I: Dhamsagaafi Dhamjeha” jedhu barreesseen Univarsitiin Finfinnee keessatti waajira qorannoo akka barattootaaf maxxansiisuun gaafadhe. Waajirichis kitaabicha bara 2003tti ji’a Eeblaa keessa maxxansiisee barattootaa digrii jalqabaaa barataniif raabse
(barattoonni digirii jalqabaaf hamma ammaatti kitaabichaan barachaa jiru). Itti aansee kitaaba barnootaa “Caacculee Shanan Dubbisuu Afaan Oromoo” jedhu bara 2009tti ji’a
waxabajjii keessa maxxansiiseen barattootaa digrii lammaffaaf AO barataniifan dhiyeessee (barattoonnis kitaaba kanaan barachaa jiru). Kitaabni “Caasluga Afaan Oromoo” jedhu kun
isa sadaffaa ta’uu isaati. Kitaabni kun qabiyyeewwan kitaabilee duraan qophaa’an of keessatti hammachuu irra darbee dhimmoota dabalataa hedduu qaba; akkasumas bal’inaafi gad
fageenya qaba.

Kitaaba kana akkan qopheessuuf kan na kakaasan sababoota addaddaatu jiru. Sababni inni jalqabaa fedhii dhuunfaakooti. Wixinneen caasluga AO guutuun isaa inni adeemsa keessa
fooyya’aa tureeru harkakoo jira. Kanaaf wixinee kana gara kitaabaatti jijjiiree barsiisotaafi barattootaa biraan ga’uuf fedhiin qabu galmaan ga’uuf jecha kitaabni kun qophaa’e.
Sababni biroon hir’ina caasluga AOti. Caaslugni AO haala ammayyoomeen guutuu ta’ee afaanichaan barreeffamee gabaarras ta’e manneen kitaabaa keessatti hinargamu. Kana
jechuun caaslugni AO hojiiwwan beektota biyya alaafi biyya keessaa AO irratti hojjetaman irratti bu’uureffatee akkasumas yaaxina irratti hundaa’ee kan dhiyaateeru hinjiru jechuudha.
Hir’ina kana hamma tokko guutuuf jecha kitaabni kun qophaa’e. Sababni biroon gaaffiilee namootaati. Keessumattuu barattoon digrii lammaffaaf AO baratan kitaaba caasluga AO ilaalchisee
hir’inni akka jirufi akkan ani kitaaba caaslugaa kana qopheessu irra deebii’dhaan na gaafachaa turaniiru. Kanaaf gaaffii namootafi barattootakoo na gaafachaa turaniif hamma tokko
deebii kennuuf kitaabni kun qophaa’e.

Kaayyoon kitaaba kanaa caasluga AO addeessuudha. Kana jechuun sagaleewwan AO (dubbachiiftotaafi dubbifamtoota) ni’ibsa. Xindhamsaga AO beektota biyya alaafi keessaa irratti
hundaa’uun gadfageenyaan ni addeessa. Xindhamjecha AO ni addeessa. Himni maal akka ta’eefi akaakuuwwan himootaa ni addeessa. Akkasumas yaaxinoota maddaafi tajaajilaa irratti
bu’uureffachuun caashima AO niqaaccessa.

Faayidaan kitaaba kanaa inni guddaan akka kitaaba barnoota caasluga AOtti fayyaduudha. Kessumattu xiyyeeffannoon isaa Univarsitii keessatti digirii lammaffaafi sadaffaaf kitaaba
barnootaa ta’ee gumaacha ol aanaa kenna. Akkasumas digrii jalqabaa, leenjii barsiisotaafi barnoota sadarkaa lammafaaf akka kitaaba wabii ta’ee tajaajiluu danda’a.

Kitaabni kun kutaalee gurguddoo sadii qaba. Kutaan tokko dhimmoota Xinsagaafi Xindhamsaga dhiyeessa. Qabiyyeen kutaan kanaa kitaaba “Seera Afaan Oromoo I: Dhamsagaafi Dhamjecha” jedhu waliin wlitti siqa. Garuu inni kun gadfageenya qaba. Haaluma kanaan kutaan kun boqonnaawwan lamatti qoodama. Isaanis boqonnaa Xinsagafi boqonnaa xindhamsagaati. Boqonnaan xinsagaa sagaleewwan AO bakka kamitti akka uumaman, haala kamiin akka uumaman, haala dibbee sagalee, sosso’ina arrabaa, saaqama afaaniifi amartaa’uu/diriiruu hidhii qaaccessa. Boqonnaan Xindhamsagaammoo hojiiwwan beektota biyyaa keessaafi alaa kanneen akka\cite{lloret1988gemination,griefenow2001grammatical,owens1985grammar,tolemariam2011,wako1981,biniyam1988,kebede1994}fi kanneen biroo irratti bu’uureffata. Boqonnichi qindoomina dubbifamtootaafi
dubbachiiftotaa qaacceessa. Sagaleen cimdii tokko jechuun maal akka ta’e, dhamasagniifi firsagni maal akka ta’an adda baasee ibsa. Akkasumas irra butaa, adeemsota dhamsagaa
kanneen akka gosagalomii garduree, garduubee, guutuufi gamisa addeessaa. Akkasumas adeemsota dhamsagaa addeessee seera adeemsa dhamsagaatiin ibsa. Kana malees
akaakuuwwan birsagaafi caasaa birsagaa AO addeessa.

Kutaan lama dhimmoota Xindhamjechaafi himaa dhiyeessa. Kutaan kun boqonnaawwan sagalitti qoqqoodama. Kutaan lama hojiiwwan beektota biyya keessaafi alaa akka\cite{abera1982,temesgen1993,Addunya2018,owens1985grammar,gragg1976oromo,beekamaa1996,griefenow2001grammatical,file2015,baye1988focus,tolemariam2011,Abdusamad1994,aadaa1995}fi kanneen biroo ka’umsa godhateera. Kutaan kun jalqabarratti dhamjechi maal jechuu akka ta’e addeessa. Boqonnaan maqaa dhimmoota addaddaa kanneen akka maqaa waloo, maqaa dhalatoo, maqaa uumamteefi maqaa dhuunfaafi fufilee maqaa irratti fufaman
addeessa. Boqonnaawwan itti aananii dhufan maqibsa, agarsiistotaa, lakkoofsafi bamaqaalee addeessa. Boqonnaa gochimaa keessatti gochima ce’aa, hafoo, raawwatmaa, giddugalaafi fufilee gochimarratti fufaman gad fageenyaan ibsamaniiru. Akkasumas kutaan kun jechoota tajaajilaa, gochimibsafi hima dhiyeessa.


Dhumarrattis kitaabni barnootaa tokko looga waalta’e bu’uura godhachuu qaba. Qophiin kitaaba kanas hamma tokko looga AO waalta’e bu’uura godhateera. Hamma danda’ametti
jechootaaf hojiiwwan\cite{aadaa1995}fi kanneen biroo faayidaarra oolaniiru. Haa ta’u malee ilaalcha sabdaneessaan loogni kamuu moggeeffamuu hinqabu. Haaluma kanaan
kitaaba kana keessatti loogotni AO gara garaa faayidaarra oolaniiru. Looga Harar, looga Maccaa, Kamisee, Tuulamaafi Booranaa hammatamaaniiru\cite{biniyam1988,griefenow2001grammatical,kebede1994}.


\chapter{Xinsaga}
\setlength{\parindent}{3em}
\subsubsection{Qabiyyee}
\begin{itemize}
  \item Dubbifamtoota
  \item Dubbachiiftota
\end{itemize}

\subsubsection{Gaaffilee Ka'umsaa}
\begin{enumerate}
  \item Xinsaga Danbalii\footnote{Gaaleen 'Xinsaga Danbalii' jedhu hiika gaalee Afaan Inglizii 'Acoustic Phonetics' jedhuuf kan keennamedha.} jechuun maal jechuudha?
  \item Xinsaga Dhageeffannoo\footnote{Gaaleen 'Xinsaga Dhaggeeffannoo' jedhu hiika gaalee Afaan Inglizii 'Auditory Phonetics' jedhuuf kan kennamedha.} jechuun maal jechuudha?
  \item Xinsaga Dubbannoo\footnote{Gaaleen 'Xinsaga Dubbannoo' jedhu gaalee Afaan Inglizii 'Articulartory Phonetics' tiif kennamedha.} jechuun maal jechuudha?
\end{enumerate}

Xinsaga jechuun maal jechuudha? Saayinsiin xinqooqaa sagaleewwan dubbii afaan tokko keessatti fayyadan ittiin adda baasaanii baran yookiin barsiisan xinsaga jedhama. Xinsagni sagaleewwan afaan tokko keessa jiran adda baasee hubachuuf dubbii ykn barruu, hima, jechaafi birsaga xiinxala. Dhimmi isaas sagaleewwan afaan tokko keessa jiran sirriitti adda baafachuudha. Dubbii ykn barruu keessatti sagaleen qaama xiqqaadha; kana jechuunis sadarkaa dubbiitti sagaleen tokko qaama birootti hinqoqqoodamu ykn hincaccabu jechuudha. Sagaleewwan walitti makamuun birsaga uumu. Birsagoonni ammoo walitti makamuun jecha uumu. Jechoonni walitti makamuun hima ijaaru. Himoonni walitti makamuun barruu ijaaru. Barruu alphabetii keessatti sagaleen tokko qubee tokkoon bakka buusama. AO keessatti qubeen tokko sagalee tokko bakka bu’a (kun qubee dachaa hin’ilaallatu). Fakkeenyaaf,jecha mana jedhu keessatti qubeewwan /m/, /a/, /n/fi /a/n walitti makamuun jecha tokko uumaniiru.

Qorannoon xinsagaa dhaggeeffachuu irraa ka’a. Barruu dhaggeeffatanii himatti qoqqooduu, hima dhaggeeffatanii jechatti qoqqooduu, jecha dhaggeeffatanii birsagatti qoqqooduufi birsaga dhaggeeffatanii sagaleetti qoqqooduu gaafata. Fakkeenyaaf hima armaan gadii haa ilaallu:
Ganamoon hoolaa adii gabaatii bite.
Hima armaan olii kana jechatti yoo qoqqoodnu akka armaan gadii ta’a:
Ganamoon-hoolaa-adii-gabaatii-bite.
Jechoota hima armaan olii keessa jiran ammoo yoo birsagatti
qoqqoodnu akka armaan gadii ta’a:
Ga-na-moon-hoo-laa-a-dii-ga-baa-tii-bi-te.
Birsagoota armaan olii gara sagaaleetti yoo qoqqoodnu ammoo akka armaan gadii ta’a:
G-a-n-a-m-oo-n-h-oo-l-aa-a-d-ii-g-a-b-aa-t-ii-b-i-t-e.

Toftaan dubbii ykn barruu qoqqooduu kun himicha keessa sagaleewwan meeqa akka jiran adda baasee agarsiisa. Barruun tokko sagaleewwan maaliin akka ijaarame adda baafachuun
ga’aa miti. Barataan xinsagaa tokko, sagaleewwan adda baafate sanneen sagaleesse hubachuu qaba. Kana jechuun sagaleewwan baafachuun barbaachisa jechuudha. Kunis toftaa mataa isaa
qaba. Fakkeenyaaf jecha Ganamoon jedhu fudhannee jalqaba, gidduufi dhuma jechaatti sagalee maal akka dhaggeeffanne dubbannee addaan baafachuu keenya mirkaneeffachuu qabna.

Waliigalaan, xinsagni karaalee sadii sagaleewwan qorata. Karaaleen kunneenis amala danbalii, amala dhageeffannoofi amala dubbannooti. Amalli danbalii sagaleen afaan nama dubbatu irraa ka'ee hamma gurra nama isa dhageeffatuutti wayita qilleensa keessa darbu amala maalii akka qabaatu qorachuu irratti xiyyeeffata. Amallii dhageeffannoo ammoo sagaleen erga gurra nama isa
dhageeffatuu ga’ee booda ergaan akkamiin gara sammuutti akka darbu hubachuu irratti xiyyeeffata. Amalli danbaliifi amalli dhageeffannoo meeshaa sagalee qoratutti fayadamuun
malee haala salphaatiin hinhubataman. Amallii dubbannoo garuu meeshaa malee hubatama. Amala dubbannoo kanas kutaawwan itti aananii dhufan keessatti gad fageenyaan ilaalla.

\section{Qaamolee sagalee dubbii}

Qaamni sagalee dubbii sagalee uumuuf tajaajila (fakkii fuula 12 irraa ilaalaa). Qaamni kun sagalee uumuuf qilleensa somba keessaa gara diidaatti ba’uun fayyadama. Akkuma afaanota
hedduu addunyaa kana irrattii argamanii AOs sagalee dubbii kan uumu afuura somba keessaa gara diidaatti ba’uun haa ta’u malee, qilleensa diidaa gara sombaatti galuunis sagaleen
ni’uumama. Qilleensi somba keessaa gara diidaatti wayita ba’u qaama sagalee dubbii uumu waliin walquunnamtii adda addaa taasisa. Qaamni sosso’ina adda addaa gochuudhaan sagalee
adda addaa uuma. Qaamni sagalee dubbii uumu bakka adda addaatti qoodamee ilaalamuu danda’a.Qaamni sagaleewwan dubbii uuman kokkee, qoonqoo, daandii afaaniifi daandee funyaaniiti.
\begin{itemize}
\item[•]Kokkee: Qilleensi somba keessaa gara diidaatti bahu yeroo calqabaaf kan inni quunnamtii godhu kokkee waliini. Kokkeen lafee saanduqa fakkaatu ta’ee qaama qilleensa gara sombaatti
galshuufi baasu ofirraa qaba. Qaamni qilleensa gara sombaatti galshuufi somba keessaa baasu kunis nicufama; nibanamas. Qaamni cufamee banamu kun dibbee sagalee (goongira)
jedhama. Dibbeen sagalee kun ribuuwwan qaqal’oodha. Ribuuwwan kunneen yeroo banamanii qilleensa gadi lakkisan amaloota lama agarsiisu. Amalli inni tokko banamanii qilleensa
gidduusanii osoo hinxiixiin hulluuqsisuudha. Amalli inni lamaffaan ammoo banamanii hollachaa ykn xiixaa qilleensa gadi lakkisuudha.Qoonqoo irratti sagaleewwan uumaman jiru. Kana
malees qoonqoon nyaanniifi dhugaatiin gara sombaatti akka hinseenne eegumsa godha. Kana jechuunis nyaanni gara garaachaatti akka goru taasisa jechuudha.

\item[•] Daandii Afaanii: Bakki qilleensi somba keessaa bahe jijjiirama guddaa itti agarsiisu yoo jiraate daandii afaaniiti. Afaan keessatti sosso’inni arrabxiqqee, hidhii jalaafi sosso’inni
arrabaa ni mul’ata. Tajaajilli arrabxiqqee bal’inaan waan mul’atee ibsamu miti. Hidhiin jalaa garuu hidhii irraafi ilkaan irraa waliin walquunnamtii adda addaa taasisuudhaan
sagaleewwan adda addaa uuma. Hunda caalaa sagalee dubbii uumuuf qaamni iddoo guddaafi sosso’ina bal’aa qabu arraba. Haala kanaan arrabni bakkeewwan afuritti qoqqoodamee
ilaalama.Isaanis fiixee arrabaa, fuuldura arrabaa, qixxelamaan arrabaafi duuba arrabaati. Arrabni gara fuulduraafi gara duubaatti sosso’ee, ol ka’ee gadis bu’ee sagaleewwan adda
addaa uumuuf tajaajila.

\item[•] Daandii Funyaanii: Qaawwi funyaanii qoonqootii kaasee hamma funyaaniitti kan deemu daandii holqa fakkaatu qaba. Duubni daandii afaanii qaama lallaafaa duuba harsassee
jedhamu qaba. Duubni harsassee Jun daandii funyaanii cufuufi banuuf tajaajila. Duubni harsassee gara boodaatti yeroo harkifamu qilleensi somba keessaa ba’u hundi karaa dandii
afaanii akka ba’uuf dirqama. Duubni harsassee yeroo gara fuulduraatti deebi’ummoo daandiin funyaanii nibanama. Yeroo kana qilleensi somba keessaa gara diidaatti yaa’u karaa daandii
funyaanii bahuudhaaf carraa argata. Qilleensi karaa daandii funyaanii gara diidaatti bahu unis sagalee dubbii uumuuf tajaajila.

\end{itemize}


\section{Dubbifamtoota}

Dubbifamtoonni akkamiin uumamu? Walumagalaan uumamni dubbifamtootaa ulaagaa irratti hundaa’a.  Ulaagaawwan kunneenis dibbee sagalee, bakka itti sagaleen uumamuu, akkaataa itti sagaleen uumamuufi kallattii daandii qilleensaati (fakkii fuula 12 irraa ilaalaa). Qabxiiwwan kanneennis akka armaan gadiitti ilaalla: 

\begin{itemize}

\item[•] Dibbee Sagalee: Duraan dursinee dibbeen sagalee banamee akka cufamu ibsineerra.  Dibbeen sagalee banamee qilleensa ofkeessa hulluuqisisee yeroo dbarsu 
sagaleen uumamu mitxiixaa/qooqa dhabeessa jedhama.  Dibbeen sagalee banamee qileensa of keessa hollachaa ykn hurgufamaa yeroo dabarsuu sagaleen uumamummoo xiixaa/qooqa qabeessa jedhama. Yeroo tokko tokkommoo dibbeen sagalee cufame tasuma banamuudhaan dho’ee sagaleen akka uumamu taasisa.  Sagaleen haala kanaan uumamus amala dho’uu waan agarsiisuuf dhootuu jedhamee waamama.  Dimshaashumatti haala dibbeen saglaee qilleensa somba keessaa gara diidaatti bahu waliin walquunnamtii godhuun sagaleewwan AO bakkoota sadiitti qooduu dandeenya.  Isaanis miti xiixaa, xiixaafi dhootuu dha. 

\item[•] Bakkatti Sagaleen Uumamu: Sagaleewwan AO keessatti argaman bakkoota jaharratti uumamu.  Bakka sagaleen itti uumamu jechuunis qaamonni sagalee dubbii warri sosso’aniifi warri hinsossoone walxuqanii bakka sagalee itti uuman jechuudha. 
Hidh-Lamee: Sagaleen hidh-lamee jedhamu sagalee yeroo hidhiin irraafi hidhiin jalaa walxuqan ykn bay’ee walitti siqan uumama. AO keessatti sagaleewwan bakka kanatti uumaman [ph, b, p, m, w] dha.  [p] n sagalee ergisaati.

\item[•] Hidh-Ilkee: Sagaleen hidh-ilkee kan uumamu yeroo hidhiin jalaafi ilkaan irraa wal xuqan, ykn baay’ee walitti siqanii hidhiin jalaafi ilkaan irraa yeroo walitti maxxanan. Yeroo kana gidduu isaanii afuura hulluuqsisanii baasu. Sagaleen yeroo kana uumamu [f] fi [v]dha. [v] n sagalee ergisaati. 

\item[•] Irgee: Sagaleen irgee yeroo arrabniifi irgeen walxuqan uumama.  Irgeefi fuldurri arrabaa walitti siquudhaan qilleensa gidduu isaanii loosanii dabarsuudhaan, ykn cufamuudhaan sagalee uumu.  Arrabniifi irgeen yeroo walxuqan sagaleewwan uumaman [x, d, t, dh, s, z, n, l, r] dha.  

\item[•] Laagee: Sagaleen laagee kan uumamu yeroo qixxelamaan arrabaa laagee waliin walxuqee qilleensa gara diidaatti yaa’u darba dhowwu.  Sagaleewwan laagee kanneen AO keessatti argamani [sh, ch, j, c, ny, y] dha.  

\item[•] Harsassee: Sagaleen harsassee yeroo duubni arrabaa ol ka’ee harsassee waliin walxuqu uumama.  Sagaleewwan harsassee [k], [g] fi [q]dha.

\item[•] Qoonqoo: Qilleensi somba keessaa bahu qoonqoo keessatti karaan itti cufamee yeroo gad lakkifamu sagaleen uumamu sagalee qoonqooti. 
AO keessaa sagaleewwan qoonqoo lama; Oceanus hudhaa['(hudhaa)] fi [h] dha.

Sagaleewwan afaan tokko keessa jiran adda addaan baasanii baruuf qooqa sagalee sanaafi bakka itti sagaleen sun uumamu qofaa baruun gahaa miti.  Fakkeenyaaf sagaleewwan hidh-lamee lama haa ilaallu.  [b] fi [m]n qooqa qabeeyyiidha.  Akkasumas sagaleewwan kunneen lamaanuu sagaleewwan hidh-lameeti.  Kana jechuunis sagaleewwan kunneen bakka tokkotti uumamu jechuudha.  Karaa kallattii qilleensaa yoo ilaalles sagaleewwan lamaanuu qilleensa somba keessaa gara diidaatti bahuun uumamu.  Kanaaf sagaleewwan kunneen karaa qooqaa, karaa bakka itti uumamsaafi karaa kallattii qilleensaatiin addaan hinbahani.  Sagaleewwan kanneen addaan baasuuf ulaagaawwan  biroon barbaachisa; innis akkaata itti sagaleen uumamu hubachuudha. 

\item[•] Cufaa: Sagaleen cufaa kan uumamu qilleensi somba keessaa gara diidaatti bahu bakka itti uumamurratti guutumaan guutuutti dhowwamee yeroo gadi lakkifamu.  Sagaleewwan cufaa [b, p, ph, d, dh, t, x, g, k, q,’(hudhaa)] dha. 

\item[•] Lootuu: Sagaleen lootuu kan uumamu bakkoonni sagalee uuman lamaan walitti siqanii qilleensa gidduu isaanii loosanii yeroo dabarsani.  Sagaleewwan lootuu AO keessatti argaman [f, s, v, z, sh, h] dha.

\item[•] Rigataa: Sagaleen rigataa kan uumamu arrabni bakka itti sagaleen uumamu xuqee suuta suutaan bakka sana qilleensa somba keessaa dhufuuf yeroo gadhiisu.  Haala kanaan [ch] fi [j] tu uumamu. 

\item[•] Funyee: Sagaleen funyee kan uumamu duubni harsassee banamee qilleensi somba keessaa bahu karaa afaaniifi karaa daandii funyaanii akka darbu yeroo taasisu.  Sagaleewwan funyee sadii qofaadha.  Isaanis [m], [n] fi [ny] dha.  Sagaleewwan kunneen yeroo sagaleeffamani qilleensi karaa daandii funyaanii darbuuf dirqama.  Namni tokko funyaansaa cuqqaaluudhaan sagaleewwan kanneen sirriitti sagleessuu hindandahu. 

\item[•] Maddee: Sagaleen maddee kan uumamu fiixeen arrabaa gara irgeetti ol ka’ee karaa cufee, qileensi karaa maddii arrabaa yaa’ee sagalee akka uumu yeroo taasisu.  Sagaleen maddee AO keessatti argamu [l] dha. 

\item[•] Rom’aa: Rom’aan kan uumamu fiixeen arrabaa irgee xuqee ykn xuxxuqee yeroo deebihu.  Sagaleen rom’aa kun [r] dha. 

\item[•] Gamduubee: Sagaleewwan gamduubee amala dubbachiiftotaafi amala dubbifamtootaa qabu.  Sagaleewwan Kunneen yeroo sagaleeffamani akka dubbachiiftotaa afaan nama bansiisu.  Garuu akka dubbifamtootaa ofiisaaniitin of dandahanii hin dhaabbatani.  Kunis dubbifamtoota wajjin walisaan fakkeessa.  Sagaleewwan kunneen [w] fi [y] dha. 

\end{itemize}

\section{Dubbachiiftota}

Akkuma beekamutti afaan tokko keessatti madabiiwwan sagaleewwan kanneen bu'uura tahan lamatu argamu.  Madabiiwwan kunneenis dubbifamtootaafi dubbachiiftota.  Akka kutaa darbe keessatti hubannetti sagaleewwan  dubbifamtuu tahan wayita uumaman qilleensi somba keessaa  gara diidaattii yaa'u afaan keessatti bakkeewwan adda addaa  irratti dhowwamee gadhiifama. Wayita sagaleewwan  dubbachiiftuu tahan uumaman garuu qilleensi somba keessaa  gara diidaatti yaa'u bakka kamittuu hindhowwamu. Kanaaf  jecha dubbachiiftota bakkeewwan sagaleewwan itti uumamaniif  akkaataa sagaleewwan itti uumaman irratti hundoofnee adda  addaa qooduu hindandeenyu. Haala dibbee sagalee irratti  hundoofnees dubbachiiftonni xiixaa dha ykn miti xiixaa dha  jechuu hin dandeenyu; sababni isaas dubbachiiftonni hundi  xiixaa waan tahaniif. Dubbachiiftonni karaawwan sadii ibsamuu  dandahu: sosso'ina arrabaa, qaama arrabaafi haala hidhiiti.  Wayita dubbachiiftuun tokko uumamu sadarkaan sosso'ina  arrabaa ol, gidduu ykn gad tahuu dandaha. Ol jechuun arrabni  ol ka'eera jechuudha; gidduu jechuunammoo arrabni baayyee soo baayyee ol hinfagaatiin, osoo gadis hinbu'iin  giddugaleessairratti kan uumamu jechuudha; gad  jechuunammoo arrabni wayita gad bu'ee dubbachiiftuu sana  uumu mul'ata (fakkii fuula 13 irratti kenname ilaalaa). 

Qaama arrabaa jechuunammoo wayita sagaleen sun uumamu  qaama arrabaa isa caalmaatiin sosso'u ilaallata. Qaamni  arrabaa fuldura, qixxelamaanfi duuba jedhamee bakkeewwan  sadiitti qoodama. Haalli hidhii diriiraa ykn ammoo amartii  taha. Fakkeenyaaf dubbachiiftonni [u] fi [i]n wayita uumaman  arrabni baayyee ol ka'a. Haa tahu malee qaamni arrabaa ol  ka'u sagaleewwan lamaaniifuu adda adda. [u]n wayita  uumamu qaamni arrabaa inni ol ka'u duuba arrabaa yoo tahu,  [i]n wayita uumamuummoo qaamni arrabaa inni ol ka'u  fuuldura arrabaati. [e] fi [o]n wayita uumaman ol ka'umsi  arrabaa giddugaleessa taha. [e]n wayita uumamu qaamni  arrabaa kan giddugaleessa tahu fuuldura araabaa yoo tahu,  [o]n wayita uumamu ammoo qaamni arrabaa kan bakka  giddugaleessa qabatu duuba arrabaati. Dubbachiiftuu [a]n  wayita sagaleessamu qixxelamaan arrabaa gad bu'a. 

Dubbachiiftonni hidhii naannessiisani [u] fi [o] qofa.  Dubbachiiftonni warri hafan hundi wayita sagaleessaman hidhii  hinnaanneessiisan. Sosso'ina aarrabaafi qaama arrabaa irratti  bu'uureffachuun dubbachiiftota AO gabateetiin kutaa dhufu  keessatti ilaalla. 

\subsubsection{Gaaffilee Boqonnichaa}

\begin{enumerate}
  \item Qaamni sagalee dubbii uuman maalfa’i?
  \item Bakka itti sagaleewwan AO uumaman tarreessi.
  \item Haala itti sagaleewwan AO uumaman tarreessi.
  \item Tajaajilli dibbee sagalee maali?
  \item AO keessatti dubbifamtoonni ergisaa kam fa’i?
  \item AO keessa dubbachiiftonni haal kamiin akka uumaman ibsi.
  \item Hudhaan akka dubbifamtuu tokkootti ilaalamaa? Maaliif?
  \item Fakkii qaama sagalee dubbii kaasiitii maqaa isaanii  barreessi.
  \item Gabatee sagaleewwan AO kaasiitii sagaleewwan keessatti  guutii barreessi.
  \item Sagalee AO keessa jiru kan Afaan gara biroo ati beektu  tokko keessa hinjirre ibsi.
  \item Sagalee Afaan gara biroo ati beektu tokko keessa jiru,  garuu kan AO keessa hinjirre tokko ibsi.
  \item Sagaleewwan AO keessas Afaan gara biroo ati beektu tokko  keessa jiran ibsi.
  \end{enumerate}

\chapter{Xindhamsaga}
\setlength{\parindent}{3em}
\subsubsection{Qabiyyee}

\begin{itemize}
  \item Addeessa Dubbifamtootaa
  \item Addeessa Dubbachiiftotaa
  \item Dhamsaga
  \item Firsaga
  \item Adeemsa Dhamsagaa
  \item Birsaga
\end{itemize}

Hayyoonni sagaleewwan dubbii AO addeessaniiru\cite{griefenow2001grammatical,owens1985grammar,wako1981}. Kutaan kun 2.1.  keessatti addeessa dubbifamtootaa dhiyeessa. Akkasumas  faca’insa, jabinaafi irra butaa ibsa. Kana malees dhamsagaafi  firsaga irratti ibsa kenna. Kutaa 2.2. keessatti dubbachiiftota  addeessa. Dubbachiiftonni amaloota adda addaan, laafinaa,
dheerinaafi faca’insaan ni’ibsamu. Kutaa 2.3. keessatti  adeemsa dhamsagaa garagaraa dhiyeessa.
  
\section{Addeessa Dubbifamtootaa}

Kutaa kana keessatti ibsa dubbifamtuu, addeessa dubbifamtuu  cimdii tokko, ibsa irra butaa, ibsa faca’insa dubbifamtuu, ibsa  dhamsagaafi firsagaafi ibsa jabina dubbifamtuu dhiyeessina.

Ibsi dubbifamtotaa bakka, haalaafi dibbee sagaleerratti  bu’uureffata. Fakkeenyonni sagaleewwan jalatti eeraman,  sagalichi jalqaba, gidduufi dhuma jechaatti argamuu isaa  mirkaneessu. Akka waliigalaatti garuu dubbifamtoonni AO  heddumminnaan dhuma jechaarratti hinargamani. 

AO sagaleewwan 58 qaba. Kanneen keessaa sagaleewwan 48 dubbifamtoota dha. Sagaleewwan 10 ammoo dubbachiiftota. Dubbifamtoota keessaa 28 sagaleewwan lafaanii barreeffamn. 

\begin{table}[h!]
	\centering
	\caption{Dubbifamtoota AO Jabaatanii Barreeffaman}
	\begin{tabular}{c |c c c c c c c}
		\hline
		Haala & Hidhlamee & Hidhirgee & Irgee & Laagee & Harsassee & Qoonqoo \\
		\hline
		Cufaa \\
		 & bb & & dd & & gg & & \\
		 & pp & & tt & & kk & & \\
		 & & & xx & & qq & & \\
		 & & & &\cr
		\hline
		Lootuu \\
		 & & vv & zz & & \\
		 &  & ff & ss & & & &  \\
		 & & & &\\
		\hline
		Rigataa \\
		 & & & & jj \\
		 & & & & &\\
		 & & & & cc\\
		\hline
		Funyee & & nn \\
		\hline
		Maddee & & ll\\
		\hline
		Rom'aa & & rr\\
		\hline
		G.D. & ww & & & yy\\
		\hline		
	\end{tabular}
\end{table}

Sagaleewwan 20 ammoo jabaatanii barreeffamu. Sagaleewwan 7 dachaa waan ta'aniif akka sirna barreessuu AOtti jabaatanii hinbarreeffaman. Akkasumas hudhaafi /h/n jabaatanii hinbarreeffaman.

\begin{table}[h!]	
	\caption{Dubbifamtoota AO Laafanii Barreeffaman}
	\begin{tabular}{c c c c c c c c}
		\hline
		Haala & Hidhlamee & Hidhirgee & Irgee & Laagee & Harsassee & Qoonqoo \\
		\hline
		Cufaa \\
		 & b & & d & & g & & \\
		 & p & & t & & k & ' \\
		 & ph & & x & & q & & \\
		 & & & dh\cr
		\hline
		Lootuu \\
		 & & v & z & zy \\
		 &  & f & s & sh & & h\\
		 & & & ts\\
		\hline
		Rigataa \\
		 & & & & j \\
		 & & & & ch\\
		 & & & & c\\
		\hline
		Funyee & & & n & ny\\
		\hline
		Maddee & & & l\\
		\hline
		Rom'aa & & & r\\
		\hline
		G.D. & w & & & y\\
		\hline		
	\end{tabular}
\end{table}


Addeessi dubbifamtotaa bakka, haalaafi dibbee sagaleerratti  bu’uureffata. Fakkeenyonni sagaleewwan jalatti eeraman,  sagalichi jalqaba, gidduufi dhuma jechaatti argamuu isaa  mirkaneessu. Akka waliigalaatti garuu dubbifamtoonni AO  heddumminnaan dhuma jechaarratti hinargamani.
\begin{itemize}
  \item[p] Mitxiixaa, hidhlamee, cufaa. Sagalee ergisaati. Badaa  hinargamu. Fkn,poolisii-jalqaba jechaarratti argama; olompikii-gidduu jechaarratti argama
  \item[b] Xiixaa, hidhlamee, cufaa. Fkn, bani-jalqaba jechaarratti; gobaa-gidduu jechaarratti; gobbaa-gidduu jechaarratti jabaatee argama.
  \item[ph] Hidhlamee, dhootuu, cufaa. Fkn,phaphaasii- jalqaba jechaarratti; haphii-gidduu jechaarratti; hoph jedhe- dhuma jechaarratti argama.
  \item[m] Xiixaa, hidhlamee, funyee. Fkn, mana- jalqaba jechaarratti; lama- gidduu jechaarratti; lammii- jabaatee argama. 
  \item[w] xiixaa, hidhlamee, gamduubee. Fkn, waaqa- jalqaba jechaarratti; gawuu-gidduu jechaarratti; gowwaa- jabaatee argama.
  \item[t] xiixaa, irgee, cufaa. Fkn,tokko- jalqaba jechaarratti; atoo- gidduu jechaarratti; kottee-jabaatee argama. 
  \item[d] xiixaa, irgee, cufaa. Fkn, daree- jalqaba jechaarratti; raada- gidduu jechaarratti; guddaa-jabaatee argama.
  \item[x] Dhootuu, irgee, cufaa. Fkn, xurii- jalqaba jechaarratti; ixaa- gidduu jechaarratti; maxxannee- jabaatee argama.
  \item[dh] Gad dhootuu, irgee, cufaa. Fkn, dhaqi- jalqaba jechaarratti; hoodhu- gidduu jechaarratti; hodhi- jabaatee argama.
  \item[f] mitxiixaa, hidhilkee, lootuu. Fkn,fufaa- jalqaba jechaarratti; lafa- gidduu jechaarratti; of laaffise- jabaatee argama.
  \item[v] Xiixaa, hidhilkee, lootuu, hinjabaatu. Sagalee ergisaati.  Fkn, vidiyoo- jalqaba jechaarratti; televizyinii- gidduu jechaarratti argama.
  \item[s] Mitixiixaa, irgee, lootuu. Fkn,seene- jalqaba jechaarratti; isa- gidduu jechaarratti; keessa- jabaatee argama. 
  \item[z] Xiixaa, irgee, lootuu. Sagalee ergisaati. Fkn,zayituuna- jalqaba jechaarratti; muuzii- gidduu jechaarratti argama.
  \item[ts] Dhootuu, irgee, lootuu. Fkn, Tsahay -jalqaba jechaarratti argama.
  \item[n] Xiixaa, irgee, funyee. Fkn, nama- jalqaba; mana- gidduu jechaarratti; ganna- jabaatee argama.
  \item[r] Xiixaa, irgee, rom’aa. Fkn, rooba- jalqaba; moora- gidduu jechaarratti; irra- jabaatee argama.
  \item[l] Xiixaa, irgee, maddee. Fkn, lama- jalqaba; mala- gidduu jechaarratti; mulluu- jabaatee argama.
  \item[sh] Qooqa dhabeessa, laagee, lootuu. Fkn,shaashii- jalqaba; bishaan- gidduu jechaarratti; bishii- jabaatee argama.
  \item[zy] Xiixaa, laagee, lootuu. Sagalee ergisaati. Badaa  hinargamu. Fkn,televizyinii- gidduu jechaarratti argama.
  \item[ny] Xiixaa, laagee, funyee. Fkn, nyaate- jalqaba; keenya- gidduu jechaarratti; funyaan-dhuma jechaarratti argama. 
  \item[j] Xiixaa, laagee, rigataa. Fkn, joore- jalqaba; majii- gidduu jechaarratti; gamoojjii- jabaatee mul'ata.
  \item[ch] Mitxiixaa, laagee, rigataa. Fkn,bilcheessi- gidduufi bachoo- jabaatee mul'ata.
  \item[c] Dhootuu, laagee, rigataa. Fkn,caalaa- jalqaba; leenca- gidduu jechaarratti; loccuu- jabaatee mul'ata.
  \item[y] Xiixaa, harsassee, cufaa. Fkn, yoom- jalqaba; bayeessa- gidduu jechaarratti; biyya- jabaatee argama.
  \item[k] Xiixaa, harsassee, cufaa. Fkn, koree- jalqaba; ilkaan- gidduu- jechaarratti; lukkuu- jabaatee mul'ata.
  \item[g] xiixaa, harsassee, cufaa. Fkn, gadaa- jalqaba; muguu- gidduu jechaarratti; moggaa- jabaatee argama.
  \item[q] Dhootuu, harsassee, cufaa. Fkn, qaawwa- jalqaba; loqoda- gidduu jechaarratti; baaqqee- jabaatee argama.
  \item[’] Mitxiixaa, qoonqoo, cufaa. Darbee darbee  jabaata. Fkn, re'ee- gidduu; o’’e- jabaatee mul'ata.
  \item[h] Qooqa dhabeessa, qoonqoo, lootuu. Sagaleen kun  hinjabaatu. Fkn, harmee- jalqaba; baha- gidduu jechaarratti mul'ata.  
\end{itemize}

  \subsection{Cimdii Dhamsagaa}
  
Akka qajeelfamaatti, sagaaleewwan cimdii tokko tartiiba  qubee walfakkaatu keessatti bakk waljijjiiranii/ walmorkatanii yoo addaddummaa hiikaa fidan cimdii tokko  jedhamu. Cimdii kana keessattis sagaleewwan kunneen adda  adda ta’anii addaddummaa hiikaa waan fidaniifis dhamsaga  jedhamu. Kaayyoon fakkeenyota jechoota armaan gadiis  sagaleewwan AO dhamsagoota ta’uu isaanii addeessuudhafi  mirkaneessuudha. Addaddummaan sagaleewwan kanneenis  bakka ykn haala sagaleen uumamu ta’uu danda’a. Akkasumas  addaddummaan haala dibbee sagalee (xiixaa ykn miti xiixaa  ta’uun) addaddummaa hiikaa fiduu danda’a. Sagaleewwan  cimdii tokko AO keessaa kanneen armaan gadii isaan  muraasa.
 
\begin{itemize}
        \item/m,b/ fkn, malaa– Dhullaan malaa qaba; balaa– Balaa adda addaarraa of eeguun gaariidha.
        \item/t,d/ fkn, teesse– Isheen eessa teesse? deesse– Isheen maal deesse? 
        \item/d,dh/ fkn,haada –Inni mataa haada; haadha –Inni haadha qaba. 
        \item/k,g/ fkn, kore - Gurbaan muka kore; gore - Gurbaan gara kee gore. 
        \item/g,q/ fkn, gara - Nuti gara manaa deemna; qara - Haaduun kun qara hinqabu. 
        \item/l,m/ fkn, gama - Tulluu gama manni jira; gala - Inni yeroon manatti gala. 
        \item/r,dh/ fkn, roobe - Halkan edaa cabbii roobe; dhoobe - Inni dhoqqee mukatti dhoobe. 
        \item/w,d/ fkn, waasii - Namni hidhame waasii waamee ba’a; daasii - Daasiin mana biratti ijaarama. 
        \item/y,g/ fkn, yaraa - Gaariifi yaraa adda baafachuun sirriidha; garaa - Kan garaa garaan haa beeku jedhu. 
        \item/m,g/ fkn, maraa - Maraa wadaroo lafa kaa’i; garaa - Kan garaa garaan haa beeku jedhi.  
        \item/j,d/ fkn, jiruu - Jiruu fi amatii hintuffatan jedhu; diruu - Lookoo mukatti diruu-n gaariidha. 
        \item/f,j/ fkn, fira - Namni namaaf fira; jira - Uummanni haala gaarii keessa jira. 
        \item/ny,n/ fkn, nyaate - Inni ciree nyaate; naate - Ati garuu maaliif naate?
        \item/c,t/ fkn, baacaa - Inni baacaa abbaa qarshiiti; baataa - Fardi kun collee, kunammoo baataa dha. 
        \item/sh,d/ fkn, shaamaa - Shaamaa uffachuun dur hafe; daamaa - Daamaa taphachuun gaariidha. 
        \item/z,w/ fkn, zayitii - Zayitii nyaataa eessaa bitattu? wayitii - AO torbanitti wayitii meeqa barattu?; 
        \item/p,q/ fkn, poostaa - Xalayaa poostaa keessa kaa’u; qoosxaa- Qoosxaa nyaachuun fayyaaf gaariidha. 
        \item/h,l/ fkn, haphee - Laaftoon haphee qaba; laphee - Lapheen qaama namaa keessaa tokko. 
        \item/r,f/ fkn, haruu - Lafa haruu-n qulqulleessuuf; hafuu - Beellama hafuu-n safuudha.
        \item/x,d/ fkn, fixe - Abbaltiikoo hojjedhee fixe; fide - Qarshii hamma kana eessaa fide? 
        \item/’,f/ fkn, oo’e - Oo’e jechuun gadoode jechuudha; oofe - Horii eessatti oofe? 
        \item/t,l/ fkn, gatii - Gatii-n nyaataa dabaleera; galii - Oomihsarraa garuu galii-n hindaballe.

\end{itemize}  

\subsection{Irra Buta}

Irra buta jechuun walitti aananii dhufaatii dhamsagoota gosa  adda addaa jechuudha. Dhaamsagoonni wayita waliin hiriiran  aantee qabu. Keessumattuu dubbifamtoonni yeroo walitti aananii dhufan dursaafi itti aanaa qabu.AO keessatti irra butaan akaakuuwwan 64 qaba. Isaan keessaa  47 sagaleewwan funyee, maddeefi rom’aa jalqaba irra butaa  taasisu\cite{lloret1988gemination}:  
\begin{table}[h!]
	\caption{Faca'insa Irra Butaa}
	\begin{tabular}{c c c c c c c c c c c c c c}
		\hline\hline
		& k & b & g & q & f & s & m & n & ny & l & r & w & y \\
		t & + & - & - & - & - & - & + & + & -  & + & + & - & -\\
		c & - & - & - & - & - & - & - & - & -  & + & - & - & -  \\
		k & - & - & - & - & + & - & - & + & -  & + & + & - & - \\
		' & - & - & - & - & - & - & + & + & +  & + & + & + & + \\
		b & - & - & - & - & - & - & + & + & -  & + & + & - & - \\
		d & - & + & + & - & - & - & - & + & -  & + & + & - & -  \\
		j & - & + & - & - & - & - & - & + & -  & - & + & - & -  \\
		g & - & - & - & - & - & - & - & + & -  & + & + & - & -  \\
		p & - & - & - & - & - & - & - & - & -  & + & + & - & -   \\
		x & - & - & - & - & - & - & + & + & -  & + & + & - & -  \\
		ch & - & - & - & - & - & - & - & + & - & - & + & - & -  \\
		q & - & - & - & - & + & - & - & + & -  & - & + & - & - \\
		dh & - & - & - & - & - & - & - & + & - & - & - & - & -   \\
		f & - & - & - & - & - & - & - & + & -  & + & + & - & -  \\
		s & + & + & - & + & + & - & + & + & -  & + & + & - & - \\
		sh & - & - & - & - & + & - & - & + & - & - & - & - & -   \\
		m & - & - & - & - & - & - & - & - & +  & + & - & - & -  \\
		n & - & + & + & - & - & - & + & - & -  & - & - & - & - \\
		r & - & - & - & - & - & + & - & - & -  & - & - & - & - \\
		\hline\hline  
	\end{tabular}
\end{table}

\begin{table}[h!]
\caption{Fakkeenyota Irra Butaa}
\centering
\begin{tabular} {c c c} \\
  \hline\hline
  fakkeenya 1 & fakkeenya 2 & fakkeenya 3 \\
  beekta & alanfadha & salaksa \\
  balaansoofii & abdii & inshilaala \\
  abjuu & many'ee & obsa \\
  alaltuu & diinagdee & baalca \\
  dhana & halkan & qooqsa \\
  bal'oo & doqna & tinjii (dangaa)\\
  hordofaa & diftii & aangoo \\
  arjaa & tafkii & firinxiixaa \\
  arga & dafqa & hirphoo \\
  baafsaa & hanqaaquu & qurxummii \\
  afshaala & baay'ee & barcuma \\
  albee & birqii & gamtaa \\
  baaldii & afraasaa & dhim'uu \\
  bumbee & salphaa & birmaduu \\
  lamxii & agamsa & ulfa \\
  handhuura & gamana & filsataa \\
  gatantara & elma & amartii \\
  xannacha & burkutaa & balanballee \\
  bir'aa & handaqii & arba \\
\hline\hline
\end{tabular}
\end{table}

Hundee jechaa ilaalcha keessa galchinee yoo ilaalle, irra buta sagaleewwan laagee irratti daangaan jira. Isaanis: 1. Sagaleewwan /ny/ fi /y/ hudhaa isaanitti aanee dhufu waliin malee irra buta hin’uuman. 2. /sh/n sagalee kam faanayu jalqaba irra butaa ta’ee hindhufu. 3. Sagaleewwan rigataa laagee irratti uumaman jalqaba irra butaa ta’anii hindhufan. Akkuma waliigalaatti sagaleewwan rigataa laagee irratti uumaman sagalee kamiinuu hordofamanii irra butaa hin’uuman yoo hudhaa ta’e malee \cite{lloret1988gemination}.

Akkasumas, irra buta jiran keessaa fC (sagaleen /f/ dura dhuftee jechuudha) heddumminaan mul’ata. Garuu sC (dhamsagni /s/ dura dhufee jechuudha jechoota ergisaa
keessatti argama, Fkn, maskootii (foddaa), masgiida, bataskaana. Irra butawwan armaan gadiis hundee keessatti akka garaa hinargaman:
\begin{itemize}
        \item[kt], makte (-te n fufiidha malee hundee miti)
        \item[ks], salakse (-se n fufiidha malee hundee miti)
        \item[qs], qooqse (-se n fufiidha malee hundee miti)
        \item[qn], waaqni (-ni n fufiidha malee hundee miti)
        \item[bs],c abse (-se n fufiidha malee hundee miti)
        \item[bd], dhabde (-de n fufiidha malee hundee miti)
        \item[bj], abjuu
        \item[gd], dugda
        \item[gn], lugna (Lloret, 1988: 23)
\end{itemize}

\subsection{Dhamsagootaafi Firsagoota}

Dhamsagni tokko akaakuwwan sagaleewwan lama ykn lamaa ol  qabaachuu danda’a. Akaakuuwwan sagaleewwan kunneen  firsga dhamsagichaati. Dhamsaguma tokkotu sagalee gara  biraa fakkaate bakka adda addatti dhugooma. Fakkeenya firsagootaa mursaa isaanii akka armaan gadiitti addeessina. 
\begin{itemize}
        \item \textipa{[M]} qooqa qabeessa, hidhlamee, funyee. Jecha keessatti /n/n sagalee /f/ dursee yoo dhufe \textipa{[M]} ta’ee  mul’ata; Fkn, [go\textipa{[M]}faa] , gonfaa. Kanaaf \textipa{[M]} n firsaga  dhamsaga /n/ ti.
        \item \textipa{[N]} qooqa qabeessa, irgee, funyee. Jecha keessatti  /n /n , /k /,/g/ ykn /k’/ (q) dursee yoo dhufe \textipa{[N]} ta’ee  mul’ata; Fkn, [ma\textipa{[N]}kuusa], mankuusa. Kanaaf \textipa{[N]}n  firsaga dhamsaga /n/ti.
        \item \textipa{[l]} qooqa qabeessa, irgee, maddee. [l] Jecha keessatti  /n/n sagalee /l/ dursee yoo dhufe [l] ta’ee mul’ata;  Fkn, [hillixu], . Kanaaf [l] firsaga dhamsaga /n/ti.
        \item \textipa{[r]} qooqa qabeessa, irgee, rom’aa. Jecha keessatti /n/ sagalee /r/dursee yoo dhufe [r] ta’ee mul’ata; Fkn, [hrrafu] , hinrafu. Kanaaf [r]n firsaga dhamsaga /n/ ti.Walumaa galatti dhamsagni /n/ firsagoota adda addaa  qaba jechuudha.
        \item \textipa{[t]} qooqa qabeessa, irgee, cufaa. Jecha keessatti  /t/n sagalee /n/ dursee yoo dhufe [n] ta’ee mul’ata;  Fkn,[rukunne] , rukutne. Kanaaf /firsaga / t/ti.
        \item \textipa{[t\super h]} qooqa dhabeessa, afuura qabeessa, urge, cufaa. Jecha  keessatti /t/n jalqaba jechaarra yoo dhufe afuura qa beessa ta’ee mul’ata; fkn. [\textipa{[t\super h]}ure], ture. Kanaaf \textipa{[t\super h]} n  firsaga /t/ ti jechuudha.
        \item \textipa{[k\super h]}qooqa qabeessa, afuura qabeessa,  harsassee, cufaa. Jecha keessatti /k/n jalqabarra yoo  dhufe afuura qabeessa ta’ee mul’ata; Fkn,[\textipa{[k\super h]}utaa]. Kanaaf \textipa{[k\super h]}n firsaga /k/ti jechuudha.
        \item \textipa{[k']} ([q]) qooqa dhabeessa, harsassee, cufaa. Jecha  keessatti /q/n sagalee [k] dursee yoo dhufe [k] ta’ee  mul’ata; Fkn, [haksiise], haqsiise. Kanaaf [k]n firsaga  dhamsaga /q/ ti jechuudha.

\subsection{Jabeessuu}

AO keessatti dubbifamtuu jabeessuun jijjiirraa hiikaa fida.  Sababa kanaafis jabeenyi dubbifamtootaa dhamsaga jedhama.  Dubbifamtoota jabeessuun kunis heddumminaan mul’ata.  Kanaaf jabinnii dubbifamtootaa bu’uura AOti. Haala kana  ibsuuf fakkeenyota armaan gadii haa ilaallu: 
\begin{itemize}
        \item[t: tt]
        \item/baati/ Jiini yoom baati?
        \item[baatti] Huummoon morma isheetti maal baatti?
        \item[b: bb]
        \item/gaabii/ Gaabii uffachuun qorraaf gaariidha. 
        \item/gaabbii/ Cubbuu dalaguun gaabbii fida. 
        \item[g: gg]
        \item/gadi/ Manaa gadi lagatu jira. 
        \item/gaddi/ Gaddi nama miidha.
        \item[l: ll] 
        \item/gala/ Bineensi daggala keessa gala. 
        \item/galla/ Nuti garuu gamoo keessa galla. 
        \item[r: rr]
        \item/bare/ Inni konkolaataa ofuu yoom bare? 
        \item/barre/ Nuti har’a wal barre.
        \item[d: dd]
        \item/sagaduu/ Sagaduu-n gaariidha.
        \item/sagadduu/ Sagadduun nama sagadu jechuudha. 
        \item[q: qq]
        \item/baqe/  Dhadhaan aduutiin baqe.
        \item/baqqe/ Ani hamminan baqqe.
        \item[x: xx] 
        \item/fixe/ Ani hojiikoo fixe.
        \item/fixxe/ Atis abbaltiikee fixxe. 
        \item[n: nn]
        \item/gana, ganna/ - Maaltu ganna nama gana?
\end{itemize}

\section{Addeessa Dubbachiiftotaa}

Dubbachiiftonni ta’umsa arrabaa, saaqa afaaniif amartaa’uu hidhii irratti hundaa’ani bakka adda addaatti qoqqoodamu.  Ta’umsi arrabaa bakkeewwan sadiitti qoodama. Isaanis  fuuldura, walakkaafi duuba jedhamu. Fuuldura jechuun  dubbachiiftota fuuldura/fiixee arrabbaa irratti uumaman  jechuudha. Akkasumas kanneen walakkaa arrabaa irratti  uumamaniif duuba ykn hundee arrabaa irratti uumaman jiru  (fakkii dubbachiiftotaa fuula 13 irraa ilaalaa).Karaa saaqaa afaanii dubbachiiftonni bakkeewwan sadiitti  qoodamu. Isaanis dubbachiiftota olii, dubbachiiftota gidduufi  dubbachiiftota gadii jedhamu. Yeroo afaan xinnooshee  saaqamu sagaleewwan uumaman dubbachiiftota olii, yeroo  afaan giddugaleessaan saaqamu sagaleewwan uumaman  dubbachiiftota gidduu, yeroo afaan sirriitti saaqamu  sagaleewwan uumaman dubbachiiftota gadii jedhamu. Haala  amartii hidhiitiin dubbachiiftonni bakka lamatti qoodamu.  Isaanis dubbachiiftota amartaa’uufi kanneen hinamartoofne  jedhamu. Dubbachiiftonni AO kudhani; dhedheeroo shaniifi gaggabaaboo  shan; isaanis fuuldura arrabaa, walakkaa arrabaafi duuba  arrabaa irratti uumamu. Karaa sadarkaa banama afaanii  sagaaleewwan olii, gidduufi gadii jedhamu. Sagaleewwan  duubaa hundi amartii jedhamu.

\begin{table}[h!]
	\centering
\caption{Dubbachiiftota Dhedheeroo AO}
\begin{tabular}{c c c c}\\
  \hline\hline
  Afaan & Fuuldura & Walakkaa & Duuba\\
  Ol & ii & - & uu\\
  Gidduu & ee - & - & oo\\
  Gad & - & aa & -\\
  \hline\hline
  \end{tabular}
\end{table}

\begin{table}[h!]
	\centering
	\caption{Dubbachiiftota Gaggabaaboo AO}	
	\begin{tabular}{c c c c} \\
		\hline\hline
		Afaan & Fuuldura & Walakkaa & Duuba\\
		Ol & i & - & u\\
		Gidduu & e - & - & o\\
		Gad & - & a & -\\
		\hline\hline
	\end{tabular}
\end{table}
  
\begin{itemize}
        \item[i] fuuldura, ol, gabaabaa, hin’amartaa’u. Fkn,ilkaan- jalqaba; bite- gidduu; biti-dhuma jechaarratti argama.
        \item[ii] fuuldura, ol, dheeraa, hin’amartaa’u. Fkn, miira- gidduu; bultii- dhuma jechaarratti argama.
        \item[e] fuuldura, gidduu, gabaabaa, hin’amartaa’u. Fkn, erbuu- jalqaba; bebbeeka- gidduu; rafe- dhuma jechaarratti mul'ata.
        \item[ee] fuuldura, gidduu, dheeraa, hin’amartaa’u. Fkn, eebba- jalqaba; beela- gidduu; re’ee- dhuma jechaarratti argama.
        \item[a] gad, walakkaa, gabaabaa, hin’armartaa’u. Fkn, ana-jalqabaafi dhuma; mana- gidduufi dhuma jechaarratti mul'ata.
        \item[aa] gad, walakkaa, dheeraa, hin’amartaa’u. Fkn, aarii-jalqaba; maatii-giddy; mataa- dhuma jechaarratti argama.
        \item[u] ol, duuba, gabaabaa, amartaa’aa. Fkn, ulfa- jalqaba; bultii- gidduu; rafu-dhuma jechaarratti argama.
        \item[uu] ol, duuba, dheeraa, amartaa’aa. Fkn, uumaa- jalqaba; buusa- gidduu; kaasuu dhuma jechaarratti argama.
        \item[o] gidduu, duuba, gabaabaa, amartaa’aa. Fkn, ol- jalqaba; boru- gidduu; Ganamo- dhuma jechaarratti argama.
        \item[oo] gidduu, duuba, , amartaa’aa. Fkn, booka- gidduu; lookoo- gidduufi dhuma jechaarratti argama.

\end{itemize}

\section{Adeemsa Dhamsagaa}

Adeemsa dhaamsagaa jechuun dhamsagni walitti makamee  sagalee haaraa uumuu jechuudha. Dhamsagni tokko ollaa  dhamsaga biroo yeroo galu nijijjiirama ykn nijjiira. Kana  jechuun sagaleen tokko sagalee biroorra dhiibbaa geessisa ykn  dhiibbaan irra ga'a. Adeemsi dhamsagaa akaakuu adda addaa  qaba. Kutaa kana keessatti gosagaloomii, hidheessuu, laagessuu, harsasseessuu, amarteessuu, funyeessuu, qooqa  fudhachuu, qooqa dhabuu, saaguu, waldarbuu (waljaafuu),  haquu, algosagaloomuu, dheeressuu, ol kaasuufi kkf ilaalla. 

\subsection{Gosagaloomii/Firoommii} 
\setlength{\parindent}{3em}

Jechiifi jechi ykn fufiifi jechii yeroo walitti dhufani, daangaa  walitti dhufeenyaa isaanii irratti jijjiiramni dhamsagaa  ni’uumama. Jijjiirraa haala kanaan dhufu keessaa inni tokko  gosagaloomii jedhama. Gosagaloomiin sagaleen tokko sagalee  isa ollaa isaa jiru akka fakkaatu godha. AO keessatti  gosagaloomiin karaawwan mul’atan keessaa inni tokko  maxxantun duraa inni dubbifamtuutiin dhaabbatu jecha  dubbifamtuutiin calqabutti yeroo maxxanu.Akkasumas fufiin  boodaa dubbifamtuutiin calqabu, jecha dubbifamtuutiin  xumuramutti yeroo maxxanu gosagaloomiin ni  argama. Fakkeenyaaf jecha hinlixnu jedhu wayta hillixnu  jennee barreessinu sagaleen /n/ inni /hin/ keessatti argamu,  sagalee /l/ isa /lix-/ keessa jiru wajjin wal fakkaate. Kunis  gosagaloomii agarsiisa. Kana malees gosagaloomiin karaawwan  adda addaa mul'ata. Isaaniinis kutaawwan armaan gadii  keessatti hubatna. Akaakuuwwan gosagaloomii keessaa tokko gosagaloomii  gardureeti. Gosagaloomii garduree kan jedhamu yeroo sagaleen  tokko sagalee isatti aanee dhufe waliin walfakkaate ykn gara sagalee isatti aanee dhufetti jijjiiramudha. Fakkeenyaaf jecha /dugda/ jedhu haa ilaallu. Jecha kana keessatti sagaleen /g/ sagalee /d/ dursee dhufeera. Sagaleen /g/ kun gara sagalee isatti aanee dhufeetti, jechuun gara sagalee /d/tti gosagalooma; kanaaf jechi /dugda/ jedhu unki isaa jijjiiramee /dudda/ ta'ee dubbatama. Jecha /malaanmaltuu/ jedhus haa ilaallu. Jecha kana keessatti sagaleen /n/ fi /m/n walitti aananii dhufaniiru; sagaleen /n/ sagalee /m/ dursitee dhufteetti; sagaleen /m/ sagalee /m/ sagalee /n/tti aantee dhufteetti. Garuu sagaleen ollaa ishee irratti dhiibbaa geessifte sagalee /m/dha; sagaleen /n/ gara sagalee /m/tti jijjiiramteetti; kanaaf jechi /malaanmaltuu/ jedhu /malaammaltuu/ ta'ee mul'ateera. Karaa jecha gara biroo sagaleen /m/n laaftuu turte jabaattee /mm/ taatee mul'atteetti.

Gosagaloomiin garduree faallaa qaba. Faallaan gosagaloomii garduree,gosagaloomii garduubee jedhama. Gosagaloomii garduubee kan jedhamu wayita  sagaleen tokko gara saglee isa dursee dhufeetti  gosagaloomudha. Fakkeenyaaf jecha /gallne/ jedhu haa ilaallu. Hundeen jecha kana /gal-/ kan jedhu. Hundee jecha kanaa irratti fufiin ramaddii tokkoffaa qeentee {-ne} wayita ida'amtu jechi /gal-ne/ jedhu ijaarama. Yeroo kanatti sagaleen /l/ fi /n/ ollaa walii ta'uuf carraa argatu. Yeroo sagaleewwan kunneen walitti aananii dhufan sagaleen dura dhufte jechuun /l/n sagalee isheetti aantee dhufte jechuun /n/ irratti dhiibbaa geessifti; kanaaf sagaleen /n/ gara sagalee /l/tti jijjiiramti. Haaluma kanaan jechi /galne/ jedhu jecha /galle/ jedhu ta'ee mul'ata jechuudha.

Akkasumas gosagaloommiiwwan guutuufi gamisaa jiru. Gosagaloomii  guutuu kan jedhamu yeroo sagaleen gosagalomu sun gutumaa  guutuutiin jijjiirame sagalee isatti aanee dhufe ykn sagalee isa  dursee dhufe fakkaatudha. Fakkeenyonni gosagaloomii  garduree fi gosagaloomii garduubee keessatti argaman  gosagaloomii guutuu agarsiisuu danda’u. Fakkeenyaaf,\newline
a. hinlixu→hillixu\newline  
b. hinrafu→hirrafu\newline  
c. hinwaamu→hiwwaamufi kkf ilaaluu dandeenya.

Kana malees gosagaloomiin gamisaa jira. Gosagaloomii gamisaa  kan jedhamu sagaleen tokko sagalee isa dursee ykn sagalee  isatti aanee dhufe waliin gutumaa guutuutti osoo hintahiin  gamisaan yeroo wal fakkaatudha. Fakkeenyaaf gosagaloomii  armaan gadii haa hubannu.\newline
a. hinbadu→himbadu\newline
b. hinbeekuu→himbeeku\newline  
c. hinciisu→hinyciisufi kkf.

Fakkeenya kanneen irraa gosagaloomii gamisaa hubachuu  dandeenya. Sagaleen /n/ inni fufii duraa /hin-/ keessatti  argamu gara /m/ fi /ny/ tti jijjiirameera malee gutumaan  guutuutti gara /b/ fi /c/ tti hinjijjiiramne.

Kanatti aansinee gad fageenyaan wayita ilaallu, hidheessuu, laagessuu,  harsasseessuu, amarteessuu, funyeessuu, qooqa fudhachuufi qooqa dhabuun akaakuuwwan gosagaloomii beekamani. Qabxiiwwan kanneenis akka armaan gadiitti tokko  tokkoon ilaalla.  

\subsubsection{Hidheessuu}

Sagaleen bakki uumamsisaa hidhii hintaane ollaa sagalee hidhii  irratti uumamu galee amala hidhi yoo fudhate hidheessuu  jedhame waamama. Fakkeenyaaf,\newline
a. /hinbeeku/ = [himbeeku]\newline  
b. /isinfaa/ = [isi\textipa{M}faa]fi kkf.

Akka (a) irratti hubannu sagaleen /n/ ollaa sagalle /m/ waan  dhufteef gara sagalee /m/tti jijjiiramte. Kana jechuun  sagaleen /n/ bakka uumamsaa ishee irgee irraa gara hidhi  lameetti jijjiirratte jechuudha. Akkasumas akka (b) irraa  hubannutti sagaleen /n/ irgee ta’uu dhiiftee sagalee hidh-illkee  taateetti; sababni isaas ollaa sagalee /f/ waan dhufteef.  Adeemsa dhamsagaa akka kanneeni seera dhamsagaan  agarsiisuun nidandahama.\newline
a. /n/ → [m] / - /b/  \newline
b. /n/ → \textipa{[M]}/- /f/  

Odeeffannoowwan seera dhamsagaa irraa argman sadii:  sagalee adeemsa dhamsagaan tuqamu, jijjiirraa sagalee  mul'atuuf bakka itti jijjiirraan kun mul'atu. Akka (a) irratti  mul'atu kana /n/n sagalee isa adeemsa dhamsagaan tuqamu.  Mallattoon → kun kallatti jijjiirama sagalee agarsiisa. Kana  jechuunis sagaleen /n/ gara firsaga[m] tti jijjiiramteetti jechuudha. Mallattoon [ ] jedhu odeeffannoo lammafaadha;  innis jijjirama sagalee agarsiisa. Mallattoon / kun naannoo  jijjiiramni sagalee itti mul'ate agarsiisa. Karaa jecha gara  biroo /n/ n gara [m] tti kan jijjiiramte wayita sagalee /b/  dursitee dhuftu jechuudha. Akkasumas (b)n haaluma walfakkaatu agarsiisa, sagaleen /n/ wayita /f/ dursitee dhuftu  gara firsaga \textipa{[M]}tti jijjiiramti jechuudha.

\subsubsection{Laagessuu}

Laagessuun, dubbifamaan tokko dubbachiiftota /i/ fi /e/  dursee dhufuun amala laagaa wayita argatu uumama ykn  mul'ata. Adeemsa dhamsagaa akaakuu kanaa akka armaan  gadiitti ilaalla:\newline
a. /leemmana/ = [\textipa{l\super j}eemmana]  \newline
b. /mootii/ = [moo\textipa{t\super j}ii]  \newline
c. /re’ee/ = [\textipa{r\super j}e\textipa{’\super j}ee]  

Akkuma fakkeenya armaan oliirratti mul'atutti dubbifamtoonni  kanneen dubbachiiftota /e/ fi /i/ dursanii dhufan amala  laageffamuu argataniiru. Amalli laageffamuu kunis mataa  dubbifamtuu sana irratti mallattoo “{\super j} " barreessuun  agarsiifama. Adeemsa laagessuu kana seera dhamsagaan  agarsiisuu nidandeenya:  \newline
a. /t/ → [\textipa{t\super j}]/-/i/  \newline
b. /r/ → [\textipa{r\super j}]/-/e/  


Akka seerri armaan olii irratti eerame ibsuti sagaleen /t/ fi /r/ n dubbachiiftota /i/ fi /e/ tiin dursanii waan dufaniif amala  laagessamuu argataniiru.

\subsubsection{Harsasseessuu}

Sagaleen harsassee irratti hinuumamne tokko ollaa sagalee  harsassee galuun amala harsassee wayita fudhatu  harsasseessuu jedhama. \newline
a. /k'oonk'oo/ = [k’oo\textipa{N}k’oo]  \newline
b. /manguddoo/ = [ma\textipa{N}gddoo]  \newline
c. /gaangee/ = [gaa\textipa{N}gee] \newline
d. /sangaa/ = [sa\textipa{N}gaa] 

Akka fakkeenyota armaan olii irraa hubanutti /n/ n ollaa  sagaleewwan harsassee galuun amala harsassee  fudhateetti. Adeemsa kanas seeraan akka armaan gadiitti  kaa’uu dandeenya:  \newline
/n/→ \textipa{[N]/}- /k'(g)/  

\subsubsection{Amarteessuu}

Dubbifamtoonni sagaleewwan dubbachiiftota boodaa dursanii  dhufan amala amartaa’uu dubbachiiftota kanneen irraa argatu.  \newline
a. /bona/ → [\textipa{b\super w}ona]  \newline
b. /kutaa/ → [\textipa{k\super w}utaa] \\
Akkuma armaan olitti mul'atetti /k/ n wayita /u/ tiin dursee  dhufu ni’amartaa'a. /k/n wayita amartaa’u /w/n mataa /k/ irratti barreeffamti. Haala kanaan /kutaa/ inni sarara lama  gidduutti barreeffame jecha nuti dubbachuuf kaane. Inni  ammoo hammattuu keessatti barreeffame, jechuunis, [\textipa{k\super w}utaa]  waan nuti qabatamaan sagaleessine. Akkasumas (b) nis  haaluma (a) waliin wal fakkaata. Fakkeenyota kannenis akka  armaan gadiitti seera dhamsagaan agarsiisuu dandeenya: \newline
a. /k/→ [\textipa{k\super w}] / - /u/  \newline
b. /b/→ [\textipa{b\super w}]/- /o/ \\
 
Akka (a) fi (b) irratti mul’atutti /k/n wayita dubbachiiftuu /u/  dursee dhufu gara [\textipa{k\super w}] tti geeddarama; /b/n wayita  dubbachiiftuu /o/ dursee argamu gara sagalee [\textipa{b\super w}]tti  jijjiirama.  

\subsubsection{Funyeessuu}

Dubbachiiftonni hundi wayita ollaa sagaleewwan funyee dhufan, ollaa isaanii irraa amala funyee argatu. Fakkeenyaaf, \newline
a. /ana/ → [an\textipa{\~a}]  \newline
b. / muka/ → [m\textipa{\~u}ka]  

Akkuma armaan ol irratti mul'atutti dubbachiiftonni ollaaa  sagaleewwan funyee galan nifunyeeffamu. Amalli  funyeeffamuu kunis sarara daddabduu mataa dubbachiiftuu  funyeeffamte irra kaa'uun agarsiifamti. Haalli kunis seera  dhamsagaatiin taa'uu mala:  \newline
a. /a/ → [\textipa{\~a}] //n/-  \newline
b. /u/ → [\textipa{\~u}] / /m/- 

Dubbachiiftuu /a/n sagalee funyeetti aantee dhuftee amala  funye argachuu ishee, akkasumas dubbachiiftuu /u/n sagalee  funyee /m/tti aantee dhuftee amala funee argachuu ishee (a) fi  (b) irraa hubachuu dandeenya. 

\subsubsection{Qooqa Fudhachuu}

Sagaleen qooqa dhabeessi tokko ollaa sagalee qooqa qabeessaa  galee wayita qooqa argatu adeemsisaa qooqa fudhachuu  jedhama.  \newline
a. /gat-na/ →[gan-na]  \newline
b. /dhg-te/→[dhug-de]  

Sagaleen /t/ qooqa dhabeessa. Sagaleen kun ollaa sagalee  qooqa qabeessa wayita galetti qooqa argata ykn qooqa itti aanee wayita dhufu sagalee qooqa qabeessa ta’a jechuudha. 

\subsubsection{Qooqadhabuu}

Sagaleen qooqa qabeessa ta’e tokko ollaa sagalee qooqa  dhabeessa ta’ee galee qooqa dhabeessa ta’a. Fakkeenya,  \newline
a. /fiigse/→ [fiikse]  \newline
b. /\textipa{\!d}iigse/→[\textipa{\!d}iikse]  (dhiigse/dhiikse)\newline
c. /hinseenu/ → [hisseenu]  

Akka (a)fi (b) irratti hubatamutti dhamsagni /g/ gara [k]tti, (c)  irrattammoo /n/n gara sagalee [s]tti jijjiramaniiru.  Fakkeenyota kanneeniif seera dhamsagaa akka armaan gadiitti  keenya: 


/qooqa qabeessa/ → [qooqa dhabeessa] /-- qooqa dhabeessa 

\subsubsection{Saaguu}

Saaguu jechuun dubbifamtoota sadii ta’anii walitti aananii  dhufanii irrabuta uumanii gidduu, dubbachiiftuu galshuu ykn  dubbachiiftota lama ol ta’nii walitti aananii dhufan gidduu  hudhaa galchuudha. Akkuma beekamutti dubbifamtoonni sadii  walitti aananii dhufun seera AO keessatti  hinhayyamamu. Akkasumas irra butaan calqabaaf dhuma  jechaa irratti hinbeekamu. Haalonni kunneen yeroo nama  quunnaman garuu, dubbachiiftuu saaguudhan haala dhibsiisaa  tahe sana furuun nidandahama. Fakkeenyaaf jecha /argine/ jedhu fudhannee haa ilaallu:\newline
a. arg + ne -i- arg-i-ne \\
b. jibb + ne -i- jib-i-ne \\
c. dhoks + ne -i- dhoks-i-ne 

Hundeen jecha /argine/ jedhuu {arg-} dha. Hundee jecha kanaarratti fufiin ramaddii tokkoffaa hedduu {-ne} wayita ida'amtu jechi /argne/ jedhu ijaarama. Haa ta'u malee jechi /argne/ jedhu dubbifamtoota sadii waan walitti aansee fideef AO keessatti seermaleedha. Seera dabe kana sirreessuu sagaleen [i] gidduu saagamte. Haala kanaan jechi /arg-i-ne/ jedhu ijaarame. Jecha /dhoksine/ jedhuus haaluma wal fakkaatuun qaacceffama. 

Akka hayyoonni AO jedhanitti, walumaagalan saaguun seerota sadii qaba\cite{owens1985grammar}. Isaanis:\\ 
1. Dubbifamtoonni sadii walitti aananii dhufaniiran  keessaa dubbifamtuun lammaffaa /l/ ykn /r/ yoo taatee,  sagaan dubbachiiftuu /a/dha. Fkn,  \\
a. kofl-te=kofalte \\
b. kofl+siis=kofalsiis  \\
c. dabr+tan=dabartan  \\
d. dubr+tii=dubartii  

Filannoo gara biroon ammoo waljaafuu fi /i/ saaguudha. Fkn, \\
a. kolfite \\
b. kolfisiise \\
c. dabritan \\
2. Seerri sagaa inni lammaffaa akksi jedha: dubbifamtoota sadii  walitti aananii dhufanii jiran keessaa dubbifamtuun inni  lammaffaa /l/ ykn /r/ miti yoo ta’e /i/tu saagama. Fkn,  \\
a. gudd + s= guddis \\
b. cabs + ta= cabsita \\
c. sirb + te= sirbite \\
d. tiss + te= tissite \\
3. Dubbachiftonni lamaa ol yoo walitti aananii dhufaniiru ta’e  hudhaatu saagama. Fkn,  \\
a. ani + ifa= ni’ifa; \\
b. ni + ilaala= ni’ilaala;\\
c. ni + aara= ni’aarafi kkf. 

\subsubsection{Waldabruu/Waljaafuu}

Waldarbuu/waljaafuu jechuun akkuma maqaan isaa himutti  qubee waldabarsanii dubbachuu ykn barreessuudha.  Fakkeenyaaf, \\
a. afraffaa → arfaffaa  \\
b. darbaa → dabraa  \\
c. salgaffaa → saglaffaa  \\
d. qabarichoo → qarabichoo 

Aramaan olitti sagaleewwan /r/fi /f/n; /g/ fi /l/n; /b/ fi /r/n  akkasumas /q/ fi /r/n bakka walijjiiraniiru. Bakka waljijjiiruun  sagaleewwanii kunis waljaafuu jedhama. Sagaleewwan wal  irraa fagoo jiraatanii jecha tokko ijaaranillee haaluma kanaan  bakka waljijjiiruu dandahu. Dabalataan fakkeenya armaan gadii  haa ilaallu: \\
a. jaldeessa → daljeessa  \\
b. qamalee → qalamee  \\
c. arge → agre \\

Fakkeenya armaan olii (a) irratti /j/ fi /d/ gidduu sagaleewwan  lama jiru. Haa ta’u malee /j/ fi /d/ n bakka wal jijjiiraniiru; (b)  irratti ammoo /m/ fi /l/ gidduu qubee tokko qofaatu jira;  isaansi bakka waljijjiiraniiru; (c) irratti /r/fi /g/n bakka  waljijjiiraniiru.  

\subsubsection{Haquu}

Haquu jechuun adeemsa dhamsagaa keessatti sagalee tokko  jecha keessaa baasuu ykn gatuu jechuudha. AO keessatti  haquun dubbifamtoota irras dubbachiifttota irras  gaha. Fakkeenyaaf lakkoofsa /afur/ fi /sagal/ irratti fufiin  boodaa /-affaa/ yeroo ida’amu dubbachiiftonni badan jiru. Fakkeenyaaf, jecha /nama/ jedhutti fufiin {-oota} wayita ida'amu jecha /nama/ jedhurraa dubbachiiftuun boodaa /a/ haqamti. Haala kanaan jechi /nam-oota/ jedhu ijaarama. 

\subsubsection{Algosagaloomuu}

Algosagaloomuun faallaa gosagaloomuuti. Gosagaloomuun  sagaleen ollaa fakkaachuu yoo ta’u, algosagaloomuun ammoo  ollaa irraa adda ta’uudha. Algosagaloomuun karaalee adda  addaa mul’ata. Karaan tokko, fufiin boodaa inni dubbachiiftuu dheeraa qabu jirma jechaa isa dubbachiiftuu dheeraa qabu  irratti yeroo maxxanuu dubbachiiftuu dheeraa sana gabaabsa;  walfaana hindheeratan. Fakkeenyaaf {nam-} irratti {-oota}n  yeroo ida’amu namoota taha. Jirmi jechaa {nam-} jedhu waan  dubbachiiftuu gabaabduu qabuuf {-oota} fufii sagalee dheeraa  qabu fudhata. Haaluma kanaan {cab-} irratti {-siise} yoo  idaane cabsiise taha. Jirma jechaa sagalee dheeraa qabu  irratti garuu adeemsi kun jijjiiramee algosagaloomii  agarsiisa. Fakkeenyaaf fufii heddumminaa {-oota} jedhu haa ilaallu. Fufiin kun kan inni maxxanu hundee jechaa dubbachiiftuu gabaabduu qabdu irratti. Fakkeenyaaf hundeen jechaa {nam-} dubbachiiftuu gabaabduu qabdi kanaaf fufiin heddumminaa dubbachiiftuu dheeraa qabu jechuun {-oota}n itti fufame. Garuu hundeen jechaa dubbachiiftuu dheertuu qabdi yoo ta'e kan fufamu fufii heddumminaa isa dubbachiiftuu gabaabduu qabu, jechuun {-ota}dha. Fakkeenyaaf jecha /hoolaa/ jedhu irratti kan fufamu {-ota}dha. 

\subsection{Caasaa Birsaga}
\setlength{\parindent}{3em}

Caasaan birsagaa qaamota sadii qaba. Isaanis saaqxuu,  utubaafi cuftuu jedhamu. Saaqxuun birsaga keessatti  dubbifamtuu jalqaba dhuftudha. Utubaan bantii ykn  dubbachiiftuu saaqxuutti aantee dhuftu qaba. Cuftuun  dubbifamtuu dhuma birsagaa dhuftudha. AO keessatti bakki  saaqxuu sagalee dubbifamtuu tokkoon guutama.  Dubbifamtoonni lama bakka saaqxuu galuun dhorkaadha.  Caasaa birsagaa keessa yoo dubbifamtuun hinjiraanne garuu  bakkichi duwwaa ta’uu danda’a. Bakki bantii garuu duwwaa  ta’uu hindanda’u. Bakki bantii dubbachiiftuu tokkoon ykn  dubbachiiftota lamaan guutamuu qaba. Bakki cufaa  dubbifamtuu tokkoon guutama; yoo dubbifamtuu hinqabu ta’e  bakkichi duwwaa ta’uu danda’a; garuu dubbifamtoota lamaan  guutamuu hindanda’u. AO keessatti caasaan birsagaa inni guddaan isa bakka saaqxuu dubbifamtuu tokko, bakka bantii dubbachiiftota lamafi bakka cuftuu dubbifamtuu tokko qabudha. 

\begin{forest}
	[Birsaga  [Saaqxuu (S) [C]] [Utubaa (U) [Bantii (B) [V]] [Cuftuu (C)[C]]]]
	]
\end{forest}  

Akaakuuwwan caasaa birsagaa AO hayyoota gara garaan qaacceeffamera\cite{Addunya2018,griefenow2001grammatical,gragg1976oromo}. Keessumattuu Owns fi Griefenow-Mewis  qorannoo xinqooqaan sadarkaa idil adunyaatti beekamtii ol  aanaa qabu; lamaan isaaniyyuu caasluga AO irratti qorannoo  taasisanii kitaabilee isaanii dhiyeessaniiru. Akka hayyoonnii tokko tokko jedhanitti\cite{griefenow2001grammatical} AO keessa birsagni  dubbachiiftuutiin jalqabamu hinjiru. Fakkeenyaaf jechoota  iriyaa, antuuta, angafa fi kkf wayita sagaleessinu /h/ dura  fidna; yoo /h/ dhiifnemmoo hudhaa dura fidna. Dura dhufaatiin  hudhaa garuu nama meeshaafi gurraan adda baafachuu irratti  leenjii qabuun adda baafama. Hayyoonni hedduun yaada kana  deeggaru. Kanaaf nutis caasaa birsagaa AO keessatti jechi  dubbachiiftuutiin jalqabu hinjiru jenna (Garuu seera qubee AO  keessatti jechi kamuu hudhaatiin hinjalqabu). Karaa gara biroo  ammoo jechuma keessattillee /h/ fi hudhaan bakka waljijjiranii  dhufu. Fakkeenyaaf, jechoota dandaha/danda’a; dhagahe/ dhaga’e jedhan hubachuu dandeenya\cite{griefenow2001grammatical}. 

Kana malees qaaccessa birsagaa keessatti sagaleen jabaateru  tokko bakka lamatti qoodama; sagaleen dheerateeru garuu  lamatti hinqoodamu; irra butaan bakka lamatti qoodama\cite{griefenow2001grammatical}. \\
1. \textipa{P}V (saaqxuu hudhaafi dubbachiiftuu tokko)  \\
 a-na  \\
 i-sa  \\
2. \textipa{P}VV (Saaqxuu hudhaafi sagalee dheeraa)  \\
 ee-boo  \\
 aa-ra  \\
3. CV (saaqxuufi dubbachiiftuu tokko)  \\
 ma-na  \\
 la-ma  \\
4. \textipa{P}VC (saaqxuu hudhaafi dubbachiituu tokkoof cufaa)  \\
 ol  \\
 of \\
5. CVC (saaqxuu, bakka utubaa dubbachiiftuu tokkoofi cufaa)  \\
 gad-da  \\
 gam-na  \\
6. CVVC (saaqxuu, sagalee dheeraafi cufaa)  \\
 maal  \\
 kaan  \\
7. \textipa{P}VVC (saaqxuu hudhaa, sagalee dheeraafi cufaa)  \\
8. CVV (saaqxuufi bakka utubaa sagalee dheeraa)  \\
 boo-kee  \\
 laa-faa \\

Fakkeenyota saddettan kanneen yoo walitti cuunfine akaakuu  caasaa birsagaa afur qofa arganna. Akka Griefenow-Mewis  (2011:21-22) jettutti AO keessa caasaa birsagaa CV, CVV,  CVC fi CVVC tu jiru; jechi hoboombolleetti jedhus hunda  isaanii of keessaa qaba. Akka Owns (1985: 16) jedhutti  naannoo Hararitti caasaaleen birsagaa armaan gadii hundee  jechaa uumuf tajaajilu.  \\
 1. CVC= nam- (nama) \\
 2. VC=if (looga biraatti of) \\
 3. VCC=add (adda) \\
 4. VVC=uum (uume) \\
 5. CVVCC=moorm (looga biraatti morma) \\
 6. CVVC=daar (daaraa) \\
 7. VCVVC=adeer(looga biraatti eessuma) \\
 8. VCVC=afur \\
 9. VCCVC=obbol \\
 10. VCVCCV=ibiddi\\
 11. VVCVC=eegal \\
 12. CVCVVC=ciniin, sakaal\\
 13. CVCVCC=sogidd \\
 14. CVCCVCC=shimbirr \\

AO keessatti dhimmi caasaa birsagaa waliin walitti dhufeenya  cimaa qabu haala itti dubbifamtoonni qindaa’an. AO keessatti  dubbifamtoonni adda addaa lama jalqaba jechaarratti dhufuu  hindanda’an. Kanaaf bakki saaqxuu dubbifamtoota lamaan  guutamuu hindanda’u. Akkasumas jidduu jechaatti  dubbifamtoonni akaakuu adda addaa lamaa ol dhufuu  hindanda’an. Kanaaf irra butaan sagaleewwan lama qofa  qabaachuu qaba; caasaa birsagaa keessatti irra butaanis ta’e  sagalee dheeraan bakka lamatti qoodamanii caasaa birsagaa  adda addaa keessa seenu. Akkasumas AO keessatti  dubbifamtoonni akaakuu adda addaa lama dhuma jechaarra  galuu hindanda’an. Waan kana ta’eefis caasaa birsagaa  keessatti cuftuun dubbifamtuu tokko qofaan guutamuu qabdi.

Caasaan birsagaa karaa irra deebii birsagaa jijjiiramuu danda'a. Irra deebii jechuun waan tokko, haala tokko dabaluudha. Irra  deebii birsagaa jechuun birsaga tokko irra deebi’anii  dubbachuu ykn barreessuu jechuudha. AO keessatti irra  deebiin bal’inaan hojiirra oola. Keessumattuu hundee jechootaa maqibsafi gochima keessatti birsagni jalqabaa irra deebi’ama.  Kana jechuun birsagni jalqaba hundee jechaa gochimaafi  maqibsa irra deebi’ama. Hundee gochimaa ykn maqibsa irra  deebi’uun hiika dabalataa kenna. Fakkeenyaaf, jechoonni diddiimaa, guguraacha, babal'aa fi kkf irra deebii birsagaa agarsiisuu. Irra deebiin birsagaa caasaa birsagaa jecha tokkoo jijjiira. 

\subsubsection{Gaaffilee Boqonnichaa}

Gaaffilee armaan gadii deebisi.
\begin{enumerate}
  \item AO olka'insaafi gad bu'insa sagaleetiin jijjiirraa hiikaa fiduu danda'a?
  \item AO irratti hudhaan jalqaba jechaafi dhuma jechaa dhufuu danda'aa? Maaliif?
  \item AO keessatti sagaleewwan qonqoo irratti uumaman jabaachuu danda'uu? Maaliif?
  \item AO saagduu meeqa qaba? 
  \item Akaakuuwwan adeemsaa dhamsagaa hundaaf fakkeenyota dhuunfaakee kenni.
  \item Akaakuuwwan irra butaa fakkeenya dhuunfaakeetiin ibsi.
\end{enumerate}

\chapter{Xindhamjecha}
\setlength{\parindent}{3em}

\subsection{Seensa}

\subsubsection{Qabiyyee Boqonnaa}

\begin{itemize}
  \item Maqaa
  \item Maqibsa
  \item Gochima
  \item Gochimibsa
  \item Durduubee
  
\end{itemize}

\subsubsection{Gaaffilee Ka'umsaa}

\begin{enumerate}
  \item Addaddummaan birsagaafi jecha gidduu jiru maali?
  \item Jechi tokko hiikaafi unka qaba; dhimma kana fakkeenyaan ibs.
  \item Jechi tokko galmee jechootaa keessatti galmaa’a. Galmeen jechootaa jecha tokkof odeeffannoowwan maal maal kennuutu irraa eegama? Ibsi.
  \item Jechoonni qabiyyee maali?
  \item Jechoonni tajaajilaa maali?
  \item Dhamjecha jechuun maal jechuudha?
  \item Dhamjecha of danda’a jechuun maal jechuudha?
  \item Dhamjecha hirkataa jechuun maal jechuudha?
  \item Hundeen jechaa maali?
  \item Jirmi jechaa maali?
  \item Fufiin hormaataa maali?
  \item Fufiin dhalatoo maali?

\end{enumerate}

Xindhamjechi qorannoo jechootaa irratti xiyyffata. Akkaataatti jechoonni uumaman ibsa. Akkuma kutaa darbe keessatti ilaalletti, dhamsagoonni walitti makamuun birsaga uumu. Birsagni unka qaba; garuu hiika hinqabu. Birsagaoonni ammoo walitti makamanii dhamjecha uumu. Dhamjechi unka qaba; hiikas qaba. Dhamjechi ammoo walitti makamee jecha uuma. Boqonnaawwan itti aananii dhufan keessatti yadrimeewwan armaan gadii faayidaarra oolu. Kanaaf yadrimeewwan kanneen hubachuun gaariidha. 

\begin{itemize}

\item[•] Dhamjecha: xiinxala xindhamjechaa keessaatti dhamjechi maal jechuudha? Dhamjechi qindoomina sagaleewwanii tahee hiika qabeessa isa xiqqaadha. Qindoomina sagaleewwaniiti jechuun sagaleefi sagaleen walitti dhufee dhamjecha ijaara; kanaaf dhamjechi unka qaba jechuudha. Dhamjechi hiika qabeessa isa xiqqaadha jechuun ammoo unki dhamjechaa sun yoo qoqqoodame hiika dhaba jechuudha. Dhamjechi haala kuusaa jechootaa keessatti dhaabbatuun bakkeewwan lamatti qoodama. Akaakuuwwan kunneenis dhamjecha ofdandahaa (dhamjecha bilisaa) fi dhamjecha hirkataa jedhamu. 

\item[•] Dhamjecha ofdandahaa: kun isa ofiisaatiin of dandahee kuusaa jechootaa keessa dhaabbatu.Kana kan jennus haala dhamjechi sun fufii fudhachuufi dhiisuu isaatiini.Dhamjechi ofdandahan maqaa, maqibsa, firoomsee ykn Gochimibsa tahuu dandaha.  Fakkeenyaaf, jechoonni dheedhii, gowwaa, guutuu jedhan maqibsa jedhamu; jechoonni aannan, bishaan, qoraanfi kkf ammoo maqaa jedhamu. Jechoonni kunneen fufii osoo ofitti hin ida’atiin ofdandahanii kuusaa jechootaa keessatti galmaahanii argamuu waan dandahaniif dhamjecha ofdandahaa jedhamanii beekamu.

\item[•] Dhamjecha hirkataa: dhamjechi hirkataan garuu fufii ofitti ida’ata. akka latoo ofdandahaa ofisaatiin ofdandahee kuusaa jechootaa keessatit hinargamu.Dhamjechi hirkataan hundee jechaa yknfufii tahuu dandaha.

\item[•] Hundee jechaa:  hundeen jechaa akaakuu dhamjecha hirkataati. jechaa jechuunis dhamjecha qaama jechaa isa fufiin erga irraa molqamee dhumee argamudha.  Karaa jecha gara biroo, hundeen jechaa unka jechaa isa fufiin osoo itti hinida’amiin dura argamudha. keessatti maqaa, maqibsa, gochimafi gochim ibsi hundee jechaa qabu.  Fakkeenyaaf jechoota armaan gadii hundi hundee gochimaati: adams-, arg-, bar-, booy-, bu’-fi kkf. Hundeewwan kunneen hundeewwan gochimaati. kanneen irratti fufii \{-e\} yoo idaane gochima yeroo dabre keessatti raawwatame hubanna: adamse, arge, bare, bu’efi kkf. Hundeewwan kunneen gochimoota waan ta'aniif ramaddiiwwan hundaaf nibay'atu; fkn, arge, argite, argine, argitan, arganfi kkf. 

\item[•] Fufilee: fufilee/maxxantoonni akkaataa tajaajila isaanittiin bakkoota lamatti qoodamanii ilaalamu. : fufilee hormaataa fi fufilee dhalatooti.  Kallattii jechatti fufamniin ammoo fufii duraa, fufii boodaafi fufii gidduu jechamu. Fufiin duraa hundee jechaa dursee dhufa; fufiin boodaa karaa boodaa maxxana; fufiin gidduu ammaa saagaa yoo ta’u gidduutti fufama. 

\item[•] Fufileen hormaataa: fufileen hormaataa jechootarratti haala adda addaatiin maxxanuudhaan lakkoofsa, korniyaa, maayii, ramaddiifi kan kanneen fakkaatani agarsiisu. Fufileen kunneen maqaa, maqibsafi gochimarratti ida’amu (boqonnaawwan 4, 5,6,7fi 8 ilaalaa).

\item[•] Fufilee dhalatoo: fufileen dhalatoo kanneen jedhaman fufilee garee dubbii jijjiiruu dandahan. Kana jechuunis maqibsa gara maqaatti, gochima gara maqaatti, maqaa gara gochimaattifi kkf jijjiiru dandahu. Kana jechuunis garee dubbii tokko irraa garee dubbii biroo uumu jechuudha. Karaa gara biroommoo garee dubii jijjiiruu dhiisee garee dubbii tokko keessaa akaakuu dubbii isa mataa isaa fakkaatu biroo baasa. Fakkeneyaaf, fufiin dhalatoo tokko maqaa 
irraa maqaa uuma; nama jedhee namummaa wayita jedhu jechuudha (boqonnaawwan 4, 5,6,7fi 8 ilaalaa).
\end{itemize}

Garee jechaa: akkuma jalqabarratti eerretti qubeewwan adda addaa walitti makamanii jechoota ijaaru. Jechoonni adda addaa ammoo walitti makamanii hima ijaaru. Karaa jecha biroo himni tokko akaakuujechootaa gara garaa irraa ijaarama. Jechoonni hima ijaaran kunneen akkaataa unka isaaniif tajaajila isaaniitiin bakka adda addaatti qoodamu. Qoqqooddiin jechootaa kunis  garee jechaa jedhama. Karaa jecha gara biroo garee jechaa jechuun jechoota qabiyyee jechuudha (kan durduubee irraa hafe jechuudha; durduubeen jechoota qabiyyee miti). Gareen jechaa MAQAA, MAQIBSA, GOCHIMA, GOCHIMIBSA fi DURDUUBEE ofkeessaa qaba; isaaninis kutaawwan kanatti aanani dhufan keessatti tokko tokkoon ilaalla. 

\section{Maqaa}

Kutaa kana keessatti maqaa bu'uuraa, maqaa dhalatoo, maqaa dhuunfaa, maqaa waloofi fufilee maqaa irratti fufaman ilaalla. Maqaan jechoota qabiyyee keessaa tokko. Akkasumas maqaan garee jechootaa keessaa tokko. Maqaan namoota dabalatee, waantota lubbuu qabaniif hinqabne hunda bakka bu'a. Maqaan waantota qabatamaan mul'ataniif kanneen lakkawwaman ykn kanneen hinlakkwwamnes bakka bu’a. Maqaan akaakuu adda addaatti qoodama. Isaan keessaa maqaa dhuunfaa, waloo, lakkaa’amu, hinlakkaa’amne, bu’uuraa, dalatoofi kkf jedhamu. 

\item[•] Maqaa dhuunfaa: Maqaan dhuunfaa kan namni tokko, bakki tokko, bineeldi tokko, lagani tokko, tulluun tokko, magaalli tokko, biyyi tokko dhuunfaadhaan ittiin waamamudha.Fakkeenyaaf Abbayya maqaa lagaati. Maqaan namaa adeemsa ykn jijjiirama siyaasaa, amantii,dinagdeefi hawaasummaa waliin walqabatee moggaafama.Fakkeenyaaf moggaasni maqaa Akka lakkoofsa Itoophaatti bara 1983 as moggaafamaa jiran kan duriiraa adda\cite{sinqinash2018}. Dhimma kana xinnoo gad fageessinee ilaaluun yaada kana qabatama taasisa.Moggaasa maqaa daa’immanni mo’icha, dinagdee, dhiibbaafi firooma agarsiisan kan bara 1983 asittii\cite{sinqinash2018}. Akka Sinqinesh jettutti maqaaleen Didiyaa, Falmataa, Hamarasanfi kkf injifannoo agarsiisu. Maqaaleen kannen akka Ankeet, Ansiif, Horrenusiif kkf dinagdee agarsiisu. Maqaaleen kanneen akka Didiyaa, Maafloogan, Nugataniif kkf dhiibbaa agarsiisu. Maqaaleen kanneen akka Atikooti, Asaantuu, Firaanoliif kkf firooma agarsiisu.

\item[•] Maqaa waloo: maqaan kun kan namoonni, bineeldonni,bakkeewwan, waantonnifi kkf gamtaan ittiin waamaman.Fakkeenyaaf maqaalee waloo armaan gadii haa ilaallu:\\
a. Nama= dhiira, dubartii, jaarsa, jaartii...hundaaf maqaa walooti.\\
b. Saree=akaakuu saree hundaafi maqaa walooti.\\
c. Muka=akaakuu muka kamu of keessatti qabata.\\
d. Laga=akaakuu laga hundaaf maqaa walooti.

\item[•] Amala lakkoofsaan maqaan bakka lamatti qoodama. Isaanis maqaa lakkaa'amuufi kan hinlakkaa'mnedha.\\
Maqaa lakkaa’amu: maqaan lakkaa'amu hammam akka ta'e qabatamaan adda baafama. Fakkeenyaaf maqoota armaan gadii haa ilaallu:\\
a. nama-namoota lama\\
b. mana - manneen sadii\\
c. lukkuu - lukkuu sagalfi kkf.

\item[•] Maqaa hinlakkaa’amne: maqaan hinlakkaa'amne kan tokko lama jennee ibsu hindandeenyedha. Fakkeenyaaf maqoota armaan gadii haa ilaallu:\\
a. xaafii - xaafii lama hinjedhamu;\\
b. bishaan - bishaan afur hinjedhamu;\\
c. aannan - aannan lama hinjedhamu; kkf.
\end{itemize}

Madda irratti hundoofnee maqaa bakka lamatti qooduu dandeenya. Isaanis maqaa bu'uuraa\footnote{Gaaleen 'maqaa bu'uuraa' jedhu gaalee Afaan Inglizii'basic nominal' jedhu bakka bu'a.}fi maqaa dhalatoo\footnote{Gaaleen 'maqaa dhalatoo' jedhu gaalee Afaan Inglizii'derived nominal' bakka bu'a.} jedhamu. Maqaan bu’uuraa durumaan kaasee afaanicha keessa jira. Fakkeenyaaf, nama, hoolaa, muka, bishaan, kkf. Maqaan dhalatoo garuu maqaa irraa ykn garee jecha biroorraa uumama. Dhimma kana gad fageenyaan ilaalla.

\subsection{Maqaa Dhalatoo}

Maqaan dhalatoo jechuun afaanichuma keessatti kan maqaa ykn gochima ykn maqibsa irraa dhalatee tajaajila maqaa kennudha. Fufiileen adda addaa hundee maqaa, maqibsa ykn.gochimaatti fufamuun maqaa adda addaa uumu. Isaan keessaa muraasa isaanii akka armaan gadiitti ilaalla.\\
\begin{itemize}

\item[•] Maqaa dudhaa: fufileen/maxxantoonni \{-umma(a)\}, \{-oma\},\{-eenya\}fi \{-inaan\} hundee maqaa ykn maqibsa irratti maxxanuudhaan maqaa dudhaa uumu. Maqoonni dhalatoo haala kanaan uumaman kanneen sammuu keessatti xinxalamanii dudhaafi aadaa uummataa calaqqisiisanii malee kanneen qabatamaan mul'atn miti; amala dhuunfaa osoo hintane uummata keessatti kan gatii qabu agarsiisu. Fufiin {–ummaa} bu'uuraa irratti ida’ame maqaa dhalatoo dudhaa agarsiisu uuma. Fakkeenyaaf jechootni nam-ummaa, ijooll-ummaafi gooft-ummaa haala kanaan ijaaraman. Akkasumas fufiileen \{–eenya\},\{-oma\},\{-ina\} fi \{–insa\} maqibsa irratti ida’amanii maqaa dudhaa uumu. Fakkeenyaaf, jechoonni kan akka jab-eenya, gamn-ooma, diim-inafi jibb-insa jiran haala kana hordofanii uumaman. 

\item[•] Maqaa adeemsaa: fufileen \{-sa(a)\}, \{-a(a)\} fi \{-taa\}n gochima irratti fufamanii maqaa dhalatoo adeemsa raawwii hojii agarsiisu uumu. Jechoonni kanneen akka adeem-sa, bit-taa, gurgur-taafi cabs-aa jiran fakeenya maqaa adeemsaati. Maqoota dhalatoo kanneen hima keessatti ilaaluun gaariidha.\\
a. Ogganaan tokko adeemsa hojii to’ata.\\
b. Gabaa keessatti bittaa gurgurtaan jira.\\
c. Bakki lafee cabsaa jedhamu jiraa?\\
d. Dabsaan sibilaa ogummaa qaba.

\item[•] Maqaa bu’aa: Akkuma maqaan isaa ibsutti maqaan bu'a, sababa waan tokko ta’eeruuf bu’aan tokko mul’achuu garsiisa. Maqaa bu’aa uumuuf fufilee gara garaa \{-umsa,-sa, -aa, -tee, -ii\}n gochimarratti ida’amu. Akka fakkeenya maqaa bu'aatti jechoota abaar-sa, beek-aa, dadhabb-ii, kenn-aafi mur-tee fudhachuu dandeenya.

\item[•] Maqaa mataduree: Maqaan hubannoo, hojjaa ykn gocha tokkoof akka matadureeti tajaajila. Fufii \{–uu\} n maqaa mataduree uumuuf hundee gochimaa irratti ida'amti. Fakkeenyaaf, bituu, jjibbuu, jaallachuufi kkf. maqoota matadureeti. 

\item[•] Maqaa akkaataa/haalaa: Maqaan akkaataa mala ykn haala ittiin adeemsi ykn gochi tokko raawwatu agarsiisa. Maqaan kun yeroo gochima irratti fufileen \{–ii\} , \{-umsa\} fi \{-aati\} n maxxanani uumama. Fakkeenyaaf jechoonni akka dhug-aatii, qal-umsafi ijaajj-ii maqoota akkaataati.

\item[•] Maqaa meeshaa: maxxantoonni kanneen akka \{–ata\} , \{-aa\}fi \{-(i)tuu\}n hundee gochimaa irratti maxxanuudhaan maqaa meeshaa uumu. Maqoonni kanneen akka har-ataa, hafars-aa fi fur-tuu jiran haala kanaan ijaaraman.

\item[•] Maqaa abgochaa: maqaan abgochaa nama hojjaa tokko hojjechuuf dandeettii qabu agarsiisa. Fufileen \{-aa\} fi \{-tuu\}n gochima irratti fufamuun maqaa abgochaa uumu. Fakkeenyaaf, ajjeess-aafi barsiis-tuun haala kanaan ijaaramu.
\end{itemize}

\subsection{Maqaafi Korneyaa}

Jechi 'korniyaa' jedhu nama ykn waan jechi tokko bakka bu'eeru dhiira ykn dubartii ta'uu kan agarsiisu unka yookiin fufii maqaa irratti maxxanu. Korniyaan namni tokko ykn waan tokko uumaatiin dhiira ykn ubartii ta'uu agarsiisa. AO keessatti fufiin korniyaa dhiiraa agarsiisu ichi/icha yoo ta'u, fufiin korniyaa dubartii agarsiisu ammoo \{–ittiin/ittii\} dha. Maqaan tokko hima keessatti bakka matimaa galee korniyaa dhiiraa agarsiisa yoo ta'e, fufii \{–ichi\} jedhu fudhata; bakka antima galee korniyaa dhiiraa agarsiisa yoo ta'e garuu fufii \{–icha\} fudhata. 

Akkasumas, maqaan tokko bakka matimaa galee korniyaa dubartii agarsiisa yoo ta'e fufii \{–ittiin\} yoo fudhatu, bakka antimaa galee korniyaa dubartii agarsiisa yoo ta'e garuu fufii \{–ittii\} fudhata. Fakkeenyaa,\\
1. -ichi\\
a. Sareen dhufe\\
b. Sar-ichi dhufe.\\
2. -icha\\
a. inni saree arge.\\
b. inni sar– icha arge.\\
3. -ittiin\\
a. Sareen dhufte.\\
b. Sar-ittiin dhufte.\\
4. -ittii\\
a. Isheen saree argite.\\
b. Isheen sar– ittii argite. 

Kana malees, maqaaleen uumaatiin dhiiraa ykn dubartii mul'isan jiru. Fakkeenyaaf,\\
1. Dhiira/kormaa\\
a. sangaa\\
b. korma\\
c. korbeessa\\
d. muka\\
e. qilleensa\\
f. samiifi kkf.\\
2. Dubartii/dhaltuu\\
a. sa'a\\
b. goromsa\\
c. raada\\
d. biiftuu\\
e. ji’a\\
f. billaachafi kkf.

\subsection{Maqaafi Lakkoofsa}

Lakkoofsa heddummina yookiin qeentee agarsiisuuf fufii maqaa irratti maxxansina. AO keessa, fufiiwwan maqaa irratti ida'amanii lakkoofsa heddumminaa agarsiisan jiru. Isaan keessaa fufiiwwan \{–oota/ota\} \{-wwan\}, \{-een\}, \{-lee\}, \{-olii\}, \{-olee\} fi \{–aan\} ilaalla. Fufiin \{-oota/-ota\} jedhu heddummina agarsiisuun beekamaadha\cite{griefenow2001grammatical}. Fufiin kun birsaga dubbachiiftuu gabaabaa qabuun dursameera yoo ta'e {-oota} ta'a, /o/n
nidheerata jechuudha. Garuu birsaga dubbachiiftuu dheeraatiin yoo dursame {-ota} ta'a; /o/n nigabaabbata jechuudha . Fakkeenyaaf himoota armaan gadii haa ilaallu: \\
a. Hattuun tokko dhufe.\\
b. Inni hatt-oota lama arge. \\

Hima (a) maqaan hattuun jedhu qeenteedha. Hima (b) irratti garuu maqaan hattoota jedhu hedduudha; fufiin \{-oota\} jedhu birsaga dubbachiiftuu gabaabduu qabuun dursameera. \\
a. Inni afaan tokko dubbata.\\
b. Isheen garuu afaan-ota lama beekti.

Hima (a) irratti akka hubanutti maqaan afaan jedhu qeenteedha. Hima (b) irratti garuu maqaan afaanota jedhu hedduudha; fufiin \{-ota\} jedhu birsaga sagalee dheeraatiin waan dursameef sagaleen /o/ abaabbateera. Fufiin \{-wwan\} jedhus heddummina agarsiisa. Kanas fakkeenya armaan gadii irraa hubachuu dandeenya. \\
a. Lammiin koo dhufe.\\
b. Lammiiwwan koo dhufan.

Akkasumas \{–een\} fi \{-lee\} n lakkoofsa hedduu agarsiisuu. \\
a. Isheen farda tokko qabdi.\\
b. Inni fardeen hedduu qaba.\\
c. Ani jabbii tokko qaba.\\
d. Isheen jabbilee hedduu qabdi. \\

Yeroo baayyee fufii \{–een\} wayita fufamu sagaleen nijabaata.Fakkeenya, maka-mukkeen; laga-laggeen.Fufileen \{–olee\}, \{-olii\} fi \{–aan\} heddummina agarsiisu. Fakkeenaaf,
gaangolii, jarsolee, ilmaan fi kkf tilmaamuun danda'ama.

\subsection{Maqaa Beekamaafi Dhokataa}

AO keessatti maqaan tokko beekamaa ykn dhokataa ta'uu danda'a. Beekamaa
jechuun namni dubbattuufi namni dhaggeeffatu nama waa’een isaa/ishee dubbatamu beeku jechuudha. Yoo hinbeekne ta’e dhokataa jedhama. Maqaan tokko dhokataa yoo ta'e fufii ittiin dhokataa agarsiisu hinqabu.Kana jechuunis maqaa dhokataan fufiitiin dhokataa \{Ø\} qaba jechuudha. Garuu maqaan tokko beekamaa yoo dhiiraaf fufii \{-icha\}, dubartiif \{-ittii\}maxxanfata. Fakkeenyaaf,\\
a. Poolisiin hattuu qabe.\\
b. Poolisiin hatticha qabe. 

Fakkeeny (a) irratti eerame maqaan hattuu jedhu dhokataa ta'uu agarsiisa; hattuun hinbeekamu. Hima (b) irratti eerame keessatti garuu maqaan hatticha jedhu sun beekamaa ta'uu agarsiisa. Fufiin \{-icha\} maqaa korniyaa dhiraatti maxxane. 
a. Isaan dubartii argan.\\
b. Isaan dubartittii waaman. \\

Hima (a) irratti maqaan dubartii jedhu dhokataadha. Hima (b) irratti garuu maqaan dubartittii jedhu nama dubbatuufis ta'e nama dhaggeeffatuuf beekamtuudha. Fufiin \{-ttii\} jedhu kun maqaa dubartii agarsiisutti maxxane. AO keessatti maqaan tokko waan hedduu bakka bu'a yoo ta'e fufiin beekamaas dhokataas itti hinmaxxanu.\\
a. Poolisiin hattoota qabe.\\
b. Poolisiin hattooticha qabe. (hinjedhamu)\\
c. Poolisiin hattootattii qabe. (hinjedhamu)\\

Akkuma himoota armaan olii irraa argamutti, maqaan hattoota jedhu hedduu agarsiisa; sababni isaas fufiin \{–oota\} jedhu waan maqaa irratti maxxaneef. AO keessatt fufii \{-oota\} tti aanee fufiin waan dhufuu inqabneef himootni (b)fi (c) irratti agarsiisaman fudhatama hinqaban; seera AO waan diiganiif. Gabeteen armaan gadii odeeffannoo fufilee beekamaafi dhokataa gabaabumatti lafa kaa'a:\\

\subsection{Maayii Maqaa}

Fufileen adaddaa maqaa irratti ida'amuun maqaan sun hima keessatti tajaajila maal akka kennu agarsiisu. Maqaan tokko hima keessatti akka matimaa, akka antimaatti, akka meeshaattifi kkf tajaajiluu mala. Tajaajilli maqaan hima keessatti kennus fufii maqaatti fufamuun adda ba'ee beekama. Fufilee maayiis kutaa kana keessatti ilaalla.

Maayii matimaa: Fufiin maayii mathimaa agarsiisu maqaa fi maqibsa irratti maxxana. AO keessatti unka matimaaf seeronni armaan gadii hojjetu\cite{griefenow2001grammatical}: \\
1. Jechi birsaga CV tiin gochima yoo ta’e fufiin matimaa \{–ni\} dha. Fakeenyaaf jecha 'nama' jedhuuf matimni isaa 'nam-ni'dha.\\
2. Jechi birsa CCV tiin gochima yoo ta’e fufiin matima agarsiisu \{–i\} dha. Fakkeenyaaf jecha 'obboleessa' jedhuuf unki matimaa 'obboleess-i'dha.\\
3. Jechi dubbachiiftuu dheeraatiin xumura yoo ta’e, fufiin matimaa \{–n\} dha. Fakkeenyaaf jecha 'mataa' jedhuuf unki matimaa 'mataa-n'dha.\\
4. Jechi dhamsaga /n/tiin gochima yoo ta’e, fufii matimaa homaa hin’ida’atu. Fakkeenyaaf jecha 'aannan' jedhuuf unki matimaa 'aannan'dha.\\
5. Maqaaleen dubartii tokko tokko fufii matimaa \{–ti\} ida’atu (kun looga Hararitti mul’ata). Fakkeenyaaf jecha 'lafa' jedhuuf unki matimaa 'laf-ti'dha.

Maayii kennaa:  Fufiin maayii kennaa AO keessatti bifa adda addaa jechuunis \{-dhaaf\}, \{-tiif\}, \{-ii\}, \{-iif\}, \{-a\}, \{-f\}, qaba. Bifoota kanneenis
fakkeenyota armaan gadiirraa hubachuu dandeenya:
a. Inni kitaaba barsiisaa-f kenne.\\
b. Inni kitaaba barsiisaa-dhaaf kenne.\\
c. Inni kitaaba barsiisaa-tiif kenne.\\
d. Inni soogidda loon-ii kenne.\\
e. Inni soogidda loon-iif kenne.\\
f. Isheen qarshii nama-a kennite.\\
g. Isheen qarshii nama-af kennite.

Akkuma armaan olitti ilaalutti maqaan dubbachiiftuu dheertuutiin xumuru maayii kennaa \{-dhaaf\}, fi \{-tiif\} maxxanfata. Maqaan dubbachiiftuu gabaabduu dhumarratti qabummoo dubbachiiftuu dheereffachuu ykn dubbachiiftuu dheereffatee \{–f\} ida'achuun maayii kennaa agarsiisa. Maqaan dubbifamtuu dhumarratti qabu garuu \{–ii\} ykn \{-iif\} maxxanfatee maayii kennaa agarsiisa. 

Maayiin meeshaa waan tokko maaliin akka hojjetame mul'isa.Maayii meeshaa agarsiisuuf maqaa irratti maxxantoonni \{-dhaan\}, \{-an\}, \{-tiin\}, \{-iin\} fi \{-n\}, ida'amu. Fakkeenyaaf himoota armaan gadii haailaallu:
a. Marqaa aannan-iin marqite.\\
b. Marqicha fal’aana-an nyaatte.\\
c. Adamsaan qawwee-tiin/dhaan warabboo ajjeesse.

Akka fakkeenyota kana irraa hubannutti maqaan dubbachiiftuu dheeraa dhumaa qabu \{-tiin/-dhaan\}, maqaan dubbachiiftuu gabaabduun fixu \{–an\}, maqaan dubbifamtuun raawwatu ammoo \{-iin\} maxxanfatanii maayii meeshaa agarsiisu.


\newpage
\bibliographystyle{apacite}
\bibliography{to}





\end{document}
