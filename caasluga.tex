\documentclass[11pt,b5paper]{book}
\renewcommand{\chaptername}{Boqonnaa}
\renewcommand{\tablename}{Gabatee}
\renewcommand{\figurename}{Caasaa}
\usepackage[linguistics]{forest}
\usepackage{tikz}
\usepackage{tikz-qtree}
\usepackage{graphicx}
\usepackage{amssymb}
\usepackage{ragged2e}
\usepackage[T1]{fontenc}
\usepackage[utf8]{inputenc}
\usepackage[mathletters]{ucs}
\usepackage{tipa}
\usepackage{tipx}
\usepackage{blindtext}
\usepackage[]{acronym}
\usepackage{fancyhdr}
\fancyhf{}
\fancyhead[LE]{\leftmark}
\fancyhead[RO]{\nouppercase{\rightmark}}
\fancyfoot[LE,RO]{\thepage}
\pagestyle{fancy}
\renewcommand{\headrulewidth}{3.6pt}
\usepackage{apacite,pslatex}
\usepackage{float}
\usepackage{imakeidx}\makeindex[columns=3]
\renewcommand{\indexname}{Indeeksii Mataduree}
\usepackage[acronym]{glossaries}
\makeglossaries
\begin{document}
\renewcommand\refname{Kitaabilee Wabii} 
\begin{titlepage}
\centering
\thispagestyle{empty}
{\Huge Caasluga Afaan Oromoo}\\
\vspace{3.0\baselineskip}
\vspace{3.0\baselineskip}
\vspace{3.0\baselineskip}
{\Large Tolemariam Fufa Teso}\\
\vspace{3.0\baselineskip}
\vspace{3.0\baselineskip}
\vspace{3.0\baselineskip}
\vspace{3.0\baselineskip}
\vspace{3.0\baselineskip}
\vspace{3.0\baselineskip}
{\small Finfinnee/Addis Ababa 2015}\\
\vspace{3.0\baselineskip}
\date{}
\end{titlepage}
\newpage

\thispagestyle{empty}
  Copyright: \copyright 2015 Tolemariam Fufa Teso\\
  % Email: \email tolemariam.fufa@aau.edu.et\\
  % Mobile Phone: \phone +2519607492\\
\newpage
\thispagestyle{empty}
Yaadannoon isaa hiriyootakoof haa ta'u!
\newpage

\section*{Gabaajeewwan}

\begin{acronym}
	\acro{AO}{Afaan Oromoo}
	\acro{BBO}{Biiroo Barnoota Oromiyaa}
	\acro{C}{Dubbifamtuu (Consonant)}
	\acro{D}{Durduubee}
	\acro{Dh.} {Dhiira}
	\acro{Du.}{Dubartii}
	\acro{f}{Faana Gaalee}
	\acro{G.D.}{Gamduubee}
	\acro{G.Dho.}{Gad dho'aa}
	\acro{G}{Gochima}
	\acro{GD}{Gaalee Durduubee}
	\acro{GD1}{Gaalee Durduubee Xiqqaa}
	\acro{GDG}{Gaalee Durduubee Guddaa}
	\acro{GF}{Gocha Fufaa}
	\acro{GG1}{Gaalee Gochimaa Xiqqaa}
	\acro{GGM}{Gaalee Gochimaa Guddaa}
	\acro{GIG}{Gaalee Gochimibsaa}
	\acro{GIGI}{Gaalee Gochimibsaa Xiqqaa}
	\acro{GIGG}{Gaalee Gochimibsaa Guddaa}
	\acro{GIM}{Gaalee Maqibsaa}
	\acro{GIM1}{Gaalee Maqibsaa Xiqqaa}
	\acro{GIMG}{Gaalee Maqibsaa Giqqaa}
	\acro{GM}{Gaalee Maqaa}
	\acro{GM1}{Gaalee Maqaa Xiqqaa}
	\acro{GMG}{Gaalee Maqaa Guddaa}
	\acro{Hed.}{Lakkoofsa Hedduu}
	\acro{IM}{Ibsa Maqaa}
	\acro{KATO}{Komishinii Aadaafi Turzimii Oromiyaa}
	\acro{M}{Maqaa}
	\acro{M.X}{Miti Xiixaa (voiceless)}
	\acro{MU}{Murteessituu}
	\acro{Qen.}{Lakkoofsa Qeentee}
	\acro{V}{Dubbachiiftuu}
	\acro{X.}{Xiixaa}
	\acro{1}{Ramaddii Tokkoffaa}
	\acro{2}{Ramaddii Lammaffaa}
	\acro{3}{Ramaddii Sadaffaa}
\end{acronym}

\newpage
\section*{Jibsoo AO irraa gara Afaan Ingliziitti}
\begin{itemize}
	\item Abgocha-Agentive
	\item Adeemsa Dhamsagaa-Phonological Process
	\item Agarsiistota-Demonstratives
	\item Ajaja-Gerund
	\item Akkaatee-Manner
	\item Aldhumataa-Infinite
	\item Algosagaloomuu-Dissimilation
	\item Danbalii-Acoustic
	\item Dhaggeeffannoo-Auditory
	\item Dubbannoo-Articulatory
	\item Amarteessuu-Rounding
	\item Ammeennaa-Simple Present Tense
	\item Amsiqaa-Present Continuous Tense
	\item Antima-Object
	\item Bamaqaa-Pronoun
	\item Beekamaa-Definite
	\item Caasaa Birsagaa-Syllable Structure
	\item Caasima-Syntax
	\item Caasluga-Grammar
	\item Caasluga Maddaa-Generative Grammar
	\item Caasluga Tajaajilaa-Functional Grammar
	\item Cimdii Sagalee/Dhamsagaa-Minimal Pairs
	\item Cufaa-Coda
	\item Dabrennaa-Simple Past Tense
	\item Dhalatoo-Derivation
	\item Dhokataa-Indefinite
	\item Dhumataa-Finite
	\item Duranaa-Future Tense
	\item Eentaa-Positive
	\item Faca'insa-Distribution
	\item Firsaga-Allophone
	\item Funyeessuu-Nasalizaion 
	\item Garlamee/Waliyoo-Reciprocal
	\item Giddugala-Middle
	\item Gochima-Verb
	\item Gosagaloomii/Firoomii-Assimilation
	\item Handhuura-Nucleus
	\item Haquu-Deletion
	\item Harsasseessuu-Velarization
	\item Hi'eentaa-Negative
	\item Hidheessuu-Labialization
	\item Hirkataa-Dependent
	\item Irrabuta-Cluster
	\item Irradeebii-Reduplication
	\item Kennaa-Dative
	\item Kutima-Predicate
	\item Laagessuu-Palatalization
	\item Hammamtaa-Ordinal Numbers
	\item Hiikcaasima-Functional Grammar
	\item Meeqantaa-Cardinal Numbers
	\item Maayii-Case
	\item Matima-Subject
	\item Mitabgocha-Non-agentive
	\item Mitixiixaa-Voiceless
	\item Meeshaa-Instrument
	\item Murannaa-Future Definite
	\item Muranaala-Future Indefinite
	\item Muuxataa-Experiencer
	\item Qabeentaa-Posessive
	\item Qooqadhabuu-Devoicing
	\item Qooqafudhuu-Voicing
	\item Raawwatamaa-Passive
	\item Raawwatamaa Dhuunfaa-Personal Passive
	\item Raawwatamaa Mitidhuunfaa-Impersonal Passive
	\item Raawwima Ammeennaa-Present Perfect Tense
	\item Raawwima Dabrennaa-Past Perfect Tense
	\item Ramaddii-Person
	\item Saaguu-Ephenthesis
	\item Sagalee Dubbii- Speech Sound
	\item Saaqxuu-Onset
	\item Taasisuu-Causative
	\item Taasisuu Hafoo-Intransitive Causative
	\item Taasisuu Lammeessoo-Double Causative
	\item Taasisuu Mitidhuunfaa-Impersonal Causative
	\item Taasisuu Qeentee-Single Causative
	\item Taasisuu Sadeessoo-Triple Causative
	\item Tarree Gochimaa-Serial Verbs
	\item Tarsiqa-Past Continuous Tense
	\item Tumsii-Helping Verb
	\item Waldarbuu/Waljaafuu-Metathesis
	\item Walmorkii-Contrast
	\item Xiixaa-Voiced
	\item Xindhamjecha-Morphology
	\item Xindhamsaga-Phonology
	\item Xinsaga-Phonetics
	\item Yaaxina-Theory
	
	
\end{itemize}


\newpage
\thispagestyle{empty}
\renewcommand{\contentsname}{Baafata}
\tableofcontents

\newpage
	


\chapter{Seensa}
\setlength{\parindent}{3em}

Kitaabni kun yeroo dheeraa keessatti qophaa’e. Qophii isaaf dhiibbaa kan taasisaan keessaa inni guddaan fedhii barattoota kooti. Kaayyoon kitaaba kanaa caasluga AO gadifageenyaan
addeessuudha. Faayidaan isaas barnoota AO digrii lammaffaaf sadaffaaf akka kitaaba barnootaatti tajaajiluudha. Qabiyyeen isaas gadfageenya qaba; bu’uurrii isaas hojiiwwan hayyoota
biyya keessaafi alaati.

Qophiin kitaaba knaa waggoota dheeraa fudhateera. Wixineen kitaaba kana yeroo jalqabaaf kan qophaa’ee bara 1990ti. Yeroo sana digrii lammaffaatiin eebbifameen sirna
barnootaa keessatti akka eksipartii ta’ee osoon hojjechaa jiruu hanqinni caasluga AO jiraachuu isaa hubadheen wixinee kitaaba caasluga AO qopheesse. Matadureen kitaabichaas,
“Seera Afaan Oromoo” kan jedhuu ture. Wixinichis sagaleewwan, dhamjechootaafi caashima AO qaacceessa ture. Yeroo muraasa booda Univarsitii Finfinnee keessatti AO
barsiisuuf carraan argadhe. Wixinichis ittiin barsiisuuf na gargaare. Adeemsa keessas duubdeebii barattootaafi kitaabilee dabalataa biyya alaafi waraqaa qorannoo biyyaa
keessaa dubbisuun wixinicha fooyyeessuun itti fufe. Itti aansee, bara 1996tti qorannoo Ph.D. gaggeessuuf biyya Awuroopaa deemuuf carraan argadhe. Qorannoon koos dhalatoo
gochimaa Afaanota Itoophiyaa keessatti argaman (Verbal Derivation in Ethiopian Afro-Asiatic Languages) irratti xiyyeeffata ture. Dhimma qorannoo kanaaf afaanota handhuura ta’an keessaa AO isa jalqabaa ture. Carraan qorannoo kunis caashimaafi hiika AO gadfageenyaan akkan hubadhu baayyee na gargaare. Akkan Ph.D. koo xumuree biyyatti deebi’eenis wixineen qopheessaa ture keessaa hamma tokko gara kitaabatti jijjiiree ittiin barsiisuun jalqabe. Haaluma kanaan kitaaba “Seera Afaan Oromoo I: Dhamsagaafi Dhamjeha” jedhu barreesseen Univarsitiin Finfinnee keessatti waajira qorannoo akka barattootaaf maxxansiisuun gaafadhe. Waajirichis kitaabicha bara 2003tti ji’a Eeblaa keessa maxxansiisee barattootaa digrii jalqabaaa barataniif raabse
(barattoonni digirii jalqabaaf hamma ammaatti kitaabichaan barachaa jiru). Itti aansee kitaaba barnootaa “Caacculee Shanan Dubbisuu Afaan Oromoo” jedhu bara 2009tti ji’a
waxabajjii keessa maxxansiiseen barattootaa digrii lammaffaaf AO barataniifan dhiyeessee (barattoonnis kitaaba kanaan barachaa jiru). Kitaabni “Caasluga Afaan Oromoo” jedhu kun
isa sadaffaa ta’uu isaati. Kitaabni kun qabiyyeewwan kitaabilee duraan qophaa’an of keessatti hammachuu irra darbee dhimmoota dabalataa hedduu qaba; akkasumas bal’inaafi gad
fageenya qaba.

Kitaaba kana akkan qopheessuuf kan na kakaasan sababoota addaddaatu jiru. Sababni inni jalqabaa fedhii dhuunfaakooti. Wixinneen caasluga AO guutuun isaa inni adeemsa keessa
fooyya’aa tureeru harkakoo jira. Kanaaf wixinee kana gara kitaabaatti jijjiiree barsiisotaafi barattootaa biraan ga’uuf fedhiin qabu galmaan ga’uuf jecha kitaabni kun qophaa’e.
Sababni biroon hir’ina caasluga AOti. Caaslugni AO haala ammayyoomeen guutuu ta’ee afaanichaan barreeffamee gabaarras ta’e manneen kitaabaa keessatti hinargamu. Kana
jechuun caaslugni AO hojiiwwan beektota biyya alaafi biyya keessaa AO irratti hojjetaman irratti bu’uureffatee akkasumas yaaxina irratti hundaa’ee kan dhiyaateeru hinjiru jechuudha.
Hir’ina kana hamma tokko guutuuf jecha kitaabni kun qophaa’e. Sababni biroon gaaffiilee namootaati. Keessumattuu barattoon digrii lammaffaaf AO baratan kitaaba caasluga AO ilaalchisee
hir’inni akka jirufi akkan ani kitaaba caaslugaa kana qopheessu irra deebii’dhaan na gaafachaa turaniiru. Kanaaf gaaffii namootafi barattootakoo na gaafachaa turaniif hamma tokko
deebii kennuuf kitaabni kun qophaa’e.

Kaayyoon kitaaba kanaa caasluga AO addeessuudha. Kana jechuun sagaleewwan AO (dubbachiiftotaafi dubbifamtoota) ni’ibsa. Xindhamsaga AO beektota biyya alaafi keessaa irratti
hundaa’uun gadfageenyaan ni addeessa. Xindhamjecha AO ni addeessa. Himni maal akka ta’eefi akaakuuwwan himootaa ni addeessa. Akkasumas yaaxinoota maddaafi tajaajilaa irratti
bu’uureffachuun caashima AO niqaaccessa.

Faayidaan kitaaba kanaa inni guddaan akka kitaaba barnoota caasluga AOtti fayyaduudha. Kessumattu xiyyeeffannoon isaa Univarsitii keessatti digirii lammaffaafi sadaffaaf kitaaba
barnootaa ta’ee gumaacha ol aanaa kenna. Akkasumas digrii jalqabaa, leenjii barsiisotaafi barnoota sadarkaa lammafaaf akka kitaaba wabii ta’ee tajaajiluu danda’a.

Kitaabni kun kutaalee gurguddoo sadii qaba. Kutaan tokko dhimmoota Xinsagaafi Xindhamsaga dhiyeessa. Qabiyyeen kutaan kanaa kitaaba “Seera Afaan Oromoo I: Dhamsagaafi Dhamjecha” jedhu waliin wlitti siqa. Garuu inni kun gadfageenya qaba. Haaluma kanaan kutaan kun boqonnaawwan lamatti qoodama. Isaanis boqonnaa Xinsagafi boqonnaa xindhamsagaati. Boqonnaan xinsagaa sagaleewwan AO bakka kamitti akka uumaman, haala kamiin akka uumaman, haala dibbee sagalee, sosso’ina arrabaa, saaqama afaaniifi amartaa’uu/diriiruu hidhii qaaccessa. Boqonnaan Xindhamsagaammoo hojiiwwan beektota biyyaa keessaafi alaa kanneen akka\cite{lloret1988gemination,griefenow2001grammatical,owens1985grammar,tolemariam2011,wako1981,biniyam1988,kebede1994}fi kanneen biroo irratti bu’uureffata. Boqonnichi qindoomina dubbifamtootaafi
dubbachiiftotaa qaacceessa. Sagaleen cimdii tokko jechuun maal akka ta’e, dhamasagniifi firsagni maal akka ta’an adda baasee ibsa. Akkasumas irra butaa, adeemsota dhamsagaa
kanneen akka gosagalomii garduree, garduubee, guutuufi gamisa addeessaa. Akkasumas adeemsota dhamsagaa addeessee seera adeemsa dhamsagaatiin ibsa. Kana malees
akaakuuwwan birsagaafi caasaa birsagaa AO addeessa.

Kutaan lama dhimmoota Xindhamjechaafi himaa dhiyeessa. Kutaan kun boqonnaawwan sagalitti qoqqoodama. Kutaan lama hojiiwwan beektota biyya keessaafi alaa akka\cite{abera1982,temesgen1993,Addunya2018,owens1985grammar,gragg1976oromo,beekamaa1996,griefenow2001grammatical,file2015,baye1988focus,tolemariam2011,Abdusamad1994,aadaa1995}fi kanneen biroo ka’umsa godhateera. Kutaan kun jalqabarratti dhamjechi maal jechuu akka ta’e addeessa. Boqonnaan maqaa dhimmoota addaddaa kanneen akka maqaa waloo, maqaa dhalatoo, maqaa uumamteefi maqaa dhuunfaafi fufilee maqaa irratti fufaman
addeessa. Boqonnaawwan itti aananii dhufan maqibsa, agarsiistotaa, lakkoofsafi bamaqaalee addeessa. Boqonnaa gochimaa keessatti gochima ce’aa, hafoo, raawwatmaa, giddugalaafi fufilee gochimarratti fufaman gad fageenyaan ibsamaniiru. Akkasumas kutaan kun jechoota tajaajilaa, gochimibsafi hima dhiyeessa.


Dhumarrattis kitaabni barnootaa tokko looga waalta’e bu’uura godhachuu qaba. Qophiin kitaaba kanas hamma tokko looga AO waalta’e bu’uura godhateera. Hamma danda’ametti
jechootaaf hojiiwwan\cite{aadaa1995}fi kanneen biroo faayidaarra oolaniiru. Haa ta’u malee ilaalcha sabdaneessaan loogni kamuu moggeeffamuu hinqabu. Haaluma kanaan
kitaaba kana keessatti loogotni AO gara garaa faayidaarra oolaniiru. Looga Harar, looga Maccaa, Kamisee, Tuulamaafi Booranaa hammatamaaniiru\cite{biniyam1988,griefenow2001grammatical,kebede1994}.


\chapter{Xinsaga}
\setlength{\parindent}{3em}
\subsubsection{Qabiyyee}
\begin{itemize}
  \item Dubbifamtoota
  \item Dubbachiiftota
  \item Qaama Sagalee Dubbii
\end{itemize}

\subsubsection{Gaaffilee Ka'umsaa}
\begin{enumerate}
  \item Xinsaga\index{xinsaga} Danbalii\footnote{Gaaleen 'Xinsaga Danbalii' jedhu hiika gaalee Afaan Inglizii 'Acoustic Phonetics' jedhuuf kan keennamedha.} jechuun maal jechuudha?
  \item Xinsaga Dhageeffannoo\footnote{Gaaleen 'Xinsaga Dhaggeeffannoo' jedhu hiika gaalee Afaan Inglizii 'Auditory Phonetics' jedhuuf kan kennamedha.} jechuun maal jechuudha?
  \item Xinsaga Dubbannoo\footnote{Gaaleen 'Xinsaga Dubbannoo' jedhu gaalee Afaan Inglizii 'Articulartory Phonetics' tiif kennamedha.Kitaaba tokko keessatti 'xinsaga uumamsaa, darbiinsaafi dhageetti jedhamaniiru \cite[p.39]{Addunya2018}} jechuun maal jechuudha?
\end{enumerate}

Xinsaga jechuun maal jechuudha? Saayinsiin xinqooqaa\index{xinqooqa} sagaleewwan dubbii\index{sagaleewwan dubbii} afaan tokko keessatti fayyadan ittiin adda baasaanii baran yookiin barsiisan xinsaga jedhama. Xinsagni sagaleewwan afaan tokko keessa jiran adda baasee hubachuuf dubbii ykn barruu, hima, jechaafi birsaga xiinxala. Dhimmi isaas sagaleewwan afaan tokko keessa jiran sirriitti adda baafachuudha. Dubbii ykn barruu keessatti sagaleen\index{sagalee} qaama xiqqaadha\index{qaama xiqqaa}; kana jechuunis sadarkaa dubbiitti sagaleen tokko qaama birootti hinqoqqoodamu ykn hincaccabu jechuudha. Sagaleewwan walitti makamuun birsaga\index{birsaga} uumu. Birsagoonni ammoo walitti makamuun jecha\index{jecha} uumu. Jechoonni walitti makamuun hima\index{hima} ijaaru. Himoonni walitti makamuun barruu\index{barruu} ijaaru. Barruu alphabetii keessatti sagaleen tokko qubee\index{qubee} tokkoon bakka buusama. AO\index{AO} keessatti qubeen tokko sagalee tokko bakka bu’a (kun qubee dachaa\index{qubee dachaa} hin’ilaallatu). Fakkeenyaaf,jecha mana jedhu keessatti qubeewwan /m/, /a/, /n/fi /a/n walitti makamuun jecha tokko uumaniiru.

Qorannoon xinsagaa dhaggeeffachuu irraa ka’a. Barruu dhaggeeffatanii himatti qoqqooduu\index{qoqqooduu}, hima dhaggeeffatanii jechatti qoqqooduu, jecha dhaggeeffatanii birsagatti qoqqooduufi birsaga dhaggeeffatanii sagaleetti qoqqooduu gaafata. Fakkeenyaaf hima armaan gadii haa ilaallu:
Ganamoon hoolaa adii gabaatii bite.
Hima armaan olii kana jechatti yoo qoqqoodnu akka armaan gadii ta’a:
Ganamoon-hoolaa-adii-gabaatii-bite.
Jechoota hima armaan olii keessa jiran ammoo yoo birsagatti
qoqqoodnu akka armaan gadii ta’a:
Ga-na-moon-hoo-laa-a-dii-ga-baa-tii-bi-te.
Birsagoota armaan olii gara sagaaleetti yoo qoqqoodnu ammoo akka armaan gadii ta’a:
G-a-n-a-m-oo-n-h-oo-l-aa-a-d-ii-g-a-b-aa-t-ii-b-i-t-e.

Toftaan dubbii ykn barruu qoqqooduu kun himicha keessa sagaleewwan meeqa akka jiran adda baasee agarsiisa. Barruun tokko sagaleewwan maaliin akka ijaarame adda baafachuun
ga’aa miti. Barataan xinsagaa tokko, sagaleewwan adda baafate sanneen sagaleesse hubachuu qaba. Kana jechuun sagaleewwan baafachuun barbaachisa jechuudha. Kunis toftaa mataa isaa
qaba. Fakkeenyaaf jecha Ganamoon jedhu fudhannee jalqaba, gidduufi dhuma jechaatti sagalee maal akka dhaggeeffanne dubbannee addaan baafachuu keenya mirkaneeffachuu qabna.

Waliigalaan, xinsagni karaalee sadii sagaleewwan qorata. Karaaleen kunneenis amala danbalii\index{amala danbalii}, amala dhageeffannoofi\index{amala dhaggeeffannoo} amala dubbannooti\index{amala dubbanno}. Amalli danbalii sagaleen afaan nama dubbatu irraa ka'ee hamma gurra nama isa dhageeffatuutti wayita qilleensa keessa darbu amala maalii akka qabaatu qorachuu irratti xiyyeeffata. Amallii dhageeffannoo ammoo sagaleen erga gurra nama isa
dhageeffatuu ga’ee booda ergaan akkamiin gara sammuutti akka darbu hubachuu irratti xiyyeeffata. Amalli danbaliifi amalli dhageeffannoo meeshaa sagalee qoratutti fayadamuun
malee haala salphaatiin hinhubataman. Amallii dubbannoo garuu meeshaa malee hubatama. Amala dubbannoo kanas kutaawwan itti aananii dhufan keessatti gad fageenyaan ilaalla.

\section{Qaamolee sagalee dubbii}

Qaamni sagalee dubbii sagalee uumuuf tajaajila. Qaamni kun sagalee uumuuf qilleensa somba keessaa gara diidaatti ba’uun fayyadama. Akkuma afaanota
hedduu addunyaa kana irrattii argamanii AOs sagalee dubbii kan uumu afuura \index{afuura} somba keessaa gara diidaatti ba’uun haa ta’u malee, qilleensa diidaa gara sombaatti galuunis sagaleen
ni’uumama. Qilleensi somba keessaa gara diidaatti wayita ba’u qaama sagalee dubbii \index{qaama sagalee dubbii} uumu waliin walquunnamtii adda addaa taasisa. Qaamni sosso’ina adda addaa gochuudhaan sagalee
adda addaa uuma. Qaamni sagalee dubbii uumu bakka adda addaatti qoodamee ilaalamuu danda’a.Qaamni sagaleewwan dubbii uuman kokkee,\index{kokkee} qoonqoo,\index{qoonqoo} daandii afaaniifi \index{daandii afaanii} daandee funyaaniiti\index{daandii funyaanii}.
\begin{itemize}
\item[•]Kokkee: Qilleensi somba keessaa gara diidaatti bahu yeroo calqabaaf kan inni quunnamtii godhu kokkee waliini. Kokkeen lafee saanduqa fakkaatu ta’ee qaama qilleensa gara sombaatti
galshuufi baasu ofirraa qaba. Qaamni qilleensa gara sombaatti galshuufi somba keessaa baasu kunis nicufama; nibanamas. Qaamni cufamee banamu kun dibbee sagalee (goongira)
jedhama. Dibbeen sagalee kun ribuuwwan qaqal’oodha. Ribuuwwan kunneen yeroo banamanii qilleensa gadi lakkisan amaloota lama agarsiisu. Amalli inni tokko banamanii qilleensa
gidduusanii osoo hinxiixiin hulluuqsisuudha. Amalli inni lamaffaan ammoo banamanii hollachaa ykn xiixaa qilleensa gadi lakkisuudha.Qoonqoo irratti sagaleewwan uumaman jiru. Kana
malees qoonqoon nyaanniifi dhugaatiin gara sombaatti akka hinseenne eegumsa godha. Kana jechuunis nyaanni gara garaachaatti akka goru taasisa jechuudha.

\item[•] Daandii Afaanii: Bakki qilleensi somba keessaa bahe jijjiirama guddaa itti agarsiisu yoo jiraate daandii afaaniiti. Afaan keessatti sosso’inni arrabxiqqee, hidhii jalaafi sosso’inni
arrabaa ni mul’ata. Tajaajilli arrabxiqqee bal’inaan waan mul’atee ibsamu miti. Hidhiin jalaa garuu hidhii irraafi ilkaan irraa waliin walquunnamtii adda addaa taasisuudhaan
sagaleewwan adda addaa uuma. Hunda caalaa sagalee dubbii uumuuf qaamni iddoo guddaafi sosso’ina bal’aa qabu arraba. Haala kanaan arrabni bakkeewwan afuritti qoqqoodamee
ilaalama.Isaanis fiixee arrabaa, fuuldura arrabaa, qixxelamaan arrabaafi duuba arrabaati. Arrabni gara fuulduraafi gara duubaatti sosso’ee, ol ka’ee gadis bu’ee sagaleewwan adda
addaa uumuuf tajaajila.

\item[•] Daandii Funyaanii: Qaawwi funyaanii qoonqootii kaasee hamma funyaaniitti kan deemu daandii holqa fakkaatu qaba. Duubni daandii afaanii qaama lallaafaa duuba harsassee
jedhamu qaba. Duubni harsassee Jun daandii funyaanii cufuufi banuuf tajaajila. Duubni harsassee gara boodaatti yeroo harkifamu qilleensi somba keessaa ba’u hundi karaa dandii
afaanii akka ba’uuf dirqama. Duubni harsassee yeroo gara fuulduraatti deebi’ummoo daandiin funyaanii nibanama. Yeroo kana qilleensi somba keessaa gara diidaatti yaa’u karaa daandii
funyaanii bahuudhaaf carraa argata. Qilleensi karaa daandii funyaanii gara diidaatti bahu unis sagalee dubbii uumuuf tajaajila.

\end{itemize}


\section{Dubbifamtoota}

Dubbifamtoonni\index{dubbifamtuu} akkamiin uumamu? Walumagalaan uumamni dubbifamtootaa ulaagaa irratti hundaa’a.  Ulaagaawwan kunneenis dibbee sagalee, bakka itti sagaleen uumamuu, akkaataa itti sagaleen uumamuufi kallattii daandii qilleensaati (fakkii fuula 12 irraa ilaalaa). Qabxiiwwan kanneennis akka armaan gadiitti ilaalla: 

\begin{itemize}

\item[•] Dibbee Sagalee\index{dibbee sagalee}: Duraan dursinee dibbeen sagalee banamee akka cufamu ibsineerra.  Dibbeen sagalee banamee qilleensa ofkeessa hulluuqisisee yeroo dbarsu 
sagaleen uumamu mitxiixaa/qooqa dhabeessa jedhama.  Dibbeen sagalee banamee qileensa of keessa hollachaa ykn hurgufamaa yeroo dabarsuu sagaleen uumamummoo xiixaa/qooqa qabeessa jedhama. Yeroo tokko tokkommoo dibbeen sagalee cufame tasuma banamuudhaan dho’ee sagaleen akka uumamu taasisa.  Sagaleen haala kanaan uumamus amala dho’uu waan agarsiisuuf dhootuu jedhamee waamama.  Dimshaashumatti haala dibbeen saglaee qilleensa somba keessaa gara diidaatti bahu waliin walquunnamtii godhuun sagaleewwan AO bakkoota sadiitti qooduu dandeenya.  Isaanis miti xiixaa, xiixaafi dhootuu dha. 

\item[•] Bakkatti Sagaleen Uumamu\index{bakka sagaleen itti uumamau}: Sagaleewwan AO keessatti argaman bakkoota jaharratti uumamu.  Bakka sagaleen itti uumamu jechuunis qaamonni sagalee dubbii warri sosso’aniifi warri hinsossoone walxuqanii bakka sagalee itti uuman jechuudha. 
Hidh-Lamee: Sagaleen hidh-lamee jedhamu sagalee yeroo hidhiin irraafi hidhiin jalaa walxuqan ykn bay’ee walitti siqan uumama. AO keessatti sagaleewwan bakka kanatti uumaman [ph, b, p, m, w] dha.  [p] n sagalee ergisaati.

\item[•] Hidh-Ilkee\index{hidh-ilkee}: Sagaleen hidh-ilkee kan uumamu yeroo hidhiin jalaafi ilkaan irraa wal xuqan, ykn baay’ee walitti siqanii hidhiin jalaafi ilkaan irraa yeroo walitti maxxanan. Yeroo kana gidduu isaanii afuura hulluuqsisanii baasu. Sagaleen yeroo kana uumamu [f] fi [v]dha. [v] n sagalee ergisaati. 

\item[•] Irgee: Sagaleen irgee yeroo arrabniifi irgeen walxuqan uumama.  Irgeefi fuldurri arrabaa walitti siquudhaan qilleensa gidduu isaanii loosanii dabarsuudhaan, ykn cufamuudhaan sagalee uumu.  Arrabniifi irgeen yeroo walxuqan sagaleewwan uumaman [x, d, t, dh, s, z, n, l, r] dha.  

\item[•] Laagee\index{laagee}: Sagaleen laagee kan uumamu yeroo qixxelamaan arrabaa laagee waliin walxuqee qilleensa gara diidaatti yaa’u darba dhowwu.  Sagaleewwan laagee kanneen AO keessatti argamani [sh, ch, j, c, ny, y] dha.  

\item[•] Harsassee\index{harsassee}: Sagaleen harsassee yeroo duubni arrabaa ol ka’ee harsassee waliin walxuqu uumama.  Sagaleewwan harsassee [k], [g] fi [q]dha.

\item[•] Qoonqoo\index{qoonqoo}: Qilleensi somba keessaa bahu qoonqoo keessatti karaan itti cufamee yeroo gad lakkifamu sagaleen uumamu sagalee qoonqooti. 
AO keessaa sagaleewwan qoonqoo lama; Oceanus hudhaa['(hudhaa)] fi [h] dha.

Sagaleewwan afaan tokko keessa jiran adda addaan baasanii baruuf qooqa sagalee sanaafi bakka itti sagaleen sun uumamu qofaa baruun gahaa miti.  Fakkeenyaaf sagaleewwan hidh-lamee lama haa ilaallu.  [b] fi [m]n qooqa qabeeyyiidha.  Akkasumas sagaleewwan kunneen lamaanuu sagaleewwan hidh-lameeti.  Kana jechuunis sagaleewwan kunneen bakka tokkotti uumamu jechuudha.  Karaa kallattii qilleensaa yoo ilaalles sagaleewwan lamaanuu qilleensa somba keessaa gara diidaatti bahuun uumamu.  Kanaaf sagaleewwan kunneen karaa qooqaa, karaa bakka itti uumamsaafi karaa kallattii qilleensaatiin addaan hinbahani.  Sagaleewwan kanneen addaan baasuuf ulaagaawwan  biroon barbaachisa; innis akkaata itti sagaleen uumamu hubachuudha. 

\item[•] Cufaa\index{cufaa}: Sagaleen cufaa kan uumamu qilleensi somba keessaa gara diidaatti bahu bakka itti uumamurratti guutumaan guutuutti dhowwamee yeroo gadi lakkifamu.  Sagaleewwan cufaa [b, p, ph, d, dh, t, x, g, k, q,’(hudhaa)] dha. 

\item[•] Lootuu\index{lootuu}: Sagaleen lootuu kan uumamu bakkoonni sagalee uuman lamaan walitti siqanii qilleensa gidduu isaanii loosanii yeroo dabarsani.  Sagaleewwan lootuu AO keessatti argaman [f, s, v, z, sh, h] dha.

\item[•] Rigataa\index{rigataa}: Sagaleen rigataa kan uumamu arrabni bakka itti sagaleen uumamu xuqee suuta suutaan bakka sana qilleensa somba keessaa dhufuuf yeroo gadhiisu.  Haala kanaan [ch] fi [j] tu uumamu. 

\item[•] Funyee\index{funyee}: Sagaleen funyee kan uumamu duubni harsassee banamee qilleensi somba keessaa bahu karaa afaaniifi karaa daandii funyaanii akka darbu yeroo taasisu.  Sagaleewwan funyee sadii qofaadha.  Isaanis [m], [n] fi [ny] dha.  Sagaleewwan kunneen yeroo sagaleeffamani qilleensi karaa daandii funyaanii darbuuf dirqama.  Namni tokko funyaansaa cuqqaaluudhaan sagaleewwan kanneen sirriitti sagleessuu hindandahu. 

\item[•] Maddee\index{maddee}: Sagaleen maddee kan uumamu fiixeen arrabaa gara irgeetti ol ka’ee karaa cufee, qileensi karaa maddii arrabaa yaa’ee sagalee akka uumu yeroo taasisu.  Sagaleen maddee AO keessatti argamu [l] dha. 

\item[•] Rom’aa\index{rom'aa}: Rom’aan kan uumamu fiixeen arrabaa irgee xuqee ykn xuxxuqee yeroo deebihu.  Sagaleen rom’aa kun [r] dha. 

\item[•] Gamduubee\index{gamduubee}: Sagaleewwan gamduubee amala dubbachiiftotaafi amala dubbifamtootaa qabu.  Sagaleewwan Kunneen yeroo sagaleeffamani akka dubbachiiftotaa afaan nama bansiisu.  Garuu akka dubbifamtootaa ofiisaaniitin of dandahanii hin dhaabbatani.  Kunis dubbifamtoota wajjin walisaan fakkeessa.  Sagaleewwan kunneen [w] fi [y] dha. 

\end{itemize}

\section{Dubbachiiftota}

Akkuma beekamutti afaan tokko keessatti madabiiwwan sagaleewwan kanneen bu'uura tahan lamatu argamu.  Madabiiwwan kunneenis dubbifamtootaafi dubbachiiftota\index{dubbachiiftuu}.  Akka kutaa darbe keessatti hubannetti sagaleewwan  dubbifamtuu tahan wayita uumaman qilleensi somba keessaa  gara diidaattii yaa'u afaan keessatti bakkeewwan adda addaa  irratti dhowwamee gadhiifama. Wayita sagaleewwan  dubbachiiftuu tahan uumaman garuu qilleensi somba keessaa  gara diidaatti yaa'u bakka kamittuu hindhowwamu. Kanaaf  jecha dubbachiiftota bakkeewwan sagaleewwan itti uumamaniif  akkaataa sagaleewwan itti uumaman irratti hundoofnee adda  addaa qooduu hindandeenyu. Haala dibbee sagalee\index{dibbee sagalee} irratti  hundoofnees dubbachiiftonni xiixaa\index{xiixaa} dha ykn miti xiixaa\index{mitixiixaa} dha  jechuu hin dandeenyu; sababni isaas dubbachiiftonni hundi  xiixaa waan tahaniif. Dubbachiiftonni karaawwan sadii ibsamuu  dandahu: sosso'ina arrabaa, qaama arrabaafi haala hidhiiti.  Wayita dubbachiiftuun tokko uumamu sadarkaan sosso'ina  arrabaa\index{sosso'ina arrabaa} ol\index{ol}, gidduu\index{gidduu} ykn gad\index{gad} tahuu dandaha. Ol jechuun arrabni  ol ka'eera jechuudha; gidduu jechuunammoo arrabni baayyee soo baayyee ol hinfagaatiin, osoo gadis hinbu'iin  giddugaleessairratti kan uumamu jechuudha; gad  jechuunammoo arrabni wayita gad bu'ee dubbachiiftuu sana  uumu mul'ata (fakkii fuula 13 irratti kenname ilaalaa). 

Qaama arrabaa jechuunammoo wayita sagaleen sun uumamu  qaama arrabaa\index{qaama arrabaa} isa caalmaatiin sosso'u ilaallata. Qaamni  arrabaa fuldura\index{fuuldura}, qixxelamaanfi\index{qixxelamaan} duuba\index{duuba} jedhamee bakkeewwan  sadiitti qoodama. Haalli hidhii diriiraa ykn ammoo amartii  taha. Fakkeenyaaf dubbachiiftonni [u] fi [i]n wayita uumaman  arrabni baayyee ol ka'a. Haa tahu malee qaamni arrabaa ol  ka'u sagaleewwan lamaaniifuu adda adda. [u]n wayita  uumamu qaamni arrabaa inni ol ka'u duuba arrabaa yoo tahu,  [i]n wayita uumamuummoo qaamni arrabaa inni ol ka'u  fuuldura arrabaati. [e] fi [o]n wayita uumaman ol ka'umsi  arrabaa giddugaleessa taha. [e]n wayita uumamu qaamni  arrabaa kan giddugaleessa tahu fuuldura araabaa yoo tahu,  [o]n wayita uumamu ammoo qaamni arrabaa kan bakka  giddugaleessa qabatu duuba arrabaati. Dubbachiiftuu [a]n  wayita sagaleessamu qixxelamaan arrabaa gad bu'a. 

Dubbachiiftonni hidhii naannessiisani [u] fi [o] qofa.  Dubbachiiftonni warri hafan hundi wayita sagaleessaman hidhii  hinnaanneessiisan. Sosso'ina aarrabaafi qaama arrabaa irratti  bu'uureffachuun dubbachiiftota AO gabateetiin kutaa dhufu  keessatti ilaalla. 

\subsubsection{Gaaffilee Boqonnichaa}

\begin{enumerate}
  \item Qaamni sagalee dubbii uuman maalfa’i?
  \item Bakka itti sagaleewwan AO uumaman tarreessi.
  \item Haala itti sagaleewwan AO uumaman tarreessi.
  \item Tajaajilli dibbee sagalee maali?
  \item AO keessatti dubbifamtoonni ergisaa kam fa’i?
  \item AO keessa dubbachiiftonni haal kamiin akka uumaman ibsi.
  \item Hudhaan akka dubbifamtuu tokkootti ilaalamaa? Maaliif?
  \item Fakkii qaama sagalee dubbii kaasiitii maqaa isaanii  barreessi.
  \item Gabatee sagaleewwan AO kaasiitii sagaleewwan keessatti  guutii barreessi.
  \item Sagalee AO keessa jiru kan Afaan gara biroo ati beektu  tokko keessa hinjirre ibsi.
  \item Sagalee Afaan gara biroo ati beektu tokko keessa jiru,  garuu kan AO keessa hinjirre tokko ibsi.
  \item Sagaleewwan AO keessas Afaan gara biroo ati beektu tokko  keessa jiran ibsi.
  \end{enumerate}

\chapter{Xindhamsaga}
\setlength{\parindent}{3em}
\subsubsection{Qabiyyee}

\begin{itemize}
  \item Addeessa Dubbifamtootaa
  \item Addeessa Dubbachiiftotaa
  \item Dhamsaga
  \item Firsaga
  \item Adeemsa Dhamsagaa
  \item Birsaga
\end{itemize}
\subsubsection{Gaaffilee Ka'umsaa}
\begin{enumerate}
	\item Dhamsaga jechuun maal jechuudha?
	\item Firsaga jechuun maal jechuudha?
	\item Akaakuu adeemsa dhamsagaa AO ibsi.
	\item Caasaa birsagaa AO ibsi.
	\item Seera irra butaa ibsi.
	\item Seera sagaa AO ibsi.
\end{enumerate}

Hayyoonni sagaleewwan dubbii AO addeessaniiru\cite{griefenow2001grammatical,owens1985grammar,wako1981}. Kutaan kun 2.1.  keessatti addeessa dubbifamtootaa dhiyeessa. Akkasumas  faca’insa, jabinaafi irra butaa ibsa. Kana malees dhamsagaafi  firsaga irratti ibsa kenna. Kutaa 2.2. keessatti dubbachiiftota  addeessa. Dubbachiiftonni amaloota adda addaan, laafinaa,
dheerinaafi faca’insaan ni’ibsamu. Kutaa 2.3. keessatti  adeemsa dhamsagaa garagaraa dhiyeessa.
\section{Addeessa Dubbifamtootaa}

Kutaa kana keessatti ibsa dubbifamtuu, addeessa dubbifamtuu  cimdii tokko\index{sagalee cimdii tokko}, ibsa irra butaa\index{irra buta}, ibsa faca’insa dubbifamtuu\index{faca'insa dubbifamtootaa}, ibsa  dhamsagaafi\index{dhamsaga} firsagaafi\index{firsaga} ibsa jabina\index{jabaachuu} dubbifamtuu dhiyeessina.

Ibsi dubbifamtotaa bakka, haalaafi dibbee sagaleerratti  bu’uureffata. Fakkeenyonni sagaleewwan jalatti eeraman,  sagalichi jalqaba, gidduufi dhuma jechaatti argamuu isaa  mirkaneessu. Akka waliigalaatti garuu dubbifamtoonni AO  heddumminnaan dhuma jechaarratti hinargamani. 

AO sagaleewwan 58 qaba. Kanneen keessaa sagaleewwan 48 dubbifamtoota dha. Sagaleewwan 10 ammoo dubbachiiftota. Dubbifamtoota keessaa 28 sagaleewwan lafaanii barreeffamn. 

\begin{table}[H]
	\centering
	\caption{Dubbifamtoota AO Jabaatanii Barreeffaman}
	\begin{tabular}{c |c c c c c c c}
		\hline
		Haala & Hidhlamee & Hidhirgee & Irgee & Laagee & Harsassee & Qoonqoo \\
		\hline
		Cufaa \\
		 & bb & & dd & & gg & & \\
		 & pp & & tt & & kk & & \\
		 & & & xx & & qq & & \\
		 & & & &\cr
		\hline
		Lootuu \\
		 & & vv & zz & & \\
		 &  & ff & ss & & & &  \\
		 & & & &\\
		\hline
		Rigataa \\
		 & & & & jj \\
		 & & & & &\\
		 & & & & cc\\
		\hline
		Funyee & & nn \\
		\hline
		Maddee & & ll\\
		\hline
		Rom'aa & & rr\\
		\hline
		G.D. & ww & & & yy\\
		\hline		
	\end{tabular}
\end{table}

Sagaleewwan 20 ammoo jabaatanii barreeffamu. Sagaleewwan 7 dachaa\index{sagalee dachaa} waan ta'aniif akka sirna barreessuu AOtti jabaatanii hinbarreeffaman. Akkasumas hudhaafi\index{hudhaa} /h/n jabaatanii hinbarreeffaman.

\begin{table}[H]	
	\caption{Dubbifamtoota AO Laafanii Barreeffaman}
	\begin{tabular}{c c c c c c c c}
		\hline
		Haala & Hidhlamee & Hidhirgee & Irgee & Laagee & Harsassee & Qoonqoo \\
		\hline
		Cufaa \\
		 & b & & d & & g & & \\
		 & p & & t & & k & ' \\
		 & ph & & x & & q & & \\
		 & & & dh\cr
		\hline
		Lootuu \\
		 & & v & z & zy \\
		 &  & f & s & sh & & h\\
		 & & & ts\\
		\hline
		Rigataa \\
		 & & & & j \\
		 & & & & ch\\
		 & & & & c\\
		\hline
		Funyee & & & n & ny\\
		\hline
		Maddee & & & l\\
		\hline
		Rom'aa & & & r\\
		\hline
		G.D. & w & & & y\\
		\hline		
	\end{tabular}
\end{table}


Addeessi dubbifamtotaa bakka, haalaafi dibbee sagaleerratti  bu’uureffata. Fakkeenyonni sagaleewwan jalatti eeraman,  sagalichi jalqaba, gidduufi dhuma jechaatti argamuu isaa  mirkaneessu. Akka waliigalaatti garuu dubbifamtoonni AO  heddumminnaan dhuma jechaarratti hinargamani.
\begin{itemize}
  \item[p] Mitxiixaa, hidhlamee, cufaa. Sagalee ergisaati. Badaa  hinargamu. Fkn,poolisii-jalqaba jechaarratti argama; olompikii-gidduu jechaarratti argama
  \item[b] Xiixaa, hidhlamee, cufaa. Fkn, bani-jalqaba jechaarratti; gobaa-gidduu jechaarratti; gobbaa-gidduu jechaarratti jabaatee argama.
  \item[ph] Hidhlamee, dhootuu, cufaa. Fkn,phaphaasii- jalqaba jechaarratti; haphii-gidduu jechaarratti; hoph jedhe- dhuma jechaarratti argama.
  \item[m] Xiixaa, hidhlamee, funyee. Fkn, mana- jalqaba jechaarratti; lama- gidduu jechaarratti; lammii- jabaatee argama. 
  \item[w] xiixaa, hidhlamee, gamduubee. Fkn, waaqa- jalqaba jechaarratti; gawuu-gidduu jechaarratti; gowwaa- jabaatee argama.
  \item[t] xiixaa, irgee, cufaa. Fkn,tokko- jalqaba jechaarratti; atoo- gidduu jechaarratti; kottee-jabaatee argama. 
  \item[d] xiixaa, irgee, cufaa. Fkn, daree- jalqaba jechaarratti; raada- gidduu jechaarratti; guddaa-jabaatee argama.
  \item[x] Dhootuu, irgee, cufaa. Fkn, xurii- jalqaba jechaarratti; ixaa- gidduu jechaarratti; maxxannee- jabaatee argama.
  \item[dh] Gad dhootuu, irgee, cufaa. Fkn, dhaqi- jalqaba jechaarratti; hoodhu- gidduu jechaarratti; hodhi- jabaatee argama.
  \item[f] mitxiixaa, hidhilkee, lootuu. Fkn,fufaa- jalqaba jechaarratti; lafa- gidduu jechaarratti; of laaffise- jabaatee argama.
  \item[v] Xiixaa, hidhilkee, lootuu, hinjabaatu. Sagalee ergisaati.  Fkn, vidiyoo- jalqaba jechaarratti; televizyinii- gidduu jechaarratti argama.
  \item[s] Mitixiixaa, irgee, lootuu. Fkn,seene- jalqaba jechaarratti; isa- gidduu jechaarratti; keessa- jabaatee argama. 
  \item[z] Xiixaa, irgee, lootuu. Sagalee ergisaati. Fkn,zayituuna- jalqaba jechaarratti; muuzii- gidduu jechaarratti argama.
  \item[ts] Dhootuu, irgee, lootuu. Fkn, Tsahay -jalqaba jechaarratti argama.
  \item[n] Xiixaa, irgee, funyee. Fkn, nama- jalqaba; mana- gidduu jechaarratti; ganna- jabaatee argama.
  \item[r] Xiixaa, irgee, rom’aa. Fkn, rooba- jalqaba; moora- gidduu jechaarratti; irra- jabaatee argama.
  \item[l] Xiixaa, irgee, maddee. Fkn, lama- jalqaba; mala- gidduu jechaarratti; mulluu- jabaatee argama.
  \item[sh] Qooqa dhabeessa, laagee, lootuu. Fkn,shaashii- jalqaba; bishaan- gidduu jechaarratti; bishii- jabaatee argama.
  \item[zy] Xiixaa, laagee, lootuu. Sagalee ergisaati. Badaa  hinargamu. Fkn,televizyinii- gidduu jechaarratti argama.
  \item[ny] Xiixaa, laagee, funyee. Fkn, nyaate- jalqaba; keenya- gidduu jechaarratti; funyaan-dhuma jechaarratti argama. 
  \item[j] Xiixaa, laagee, rigataa. Fkn, joore- jalqaba; majii- gidduu jechaarratti; gamoojjii- jabaatee mul'ata.
  \item[ch] Mitxiixaa, laagee, rigataa. Fkn,bilcheessi- gidduufi bachoo- jabaatee mul'ata.
  \item[c] Dhootuu, laagee, rigataa. Fkn,caalaa- jalqaba; leenca- gidduu jechaarratti; loccuu- jabaatee mul'ata.
  \item[y] Xiixaa, harsassee, cufaa. Fkn, yoom- jalqaba; bayeessa- gidduu jechaarratti; biyya- jabaatee argama.
  \item[k] Xiixaa, harsassee, cufaa. Fkn, koree- jalqaba; ilkaan- gidduu- jechaarratti; lukkuu- jabaatee mul'ata.
  \item[g] xiixaa, harsassee, cufaa. Fkn, gadaa- jalqaba; muguu- gidduu jechaarratti; moggaa- jabaatee argama.
  \item[q] Dhootuu, harsassee, cufaa. Fkn, qaawwa- jalqaba; loqoda- gidduu jechaarratti; baaqqee- jabaatee argama.
  \item[’] Mitxiixaa, qoonqoo, cufaa. Darbee darbee  jabaata. Fkn, re'ee- gidduu; o’’e- jabaatee mul'ata.
  \item[h] Qooqa dhabeessa, qoonqoo, lootuu. Sagaleen kun  hinjabaatu. Fkn, harmee- jalqaba; baha- gidduu jechaarratti mul'ata.  
\end{itemize}

  \subsection{Cimdii Dhamsagaa}
  
Akka qajeelfamaatti, sagaaleewwan cimdii\index{cimdii dhamsagaa} tokko tartiiba  qubee walfakkaatu keessatti bakk waljijjiiranii/ walmorkatanii yoo addaddummaa hiikaa fidan cimdii tokko  jedhamu. Cimdii kana keessattis sagaleewwan kunneen adda  adda ta’anii addaddummaa hiikaa waan fidaniifis dhamsaga  jedhamu. Kaayyoon fakkeenyota jechoota armaan gadiis  sagaleewwan AO dhamsagoota ta’uu isaanii addeessuudhafi  mirkaneessuudha. Addaddummaan sagaleewwan kanneenis  bakka ykn haala sagaleen uumamu ta’uu danda’a. Akkasumas  addaddummaan haala dibbee sagalee (xiixaa ykn miti xiixaa  ta’uun) addaddummaa hiikaa fiduu danda’a. Sagaleewwan  cimdii tokko AO keessaa kanneen armaan gadii isaan  muraasa.
 
\begin{itemize}
        \item/m,b/ fkn, malaa– Dhullaan malaa qaba; balaa– Balaa adda addaarraa of eeguun gaariidha.
        \item/t,d/ fkn, teesse– Isheen eessa teesse? deesse– Isheen maal deesse? 
        \item/d,dh/ fkn,haada –Inni mataa haada; haadha –Inni haadha qaba. 
        \item/k,g/ fkn, kore - Gurbaan muka kore; gore - Gurbaan gara kee gore. 
        \item/g,q/ fkn, gara - Nuti gara manaa deemna; qara - Haaduun kun qara hinqabu. 
        \item/l,m/ fkn, gama - Tulluu gama manni jira; gala - Inni yeroon manatti gala. 
        \item/r,dh/ fkn, roobe - Halkan edaa cabbii roobe; dhoobe - Inni dhoqqee mukatti dhoobe. 
        \item/w,d/ fkn, waasii - Namni hidhame waasii waamee ba’a; daasii - Daasiin mana biratti ijaarama. 
        \item/y,g/ fkn, yaraa - Gaariifi yaraa adda baafachuun sirriidha; garaa - Kan garaa garaan haa beeku jedhu. 
        \item/m,g/ fkn, maraa - Maraa wadaroo lafa kaa’i; garaa - Kan garaa garaan haa beeku jedhi.  
        \item/j,d/ fkn, jiruu - Jiruu fi amatii hintuffatan jedhu; diruu - Lookoo mukatti diruu-n gaariidha. 
        \item/f,j/ fkn, fira - Namni namaaf fira; jira - Uummanni haala gaarii keessa jira. 
        \item/ny,n/ fkn, nyaate - Inni ciree nyaate; naate - Ati garuu maaliif naate?
        \item/c,t/ fkn, baacaa - Inni baacaa abbaa qarshiiti; baataa - Fardi kun collee, kunammoo baataa dha. 
        \item/sh,d/ fkn, shaamaa - Shaamaa uffachuun dur hafe; daamaa - Daamaa taphachuun gaariidha. 
        \item/z,w/ fkn, zayitii - Zayitii nyaataa eessaa bitattu? wayitii - AO torbanitti wayitii meeqa barattu?; 
        \item/p,q/ fkn, poostaa - Xalayaa poostaa keessa kaa’u; qoosxaa- Qoosxaa nyaachuun fayyaaf gaariidha. 
        \item/h,l/ fkn, haphee - Laaftoon haphee qaba; laphee - Lapheen qaama namaa keessaa tokko. 
        \item/r,f/ fkn, haruu - Lafa haruu-n qulqulleessuuf; hafuu - Beellama hafuu-n safuudha.
        \item/x,d/ fkn, fixe - Abbaltiikoo hojjedhee fixe; fide - Qarshii hamma kana eessaa fide? 
        \item/’,f/ fkn, oo’e - Oo’e jechuun gadoode jechuudha; oofe - Horii eessatti oofe? 
        \item/t,l/ fkn, gatii - Gatii-n nyaataa dabaleera; galii - Oomihsarraa garuu galii-n hindaballe.

\end{itemize}  

\subsection{Irra Buta}

Irra buta jechuun walitti aananii dhufaatii dhamsagoota gosa  adda addaa jechuudha. Dhaamsagoonni wayita waliin hiriiran  aantee qabu. Keessumattuu dubbifamtoonni yeroo walitti aananii dhufan dursaafi itti aanaa qabu.AO keessatti irra butaan\index{irra buta} akaakuuwwan 64 qaba. Isaan keessaa  47 sagaleewwan funyee, maddeefi rom’aa jalqaba irra butaa  taasisu\cite[p.20-21]{lloret1988gemination}:  
\begin{table}[H]
	\caption{Faca'insa Irra Butaa}
	\begin{tabular}{c c c c c c c c c c c c c c}
		\hline\hline
		& k & b & g & q & f & s & m & n & ny & l & r & w & y \\
		t & + & - & - & - & - & - & + & + & -  & + & + & - & -\\
		c & - & - & - & - & - & - & - & - & -  & + & - & - & -  \\
		k & - & - & - & - & + & - & - & + & -  & + & + & - & - \\
		' & - & - & - & - & - & - & + & + & +  & + & + & + & + \\
		b & - & - & - & - & - & - & + & + & -  & + & + & - & - \\
		d & - & + & + & - & - & - & - & + & -  & + & + & - & -  \\
		j & - & + & - & - & - & - & - & + & -  & - & + & - & -  \\
		g & - & - & - & - & - & - & - & + & -  & + & + & - & -  \\
		p & - & - & - & - & - & - & - & - & -  & + & + & - & -   \\
		x & - & - & - & - & - & - & + & + & -  & + & + & - & -  \\
		ch & - & - & - & - & - & - & - & + & - & - & + & - & -  \\
		q & - & - & - & - & + & - & - & + & -  & - & + & - & - \\
		dh & - & - & - & - & - & - & - & + & - & - & - & - & -   \\
		f & - & - & - & - & - & - & - & + & -  & + & + & - & -  \\
		s & + & + & - & + & + & - & + & + & -  & + & + & - & - \\
		sh & - & - & - & - & + & - & - & + & - & - & - & - & -   \\
		m & - & - & - & - & - & - & - & - & +  & + & - & - & -  \\
		n & - & + & + & - & - & - & + & - & -  & - & - & - & - \\
		r & - & - & - & - & - & + & - & - & -  & - & - & - & - \\
		\hline\hline  
	\end{tabular}
\end{table}

\begin{table}[H]
\caption{Fakkeenyota Irra Butaa}
\centering
\begin{tabular} {c c c} \\
  \hline\hline
  fakkeenya 1 & fakkeenya 2 & fakkeenya 3 \\
  beekta & alanfadha & salaksa \\
  balaansoofii & abdii & inshilaala \\
  abjuu & many'ee & obsa \\
  alaltuu & diinagdee & baalca \\
  dhana & halkan & qooqsa \\
  bal'oo & doqna & tinjii (dangaa)\\
  hordofaa & diftii & aangoo \\
  arjaa & tafkii & firinxiixaa \\
  arga & dafqa & hirphoo \\
  baafsaa & hanqaaquu & qurxummii \\
  afshaala & baay'ee & barcuma \\
  albee & birqii & gamtaa \\
  baaldii & afraasaa & dhim'uu \\
  bumbee & salphaa & birmaduu \\
  lamxii & agamsa & ulfa \\
  handhuura & gamana & filsataa \\
  gatantara & elma & amartii \\
  xannacha & burkutaa & balanballee \\
  bir'aa & handaqii & arba \\
\hline\hline
\end{tabular}
\end{table}

Hundee jechaa ilaalcha keessa galchinee yoo ilaalle, irra buta sagaleewwan laagee irratti daangaan jira. Isaanis: 1. Sagaleewwan /ny/ fi /y/ hudhaa isaanitti aanee dhufu waliin malee irra buta hin’uuman. 2. /sh/n sagalee kam faanayu jalqaba irra butaa ta’ee hindhufu. 3. Sagaleewwan rigataa laagee irratti uumaman jalqaba irra butaa ta’anii hindhufan. Akkuma waliigalaatti sagaleewwan rigataa laagee irratti uumaman sagalee kamiinuu hordofamanii irra butaa hin’uuman yoo hudhaa ta’e malee \cite[p.20]{lloret1988gemination}.

Akkasumas, irra buta jiran keessaa fC (sagaleen /f/ dura dhuftee jechuudha) heddumminaan mul’ata. Garuu sC (dhamsagni /s/ dura dhufee jechuudha jechoota ergisaa
keessatti argama, Fkn, maskootii (foddaa), masgiida, bataskaana. Irra butawwan armaan gadiis hundee keessatti akka garaa hinargaman:
\begin{itemize}
        \item[kt], makte (-te n fufiidha malee hundee miti)
        \item[ks], salakse (-se n fufiidha malee hundee miti)
        \item[qs], qooqse (-se n fufiidha malee hundee miti)
        \item[qn], waaqni (-ni n fufiidha malee hundee miti)
        \item[bs],c abse (-se n fufiidha malee hundee miti)
        \item[bd], dhabde (-de n fufiidha malee hundee miti)
        \item[bj], abjuu
        \item[gd], dugda
        \item[gn], lugna (Lloret, 1988: 23)
\end{itemize}

\subsection{Dhamsagootaafi Firsagoota\index{dhamsagaafi firsaga}}

Dhamsagni tokko akaakuwwan sagaleewwan lama ykn lamaa ol  qabaachuu danda’a. Akaakuuwwan sagaleewwan kunneen  firsga dhamsagichaati. Dhamsaguma tokkotu sagalee gara  biraa fakkaate bakka adda addatti dhugooma. Fakkeenya firsagootaa mursaa isaanii akka armaan gadiitti addeessina. 
\begin{itemize}
        \item \textipa{[M]} qooqa qabeessa, hidhlamee, funyee. Jecha keessatti /n/n sagalee /f/ dursee yoo dhufe \textipa{[M]} ta’ee  mul’ata; Fkn, [go\textipa{[M]}faa] , gonfaa. Kanaaf \textipa{[M]} n firsaga  dhamsaga /n/ ti.
        \item \textipa{[N]} qooqa qabeessa, irgee, funyee. Jecha keessatti  /n /n , /k /,/g/ ykn /k’/ (q) dursee yoo dhufe \textipa{[N]} ta’ee  mul’ata; Fkn, [ma\textipa{[N]}kuusa], mankuusa. Kanaaf \textipa{[N]}n  firsaga dhamsaga /n/ti.
        \item \textipa{[l]} qooqa qabeessa, irgee, maddee. [l] Jecha keessatti  /n/n sagalee /l/ dursee yoo dhufe [l] ta’ee mul’ata;  Fkn, [hillixu], . Kanaaf [l] firsaga dhamsaga /n/ti.
        \item \textipa{[r]} qooqa qabeessa, irgee, rom’aa. Jecha keessatti /n/ sagalee /r/dursee yoo dhufe [r] ta’ee mul’ata; Fkn, [hrrafu] , hinrafu. Kanaaf [r]n firsaga dhamsaga /n/ ti.Walumaa galatti dhamsagni /n/ firsagoota adda addaa  qaba jechuudha.
        \item \textipa{[t]} qooqa qabeessa, irgee, cufaa. Jecha keessatti  /t/n sagalee /n/ dursee yoo dhufe [n] ta’ee mul’ata;  Fkn,[rukunne] , rukutne. Kanaaf /firsaga / t/ti.
        \item \textipa{[t\super h]} qooqa dhabeessa, afuura qabeessa, urge, cufaa. Jecha  keessatti /t/n jalqaba jechaarra yoo dhufe afuura qa beessa ta’ee mul’ata; fkn. [\textipa{[t\super h]}ure], ture. Kanaaf \textipa{[t\super h]} n  firsaga /t/ ti jechuudha.
        \item \textipa{[k\super h]}qooqa qabeessa, afuura qabeessa,  harsassee, cufaa. Jecha keessatti /k/n jalqabarra yoo  dhufe afuura qabeessa ta’ee mul’ata; Fkn,[\textipa{[k\super h]}utaa]. Kanaaf \textipa{[k\super h]}n firsaga /k/ti jechuudha.
        \item \textipa{[k']} ([q]) qooqa dhabeessa, harsassee, cufaa. Jecha  keessatti /q/n sagalee [k] dursee yoo dhufe [k] ta’ee  mul’ata; Fkn, [haksiise], haqsiise. Kanaaf [k]n firsaga  dhamsaga /q/ ti jechuudha.

\subsection{Jabeessuu\index{jabeessuu}}

AO keessatti dubbifamtuu jabeessuun jijjiirraa hiikaa fida.  Sababa kanaafis jabeenyi dubbifamtootaa dhamsaga jedhama.  Dubbifamtoota jabeessuun kunis heddumminaan mul’ata.  Kanaaf jabinnii dubbifamtootaa bu’uura AOti. Haala kana  ibsuuf fakkeenyota armaan gadii haa ilaallu: 
\begin{itemize}
        \item[t: tt]
        \item/baati/ Jiini yoom baati?
        \item[baatti] Huummoon morma isheetti maal baatti?
        \item[b: bb]
        \item/gaabii/ Gaabii uffachuun qorraaf gaariidha. 
        \item/gaabbii/ Cubbuu dalaguun gaabbii fida. 
        \item[g: gg]
        \item/gadi/ Manaa gadi lagatu jira. 
        \item/gaddi/ Gaddi nama miidha.
        \item[l: ll] 
        \item/gala/ Bineensi daggala keessa gala. 
        \item/galla/ Nuti garuu gamoo keessa galla. 
        \item[r: rr]
        \item/bare/ Inni konkolaataa ofuu yoom bare? 
        \item/barre/ Nuti har’a wal barre.
        \item[d: dd]
        \item/sagaduu/ Sagaduu-n gaariidha.
        \item/sagadduu/ Sagadduun nama sagadu jechuudha. 
        \item[q: qq]
        \item/baqe/  Dhadhaan aduutiin baqe.
        \item/baqqe/ Ani hamminan baqqe.
        \item[x: xx] 
        \item/fixe/ Ani hojiikoo fixe.
        \item/fixxe/ Atis abbaltiikee fixxe. 
        \item[n: nn]
        \item/gana, ganna/ - Maaltu ganna nama gana?
\end{itemize}

\section{Addeessa Dubbachiiftotaa\index{addeessa dubbachiiftotaa}}

Dubbachiiftonni ta’umsa arrabaa, saaqa afaaniif amartaa’uu hidhii irratti hundaa’ani bakka adda addaatti qoqqoodamu.  Ta’umsi arrabaa bakkeewwan sadiitti qoodama. Isaanis  fuuldura, walakkaafi duuba jedhamu. Fuuldura jechuun  dubbachiiftota fuuldura/fiixee arrabbaa irratti uumaman  jechuudha. Akkasumas kanneen walakkaa arrabaa irratti  uumamaniif duuba ykn hundee arrabaa irratti uumaman jiru  (fakkii dubbachiiftotaa fuula 13 irraa ilaalaa).Karaa saaqaa afaanii dubbachiiftonni bakkeewwan sadiitti  qoodamu. Isaanis dubbachiiftota olii, dubbachiiftota gidduufi  dubbachiiftota gadii jedhamu. Yeroo afaan xinnooshee  saaqamu sagaleewwan uumaman dubbachiiftota olii, yeroo  afaan giddugaleessaan saaqamu sagaleewwan uumaman  dubbachiiftota gidduu, yeroo afaan sirriitti saaqamu  sagaleewwan uumaman dubbachiiftota gadii jedhamu. Haala  amartii hidhiitiin dubbachiiftonni bakka lamatti qoodamu.  Isaanis dubbachiiftota amartaa’uufi kanneen hinamartoofne  jedhamu. Dubbachiiftonni AO kudhani; dhedheeroo shaniifi gaggabaaboo  shan; isaanis fuuldura arrabaa, walakkaa arrabaafi duuba  arrabaa irratti uumamu. Karaa sadarkaa banama afaanii  sagaaleewwan olii, gidduufi gadii jedhamu. Sagaleewwan  duubaa hundi amartii jedhamu.

\begin{table}[H]
	\centering
\caption{Dubbachiiftota Dhedheeroo AO}
\begin{tabular}{c c c c}\\
  \hline\hline
  Afaan & Fuuldura & Walakkaa & Duuba\\
  Ol & ii & - & uu\\
  Gidduu & ee - & - & oo\\
  Gad & - & aa & -\\
  \hline\hline
  \end{tabular}
\end{table}

\begin{table}[H]
	\centering
	\caption{Dubbachiiftota Gaggabaaboo AO}	
	\begin{tabular}{c c c c} \\
		\hline\hline
		Afaan & Fuuldura & Walakkaa & Duuba\\
		Ol & i & - & u\\
		Gidduu & e - & - & o\\
		Gad & - & a & -\\
		\hline\hline
	\end{tabular}
\end{table}
  
\begin{itemize}
        \item[i] fuuldura, ol, gabaabaa, hin’amartaa’u. Fkn,ilkaan- jalqaba; bite- gidduu; biti-dhuma jechaarratti argama.
        \item[ii] fuuldura, ol, dheeraa, hin’amartaa’u. Fkn, miira- gidduu; bultii- dhuma jechaarratti argama.
        \item[e] fuuldura, gidduu, gabaabaa, hin’amartaa’u. Fkn, erbuu- jalqaba; bebbeeka- gidduu; rafe- dhuma jechaarratti mul'ata.
        \item[ee] fuuldura, gidduu, dheeraa, hin’amartaa’u. Fkn, eebba- jalqaba; beela- gidduu; re’ee- dhuma jechaarratti argama.
        \item[a] gad, walakkaa, gabaabaa, hin’armartaa’u. Fkn, ana-jalqabaafi dhuma; mana- gidduufi dhuma jechaarratti mul'ata.
        \item[aa] gad, walakkaa, dheeraa, hin’amartaa’u. Fkn, aarii-jalqaba; maatii-giddy; mataa- dhuma jechaarratti argama.
        \item[u] ol, duuba, gabaabaa, amartaa’aa. Fkn, ulfa- jalqaba; bultii- gidduu; rafu-dhuma jechaarratti argama.
        \item[uu] ol, duuba, dheeraa, amartaa’aa. Fkn, uumaa- jalqaba; buusa- gidduu; kaasuu dhuma jechaarratti argama.
        \item[o] gidduu, duuba, gabaabaa, amartaa’aa. Fkn, ol- jalqaba; boru- gidduu; Ganamo- dhuma jechaarratti argama.
        \item[oo] gidduu, duuba, , amartaa’aa. Fkn, booka- gidduu; lookoo- gidduufi dhuma jechaarratti argama.

\end{itemize}

\subsection{Faca'insa Dubbachiiftotaa\index{faca'insa dubbachiiftotaa}}
Akka dubbachiiftuun hundee jechaa keessatti faca’insa ol aanaa qabu /a/dha \cite[p.17]{owens1985grammar}.  
\begin{table}[H]
	\centering
	\caption{Faca’insa dubbachiftotaa}
	\begin{tabular}{|c||c|c|c|c|c|}
		\hline
		& a & e & i & o & u \\
		\hline\hline
	a	& + & + & + & - & + \\
		\hline
	e	& + & - & - & - & + \\
		\hline
	i	& + & - & + & - & - \\
		\hline
	o	& - & - & - & + & + \\
		\hline
	u	& + & - & - & - & - \\
		\hline
	\end{tabular}
	\caption{}
\end{table}

\subsection{Walmorkii Dubbachiiftotaa\index{walmorkii dubbachiiftotaa}}

Dubbachiiftoonni jecha tokko keessatti walmorkachuun addaddummaa hiikaa fidu. Kanaaf dhamsagoota ta’uu isaanii
mirkaneeffanna. Dhimma kana fakkeenyota armaan gadii irraa hubachuu dandeenya: 
\begin{itemize}
	\item /i/ fi /e/: Ati hojiikee \textbf{xumuri}; Inni hojiisaa \textbf{xumure}.
	\item /i/ fi /u/: \textbf{Yakki} gaarii miti; Isaan nama \textbf{yakku}.
	\item 3. /a/ fi /u/: Ogeessi nama \textbf{yaala}; Isaan nama \textbf{yaalu}.
	\item /o/ fi /a/: \textbf{Soba} dubbachuun gaarii mit; \textbf{Saba} kabajuun gaariidha.
\end{itemize}











Akkuma gabatee armaan oliirraa hubannutti /a/n hundee jechaa keessatti walfaana gala; fakkeenyaaf jecha \textbf{sagal}- jedhu fudhachuu dandeenya. Sagaleen /a/ akkasuma hundee jechaa keessatti sagalee /e/ faana argama; fakkeenyaaf, jecha \textbf{adeer}- jedhu ilaaluu dandeenya. Sagaleen /a/ hundee jechaa keessatti /o/ faana hin'argamu. Gabatee armaan olii haaluma kanaan qaaccessinee faca'insa dubbachiiftotaa baruu dandeenya.

\section{Adeemsa Dhamsagaa\index{adeemsa dhamsagaa}}

Adeemsa dhaamsagaa jechuun dhamsagni walitti makamee  sagalee haaraa uumuu jechuudha. Dhamsagni tokko ollaa  dhamsaga biroo yeroo galu nijijjiirama ykn nijjiira. Kana  jechuun sagaleen tokko sagalee biroorra dhiibbaa geessisa ykn  dhiibbaan irra ga'a. Adeemsi dhamsagaa akaakuu adda addaa  qaba. Kutaa kana keessatti gosagaloomii, hidheessuu, laagessuu, harsasseessuu, amarteessuu, funyeessuu, qooqa  fudhachuu, qooqa dhabuu, saaguu, waldarbuu (waljaafuu),  haquu, algosagaloomuu, dheeressuu, ol kaasuufi kkf ilaalla. 

\subsection{Gosagaloomii/Firoommii\index{gosagaloomii/firoomii}} 
\setlength{\parindent}{3em}

Jechiifi jechi ykn fufiifi jechii yeroo walitti dhufani, daangaa  walitti dhufeenyaa isaanii irratti jijjiiramni dhamsagaa  ni’uumama. Jijjiirraa haala kanaan dhufu keessaa inni tokko  gosagaloomii jedhama. Gosagaloomiin sagaleen tokko sagalee  isa ollaa isaa jiru akka fakkaatu godha. AO keessatti  gosagaloomiin karaawwan mul’atan keessaa inni tokko  maxxantun duraa inni dubbifamtuutiin dhaabbatu jecha  dubbifamtuutiin calqabutti yeroo maxxanu.Akkasumas fufiin  boodaa dubbifamtuutiin calqabu, jecha dubbifamtuutiin  xumuramutti yeroo maxxanu gosagaloomiin ni  argama. Fakkeenyaaf jecha hinlixnu jedhu wayta hillixnu  jennee barreessinu sagaleen /n/ inni /hin/ keessatti argamu,  sagalee /l/ isa /lix-/ keessa jiru wajjin wal fakkaate. Kunis  gosagaloomii agarsiisa. Kana malees gosagaloomiin karaawwan  adda addaa mul'ata. Isaaniinis kutaawwan armaan gadii  keessatti hubatna. 

Akaakuuwwan gosagaloomii keessaa tokko gosagaloomii  gardureeti\index{gosagaloomii garduree}. Gosagaloomii garduree kan jedhamu yeroo sagaleen  tokko sagalee isatti aanee dhufe waliin walfakkaate ykn gara sagalee isatti aanee dhufetti jijjiiramudha. Fakkeenyaaf jecha /dugda/ jedhu haa ilaallu. Jecha kana keessatti sagaleen /g/ sagalee /d/ dursee dhufeera. Sagaleen /g/ kun gara sagalee isatti aanee dhufeetti, jechuun gara sagalee /d/tti gosagalooma; kanaaf jechi /dugda/ jedhu unki isaa jijjiiramee /dudda/ ta'ee dubbatama. Jecha /malaanmaltuu/ jedhus haa ilaallu. Jecha kana keessatti sagaleen /n/ fi /m/n walitti aananii dhufaniiru; sagaleen /n/ sagalee /m/ dursitee dhufteetti; sagaleen /m/ sagalee /m/ sagalee /n/tti aantee dhufteetti. Garuu sagaleen ollaa ishee irratti dhiibbaa geessifte sagalee /m/dha; sagaleen /n/ gara sagalee /m/tti jijjiiramteetti; kanaaf jechi /malaanmaltuu/ jedhu /malaammaltuu/ ta'ee mul'ateera. Karaa jecha gara biroo sagaleen /m/n laaftuu turte jabaattee /mm/ taatee mul'atteetti.

Gosagaloomiin garduree faallaa qaba. Faallaan gosagaloomii garduree,gosagaloomii garduubee\index{gosagaloomii garduubee} jedhama. Gosagaloomii garduubee kan jedhamu wayita  sagaleen tokko gara saglee isa dursee dhufeetti  gosagaloomudha. Fakkeenyaaf jecha /gallne/ jedhu haa ilaallu. Hundeen jecha kana /gal-/ kan jedhu. Hundee jecha kanaa irratti fufiin ramaddii tokkoffaa qeentee {-ne} wayita ida'amtu jechi /gal-ne/ jedhu ijaarama. Yeroo kanatti sagaleen /l/ fi /n/ ollaa walii ta'uuf carraa argatu. Yeroo sagaleewwan kunneen walitti aananii dhufan sagaleen dura dhufte jechuun /l/n sagalee isheetti aantee dhufte jechuun /n/ irratti dhiibbaa geessifti; kanaaf sagaleen /n/ gara sagalee /l/tti jijjiiramti. Haaluma kanaan jechi /galne/ jedhu jecha /galle/ jedhu ta'ee mul'ata jechuudha.

Akkasumas gosagaloommiiwwan guutuufi gamisaa jiru. Gosagaloomii  guutuu\index{gosagaloomii guutuu} kan jedhamu yeroo sagaleen gosagalomu sun gutumaa  guutuutiin jijjiirame sagalee isatti aanee dhufe ykn sagalee isa  dursee dhufe fakkaatudha. Fakkeenyonni gosagaloomii  garduree fi gosagaloomii garduubee keessatti argaman  gosagaloomii guutuu agarsiisuu danda’u. Fakkeenyaaf,\newline
a. hinlixu→hillixu\newline  
b. hinrafu→hirrafu\newline  
c. hinwaamu→hiwwaamufi kkf ilaaluu dandeenya.

Kana malees gosagaloomiin gamisaa\index{gosagaloomii gamisaa} jira. Gosagaloomii gamisaa  kan jedhamu sagaleen tokko sagalee isa dursee ykn sagalee  isatti aanee dhufe waliin gutumaa guutuutti osoo hintahiin  gamisaan yeroo wal fakkaatudha. Fakkeenyaaf gosagaloomii  armaan gadii haa hubannu.\newline
a. hinbadu→himbadu\newline
b. hinbeekuu→himbeeku\newline  
c. hinciisu→hinyciisufi kkf.

Fakkeenya kanneen irraa gosagaloomii gamisaa hubachuu  dandeenya. Sagaleen /n/ inni fufii duraa /hin-/ keessatti  argamu gara /m/ fi /ny/ tti jijjiirameera malee gutumaan  guutuutti gara /b/ fi /c/ tti hinjijjiiramne.

Kanatti aansinee gad fageenyaan wayita ilaallu, hidheessuu, laagessuu,  harsasseessuu, amarteessuu, funyeessuu, qooqa fudhachuufi qooqa dhabuun akaakuuwwan gosagaloomii beekamani. Qabxiiwwan kanneenis akka armaan gadiitti tokko  tokkoon ilaalla.  

\subsubsection{Hidheessuu\index{hidheessuu}}

Sagaleen bakki uumamsisaa hidhii hintaane ollaa sagalee hidhii  irratti uumamu galee amala hidhi yoo fudhate hidheessuu  jedhame waamama. Fakkeenyaaf,\newline
a. /hinbeeku/ = [himbeeku]\newline  
b. /isinfaa/ = [isi\textipa{M}faa]fi kkf.

Akka (a) irratti hubannu sagaleen /n/ ollaa sagalle /m/ waan  dhufteef gara sagalee /m/tti jijjiiramte. Kana jechuun  sagaleen /n/ bakka uumamsaa ishee irgee irraa gara hidhi  lameetti jijjiirratte jechuudha. Akkasumas akka (b) irraa  hubannutti sagaleen /n/ irgee ta’uu dhiiftee sagalee hidh-illkee  taateetti; sababni isaas ollaa sagalee /f/ waan dhufteef.  Adeemsa dhamsagaa akka kanneeni seera dhamsagaan  agarsiisuun nidandahama.\newline
a. /n/ → [m] / - /b/  \newline
b. /n/ → \textipa{[M]}/- /f/  

Odeeffannoowwan seera dhamsagaa irraa argman sadii:  sagalee adeemsa dhamsagaan tuqamu, jijjiirraa sagalee  mul'atuuf bakka itti jijjiirraan kun mul'atu. Akka (a) irratti  mul'atu kana /n/n sagalee isa adeemsa dhamsagaan tuqamu.  Mallattoon → kun kallatti jijjiirama sagalee agarsiisa. Kana  jechuunis sagaleen /n/ gara firsaga[m] tti jijjiiramteetti jechuudha. Mallattoon [ ] jedhu odeeffannoo lammafaadha;  innis jijjirama sagalee agarsiisa. Mallattoon / kun naannoo  jijjiiramni sagalee itti mul'ate agarsiisa. Karaa jecha gara  biroo /n/ n gara [m] tti kan jijjiiramte wayita sagalee /b/  dursitee dhuftu jechuudha. Akkasumas (b)n haaluma walfakkaatu agarsiisa, sagaleen /n/ wayita /f/ dursitee dhuftu  gara firsaga \textipa{[M]}tti jijjiiramti jechuudha.

\subsubsection{Laagessuu\index{laagessuu}}

Laagessuun, dubbifamaan tokko dubbachiiftota /i/ fi /e/  dursee dhufuun amala laagaa wayita argatu uumama ykn  mul'ata. Adeemsa dhamsagaa akaakuu kanaa akka armaan  gadiitti ilaalla:\newline
a. /leemmana/ = [\textipa{l\super j}eemmana]  \newline
b. /mootii/ = [moo\textipa{t\super j}ii]  \newline
c. /re’ee/ = [\textipa{r\super j}e\textipa{’\super j}ee]  

Akkuma fakkeenya armaan oliirratti mul'atutti dubbifamtoonni  kanneen dubbachiiftota /e/ fi /i/ dursanii dhufan amala  laageffamuu argataniiru. Amalli laageffamuu kunis mataa  dubbifamtuu sana irratti mallattoo “{\super j} " barreessuun  agarsiifama. Adeemsa laagessuu kana seera dhamsagaan  agarsiisuu nidandeenya:  \newline
a. /t/ → [\textipa{t\super j}]/-/i/  \newline
b. /r/ → [\textipa{r\super j}]/-/e/  


Akka seerri armaan olii irratti eerame ibsuti sagaleen /t/ fi /r/ n dubbachiiftota /i/ fi /e/ tiin dursanii waan dufaniif amala  laagessamuu argataniiru.

\subsubsection{Harsasseessuu\index{harsasseessuu}}

Sagaleen harsassee irratti hinuumamne tokko ollaa sagalee  harsassee galuun amala harsassee wayita fudhatu  harsasseessuu jedhama. \newline
a. /k'oonk'oo/ = [k’oo\textipa{N}k’oo]  \newline
b. /manguddoo/ = [ma\textipa{N}gddoo]  \newline
c. /gaangee/ = [gaa\textipa{N}gee] \newline
d. /sangaa/ = [sa\textipa{N}gaa] 

Akka fakkeenyota armaan olii irraa hubanutti /n/ n ollaa  sagaleewwan harsassee galuun amala harsassee  fudhateetti. Adeemsa kanas seeraan akka armaan gadiitti  kaa’uu dandeenya:  \newline
/n/→ \textipa{[N]/}- /k'(g)/  

\subsubsection{Amarteessuu\index{amarteessuu}}

Dubbifamtoonni sagaleewwan dubbachiiftota boodaa dursanii  dhufan amala amartaa’uu dubbachiiftota kanneen irraa argatu.  \newline
a. /bona/ → [\textipa{b\super w}ona]  \newline
b. /kutaa/ → [\textipa{k\super w}utaa] \\
Akkuma armaan olitti mul'atetti /k/ n wayita /u/ tiin dursee  dhufu ni’amartaa'a. /k/n wayita amartaa’u /w/n mataa /k/ irratti barreeffamti. Haala kanaan /kutaa/ inni sarara lama  gidduutti barreeffame jecha nuti dubbachuuf kaane. Inni  ammoo hammattuu keessatti barreeffame, jechuunis, [\textipa{k\super w}utaa]  waan nuti qabatamaan sagaleessine. Akkasumas (b) nis  haaluma (a) waliin wal fakkaata. Fakkeenyota kannenis akka  armaan gadiitti seera dhamsagaan agarsiisuu dandeenya: \newline
a. /k/→ [\textipa{k\super w}] / - /u/  \newline
b. /b/→ [\textipa{b\super w}]/- /o/ \\
 
Akka (a) fi (b) irratti mul’atutti /k/n wayita dubbachiiftuu /u/  dursee dhufu gara [\textipa{k\super w}] tti geeddarama; /b/n wayita  dubbachiiftuu /o/ dursee argamu gara sagalee [\textipa{b\super w}]tti  jijjiirama.  

\subsubsection{Funyeessuu\index{funyeessuu}}

Dubbachiiftonni hundi wayita ollaa sagaleewwan funyee dhufan, ollaa isaanii irraa amala funyee argatu. Fakkeenyaaf, \newline
a. /ana/ → [an\textipa{\~a}]  \newline
b. / muka/ → [m\textipa{\~u}ka]  

Akkuma armaan ol irratti mul'atutti dubbachiiftonni ollaaa  sagaleewwan funyee galan nifunyeeffamu. Amalli  funyeeffamuu kunis sarara daddabduu mataa dubbachiiftuu  funyeeffamte irra kaa'uun agarsiifamti. Haalli kunis seera  dhamsagaatiin taa'uu mala:  \newline
a. /a/ → [\textipa{\~a}] //n/-  \newline
b. /u/ → [\textipa{\~u}] / /m/- 

Dubbachiiftuu /a/n sagalee funyeetti aantee dhuftee amala  funye argachuu ishee, akkasumas dubbachiiftuu /u/n sagalee  funyee /m/tti aantee dhuftee amala funee argachuu ishee (a) fi  (b) irraa hubachuu dandeenya. 

\subsubsection{Qooqa Fudhachuu\index{qooqa fudhachuu}}

Sagaleen qooqa dhabeessi tokko ollaa sagalee qooqa qabeessaa  galee wayita qooqa argatu adeemsisaa qooqa fudhachuu  jedhama.  \newline
a. /gat-na/ →[gan-na]  \newline
b. /dhg-te/→[dhug-de]  

Sagaleen /t/ qooqa dhabeessa. Sagaleen kun ollaa sagalee  qooqa qabeessa wayita galetti qooqa argata ykn qooqa itti aanee wayita dhufu sagalee qooqa qabeessa ta’a jechuudha. 

\subsubsection{Qooqadhabuu\index{qooqa dhabuu}}

Sagaleen qooqa qabeessa ta’e tokko ollaa sagalee qooqa  dhabeessa ta’ee galee qooqa dhabeessa ta’a. Fakkeenya,  \newline
a. /fiigse/→ [fiikse]  \newline
b. /\textipa{\!d}iigse/→[\textipa{\!d}iikse]  (dhiigse/dhiikse)\newline
c. /hinseenu/ → [hisseenu]  

Akka (a)fi (b) irratti hubatamutti dhamsagni /g/ gara [k]tti, (c)  irrattammoo /n/n gara sagalee [s]tti jijjiramaniiru.  Fakkeenyota kanneeniif seera dhamsagaa akka armaan gadiitti  keenya: 


/qooqa qabeessa/ → [qooqa dhabeessa] /-- qooqa dhabeessa 

\subsubsection{Saaguu\index{saaguu}}

Saaguu jechuun dubbifamtoota sadii ta’anii walitti aananii  dhufanii irrabuta uumanii gidduu, dubbachiiftuu galshuu ykn  dubbachiiftota lama ol ta’nii walitti aananii dhufan gidduu  hudhaa galchuudha. Akkuma beekamutti dubbifamtoonni sadii  walitti aananii dhufun seera AO keessatti  hinhayyamamu. Akkasumas irra butaan calqabaaf dhuma  jechaa irratti hinbeekamu. Haalonni kunneen yeroo nama  quunnaman garuu, dubbachiiftuu saaguudhan haala dhibsiisaa  tahe sana furuun nidandahama. Fakkeenyaaf jecha /argine/ jedhu fudhannee haa ilaallu:\newline
a. arg + ne -i- arg-i-ne \\
b. jibb + ne -i- jib-i-ne \\
c. dhoks + ne -i- dhoks-i-ne 

Hundeen jecha /argine/ jedhuu {arg-} dha. Hundee jecha kanaarratti fufiin ramaddii tokkoffaa hedduu {-ne} wayita ida'amtu jechi /argne/ jedhu ijaarama. Haa ta'u malee jechi /argne/ jedhu dubbifamtoota sadii waan walitti aansee fideef AO keessatti seermaleedha. Seera dabe kana sirreessuu sagaleen [i] gidduu saagamte. Haala kanaan jechi /arg-i-ne/ jedhu ijaarame. Jecha /dhoksine/ jedhuus haaluma wal fakkaatuun qaacceffama. 

Akka hayyoonni AO jedhanitti, walumaagalan saaguun seerota sadii qaba\cite{owens1985grammar}. Isaanis:\\ 
1. Dubbifamtoonni sadii walitti aananii dhufaniiran  keessaa dubbifamtuun lammaffaa /l/ ykn /r/ yoo taatee,  sagaan dubbachiiftuu /a/dha. Fkn,  \\
a. kofl-te=kofalte \\
b. kofl+siis=kofalsiis  \\
c. dabr+tan=dabartan  \\
d. dubr+tii=dubartii  

Filannoo gara biroon ammoo waljaafuu fi /i/ saaguudha. Fkn, \\
a. kolfite \\
b. kolfisiise \\
c. dabritan \\
2. Seerri sagaa inni lammaffaa akksi jedha: dubbifamtoota sadii  walitti aananii dhufanii jiran keessaa dubbifamtuun inni  lammaffaa /l/ ykn /r/ miti yoo ta’e /i/tu saagama. Fkn,  \\
a. gudd + s= guddis \\
b. cabs + ta= cabsita \\
c. sirb + te= sirbite \\
d. tiss + te= tissite \\
3. Dubbachiftonni lamaa ol yoo walitti aananii dhufaniiru ta’e  hudhaatu saagama. Fkn,  \\
a. ani + ifa= ni’ifa; \\
b. ni + ilaala= ni’ilaala;\\
c. ni + aara= ni’aarafi kkf. 

\subsubsection{Waldabruu/Waljaafuu\index{waldarbuu/waljaafuu}}

Waldarbuu/waljaafuu jechuun akkuma maqaan isaa himutti  qubee waldabarsanii dubbachuu ykn barreessuudha.  Fakkeenyaaf, \\
a. afraffaa → arfaffaa  \\
b. darbaa → dabraa  \\
c. salgaffaa → saglaffaa  \\
d. qabarichoo → qarabichoo 

Aramaan olitti sagaleewwan /r/fi /f/n; /g/ fi /l/n; /b/ fi /r/n  akkasumas /q/ fi /r/n bakka walijjiiraniiru. Bakka waljijjiiruun  sagaleewwanii kunis waljaafuu jedhama. Sagaleewwan wal  irraa fagoo jiraatanii jecha tokko ijaaranillee haaluma kanaan  bakka waljijjiiruu dandahu. Dabalataan fakkeenya armaan gadii  haa ilaallu: \\
a. jaldeessa → daljeessa  \\
b. qamalee → qalamee  \\
c. arge → agre \\

Fakkeenya armaan olii (a) irratti /j/ fi /d/ gidduu sagaleewwan  lama jiru. Haa ta’u malee /j/ fi /d/ n bakka wal jijjiiraniiru; (b)  irratti ammoo /m/ fi /l/ gidduu qubee tokko qofaatu jira;  isaansi bakka waljijjiiraniiru; (c) irratti /r/fi /g/n bakka  waljijjiiraniiru.  

\subsubsection{Haquu\index{haquu}}

Haquu jechuun adeemsa dhamsagaa keessatti sagalee tokko  jecha keessaa baasuu ykn gatuu jechuudha. AO keessatti  haquun dubbifamtoota irras dubbachiifttota irras  gaha. Fakkeenyaaf lakkoofsa /afur/ fi /sagal/ irratti fufiin  boodaa /-affaa/ yeroo ida’amu dubbachiiftonni badan jiru. Fakkeenyaaf, jecha /nama/ jedhutti fufiin {-oota} wayita ida'amu jecha /nama/ jedhurraa dubbachiiftuun boodaa /a/ haqamti. Haala kanaan jechi /nam-oota/ jedhu ijaarama. 

\subsubsection{Algosagaloomuu\index{algosagaloomuu}}

Algosagaloomuun faallaa gosagaloomuuti. Gosagaloomuun  sagaleen ollaa fakkaachuu yoo ta’u, algosagaloomuun ammoo  ollaa irraa adda ta’uudha. Algosagaloomuun karaalee adda  addaa mul’ata. Karaan tokko, fufiin boodaa inni dubbachiiftuu dheeraa qabu jirma jechaa isa dubbachiiftuu dheeraa qabu  irratti yeroo maxxanuu dubbachiiftuu dheeraa sana gabaabsa;  walfaana hindheeratan. Fakkeenyaaf {nam-} irratti {-oota}n  yeroo ida’amu namoota taha. Jirmi jechaa {nam-} jedhu waan  dubbachiiftuu gabaabduu qabuuf {-oota} fufii sagalee dheeraa  qabu fudhata. Haaluma kanaan {cab-} irratti {-siise} yoo  idaane cabsiise taha. Jirma jechaa sagalee dheeraa qabu  irratti garuu adeemsi kun jijjiiramee algosagaloomii  agarsiisa. Fakkeenyaaf fufii heddumminaa {-oota} jedhu haa ilaallu. Fufiin kun kan inni maxxanu hundee jechaa dubbachiiftuu gabaabduu qabdu irratti. Fakkeenyaaf hundeen jechaa {nam-} dubbachiiftuu gabaabduu qabdi kanaaf fufiin heddumminaa dubbachiiftuu dheeraa qabu jechuun {-oota}n itti fufame. Garuu hundeen jechaa dubbachiiftuu dheertuu qabdi yoo ta'e kan fufamu fufii heddumminaa isa dubbachiiftuu gabaabduu qabu, jechuun {-ota}dha. Fakkeenyaaf jecha /hoolaa/ jedhu irratti kan fufamu {-ota}dha. 

\subsection{Caasaa Birsaga\index{caasaa birsagaa}}
\setlength{\parindent}{3em}

Caasaan birsagaa qaamota sadii qaba. Isaanis saaqxuu,  utubaafi cuftuu jedhamu. Saaqxuun birsaga keessatti  dubbifamtuu jalqaba dhuftudha. Utubaan bantii ykn  dubbachiiftuu saaqxuutti aantee dhuftu qaba. Cuftuun  dubbifamtuu dhuma birsagaa dhuftudha. AO keessatti bakki  saaqxuu sagalee dubbifamtuu tokkoon guutama.  Dubbifamtoonni lama bakka saaqxuu galuun dhorkaadha.  Caasaa birsagaa keessa yoo dubbifamtuun hinjiraanne garuu  bakkichi duwwaa ta’uu danda’a. Bakki bantii garuu duwwaa  ta’uu hindanda’u. Bakki bantii dubbachiiftuu tokkoon ykn  dubbachiiftota lamaan guutamuu qaba. Bakki cufaa  dubbifamtuu tokkoon guutama; yoo dubbifamtuu hinqabu ta’e  bakkichi duwwaa ta’uu danda’a; garuu dubbifamtoota lamaan  guutamuu hindanda’u. AO keessatti caasaan birsagaa inni guddaan isa bakka saaqxuu dubbifamtuu tokko, bakka bantii dubbachiiftota lamafi bakka cuftuu dubbifamtuu tokko qabudha. 

\begin{forest}
	[Birsaga  [Saaqxuu (S) [C]] [Utubaa (U) [Bantii (B) [V]] [Cuftuu (C)[C]]]]
	]
\end{forest}  

Akaakuuwwan caasaa birsagaa AO hayyoota gara garaan qaacceeffamera\cite{Addunya2018,griefenow2001grammatical,gragg1976oromo}. Keessumattuu Owns fi Griefenow-Mewis  qorannoo xinqooqaan sadarkaa idil adunyaatti beekamtii ol  aanaa qabu; lamaan isaaniyyuu caasluga AO irratti qorannoo  taasisanii kitaabilee isaanii dhiyeessaniiru. Akka hayyoonnii tokko tokko jedhanitti\cite[p.21]{griefenow2001grammatical} AO keessa birsagni  dubbachiiftuutiin jalqabamu hinjiru. Fakkeenyaaf jechoota  iriyaa, antuuta, angafa fi kkf wayita sagaleessinu /h/ dura  fidna; yoo /h/ dhiifnemmoo hudhaa dura fidna. Dura dhufaatiin  hudhaa garuu nama meeshaafi gurraan adda baafachuu irratti  leenjii qabuun adda baafama. Hayyoonni hedduun yaada kana  deeggaru. Kanaaf nutis caasaa birsagaa AO keessatti jechi  dubbachiiftuutiin jalqabu hinjiru jenna (Garuu seera qubee AO  keessatti jechi kamuu hudhaatiin hinjalqabu). Karaa gara biroo  ammoo jechuma keessattillee /h/ fi hudhaan bakka waljijjiranii  dhufu. Fakkeenyaaf, jechoota dandaha/danda’a; dhagahe/ dhaga’e jedhan hubachuu dandeenya\cite{griefenow2001grammatical}. 

Kana malees qaaccessa birsagaa keessatti sagaleen jabaateru  tokko bakka lamatti qoodama; sagaleen dheerateeru garuu  lamatti hinqoodamu; irra butaan bakka lamatti qoodama\cite{griefenow2001grammatical}. \\
1. \textipa{P}V (saaqxuu hudhaafi dubbachiiftuu tokko)  \\
 a-na  \\
 i-sa  \\
2. \textipa{P}VV (Saaqxuu hudhaafi sagalee dheeraa)  \\
 ee-boo  \\
 aa-ra  \\
3. CV (saaqxuufi dubbachiiftuu tokko)  \\
 ma-na  \\
 la-ma  \\
4. \textipa{P}VC (saaqxuu hudhaafi dubbachiituu tokkoof cufaa)  \\
 ol  \\
 of \\
5. CVC (saaqxuu, bakka utubaa dubbachiiftuu tokkoofi cufaa)  \\
 gad-da  \\
 gam-na  \\
6. CVVC (saaqxuu, sagalee dheeraafi cufaa)  \\
 maal  \\
 kaan  \\
7. \textipa{P}VVC (saaqxuu hudhaa, sagalee dheeraafi cufaa)  \\
8. CVV (saaqxuufi bakka utubaa sagalee dheeraa)  \\
 boo-kee  \\
 laa-faa \\

Fakkeenyota saddettan kanneen yoo walitti cuunfine akaakuu  caasaa birsagaa afur qofa arganna. AO keessa caasaa birsagaa CV, CVV,  CVC fi CVVC tu jiru; jechi hoboombolleetti jedhus hunda  isaanii of keessaa qaba \cite[pp.21-22]{griefenow2001grammatical}. Looga Hararitti caasaaleen birsagaa armaan gadii hundee  jechaa uumuf tajaajilu \cite{owens1985grammar}.  \\
 1. CVC= nam- (nama) \\
 2. VC=if (looga biraatti of) \\
 3. VCC=add (adda) \\
 4. VVC=uum (uume) \\
 5. CVVCC=moorm (looga biraatti morma) \\
 6. CVVC=daar (daaraa) \\
 7. VCVVC=adeer(looga biraatti eessuma) \\
 8. VCVC=afur \\
 9. VCCVC=obbol \\
 10. VCVCCV=ibiddi\\
 11. VVCVC=eegal \\
 12. CVCVVC=ciniin, sakaal\\
 13. CVCVCC=sogidd \\
 14. CVCCVCC=shimbirr \\

AO keessatti dhimmi caasaa birsagaa waliin walitti dhufeenya  cimaa qabu haala itti dubbifamtoonni qindaa’an. AO keessatti  dubbifamtoonni adda addaa lama jalqaba jechaarratti dhufuu  hindanda’an. Kanaaf bakki saaqxuu dubbifamtoota lamaan  guutamuu hindanda’u. Akkasumas jidduu jechaatti  dubbifamtoonni akaakuu adda addaa lamaa ol dhufuu  hindanda’an. Kanaaf irra butaan sagaleewwan lama qofa  qabaachuu qaba; caasaa birsagaa keessatti irra butaanis ta’e  sagalee dheeraan bakka lamatti qoodamanii caasaa birsagaa  adda addaa keessa seenu. Akkasumas AO keessatti  dubbifamtoonni akaakuu adda addaa lama dhuma jechaarra  galuu hindanda’an. Waan kana ta’eefis caasaa birsagaa  keessatti cuftuun dubbifamtuu tokko qofaan guutamuu qabdi.

Caasaan birsagaa karaa irra deebii birsagaa jijjiiramuu danda'a. Irra deebii jechuun waan tokko, haala tokko dabaluudha. Irra  deebii birsagaa jechuun birsaga tokko irra deebi’anii  dubbachuu ykn barreessuu jechuudha. AO keessatti irra  deebiin bal’inaan hojiirra oola. Keessumattuu hundee jechootaa maqibsafi gochima keessatti birsagni jalqabaa irra deebi’ama.  Kana jechuun birsagni jalqaba hundee jechaa gochimaafi  maqibsa irra deebi’ama. Hundee gochimaa ykn maqibsa irra  deebi’uun hiika dabalataa kenna. Fakkeenyaaf, jechoonni diddiimaa, guguraacha, babal'aa fi kkf irra deebii birsagaa agarsiisuu. Irra deebiin birsagaa caasaa birsagaa jecha tokkoo jijjiira. 

\subsubsection{Gaaffilee Boqonnichaa}

Gaaffilee armaan gadii deebisi.
\begin{enumerate}
  \item AO olka'insaafi gad bu'insa sagaleetiin jijjiirraa hiikaa fiduu danda'a?
  \item AO irratti hudhaan jalqaba jechaafi dhuma jechaa dhufuu danda'aa? Maaliif?
  \item AO keessatti sagaleewwan qonqoo irratti uumaman jabaachuu danda'uu? Maaliif?
  \item AO saagduu meeqa qaba? 
  \item Akaakuuwwan adeemsaa dhamsagaa hundaaf fakkeenyota dhuunfaakee kenni.
  \item Akaakuuwwan irra butaa fakkeenya dhuunfaakeetiin ibsi.
\end{enumerate}

\chapter{Xindhamjecha\index{xindhamjecha}}

\subsubsection{Qabiyyee Boqonnaa}
\begin{itemize}
	\item Maqaa
	\item Maqibsa
	\item Gochima
	\item Gochimibsa
	\item Durduubee
	
	
\end{itemize}
\subsubsection{Gaaffilee Ka'umsaa}

\begin{enumerate}
  \item Addaddummaan birsagaafi jecha gidduu jiru maali?
  \item Jechi tokko hiikaafi unka qaba; dhimma kana fakkeenyaan ibs.
  \item Jechi tokko galmee jechootaa keessatti galmaa’a. Galmeen jechootaa jecha tokkof odeeffannoowwan maal maal kennuutu irraa eegama? Ibsi.
  \item Jechoonni qabiyyee maali?
  \item Jechoonni tajaajilaa maali?
  \item Dhamjecha jechuun maal jechuudha?
  \item Dhamjecha of danda’a jechuun maal jechuudha?
  \item Dhamjecha hirkataa jechuun maal jechuudha?
  \item Hundeen jechaa maali?
  \item Jirmi jechaa maali?
  \item Fufiin hormaataa maali?
  \item Fufiin dhalatoo maali?

\end{enumerate}
\section{Seensa}
Xindhamjechi qorannoo jechootaa irratti xiyyffata. Akkaataatti jechoonni uumaman ibsa. Akkuma kutaa darbe keessatti ilaalletti, dhamsagoonni walitti makamuun birsaga uumu. Birsagni unka qaba; garuu hiika hinqabu. Birsagaoonni ammoo walitti makamanii dhamjecha uumu. Dhamjechi unka qaba; hiikas qaba. Dhamjechi ammoo walitti makamee jecha uuma. Boqonnaawwan itti aananii dhufan keessatti yadrimeewwan armaan gadii faayidaarra oolu. Kanaaf yadrimeewwan kanneen hubachuun gaariidha. 

\begin{itemize}

\item[•] Dhamjecha\index{dhamjecha}: xiinxala xindhamjechaa keessaatti dhamjechi maal jechuudha? Dhamjechi qindoomina sagaleewwanii tahee hiika qabeessa isa xiqqaadha. Qindoomina sagaleewwaniiti jechuun sagaleefi sagaleen walitti dhufee dhamjecha ijaara; kanaaf dhamjechi unka qaba jechuudha. Dhamjechi hiika qabeessa isa xiqqaadha jechuun ammoo unki dhamjechaa sun yoo qoqqoodame hiika dhaba jechuudha. Dhamjechi haala kuusaa jechootaa keessatti dhaabbatuun bakkeewwan lamatti qoodama. Akaakuuwwan kunneenis dhamjecha ofdandahaa (dhamjecha bilisaa) fi dhamjecha hirkataa jedhamu. 

\item[•] Dhamjecha ofdandahaa\index{dhamjecha of danda'aa}: kun isa ofiisaatiin of dandahee kuusaa jechootaa keessa dhaabbatu.Kana kan jennus haala dhamjechi sun fufii fudhachuufi dhiisuu isaatiini.Dhamjechi ofdandahan maqaa, maqibsa, firoomsee ykn Gochimibsa tahuu dandaha.  Fakkeenyaaf, jechoonni dheedhii, gowwaa, guutuu jedhan maqibsa jedhamu; jechoonni aannan, bishaan, qoraanfi kkf ammoo maqaa jedhamu. Jechoonni kunneen fufii osoo ofitti hin ida’atiin ofdandahanii kuusaa jechootaa keessatti galmaahanii argamuu waan dandahaniif dhamjecha ofdandahaa jedhamanii beekamu.

\item[•] Dhamjecha hirkataa\index{dhamjecha hirkataa}: dhamjechi hirkataan garuu fufii ofitti ida’ata. akka latoo ofdandahaa ofisaatiin ofdandahee kuusaa jechootaa keessatit hinargamu.Dhamjechi hirkataan hundee jechaa yknfufii tahuu dandaha.

\item[•] Hundee jechaa\index{hundee jechaa}:  hundeen jechaa akaakuu dhamjecha hirkataati. jechaa jechuunis dhamjecha qaama jechaa isa fufiin erga irraa molqamee dhumee argamudha.  Karaa jecha gara biroo, hundeen jechaa unka jechaa isa fufiin osoo itti hinida’amiin dura argamudha. keessatti maqaa, maqibsa, gochimafi gochim ibsi hundee jechaa qabu.  Fakkeenyaaf jechoota armaan gadii hundi hundee gochimaati: adams-, arg-, bar-, booy-, bu’-fi kkf. Hundeewwan kunneen hundeewwan gochimaati. kanneen irratti fufii \{-e\} yoo idaane gochima yeroo dabre keessatti raawwatame hubanna: adamse, arge, bare, bu’efi kkf. Hundeewwan kunneen gochimoota waan ta'aniif ramaddiiwwan hundaaf nibay'atu; fkn, arge, argite, argine, argitan, arganfi kkf. 

\item[•] Fufilee\index{fufilee}: fufilee/maxxantoonni akkaataa tajaajila isaanittiin bakkoota lamatti qoodamanii ilaalamu. : fufilee hormaataa fi fufilee dhalatooti.  Kallattii jechatti fufamniin ammoo fufii duraa, fufii boodaafi fufii gidduu jechamu. Fufiin duraa hundee jechaa dursee dhufa; fufiin boodaa karaa boodaa maxxana; fufiin gidduu ammaa saagaa yoo ta’u gidduutti fufama. 

\item[•] Fufileen hormaataa\index{fufilee hormaataa}: fufileen hormaataa jechootarratti haala adda addaatiin maxxanuudhaan lakkoofsa, korniyaa, maayii, ramaddiifi kan kanneen fakkaatani agarsiisu. Fufileen kunneen maqaa, maqibsafi gochimarratti ida’amu (boqonnaawwan 4, 5,6,7fi 8 ilaalaa).

\item[•] Fufilee dhalatoo\index{fufilee dhalatoo}: fufileen dhalatoo kanneen jedhaman fufilee garee dubbii jijjiiruu dandahan. Kana jechuunis maqibsa gara maqaatti, gochima gara maqaatti, maqaa gara gochimaattifi kkf jijjiiru dandahu. Kana jechuunis garee dubbii tokko irraa garee dubbii biroo uumu jechuudha. Karaa gara biroommoo garee dubii jijjiiruu dhiisee garee dubbii tokko keessaa akaakuu dubbii isa mataa isaa fakkaatu biroo baasa. Fakkeneyaaf, fufiin dhalatoo tokko maqaa 
irraa maqaa uuma; nama jedhee namummaa wayita jedhu jechuudha (boqonnaawwan 4, 5,6,7fi 8 ilaalaa).
\end{itemize}

Garee jechaa\index{garee jechaa}: akkuma jalqabarratti eerretti qubeewwan adda addaa walitti makamanii jechoota ijaaru. Jechoonni adda addaa ammoo walitti makamanii hima ijaaru. Karaa jecha biroo himni tokko akaakuujechootaa gara garaa irraa ijaarama. Jechoonni hima ijaaran kunneen akkaataa unka isaaniif tajaajila isaaniitiin bakka adda addaatti qoodamu. Qoqqooddiin jechootaa kunis  garee jechaa jedhama. Karaa jecha gara biroo garee jechaa jechuun jechoota qabiyyee jechuudha (kan durduubee irraa hafe jechuudha; durduubeen jechoota qabiyyee miti). Gareen jechaa MAQAA, MAQIBSA, GOCHIMA, GOCHIMIBSA fi DURDUUBEE ofkeessaa qaba; isaaninis kutaawwan kanatti aanani dhufan keessatti tokko tokkoon ilaalla. 

\newpage
\section{Maqaa\index{maqaa}}

Kutaa kana keessatti maqaa bu'uuraa, maqaa dhalatoo, maqaa dhuunfaa, maqaa waloofi fufilee maqaa irratti fufaman ilaalla. Maqaan jechoota qabiyyee keessaa tokko. Akkasumas maqaan garee jechootaa keessaa tokko. Maqaan namoota dabalatee, waantota lubbuu qabaniif hinqabne hunda bakka bu'a. Maqaan waantota qabatamaan mul'ataniif kanneen lakkawwaman ykn kanneen hinlakkwwamnes bakka bu’a. Maqaan akaakuu adda addaatti qoodama. Isaan keessaa maqaa dhuunfaa, waloo, lakkaa’amu, hinlakkaa’amne, bu’uuraa, dalatoofi kkf jedhamu. 

\item[•] Maqaa dhuunfaa\index{maqaa dhuunfaa}: Maqaan dhuunfaa kan namni tokko, bakki tokko, bineeldi tokko, lagani tokko, tulluun tokko, magaalli tokko, biyyi tokko dhuunfaadhaan ittiin waamamudha.Fakkeenyaaf Abbayya maqaa lagaati. Maqaan namaa adeemsa ykn jijjiirama siyaasaa, amantii,dinagdeefi hawaasummaa waliin walqabatee moggaafama.Fakkeenyaaf moggaasni maqaa Akka lakkoofsa Itoophaatti bara 1983 as moggaafamaa jiran kan duriiraa adda\cite{sinqinash2018}. Dhimma kana xinnoo gad fageessinee ilaaluun yaada kana qabatama taasisa.Moggaasa maqaa daa’immanni mo’icha, dinagdee, dhiibbaafi firooma agarsiisan kan bara 1983 asittii\cite{sinqinash2018}. Akka Sinqinesh jettutti maqaaleen Didiyaa, Falmataa, Hamarasanfi kkf injifannoo agarsiisu. Maqaaleen kannen akka Ankeet, Ansiif, Horrenusiif kkf dinagdee agarsiisu. Maqaaleen kanneen akka Didiyaa, Maafloogan, Nugataniif kkf dhiibbaa agarsiisu. Maqaaleen kanneen akka Atikooti, Asaantuu, Firaanoliif kkf firooma agarsiisu.

\item[•] Maqaa waloo\index{maqaa waloo}: maqaan kun kan namoonni, bineeldonni,bakkeewwan, waantonnifi kkf gamtaan ittiin waamaman.Fakkeenyaaf maqaalee waloo armaan gadii haa ilaallu:\\
a. Nama= dhiira, dubartii, jaarsa, jaartii...hundaaf maqaa walooti.\\
b. Saree=akaakuu saree hundaafi maqaa walooti.\\
c. Muka=akaakuu muka kamu of keessatti qabata.\\
d. Laga=akaakuu laga hundaaf maqaa walooti.

\item[•] Maqaa lakkaa’amu\index{maqaa lakkaa'amu}: maqaan lakkaa'amu hammam akka ta'e qabatamaan adda baafama. Fakkeenyaaf maqoota armaan gadii haa ilaallu:\\
a. nama-namoota lama\\
b. mana - manneen sadii\\
c. lukkuu - lukkuu sagalfi kkf.

\item[•] Maqaa hinlakkaa’amne\index{maqaa hinlakkaa'amne}: maqaan hinlakkaa'amne kan tokko lama jennee ibsu hindandeenyedha. Fakkeenyaaf maqoota armaan gadii haa ilaallu:\\
a. xaafii - xaafii lama hinjedhamu;\\
b. bishaan - bishaan afur hinjedhamu;\\
c. aannan - aannan lama hinjedhamu; kkf.
\end{itemize}

Madda irratti hundoofnee maqaa bakka lamatti qooduu dandeenya. Isaanis maqaa bu'uuraa\footnote{Gaaleen 'maqaa bu'uuraa' jedhu gaalee Afaan Inglizii'basic nominal' jedhu bakka bu'a.}fi maqaa dhalatoo\footnote{Gaaleen 'maqaa dhalatoo' jedhu gaalee Afaan Inglizii'derived nominal' bakka bu'a.} jedhamu. Maqaan bu’uuraa durumaan kaasee afaanicha keessa jira. Fakkeenyaaf, nama, hoolaa, muka, bishaan, kkf. Maqaan dhalatoo garuu maqaa irraa ykn garee jecha biroorraa uumama. Dhimma kana gad fageenyaan ilaalla.

\subsection{Maqaa Dhalatoo\index{maqaa dhalatoo}}

Maqaan dhalatoo jechuun afaanichuma keessatti kan maqaa ykn gochima ykn maqibsa irraa dhalatee tajaajila maqaa kennudha. Fufiileen adda addaa hundee maqaa, maqibsa ykn.gochimaatti fufamuun maqaa adda addaa uumu. Isaan keessaa muraasa isaanii akka armaan gadiitti ilaalla.\\
\begin{itemize}

\item[•] Maqaa dudhaa\index{maqaa dudhaa}: fufileen/maxxantoonni \{-umma(a)\}, \{-oma\},\{-eenya\}fi \{-inaan\} hundee maqaa ykn maqibsa irratti maxxanuudhaan maqaa dudhaa uumu. Maqoonni dhalatoo haala kanaan uumaman kanneen sammuu keessatti xinxalamanii dudhaafi aadaa uummataa calaqqisiisanii malee kanneen qabatamaan mul'atn miti; amala dhuunfaa osoo hintane uummata keessatti kan gatii qabu agarsiisu. Fufiin {–ummaa} bu'uuraa irratti ida’ame maqaa dhalatoo dudhaa agarsiisu uuma. Fakkeenyaaf jechootni nam-ummaa, ijooll-ummaafi gooft-ummaa haala kanaan ijaaraman. Akkasumas fufiileen \{–eenya\},\{-oma\},\{-ina\} fi \{–insa\} maqibsa irratti ida’amanii maqaa dudhaa uumu. Fakkeenyaaf, jechoonni kan akka jab-eenya, gamn-ooma, diim-inafi jibb-insa jiran haala kana hordofanii uumaman. 

\item[•] Maqaa adeemsaa\index{maqaa adeemsaa}: fufileen \{-sa(a)\}, \{-a(a)\} fi \{-taa\}n gochima irratti fufamanii maqaa dhalatoo adeemsa raawwii hojii agarsiisu uumu. Jechoonni kanneen akka adeem-sa, bit-taa, gurgur-taafi cabs-aa jiran fakeenya maqaa adeemsaati. Maqoota dhalatoo kanneen hima keessatti ilaaluun gaariidha.\\
a. Ogganaan tokko adeemsa hojii to’ata.\\
b. Gabaa keessatti bittaa gurgurtaan jira.\\
c. Bakki lafee cabsaa jedhamu jiraa?\\
d. Dabsaan sibilaa ogummaa qaba.

\item[•] Maqaa bu’aa\index{maqaa bu'aa}: Akkuma maqaan isaa ibsutti maqaan bu'a, sababa waan tokko ta’eeruuf bu’aan tokko mul’achuu garsiisa. Maqaa bu’aa uumuuf fufilee gara garaa \{-umsa,-sa, -aa, -tee, -ii\}n gochimarratti ida’amu. Akka fakkeenya maqaa bu'aatti jechoota abaar-sa, beek-aa, dadhabb-ii, kenn-aafi mur-tee fudhachuu dandeenya.

\item[•] Maqaa mataduree\index{maqaa mataduree}: Maqaan hubannoo, hojjaa ykn gocha tokkoof akka matadureeti tajaajila. Fufii \{–uu\} n maqaa mataduree uumuuf hundee gochimaa irratti ida'amti. Fakkeenyaaf, bituu, jjibbuu, jaallachuufi kkf. maqoota matadureeti. 

\item[•] Maqaa akkaataa/haalaa\index{maqaa akkaataa/haalaa}: Maqaan akkaataa mala ykn haala ittiin adeemsi ykn gochi tokko raawwatu agarsiisa. Maqaan kun yeroo gochima irratti fufileen \{–ii\} , \{-umsa\} fi \{-aati\} n maxxanani uumama. Fakkeenyaaf jechoonni akka dhug-aatii, qal-umsafi ijaajj-ii maqoota akkaataati.

\item[•] Maqaa meeshaa\index{maqaa meeshsaa}: maxxantoonni kanneen akka \{–ata\} , \{-aa\}fi \{-(i)tuu\}n hundee gochimaa irratti maxxanuudhaan maqaa meeshaa uumu. Maqoonni kanneen akka har-ataa, hafars-aa fi fur-tuu jiran haala kanaan ijaaraman.

\item[•] Maqaa abgochaa\index{maqaa abgochaa}: maqaan abgochaa nama hojjaa tokko hojjechuuf dandeettii qabu agarsiisa. Fufileen \{-aa\} fi \{-tuu\}n gochima irratti fufamuun maqaa abgochaa uumu. Fakkeenyaaf, ajjeess-aafi barsiis-tuun haala kanaan ijaaramu.
\end{itemize}

\subsection{Maqaafi Korneyaa\index{maqaafi korneyaa}}

Jechi 'korniyaa' jedhu nama ykn waan jechi tokko bakka bu'eeru dhiira ykn dubartii ta'uu kan agarsiisu unka yookiin fufii maqaa irratti maxxanu. Korniyaan namni tokko ykn waan tokko uumaatiin dhiira ykn ubartii ta'uu agarsiisa. AO keessatti fufiin korniyaa dhiiraa agarsiisu ichi/icha yoo ta'u, fufiin korniyaa dubartii agarsiisu ammoo \{–ittiin/ittii\} dha. Maqaan tokko hima keessatti bakka matimaa galee korniyaa dhiiraa agarsiisa yoo ta'e, fufii \{–ichi\} jedhu fudhata; bakka antima galee korniyaa dhiiraa agarsiisa yoo ta'e garuu fufii \{–icha\} fudhata. 

Akkasumas, maqaan tokko bakka matimaa galee korniyaa dubartii agarsiisa yoo ta'e fufii \{–ittiin\} yoo fudhatu, bakka antimaa galee korniyaa dubartii agarsiisa yoo ta'e garuu fufii \{–ittii\} fudhata. Fakkeenyaa,\\
1. -ichi\\
a. Sareen dhufe\\
b. Sar-ichi dhufe.\\
2. -icha\\
a. inni saree arge.\\
b. inni sar– icha arge.\\
3. -ittiin\\
a. Sareen dhufte.\\
b. Sar-ittiin dhufte.\\
4. -ittii\\
a. Isheen saree argite.\\
b. Isheen sar– ittii argite. 

Kana malees, maqaaleen uumaatiin dhiiraa ykn dubartii mul'isan jiru. Fakkeenyaaf,\\
1. Dhiira/kormaa\\
a. sangaa\\
b. korma\\
c. korbeessa\\
d. muka\\
e. qilleensa\\
f. samiifi kkf.\\
2. Dubartii/dhaltuu\\
a. sa'a\\
b. goromsa\\
c. raada\\
d. biiftuu\\
e. ji’a\\
f. billaachafi kkf.

\subsection{Maqaafi Lakkoofsa\index{maqaafi lakkoofsa}}

Lakkoofsa heddummina yookiin qeentee agarsiisuuf fufii maqaa irratti maxxansina. AO keessa, fufiiwwan maqaa irratti ida'amanii lakkoofsa heddumminaa agarsiisan jiru. Isaan keessaa fufiiwwan \{–oota/ota\} \{-wwan\}, \{-een\}, \{-lee\}, \{-olii\}, \{-olee\} fi \{–aan\} ilaalla. Fufiin \{-oota/-ota\} jedhu heddummina agarsiisuun beekamaadha\cite{griefenow2001grammatical}. Fufiin kun birsaga dubbachiiftuu gabaabaa qabuun dursameera yoo ta'e {-oota} ta'a, /o/n
nidheerata jechuudha. Garuu birsaga dubbachiiftuu dheeraatiin yoo dursame {-ota} ta'a; /o/n nigabaabbata jechuudha . Fakkeenyaaf himoota armaan gadii haa ilaallu: \\
a. Hattuun tokko dhufe.\\
b. Inni hatt-oota lama arge. \\

Hima (a) maqaan hattuun jedhu qeenteedha. Hima (b) irratti garuu maqaan hattoota jedhu hedduudha; fufiin \{-oota\} jedhu birsaga dubbachiiftuu gabaabduu qabuun dursameera. \\
a. Inni afaan tokko dubbata.\\
b. Isheen garuu afaan-ota lama beekti.

Hima (a) irratti akka hubanutti maqaan afaan jedhu qeenteedha. Hima (b) irratti garuu maqaan afaanota jedhu hedduudha; fufiin \{-ota\} jedhu birsaga sagalee dheeraatiin waan dursameef sagaleen /o/ abaabbateera. Fufiin \{-wwan\} jedhus heddummina agarsiisa. Kanas fakkeenya armaan gadii irraa hubachuu dandeenya. \\
a. Lammiin koo dhufe.\\
b. Lammiiwwan koo dhufan.

Akkasumas \{–een\} fi \{-lee\} n lakkoofsa hedduu agarsiisuu. \\
a. Isheen farda tokko qabdi.\\
b. Inni fardeen hedduu qaba.\\
c. Ani jabbii tokko qaba.\\
d. Isheen jabbilee hedduu qabdi. \\

Yeroo baayyee fufii \{–een\} wayita fufamu sagaleen nijabaata.Fakkeenya, maka-mukkeen; laga-laggeen.Fufileen \{–olee\}, \{-olii\} fi \{–aan\} heddummina agarsiisu. Fakkeenaaf,
gaangolii, jarsolee, ilmaan fi kkf tilmaamuun danda'ama.

\subsection{Maqaa Beekamaafi Dhokataa\index{maqaa beekamaafi dhokataa}}

AO keessatti maqaan tokko beekamaa ykn dhokataa ta'uu danda'a. Beekamaa
jechuun namni dubbattuufi namni dhaggeeffatu nama waa’een isaa/ishee dubbatamu beeku jechuudha. Yoo hinbeekne ta’e dhokataa jedhama. Maqaan tokko dhokataa yoo ta'e fufii ittiin dhokataa agarsiisu hinqabu.Kana jechuunis maqaa dhokataan fufiitiin dhokataa \{Ø\} qaba jechuudha. Garuu maqaan tokko beekamaa yoo dhiiraaf fufii \{-icha\}, dubartiif \{-ittii\}maxxanfata. Fakkeenyaaf,\\
a. Poolisiin hattuu qabe.\\
b. Poolisiin hatticha qabe. 

Fakkeeny (a) irratti eerame maqaan hattuu jedhu dhokataa ta'uu agarsiisa; hattuun hinbeekamu. Hima (b) irratti eerame keessatti garuu maqaan hatticha jedhu sun beekamaa ta'uu agarsiisa. Fufiin \{-icha\} maqaa korniyaa dhiraatti maxxane. 
a. Isaan dubartii argan.\\
b. Isaan dubartittii waaman. \\

Hima (a) irratti maqaan dubartii jedhu dhokataadha. Hima (b) irratti garuu maqaan dubartittii jedhu nama dubbatuufis ta'e nama dhaggeeffatuuf beekamtuudha. Fufiin \{-ttii\} jedhu kun maqaa dubartii agarsiisutti maxxane. AO keessatti maqaan tokko waan hedduu bakka bu'a yoo ta'e fufiin beekamaas dhokataas itti hinmaxxanu.\\
a. Poolisiin hattoota qabe.\\
b. Poolisiin hattooticha qabe. (hinjedhamu)\\
c. Poolisiin hattootattii qabe. (hinjedhamu)\\

Akkuma himoota armaan olii irraa argamutti, maqaan hattoota jedhu hedduu agarsiisa; sababni isaas fufiin \{–oota\} jedhu waan maqaa irratti maxxaneef. AO keessatt fufii \{-oota\} tti aanee fufiin waan dhufuu inqabneef himootni (b)fi (c) irratti agarsiisaman fudhatama hinqaban; seera AO waan diiganiif. Gabeteen armaan gadii odeeffannoo fufilee beekamaafi dhokataa gabaabumatti lafa kaa'a:\\

\subsection{Maayii Maqaa\index{maayii maqaa}}

Fufileen adaddaa maqaa irratti ida'amuun maqaan sun hima keessatti tajaajila maal akka kennu agarsiisu. Maqaan tokko hima keessatti akka matimaa, akka antimaatti, akka meeshaattifi kkf tajaajiluu mala. Tajaajilli maqaan hima keessatti kennus fufii maqaatti fufamuun adda ba'ee beekama. Fufilee maayiis kutaa kana keessatti ilaalla.

\begin{enumerate}
	\item {Maayii matimaa\index{maayii matimaa}}: Fufiin maayii mathimaa agarsiisu maqaa fi maqibsa irratti maxxana. AO keessatti unka matimaaf seeronni armaan gadii hojjetu\cite{griefenow2001grammatical}: \\
a. Jechi birsaga CV tiin gochima yoo ta’e fufiin matimaa \{–ni\} dha. Fakeenyaaf jecha 'nama' jedhuuf matimni isaa 'nam-ni'dha.\\
b. Jechi birsa CCV tiin gochima yoo ta’e fufiin matima agarsiisu \{–i\} dha. Fakkeenyaaf jecha 'obboleessa' jedhuuf unki matimaa 'obboleess-i'dha.\\
c. Jechi dubbachiiftuu dheeraatiin xumura yoo ta’e, fufiin matimaa \{–n\} dha. Fakkeenyaaf jecha 'mataa' jedhuuf unki matimaa 'mataa-n'dha.\\
d. Jechi dhamsaga /n/tiin gochima yoo ta’e, fufii matimaa homaa hin’ida’atu. Fakkeenyaaf jecha 'aannan' jedhuuf unki matimaa 'aannan'dha.\\
e. Maqaaleen dubartii tokko tokko fufii matimaa \{–ti\} ida’atu (kun looga Hararitti mul’ata). Fakkeenyaaf jecha 'lafa' jedhuuf unki matimaa 'laf-ti'dha.

	\item{Maayii kennaa\index{maayii kennaa}}:  Fufiin maayii kennaa AO keessatti bifa adda addaa jechuunis \{-dhaaf\}, \{-tiif\}, \{-ii\}, \{-iif\}, \{-a\}, \{-f\}, qaba. Bifoota kanneenis
fakkeenyota armaan gadiirraa hubachuu dandeenya:\\
a. Inni kitaaba barsiisaa-f kenne.\\
b. Inni kitaaba barsiisaa-dhaaf kenne.\\
c. Inni kitaaba barsiisaa-tiif kenne.\\
d. Inni soogidda loon-ii kenne.\\
e. Inni soogidda loon-iif kenne.\\
f. Isheen qarshii nama-a kennite.\\
g. Isheen qarshii nama-af kennite.

Akkuma armaan olitti ilaalutti maqaan dubbachiiftuu dheertuutiin xumuru maayii kennaa \{-dhaaf\}, fi \{-tiif\} maxxanfata. Maqaan dubbachiiftuu gabaabduu dhumarratti qabummoo dubbachiiftuu dheereffachuu ykn dubbachiiftuu dheereffatee \{–f\} ida'achuun maayii kennaa agarsiisa. Maqaan dubbifamtuu dhumarratti qabu garuu \{–ii\} ykn \{-iif\} maxxanfatee maayii kennaa agarsiisa. 

	\item{Maayii meeshaa}: Maayiin meeshaa\index{maayii meeshaa} waan tokko maaliin akka hojjetame mul'isa.Maayii meeshaa agarsiisuuf maqaa irratti maxxantoonni \{-dhaan\}, \{-an\}, \{-tiin\}, \{-iin\} fi \{-n\}, ida'amu. Fakkeenyaaf himoota armaan gadii haailaallu:
a. Marqaa aannan-iin marqite.\\
b. Marqicha fal’aana-an nyaatte.\\
c. Adamsaan qawwee-tiin/dhaan warabboo ajjeesse.

Akka fakkeenyota kana irraa hubannutti maqaan dubbachiiftuu dheeraa dhumaa qabu \{-tiin/-dhaan\}, maqaan dubbachiiftuu gabaabduun fixu \{–an\}, maqaan dubbifamtuun raawwatu ammoo \{-iin\} maxxanfatanii maayii meeshaa agarsiisu.

	\item{Maayii antimaa}: Maayiin antimaa\index{maayii antimaa} nama ykn qaama gochi irratti raawwatame agarsiisa. Maayiin antimaa fufii ofirratti hin’ida’atu. Kanaaf unki bu’uuraa maayii antimaa agarsiisa\cite{griefenow2001grammatical}. Fakkeenyaaf unka matimaafi antimaa walbira qabnee haa ilaallu:\\
1a. Namni dhufe (namni matima).
1b. Ati nama argiteettaa? (antimni nama)
2a. Adurreen hantuutaa nyaatte. (matimni adurree)
2b. Hantuun nyaatamte. (antuutni antima)

	\item{Maayii qabeenyaa}: Maayiin qabeenyaa\index{maayii qabeenyaa} waan tokko kan eenyuu akka ta’e agarsiisa. Tartiiba maayii qabeenyaa keessatti waan qabaatamu dura dhufa, abbaan waan sanaammoo itti aanee dhufa. Jecha abbaa qabeenyaa agarsiisu irratti karaa dhumaa sagaleen dheerata. Maayiin qabeenyaa unka matimaafi antimaa qaba (GriefcnowMewis, 2001: 43). Maayiin qabeenyaa hiika adda addaas qaba. Fakkeenyaaf, mana namaa yoo jenne qabeenya agarsiisa; buna Wallaggaa yoo jenne madda agarsiisa; muka manaa yoo jenne kaayyoo agarsiisa; barreessituu koree yoo jenne miseensa agarsiisa; naannoo keenyaa yoo jenne bakka jireenyaa agarsiisa; oggannaa BBO yoo jenne hariiroo agarsiisa; adamoo bineensaa yoo jenne matimaafi antima agarsiisa. 
	
		\item{Maayii bakka irraa dhufan}: Maayiin bakka irraa dhufan\index{maayii bakka irraa dhufan} waan tokko ykn qaamni tokko eessaa gara kamiitti akka adeemu agarsiisa. Maayiin kun dubbachiiftuu dheeressuun, fufilee –dhaa, -ii fi –tii ida’uun
		mul’ata\cite{griefenow2001grammatical}. Fakkeenyaaf himoota armaan gadii haa ilaallu:\\
		a. Isaan Amboo gara Bishooftuu dhufan.\\
		b. Irreecha kabajuuf Walisodhaa dhufan.\\
		c. Qulqullinni barnootaa yeroodhaa yerootti cimaa deema.\\
		d. Sirni barreessuu afaanii afaanitti adda adda.\\
		e. Yeroon mana barnootaatii galee matii isaa hojii gargaara. 
		
		\item{Maayii Bakkaa Itti Dhufan}: Maayiin bakka itti dhufan\index{maayii bakka itti dhufan} ykn itti adeemaan kalatti itti adeeman ykn waan irratti hojjetan agarsiisa. Maayiin kun fufii \{–tti\} dhaan agarsiisama. Fakkeenyaaf himoota armaan gadii haa ilaallu:\\
		a. Manatti na eegi.\\
		b. Akkuma manaa ba’een namatti dhufe.\\
		c. Karaan kun eessatti nama baasa?\\
		d. Biyya kanatti maaltu ta’aa jira?

\end{enumerate}

\subsubsection{Gaaffilee Kutichaa}

Yaadrimeewwan armaan gadii fakkeenyaan ibsi:\\
1. Maqaa dhuunfaa\\
2. Maqaa waloo\\
3. Maqaa dhalatoo\\
4. Maqaa uumamtee\\
5. Maqaa bu’aa\\
6. Maqaa adeemsaa\\
7. Maqaa mataduree\\
8. Maqaa abgochaa\\
9. Fufii korniyaa\\
10. Fufii lakkoofsaa\\
11. Fufii maayii\\
12. Fufii maayii matimaa\\
13. Fufii maayii kennaa\\
14. Fufii maayii bakka itti dhufan\\
15. Fufii maayii bakka irraa dhufan

\newpage
\section{Maqibsa}
Maqibsi\index{maqibsa} akkuma maqaa jecha qabiyyeeti. Akkasumas maqibsi garee jechootaa keessaa tokkodha. Maqibsa maqaa irratti
dabalamee odeeffannoo dabalataa maqaaf kenna; karaa adda addaas maqaa adda baasee agarsiisaa. Kana jechuun maqibsi
karaa tajaajilaa agarsiistuufi lakkoofsa waliin wal fakkaata. Haala maqaa agarsiisuun, maqibsi akaakuu adda addaa qaba (Addunyaa, 2018: 59). Isaan keessaa isaan ijoo ta’an ilaalla. 

1. Maqibsa hammamtaa\index{maqibsa hammamtaa}: ibsitootni maqaa armaan gadii hanga
agarsiisu: \\
a. nama guddaa \\
b. saree xinnoo \\
c. muka dheeraa \\
d. hintala gabaabduu \\
e. dubartii furdoo fi kkf. \

2. Maqibsa akaakuu\index{maqibsa akaakuu} ykn bifa waan tokkoo agarsiisu:\\
Ibsitoonni maqaa armaan gadii akaakuu agarsiisu.\\
a. barataa diimaa\\
b. nyaata mi'aawaa fi kkf. \\
c. sangaa adii\\
d. dhagaa gurraachaa 

3. Maqibsa amala\index{maqibsa amalaa} agarsiisu: Ibsitoonni maqaa armaan gadii amala agarsiisu.\\
a. nama gaarii \\
b. dubartii cimtuu \\
c. gurbaa hamaa \\
d. hintala garalaafettii 

Maqibsi madda isaatiin bakkeewwan sadiitti qoodama: maqibsa bu'uuraa, maqibsa dhalatoo, maqibsa irra deebii. Maqibsi bu’uuraa ganamuumaa kaasee afaanicha keessa jira.Fakkeenyaf guddaa, diimaa, adiifi kkf. Maqibsi irra deebiis jira. Fakkeenyaaf, diimaa-diddiimaa,gurraachaguguraachaa, bal’aa-babal’aafi kkf. Maqibsi dhalatoo gadi fageenya waan qabuuf akka armaan gadiitti addeeffama. 

\subsection{Maqibsa Dhalatoo}

Akkuma maqaan dhalatoo\index{maqibsa dhalatoo} garee jechaa irraa uumamu maqibsi dhalatoos garee jechaa irraa uumama. AO keessatti
maxxantoonni \{- a, -aa , -tuu\} n gochima irratti maxxanuudhaan maqibsa dhalatoo uumu. Fkn, bareede (gochima), bareed-aa (maqibsa), bareed-tuu/bareedduu (maqibsa); ulfaate (gochima), ulfaat-aa (maqibsa), ulfaat-tuu (maqibsa). Akkasumas fufileen \{- eessa/- eetti\} n gochima irratti maxxanuudhaan maqibsa dhalatoo uumu. Fkn, 1. soorome (gochima), soor-eessa (maqibsa), Soor-ettii (maqibsa); daboome (gochima), dab-eessa (maqibsa), dab-eettii (maqibsa); ko’oome (gochima), ko’-eessa (maqibsa), ko’-eettii (maqibsa). Fufileen kanneen akka \{–ee ,-uu , –oo\} jiranis gochima irratti ida’amanii maqibsa dhalatoo uumu. Fkn, fagaate (gochima), fagoo (maqibsa); gadhate (gochima), gadhee (maqibsa); gobbate (gochima), gobbuu (maqibsa).

\subsection{Maqibsa, lakkoofsafi korneyaa}
AO keessatti maqibsi lakkoofsa\index{maqibsaafi lakkoofsa} heddummina mul'isa. Garuu mallattoo ykn fufiin qeentee agarsiisuu hinjiru. Maqibsi
heddummina agarsiisuuf hundee jechaa irra deebi'a. Fakkeenyaaf namoota gu-guddaa, dubartoota fu-furdoo, joollee qa-qalloo fi kkf lakkoofsa heddumminaa agarsiisu. 

AO keessatti maqibsi korniyaa\index{maqibsaafi korneyaa} dhiiraafi dubartii adda baasee agarsiisa. Maqibsa irratti fufii \{-tuu\} yoo idaane, maqibsi sun korniyaa dubartii agarsiisa. Garuu fufii \{-tuu\}n kun sagalee ishee dursee dhufetti waan gosagaloomtuuf yeroo baayyee bifa jijjiirratti. Yeroo biroo ammoo fufiin \{oo\} korniyaa dubartii agarsiisti. Korniyaan dhiiraa garuu fufii \{-aa\} tiin mul'ata. Akkasumas fufiin - cha dhiira –tittii n ammoo dubartii agarsiisu (beekamaas agarsiisu, Boqonnaa 4 ilaalaa) waliin. Fakkeenyaaf, dubartii diim-tuu, dhiira diim-aa, diim-icha fi diim-tittii fakkeenya.

\subsubsection{Gaaffilee kutichaa}

	1. Addaddummaan maqaafi ibsa maqaa gidduu jiru maali?\\
	2. Tajaajilli maqibsaa maali?\\
	3. Maqibsi dhalatoo maali?\\
	4. Maqibsi uumamtee maali?\\
	5. AO keessatti maqibsi akka maqaatti tajaajila; yaada kana
	fakkeenyaan ibsi.\\
	6. Fufileen lakkoofsaa maqibsarratti ida’aman kam fa’i?\\
	7. Fufileen korniyaa maqibsarratti ida’aman kam fa’i?\\
	8. Fufileen durduubee maqibsarratti ida’aman kam fa’i?\\
	9. AO keessatti maqaan akka maqibsaatti tajaajila; yaada
	kana fakkeenyaan ibsi. 

\newpage
\section{Agarsiistotaafi Lakkoofsota}

AO keessatti jechoonni maqaa irratti ida'amanii odeeffannoo dabalataa kennanis jiru. Jehoonni odeeffannoo dabalataa kennan kunneenis agarsiistotaa\index{agarsiistota} fi lakkoofsota\index{lakkoofsota} . Wayita jechoonni kunneen maqaa irratti dabalamnii tuta jechootaa ta’an gaalee maqaatu uumama. Maqaa dura dhufee odeeffannoo dabalataa argate ammoo mataa gaalee jenna. 

\subsection{Agarsiistota}
Agarsiistonni garee jechaatiin maqibsa jalatti ramadamu. Agarsiistonni gaalee maqaa irratti ida'amanii odeeffannoo
dabalataa kennu. Kanaaf namni dubbattu akka odeeffannoo sirriitti dabarsu gargaaru. Keessumattu, waan tokko dhiyoo
ykn fagoo akka jiru namatti agarsiisu. AO keessatti agarsiistonni lakkoofsafi korniyaa agarsiisu. Agarsiitota
dhiyeenyaafi fageenya agarsiisan akka armaan gadiitti ilaalla. 

\subsubsection{Agarsiistota Dhiyeenyaa}

Agarsiistonni dhiyeenya\index{agarsiistota dhiyeenyaa} agarsiisan bakka lamatti qoodamu; isaanis qeenteefi danuu jedhamu. Agarsiistonni qeentee
agarsiisan \textbf{kana} fi tana yoo ta'an agarsiistuun hedduu ammoo kanneen jedhama. Agarsiistuun kana jedhu kornaya dhiiraa agarsiisa; kan \textbf{tana} jedhu ammoo dubartii agarsiisa. Fakkeenyaaf himoota armaan gadii haa hubannu: \\
\\
a. Isheen gurbaa kana jaalatti. (qeentee, dhiira)\\
b. Inni hintala tana jaalata. (qeentee, dubartii)\\
c. Isaan fardeen kanneen bitatan. (hedduu) \\

Agarsiistonni dhiyeenya agarsiisan kunneen karaa unka isaanii bakka lamatti qoodamu: unka antimaafi unka matimaa.
Fakkeenyonni (a-c) agarsiisaman kunneen unka antimati. Sababn isaas hima keessatti bakka antima waan galaniif.
Wayita bakka matima galan unki isaanii nijijjiirama. Haala kanas fakkeenyota armaan gadii irraa hubachuu dandeenya: \\
\\
a. Hintalli tun gurbaa kana jaalatti. (qeentee, dubartii)\\
b. Gurbaan kun hintala tana jaalata. (qeentee, dhiira)
c. Fardeen kunneen kaleessa bitaman. (hedduu)\\

Akkuma fakkeenyota armaan olii irraa hubatamu kana agarsiistonni tun, kun fi kunneen jedhan unka matimaa qabu;
sababn isaas waan hima keessatti bakka matimaa galaniif. 

\subsubsection{ Agarsiistotaa Fageenyaa}

Agarsiistonni fageenyaas\index{agarsiistota fageenyaa} karaa lakkoofsaa bakka lamatti qoodamu; isaanis qeenteefi hedduudha. Agarsiistuun qeentee
sana yoo ta'u kan hedduu ammoo sanneen dha. Agarsiistonni fageenyaa korniyaa adda hinbaasan. Agarsiistonni fageenyaa unkaa antimaafi matimaa qabu. Unki matimaa sun fi sunneen yoo ta'u unki antimaa ammoo sana fi sanneen dha.Fakkeenyaaf himoota armaan gadii haa ilaallu.\\
\\
Agarsiistonni fageenyaa wayita bakka antimaa galan:\\
a. Barattootni adurree sana ilaalu.\\
b. Barattootni adurroota sanneen ilaalu.\\
\\
Agarsiistonni fageenyaa wayita bakka matimaa galan:\\
a. Adurreen sun antuuta barbaaddi.\\
b. Barattootni sunneen dareetii bahan. \\

Walumaa galatti agarsiistonni dhiyeenyaa nama isa dubbatutti waan dhiyoo jiru agarsiisuuf tajaajilu. Agarsiistonni fageenyaa ammoo nama isa dubbattu irraa waan fagoo jiru agarsiisuuf tajaajilu. Agarsiistonni dhiyeenyaa korniyaa agarsiisu; kana (dhiiraaf) fi tana (dhalaaf). Agarsiistonni dhiyeenyaa lakkoofsas agarsiisu; innis kanneen (hedduu) dha. Dabalataanis agarsiistonni dhiyeenyaa unka matimaa qabu; kun (dhiiraaf), tun (dhalaaf) fi kunneen (hedduuf) tajaajilu jechuudha. Agarsiistonni fageenyaa korniyaa hinagarsiisan; garuu lakkoofsa agarsiisu; isaanis sun (qeentee) fi sunneen (hedduudha). Agarsiistonni fageenyaa unka matimaafi antimas qabu; unki antima sana (qeentee) fi sanneen (hedduu) dha. Unki matimaa ammoo sun (qeentee) fi sunneen (hedduu) dha.

\subsection{Lakkoofsa}

Lakkoofsi garee jechootaatin maqibsa keessatti ramadama. Lakoofsi, akkuma maqibsa, maqaa irratti id'amee odeeffannoo
dabalataa gaalee maqaaf kenna. Kanaaf, lakkoofsi tajaajilaan agarsiistota fakkaata. Lakkoofsi maqaan ibsame tokko meeqa akka ta'e adda baasee agarsiisa. AO keessa lakkoofsi akaakuu lama qaba; isaanis lakkoofsa hammamtaafi lakkoofsa meeqantaa jedhamu. Walumaa galatti lakkoofsi jecha lakkaahuuf oola jechuudha. Lakkoofsi hammamtaa ammoo
namni ykn waan waa'een isaa dubbatamu lakkoofsi isaa hammam akka ta'e jecha ibsudha. Lakkoofsa meeqantaa jechuun ammoo, namni tokko ykn haalli tokko sadarkaa inni irratti argamu jecha adda baasee agarsiisudha. 

\subsubsection{Lakkoofsa Hammamtaa}

Akkuma armaan olitti ibsame, lakkoofsi hammamtaa\index{lakkoofsa hammamtaa} waan ibsame lakkoofsaan ykn baay'inaan hammam akka ta'e kan adda baasee agarsiisuudha. Fakkeenyaaf tokko, lama, sadii, afur, shan, ja'a, torba, saddeetti, sagalfi kkf lakkoofsa hammamtaati.

\subsubsection{Lakkoofsa Meeqantaa}
Lakkoofsi meeqantaa\index{lakkoofsa meeqantaa} maqaan tokko sadarkaa ykn tartiiba inni irratti argamu adda baasee agarsiisa. Fakkeenyaaf, barataa tokkoffaa, lammaffaa, sadaffaa, ... jechuu dandeenya. Lakkoofsi meeqantaa kan uumamu lakkoofsa hammamtaa
irraati. Kana jechuunis, fufii - affaa jedhu lakkoofsa hammamtaa irratti maxxansuun lakkoofsa meeqantaa uumuu
dandeenya jechuudha. Lakkoofsi meeqantaa looga Maccaafi Tuulama keessatti fufii –affaa tiin yoo agarsiisamu looga Harar keessatti ammoo fufii –eessa/-eessitu tiin agarsiisama; akkasumas looga Booranaa keessatti fufii -
eessoo tiin agarsiisama\cite{owens1985grammar,griefenow2001grammatical}. 

\begin{table}[H]
	\begin{tabular}{cccc}
		\hline\hline
		& Macca/Tuulama & Harar & Boorana \\
		\hline
		& tokkoffaa & dura & tokkeessoo \\
		\hline
		& lammaffaa & lammeessaa & lammeessoo \\
		\hline
		& sadaffaa & sadeessaa & sadeessoo \\
		\hline
		& afuraffaa & afreessaa & afreessoo \\
		\hline
		& shanaffaa & shaneessaa & shaneessoo \\
		\hline
		& ja'affaa & ja'eessaa & ja'eessoo \\
		\hline
		& torbaffaa & torbeessaa & torbeessoo \\
		\hline
		& saddettaffaa & saddeessaa & saddeessoo \\
		\hline
		& sagalaffaa & sagleessaa & sagleessoo \\
		\hline
	\end{tabular}
	\caption{Lakkoofa meeqantaa looga Maccaa, Tuulamaa, Harariifi Booranaa}
\end{table}

\subsubsection{Gaaffilee kutichaa}

1. Agarsiistota jechuun maal jechuudha?\\
2. Agarsiistotni dhiyeenyaa maal fa’i?\\
3. Agarsiistotni fageenyaa maal fa’i?\\
4. Lakkoofsi hammamtaa maali?\\
5. Lakkoofsa meeqantaa maali?\\
6. Tartiiba ruuqolee gaalee maqaa addeessi.

\newpage
\section{Bamaqaalee}

Bamaqaaleen\index{bamaqaalee} ramaddii, eenyu dhimma tokko akka raawwate ykn gochi tokko eenyu irratti akka raawwatame agarsiisu. Kana
jechuunis bamaqaaleen matima ykn antima agarsiisu jechuudha. Kanaaf bamaqaaleen unka matimaafi antimaa qabu. Haaluma kanaan AO keessa ramaddiiwwan sadiitu jiru; isaanis,ramaddii tokkoffaa, lammaffaafi sadaffaa jedhamu.
Ramaddiiwwan kunneen korniyaa adda baasu. Akkasumas unka qeenteefi danuu qabu. Namni dubbatu ramaddii tokkoffaa
jedhama; namni itti dubbatamu ammoo ramaddii lammaffaa jedhama; namni waa'een isaa dubbatamu ammoo ramaddii
sadaffaa jedhama. Walii galatti, AO keessa bamaqaalee torbatu jiru; isaan keessaas arfan isaanii lakkoofsa qeentee agarsiisu; sadan isaanii ammoo lakkoofsa danuu\footnote{Unki hedduufi kabajaa tokkuma. Fakkeenyaaf, "isaan dhufan" yoo jenne hedduu agarsiisa; akkasumas nama tokkoo kabajuufis fayyada.} agarsiisu. 

Bamaqaaaleen \textbf{ana}, \textbf{ati}, \textbf{ishee}fi \textbf{isa} jedhan unka qeenteeti. Bamaqaaleen \textbf{nu}, \textbf{isin} fi \textbf{isaan} unka hedduti. 

\begin{table}[H]
	\centering
	\begin{tabular}{|c|c|c|}
		\hline
		Ramaddii & Qeentee & Danuu \\
		\hline
		1 & ani & nuti \\
		\hline
		2 & ati & isin \\
		\hline
		3 &  &  \\
		Du. & ishee & isaan \\
		Dh. & isa & isaan \\
		\hline
	\end{tabular}
	\caption{Bamaqalee unka matimaatiin}
\end{table}

\begin{table}[H]
	\centering
	\begin{tabular}{|c|c|c|}
		\hline
		Ramaddii & Qeentee & Danuu \\
		\hline
		1 & na & nu \\
		\hline
		2 & si & isin \\
		\hline
		3 &  &  \\
		Du. & ishee & isaan \\
		Dh. & isa & isaan \\
		\hline
	\end{tabular}
	\caption{Bamaqalee unka antimaatin}
\end{table}

\subsection{ Bamaqaalee Qabeentaa}

Unki maqaa bakka bu'ee waan tokko kan eenyuu akka ta'e agarsiisu bamaqaa qabeentaa\index{bamaqaa qabeentaa} jedhama. AO keessatti qabeenyi
karaa lama agarsiisama: tokko karaa bamaqaalee qabeentaa yoo ta'u, inni lammaffaa ammoo karaa maqaa abbaa
qabeenyaati. AO keessatti, qabeenyi kan eenyuu akka ta'e agarsiisuuf abbaan qabeenyaafi (abbaan horii eenyuu akka
ta'e) qabeenyi walitti aananii dhufu; yeroo kana maqaan inni qabeenya namaa ta'u dura dhufa, bamaqaaleen abbaa
qabeenyaa ammoo itti aanee dhufa. Bamaqaaleen qabeentaa looga adda addaa keessatti unka adda addaa qabu. \\
a. hintala koo\\
b. hintala kee\\
c. hintala ishee\\
d. ilma isaa\\
e. ilma keenya\\
f. ilma keessan\\
g. ilma isaanii


Loogni Tuulamaammoo unka biroo qaba. Fakkeeny armaan gadii looga Maccaati\cite{griefenow2001grammatical}:\\
a. ilma kiyya\\
b. hintala tiyya/too\\
c. ilma kee\\
d. hintala tee\\
e. abbaa ishii\\
f. haada ishee/ishii\\

\subsubsection{Bamaqaalee Gaaffii}

Bamaqaalee gaaffii/iyaafannoo\index{bamaqaa gaaffii/iyyaafannoo} (Addunyaa, 2018: 51) jechuun jechoota gaaffii gaafachuuf tajaajilan jechuudha. Hima
keessatti eenyu gocha tokko akka raawwate, maal akka raawwate, eessatti akka raawwate, yoom akka raawwateefi kkf
gaafachuuf kanneen nu tajaajilan bamaqaalee gaaffiiti. Bamaqaalee gaaffii AO fakkeenyota armaan gadiirraa hubachuu
dandeenya. \\
\begin{itemize}
	\item eenyu: qaama gocha raawwate
	\item maal: antima gaafata (gaaffii gaafata)
	\item maaliin:meeshaa (waanta ittiin hojjetamu) gaafata
	\item eessa/eessaa/eessatti:  bakkatti gochi itti raawwatame
	\item yoom: yerootti gochi iitti raawwatame
	\item maaliif: sababa gaafata
	\item meeqa: lakkoofsa gaafata
	\item hammam: lakkoofsa gaafata
	\item akkam/akkamiin/akkamitti: haala/akkaataa gaafata
	\item kam: kam adda baafachuu
	\item kan kan eenyuu:  abbaa gaafata
\end{itemize}
Bamaqaalee armaan oliitti fayyadamnee gaaffii adda addaa gaafachuu dandeenya. \\
\\
a. Eenyu hoolaa bite?\\
b. Boontuun hoolaa eessaa bitte?\\
c. Boontuu maal bitte?\\
d. Boontuun hoolaa eessaa bitte?\\
e. Boontuun hoolaa yoom bitte?\\
f. Boontuun hoolaa meeqaan bitte?\\
g. Boontuun hoolaa maaliif bitte?\\
h. Boontuun hoolaa eenyuuf bitte?\\
i. Boontuun hoolaa akkamii bitte?fi kkf.

\subsubsection{Bamaqaalee Ofiifee}
Bamaqaaleen ofiifee\index{bamaqaa ofiifee/ofiinaa}/ufinaa\cite{Addunya2018} namni tokko ykn qaamni tokko gocha tokko ofisaa/ofiishee mataa isaatiif/ isheetiif akkasumas murtee mataa isaatiin/isheetiin raawwachuu agarsiisa. Bamaqaaleen ofiifee AO keessatti bifa sadiin mula’atu. Isaanis of, uffi if dha\cite{griefenow2001grammatical}. Fakkeenyaaf himoota armaan gadii haa ilaallu:\\
\\
a. Isheen ofumaa dhufte; \\
b. Inni ofumaan of gargaara.\\
c. Dhukkubni osoo nama hinqabiin of irraa ittisuun baayyee gaariidha;\\
d. Hamaan ofitti bada. 

\subsubsection{Bamaqaalee Garlamee}
 
Bamaqaaleen garlamee\index{bamaqaa garlamee/waliyyoo}/waliyyoo\cite{Addunya2018} (Addunyaa, 2018: 53) namni lama ykn lamaa ol gocha waliif deebisuu agarsiisa. Bamaqaaleen garlamee AO keessatti tajaajilan wal, walii, walitti, waliif, waliin dha.
\\
a. Namni biyya tokko ijaaru wal loluu hinqabu.\\
b. Namni mana tokko ijaaru citaa wal hinsaamu.\\
c. Yeroo kamuu tanaan walii tumsuun ba’eessa.\\
d. Walii galuun hinjiru yoo ta’e guddinni hinjiraatu.\\
e. Walii galan alaa galaan.\\
f. Namani yoo alagoome walirraa fagaata.\\
g. Namni yoo wal bare walitti firooma.\\
h. Waliif miiltoo ta’uu jechuun waliin imaluu jechuudha. \\

\subsubsection{Bamaqaalee Agarsiistuu}
Bamaqaaleen agarsiistuu\index{bamaqaa agarsiistuu} waan tokko dhiyeenyatti ykn fageenyatti agarsiisuuf tajaajilu\cite{griefenow2001grammatical}. Fakkeenyaaf himoota armaan gadii ilaalla:
\\
a. Namni kun eessaa dhufe? Dubartiin kun/tun eenyu?\\
b. Dubartii kana/tana dubbisi; namicha kana dhiisi.\\

Akkasumas bamaqaaleen agarsiistuu waan fagoo jiru agarsiisan
jiru. \\
\\
a. Namoota sanneen/kanneen beektaa?\\
b. Gurbaan sun barataadha.

\subsubsection{Gaaffilee kutichaa}
Qabxiiwwan armaan gadii fakkeenyaan ibsi:\\
\\
1. Bamaqaalee ramaddii\\
2. Bamaqaalee qabeentaa\\
3. Bamaqaalee gaaffii\\
4. Bamaqaalee ofiifee\\
5. Bamaqaalee garlamee\\
6. Bamaqaalee agarsiistuu

\newpage
\section{Gochima}
AO keessatti, kutaan himaa hunda caalaa barbaachisu gochima\index{gochima}; sbabni isaas yeroo baay'ee gochimni akka himaatti of danda’ee dhaabbachuun yaada guutuu dabarsa. Gochimni
yeroo waan tokko taasisamu ykn raawwatamu agarsiisa.Gochimni taateewwan, adeemsota, haala namni ykn waan tokko keessa jiru addeessa. Fufilee adda addaa sababa adda addaatiif waan maxxanfatuuf unkan gochimaa jijjiirama guddaa
agarsiisa.

\subsection{Ramaddii Gochimarratti}
AO keessatti, matimni gochima waliin wal simachuu qaba. Wal simannaan matimafi gochimaa fufii gochima irratti maxxanuun mul'ifama. Kana jechuun, fakkeenyaaf, matimni ramaddii tokkoffaa yoo ta'eera ta'e, fufiin ramaddii\index{ramaddii} tokkoffaa agarsiistu gochima irratti maxxantee abbaa gocha raawwate agarsiisti jechuudha. Ramaddiiwwan adda addaa agarsiisuuf fufiileen unka gara garaa gochima irratti maxxanu. Fufiileen kunneenis fufii boodaa/duraa tahuudhaan hundee gochimaa irratti maxxanu. Haala kanas fakkeenyota armaan gadii irraa haa hubannu. Fakkeenyaaf himoota armaan gadii keessatti fufiiwwan dhuma gochimaa jiran ramaddiiwwan matima agarsiisan. \\
\\
a. Ani nan-raf-a\\
b. Ati ni-raf-ta\\
c. Isheen ni-raf-ti\\
d. Inni ni-raf-a\\
e. Nuti ni-raf-na\\
f. Isin ni-raf-tu\\
g. isaan ni-raf-fu

\subsection{Hennaa}

\subsubsection{Seensa}
AO keessatti, gochima irratti kanneen maxxanan fufiilee ramaddii qofa miti. Fufiileen hennaa\index{hennaa} agarsiisanis gochima
irratti maxxananii gochi tokko yoom akka raawwatame odeeffannoo kennu. Namni dubbii gaggeessu tokko gocha
gaggeefame wayita gabaasu yeroo waliin karaa sadii walitti firoomsuu danda'a. Gochi tokko akka amma raawwatamaa
jirutti gabaasuu danda'a. Akkasumas gochi tokko gara fuulduraatti kan raawwachuuf jedhu ta'uu isaa gabaasuu
danda'a. Kana malees, gochi tokko yeroo darbe keessa kan raawwate ta'uu isaa gabaasa. Gocha tokko yeroo waliin walitti firoomsuun kun yeroo raawwii gochaa jedhama. Haala kanaan AO keessatti hennaa akaakuu sadiitu argama. Isaanis dabrennaa, ammennaafi duranaa jedhamu\cite{aadaa1995,margaa2010}. Hennaan moggaasa adda addaa qaba\cite{beekamaa1996,file2015}\footnote{Mogaasinni hennaa kitaaba kanaa \cite{aadaa1995} irratti bu’uureffata.}. 

Kitaaba kana keessatti fufilee hoormaataa kanneen booda gochimaa dhufan akka fufii tokkootti ilaalamu \cite{griefenow2001grammatical,owens1985grammar,gragg1976oromo,margaa2010}. Fakkeenyaaf mee jecha \textbf{deeme} jedhu fudhannee haa addeessinu. Hundeen jecha \{deem-\} dha. \{deem- + -e = deem-e\} ta’a. Unki \textbf{deeme} jedhu ramaddii tokkoffaa qeentee (ani) fi ramaddii sadaffaa qeentee (isa)fi tokko. Kana jechuun fufiin \{-e\} yeroofi ramaddii agarsiisti jechuudha. Dabalataanis lakkoofsa qeenteefi korniyaa dhiira agarsiisti. Karaa jecha gara biroo fufii tokkittiin odeeffannoo gara garaa kenniti; yeroo dabraa, ramaddii tokkoffaa ykn sadaffaa, lakkoofsa qeenteefi kkf. Garuu namoonni tokko tokko fufii \{-e\} n waan yeroo qofa agarsiistu itti fakkaata. Kun garuu dogoggora. Fufii \{-te\} jettus haa ilaallu. Namoonni tokko tokko fufii kana lamatti baasu; \{-t-, -e\} jedhanii adda adda qoodanii \{-t-\} n ramaddii agarsiisti \{-e\} n yeroo dabaraa agarsiisti jedhu. Garuu qaaccessi akkasi dogoggora. Osoo \{-t-\}n ramaddii qofa agrsiisti taate ramaddii lammaffaa qofaa keessa jiraachuu qabdi turte (ati deem-te). 

Garuu fufiin \{-t-\}n ramaddii sadaffaa keessas jirti (isheen deem-te). AO haala kanaan qaaccessuun dogoggora guddaa fida. AO keessatti fufiileen gochimarra jiran hennaa, lakkoofsa, ramaddii, korniyaa altokkotti agarsiisu.
Fakkeenyaaf, fufii \{-te\}, dabarennaa, lakkoofsa qeenteefi korniyaa agarsiisti. Kanaaf \{-t-, -e\} adda adda qooduun sirrii miti. Fufileen ramaddii tokkoffaa hedduu, \{-ne\} (deem-ne) yoo ilaalle haala walfakkaatu hubanna. Fufii kana bakka lamatti qoodnee \{-n-e\} mee haa jennu. Fufii \{-e\}n yeroo dabraa agarsiisa haa jennu. Fufii \{-n-\}n lakkoofsa heddu agarsiisaa haa jennu. Yoo akkas ta’e ramaddiifi korniyaa kamtu agarsiisare? Kanaaf fufii \{-ne\} bakka lamatti qoqqooduun bu’aa hinqabu; akkuma jirutti dabrennaa, ramaddii tokkoffaa, lakkoofsa heddufi korniyaa hinmurtoofne agarsiisa. Fufii \{-tan\} (deemtan) yoo ilaalle akksuma. Fufiin kun akkuma jirutti dabrennaa, ramaddii lammaffaa, lakkoofsa hedduuf, korniyaa hinmurtoofne agarsiisa.  Kanaaf barnoota keessatti fufii kana akkuma jirutti ramaddii, lakkoofsa, yeroofi korniyaa agarsiisa jechuutu filatama. Fufiin, \{-an\}is (deem-an) akka fufii tokkootti fudhatama malee adda adda hinqoqqoodamu ; akkuma jirutti odeeffannoo afran ibsa. Kutaalee kanatti aananii dhufan keessatti qaacceessi hennaa haala armaan oliin gaggeeffama. 

\subsubsection{Hennaa Salphaa\index{hennaa salphaa}}

Qaacceessi henaa kutaa kanaa hayyoota gara garaa bu'uureffata \cite{aadaa1995,griefenow2001grammatical,owens1985grammar,tolemariam2011}. Hennaa salphaan akaakuuwwan gurguddoo sadii qaba. Isaanis aldhumataa, ajaajaafi dhumataa jedhamu. 

\paragraph{Aldhumataa\index{aldhumataa}}
Aladhumatni unka aldhumataa \footnote{Jechi aldhumata jedhu jecha Aafaan Inglizii gerund bakka bu'a} fi amsiqaa (KATO, 1995) of keessaa qaba. 
\begin{itemize}
	\item Aldhumataa: Aldhumata jechuun gochima fufii matima waliin hidhata hinqabneefi yeroo hinmul’isne jechuudha. AO keessatti unki yeroo aldhumataa fufii \{-uu\} tiin xumura. Fkn, deem-uu, fiig-uu, bit-uufi kkf yeroos matimas hinmul’isan.
	\item Amsiqaa: Unki hojii amma gaggeeffama jiru agarsiisuuf . Unki ramaddii hundaafi \{-aa\} dha; gochimni \{-aa\} ida’ates yeroofi matima hinagarsiisu yoo gochima \textbf{jira} jedhu waliin dhufe malee. Fkn, nyaach-aa, barreess-aa, fiig-aa fi kkf aldhumata. Garuu hima ‘Nyaachaa-aa jirra.’ jedhu keessatti ‘jirra’n aldhumata miti.
\end{itemize}

\paragraph{Ajajaa\index{ajaja}}
 Unki ajajaa yeroo hinagarsiisu. Kanaaf aldhumata jenna. AO keessatti qeenteef unki ajajaa fufii –i tiin xumura; danuufi garuu \{-aa\} tiin xumura. Haa ta’u malee, jechi fufii \{-at-\} iin kan  xumuru yoo ta’e unki ajaajaa qeenteef –adhu tiin yoo xumuru  danuuf ammoo \{-adhaa\} tiin xumura. Fakkeenyaaf jechoota kanneen akka deem-i, taa’-i, raf-i, dubbis-i, balleess-i fi kkf hubachuu dandeenya.Jechi fufii \{-at-\} dhumarraa qabu garuu unki ajajaa qeenteef \{-adhu\} dha. Fakkeenyaaf, gudd-adhu, gabb-adhu, salph-adhu, bar-adhu fi kkf.
 
 \paragraph{Dhumataa\index{dhumata}}
Gochimni dhumataan matima waliin hidhata qaba; yeroos agarsiisa. Fufiileen hima of danda’aas ta’e hima hirkataa irratti fufamanii dabrrennaas ta’e egeree agarsiisan jiru. Akka waliigalaatti rammaddi tokkoffaa qeenxeen fufii dabalataa \{-n\} kan jecha gochima dursee dhufe irratti fufamu qaba. Dhumataan fufiilee akaakuu afur qaba. 

1. Fufilee yeroo duranaa agarsiisan hima of danda’aa duranaa agarsiisu keessatti irra caalaan fufiin \{-a\} yeroo ammaa agarsiisa; kanaaf dhumata jenna. Fakkeenyaaf, ani nan rafa, ati rafta, isheen rafti, isaan rafuufi kkf.

2. Fufiilee duranaa hima hirkaataa irratti fufaman-fufiileen armaan gaditti eeraman kunneen ciroo hirkataa irratti
fufamanii yeroo ammaa agarsiisu.As keessatti walqabsiistuu akkafi fufiileen gara garaa gochimarratti ida’aman yeroo duranaa agarsiisu. Fakkeenyaaf himoota armaan gadii haa ilaallu:\\
a. As dhufuunkoo akka nama gammachiis-u beektuu?\\
b. Barachuun akka nama kabachiis-u baraa.\\
c. Hojjechuun akka nama gammachiis-u baraa.\\
Hi’eentaan unka kanaa akka hin...ne kan jedhu. \\
a. Walaalummaan akka nama hin-fayyad-ne baraa.\\
b. Barachuun akka nama hin-mii-ne baraa.

3. Fufiilee gochimarratti fufamanii dabrennaa agarsiisan fufiileen dabrennaa agarsiisan gabatee armaan gadii keessatti eeramaniiru. Fakkeenyaaf, rafe, rafte, rafne fi kkf. Hi’eentaan unka kanaa ramaddii hundaaf \textbf{hin...ne }kan jedhu. Fakkeenyaaf, ani hin-dhuf-ne. 

4. Fufiileen gochimarratti ida’amanii haayyama agarsiisanfufiileen haayyama agarsiisan as keessatti ramadamu. \\
a. Biyyaaf dalaguuf karoora haa baafat-nu.\\
b. Eessaa akka dhufu haa tilmaam-nu.\\
c. Yaada wal jijjiiruuf sirriitti wal haa caqaf-nu.\\
d. Qormaata darbuuf haa qo’at-tu fi kkf.

\subsubsection{Hennaa Dachaa}
Hennaa dachaa\index{hennaa dachaa} akaakuu addaaddaa qaba. Isaanis raawwima ammeennaa \footnote{Present perfect}, raawwima dabrennaa \footnote{Past perfect}, amsiqa \footnote{Present progressive}, tarsiqa \footnote{Past progressive}, dabrennaa \footnote{Past habitual}, muranna \footnote{Future definite}fi murannaanala \footnote{Future indefinite} of keessatti qabata.

\paragraph{Raawwima Ammeennaa}
Raawwimni ammeenaa\index{raawwima ammeennaa} jechuun hojii yeroo dabre dalagameeru amma kan dhaabbateeru; garuu haala ammaa irraatti kan
dhiibbaa qabu agarsiisa. Seerri raawwiimni Ammeennaa akka armaan gadiin taa'uu danda'a:\\
\begin{itemize}
	\item \textit{\underline{Raawwima ammeennaa = Gochima (G)+ Fufii (F) + Tumsii Ammeennaa + Fufii}}
\end{itemize}
\begin{table}[H]
	\centering
	\caption{Raawwii ammeennaa}
	\begin{tabular}{ccccc}
		\hline\hline
		Ramaddii & Hundee & Fufii & Tumsii & Fufii \\
		\hline
		Qen &  &  &  &  \\
		1	& raf- & -een & jir- & -a		\\
		2	& raf- & -tee & jir- & -a	\\
		3 \\
		Dh.	& raf- & -ee & jir- & -a		\\
		Du.	& raf- & -tee & jir- & -ti		\\
		\hline
		Hed. \\
		1   & raf- & -nee & jir- & -na     \\
		2   & raf- & -tanii & jir- & -tu      \\
		3   & raf- & -anii & jir- & -u      \\
		\hline
		
	\end{tabular}
\end{table}

\paragraph{Raawwima Dabreennaa}
Raawwimni dabreennaa\index{raawwima dabareennaa} gocha yeroo darbe fageenyatti dalagamee kan yeroo ammaa irratti dhiibbaa hinqabne agarsiisa. Raawwimni dabreennaa seera armaan gadii qaba:\\

\begin{itemize}
	\item \textit{\underline{Raawwima Dabreennaa = Gochima + Fufii + Tumsii (Dabreennaa) +Fifii}}
\end{itemize}

\begin{table}[H]
	\centering
	\caption{Raawwii dabrennaa}
	\begin{tabular}{ccccc}
		\hline\hline
		Ramaddii & Hundee & Fufii & Tumsii & Fufii \\
		\hline
		Qen &  &  &  &  \\
		1	& raf- & -een & tur- & -e		\\
		2	& raf- & -tee & tur- & -e	\\
		3 \\
		Dh.	& raf- & -ee & tur- & -e		\\
		Du.	& raf- & -tee & tur- & -te		\\
		\hline
		Hed. \\
		1   & raf- & -nee & tur- & -ne     \\
		2   & raf- & -tanii & tur- & -tan      \\
		3   & raf- & -anii & tur- & -an      \\
		\hline
		
	\end{tabular}
\end{table}

Fakkeenyaaf: \\
a. Ani rafeen ture.\\
b. Ati raftee turte.\\
c. Isheen raftee turteefi kkf.

\paragraph{Amsiqaa}
Hennaan amsiqaa\index{amsiqaa} wayita namni dubbii gaggeessuu ykn barreessu sana gocha dalagamaa jiru agarsiisa. Amsiqaan
karaalee lama agarsiifama: 

\begin{itemize}
	\item \textit{\underline{1. G + aa + Jiruu (ammeennaa)}} ykn,
	\item \textit{\underline{2. G + uu + -tti + Jiruu (ammeennaa) }}
\end{itemize}

\begin{table}[H]
	\centering
	\caption{Raawwii amsiqaatokkoffaa}
	\begin{tabular}{ccccc}
		\hline\hline
		Ramaddii & Hundee & Fufii & Tumsii & Fufii \\
		\hline
		Qen &  &  &  &  \\
		1	& raf- & -aan & jir- & -a		\\
		2	& raf- & -aa & jir- & -ta	\\
		3 \\
		Dh.	& raf- & -aa & jir- & -a		\\
		Du.	& raf- & -aa & jir- & -ti		\\
		\hline
		Hed. \\
		1   & raf- & -aa & jir- & -na    \\
		2   & raf- & -aa & jir- & -tu      \\
		3   & raf- & -aa & jir- & -u      \\
		\hline
		
	\end{tabular}
\end{table}

Karaa seera lammaffaa ammoo unki amsiqaa kan armaan gadiiti: 
\begin{table}[H]
	\centering
	\caption{Raawwii amsiqaa lammaffaa}
	\begin{tabular}{ccccc}
		\hline\hline
		Ramaddii & Hundee & Fufii & Tumsii & Fufii \\
		\hline
		Qen &  &  &  &  \\
		1	& raf- & -uuttin & jir- & -a		\\
		2	& raf- & -uutti & jir- & -ta	\\
		3 \\
		Dh.	& raf- & -uutti & jir- & -a		\\
		Du.	& raf- & -uutti & jir- & -ti		\\
		\hline
		Hed. \\
		1   & raf- & -uutti & jir- & -na    \\
		2   & raf- & -uutti & jir- & -tu      \\
		3   & raf- & -uutti & jir- & -u      \\
		\hline
		
	\end{tabular}
\end{table}

\paragraph{Tarsiqaa}
Hennaan tarsiqaa\index{tarsiqaa} dubbatichi ykn barreessichi dhimma yeroo darbe keessatti raawwatamaa tureeru agarsiisaa. Seerri isaas kan armaan gadiiti: 
\begin{itemize}
	\item \textit{\underline{G + aa + Tumsii Turuu (Dabreennaa) + Fufii }}
\end{itemize}

Unka tarsiqaa gabatee armaan gadiirra hubachuun danda'ama:
\begin{table}[H]
	\centering
	\caption{Raawwii tarsiqaa}
	\begin{tabular}{ccccc}
		\hline\hline
		Ramaddii & Hundee & Fufii & Tumsii & Fufii \\
		\hline
		Qen &  &  &  &  \\
		1	& raf- & -aan & tur- & -e		\\
		2	& raf- & -aa & tur- & -te	\\
		3 \\
		Dh.	& raf- & -aa & tur- & -e		\\
		Du.	& raf- & -aa & tur- & -te		\\
		\hline
		Hed. \\
		1   & raf- & -aa & tur- & -ne    \\
		2   & raf- & -aa & tur- & -tan     \\
		3   & raf- & -aa & tur- & -an     \\
		\hline
		
	\end{tabular}
\end{table}

\paragraph{Murannaa}

Hennaan muranaa\index{murannaa} akaakuuwwan lama qaba. Isaanis muranaafi muranaanala\index{muranala} jedhamu. Hennaan muranaa yeroo duranaa
keessatti waan hojjetamuuf jedhu agarsiisa. Waan murtoo qabu jechuudha. Seerri isaas: 
\begin{itemize}
	\item M\textit{\underline{urannaa = G + uu + fi (dha) ykn G + uu + fin + jechuu }}
\end{itemize}

\begin{table}[H]
	\centering
	\caption{Raawwii murannaa}
	\begin{tabular}{ccccc}
		\hline\hline
		Ramaddii & Hundee & Fufii & Tumsii & Fufii \\
		\hline
		Qen &  &  &  &  \\
		1	& raf- & -uufan & jedh- & -a		\\
		2	& raf- & -uuf & jet- & -ta	\\
		3 \\
		Dh.	& raf- & -uuf & jedh- & -a	\\
		Du.	& raf- & -uuf & jet- & -ti	\\
		\hline
		Hed. \\
		1   & raf- & -uuf & jet- & -na   \\
		2   & raf- & -uuf & jet- & -tan     \\
		3   & raf- & -uuf & jet- & -an     \\
		\hline
	\end{tabular}
\end{table}

\subsection{Tarree Gochimaa}
Gochi walduraa duuba hojjetama. Kanaaf, gochimni tarree qaba. Tarree gochimaa\index{tarree gochimaa} jechuun gochimoota tarreetiin wal bira naquun hima dheeraa ijaaruudha. Gochima tarreessuun gochima tokko, lama yookiin lamaa ol hima gidduutti fiduudha. Walumaa galatti, AO keessatti, bakka hima keessatti dhufan irratti hundaa'uun, gochimoota akaakuuwwan lamatti qooduu dandeenya. Akaakuun tokko gochima xumura himaa dhufudha. Akaakuun biroo ammoo gochima xumura himaa hindhufnedha; karaa jecha gara biroo isa gidduu himaa dhufudha jechuudha. Gochimni xumura himaa dhufe yeroo mara fufiilee adda addaa kanneen yeroo, ramaddii, korniyaafi lakkoofsa agarsiisan maxxanfata. Gochimni gidduu himaa dhufu garuu gocha xumura himaa dhufe dura gochi biroo raawwatamuu isaa agarsiisa; fufiin itti maxxaneerus kanuma agarsiisa. Kana jechuun gochimni gidduu himaa dhufu fufiilee akaakuu adda addaa ofirraa hinqabau jechuudha. Garuu dubbachiiftuun dhumaa nidheeratti. Fakkeenyaaf himoota armaan gadii haa hubannu:\\
\\
a. Abbaankoo manaa ba’ee, hiriyyaa isaa mari’atee, farda yaabbatee, gabaa deemee, uccuu bitee dhufe.\\
b. Namichi rafee, ka’ee, weeddisee, manaa ba’ee, konkolaataa yaabbatee, gara hojjaa deeme.

Akka fakkeenya (a) fi (b) irraa ilaalutti gochimoonni tarreettiin adeemsa nama tokkoo ibsaniiru. Unki gochimoota walfakkaata; dubbachiiftuu dhumaa dheeressuudhaan hima dheeressuudha.

Akaakuun gochimaa, himaa gidduutti, tarree galee dhufu isa gochi tokkoofi inni biroon al tokko akka gaggeeffaman kan agarsiisudha. Kana jechuun, gochoon adda addaa walduraa duuba dhufuun altokkotti hojiiwwan gaggeeffaman agarsiisuu jechuudha. Fakkeenyaaf himoota armaan gadii haa ilaallu: \\
\\
a. Osoo isheen mucaa baattee nama afeeraa jirtuu namni dheeraan tokko ishee waame.\\
b. Wayita ati barattu, dubbistu, barreessitu, inni loon tiksa.\\
c. Osoo isheen fiigaa jirtuu, inni dhufe.

Gochimni gidduu himaa dhufu inni biroon ammoo waa'ee ramaddii ykn yeroo osoo odeeffannoo hinkenniin waa'ee gocha
dhugaa ta’e tokkoo ibsa. Fakkeenyaaf kan armaan gadii haa ilaallu: \\
\\
a. Daa’imaaf aannan dhuguun barbaachisaadha.\\
b. Kitaaba dubbisuun guutuu nama taasisa.

Fakkeenyonni (a) fi (b) irratti ilaalaman kunneen gochima gara maqaatti jijjiiru; fufiin tajaajila kanaaf oolus-uu dha. Fufiin kun hundee gochimaa irratti maxxana. Kan inni maxxanus karaa boodati; Fkn, dhug-uu, dubbis-uu kkf.

\subsection{Gochima Dhalatoo\index{gochima dhalatoo}}
Akkuma waliigalaatti, AO keessatti, gochimni unkaawwan akaakuu lama qabatee mul’achuu danda'a. Unki tokko, hundeen
gochimaa fufiilee hortee (ramaddii, lakkoofsa, yeroo, korniyaa) ida’atee dhufa. Unki biroon ammoo, hundeen jechaafi fufiilee dhalatoo waliin dhufa. Kutaa kana keessatti gochima dhalatoo ilaalla. AO keessatti maxxantoonni akaakuu adda addaa haala adda addaan hundee jechaa irratti maxxanuudhaan gochimoota dhalatoo akaakuu adda addaa uumu. Fufileen tokko tokko hundee gochimaarratti fufamuun gochima dhalatoo haaraa uumu. Kanneen biroon ammoo hundee ibsa maqaarratti fufamuun gochima dhalatoo akaakuu gara garaa uumu. Akkasumas fufileen hundee maqaa irratti fufamanii gochima dhalatoo uuman jiru. Gochimoota dhalatoo haala kanaan uumaman kanneen akka taasisuu, giddugalaafi raawwatamaa akka armaan gadiitti ilaalla. 

\subsubsection{Gochimaa Taasisuu}
Gochimni taasisuu\index{gochima taasisuu} jechuun abgochi tokko isa gara biroo irra dhiibbaa geessisee wayita gocha tokko raawwachiisu agarsiisa.Hundee jechaa irratti maxxantoonni \{-s-,is-,-sis-,-siis-,-ss-,-ess–,-eess– \}ida’amani gochima taasisuu uumu \cite[p.100]{tolemariam2009}. Fufileen \{–s–,–is\} dhamjecha ceesistuus ta’uu danda’u.\\

Seerota gochima taasisuu uuman:
\begin{itemize}
	\item \textit{\underline{Gochima hafoo + is = Gochima taasisuu /ce’aa }}
\end{itemize}
Fakkeenya:\\
\\
a. rafe →raf-is-e\\
b. dammaqe → dammaq-is-e\\
\begin{itemize}
	\item \textit{\underline{Maqibsa + is = Gochima taasisuu qeentee }}
\end{itemize}
Fakkeenya:\\
\\
a. furdaa → furd-is-e \\
b. guddaa → gudd-is-e \\
\begin{itemize}
	\item \textit{\underline{Gochima hafoo + s = Gochima taasisuu/ce’aa }}
\end{itemize}
a. dabe → dab-s-e \\
b. goge → gog-s-e \\

Fufiin taasisuu \{-s-, -is-\} fufii marsaa tokkoffaati. Yoo fufii \{-is-\} booda fufiin ramaddii hindhufu ta’e \{-iis-\} fufii marsaa lamaffaa ykn fufii marsaa sadaffaa ta’a. Kana jechuunis jecha \textbf{raff-is-iis-e} jedhu keessatti fufiin marsaa lammaffaa irratti dhufe \{-iis-\} dha. Akkasumus jecha \textbf{raff-is-isiis -e} jedhu keessattis fufiin dhuma irratti dhufe \{-iis\} dha. Haa ta’u malee \{-sis\} fufii marsaa calqabaa ta’e mul’achuu danda’a. Fakkeenyaaf jechoota \textbf{deem-sis-e }fi \textbf{kolf-sis-e} keessatti \{-sis-\} fufii marsaa duraati. 

Caasaa ijaarsa gochima taasisuu akka armaan gadiitti kaa'uu dandeenya:\\
\\
\\
\begin{forest}
	[rafisisiis- [rafisis- [rafis [raf-] [-is-]] [-is-]] [-iis-]]
\end{forest}
\\
\\

\subsubsection{Gochima Gidduugalaa}

Fufileen \{-at-, -oom-, -ah-)\} maqibsa ykn gochimarraa gochima giddugalaa uumu. 

Seerri gochima gidduugalaa\index{gochima giddugalaa} uumu kan armaan gadiiti:
\begin{itemize}
	\item \textit{\underline{Gochima Giddugalaa = Maqibsa/Gochima + -at-/-om-/-ah-}}
\end{itemize}
Fakkeenya:\\
\\
a. arge → arg-at-e (gochimarraa gochima)\\
b. guddaa → gudd-at-e (maqibsarraa gochima)\\
c. gamna → gamn-oom-e (maqibsarraa gochima)\\
d. sooressa → soor-om-e (maqibsarraa gochima)\\
e. urgaa → urg-aah-e (maqaarraa gochima)\\
f. hadhaa → hadh-aah-e (maqaarraa gochima)

AO keessatti gochimni ce’aan fufii \{-at-\} ida’atee gochima giddugalaa uuma; hiiknis isaas faayidaa mataa ofiif dhimma raawwachuudha. Fakkeenyaaf, \\
\\
a. bite → but-at-e\\
b. gurgure → gurgur-at-e\\
c. mure→ mur-at-e\\
d. ijaare → ijaar-at-e fi kkf.

\subsubsection{Gochima Raawwatamaa}
Gochimni raawwatamaa\index{gochima raawwatamaa} jirma gochima ceetuu irratti fufii raawwatamaa \{-am-\} ida’uudhaan uumama.

Seerri gochima raawwatamaa uumu kan armaan gadiiti:
\begin{itemize}
	\item \textit{\underline{Gochima Raawwatamaa = Gochima Ce’aa + am}} 
\end{itemize}

Fakkeenya:\\
\\
a. bite → bit-am-e \\
b. dhaabe → dhaab-am-e\\
c. gurgure → gurgur-am-e \\
d. jaallate → jaallat-am-e \\
e. jibbe → jibb-am-e \\ 
f. mure → mur-am-e \\

\subsubsection{Gochimaa Irradeebii}
Ggochimni irra deebii\index{gochima irra deebii} kan uumamu wayita birsagni jalqaba hundee gochimaa irra deebi'amudha. Fakkeenyaaf gochimoota armaan gadii ilaaluu dandeenya:\\
\\
a. bite → bib-bite\\
b. dhaabe → dhadh-aabe (dhadh-dhaabe)\\
c. gurgure → gug-gurgure\\
d. mure → mum-mure\\
e. rukute → rur-rukute\\
f. deeme → ded-deeme 

\subsubsection{Makoo Gochima Dhalatoo\index{makoo gochima dhalatoo}}
Fufiileen gara garaa gochima dhalatoo akka horsiisan beekama. Fakkeenyaaf fufiin \{-s-\} gochima taasisuu/ceesisuu
uuma; fufiin \{-at-\} gochima giddugalaa uuma; fufiin \{-am-\} gochima raawwatamaa uuma. Fufiilleen kunneen kallattii gara garaan walitti makamuun makoo gochima dalatoo walxaxaa ta'e uumuu danda'u. 
\begin{itemize}
	\item \textbf{taasisaa + raawwatamaa}: Fufii taasisaa \{-s-\} fi fufii raawwatamaa yoo walitti makne gochima dhalatoo fufii \{-s-, -am-\} jedhu ofiirraa qabu arganna.Fakkeenyaaf, cab-s-am -e.
	\item \textbf{raawwatamaa + taasisuu}: Fufii \{-am-\} dura finnee fufii \{-s-\} yoo itti aansines gochima
	dhalatoo walxaxaa uumuu dandeenya. Fakkeenyaaf, dhabam-sis-e.
	\item \textbf{taasisaa + giddugalaa}: Fufiilee taasisaafi giddugalaa walitti maknees gochima 	dhalatoo uumuu dandeenya. Fkn, cab-s-at-e. Haa ta'u malee kallattiin taa'umsa fufiilee kanneen wal jijjiiruu hindanda’u. Fkn, *cab-at-s-e hinjedhamu. 
	\item  \textbf{raawwatamaa + giddugalaa}: Fufiileen raawwatamaa \{-am-\} fi giddugalaa \{-at-\} walitti makamuun gochima walxaxaa hin’uuman. Fakkeenyaaf, *gurgur-am-at-e (hinjedhamu)
	*gurgur-at-am-e (hinjedhamu)
\end{itemize}

\subsubsection{Gochima Tiishoo}
Gochimni tiishoo\index{gochima tiishoo} jechoota lamarraa uumama.  AO keessatti, durduubeen, Gochimibsaa ykn bamaqaan
gochimatti makamee gochima uuma. Gochimibsaafi bamaqaaleen kanneen akka gad, wal, waliin, walitti, wajjin, of, itti(in), jala fi kkf gochimatti makamuun hiika adda addaa uumu \cite[p.88]{griefenow2001grammatical}. Fakkeenyaaf, jechoota kanneen akka gad-dhiisuu, gad-ba'uu, itti-dhiisuu, jala-ba’uu, of-bulchuufi kkf ilaaluun danda'ama. 

AO keessatti hundeen jechaa gochima jechuu fi gochuu waliin walitti makamuun jecha tiishoo uumu \cite[p.89]{griefenow2001grammatical}. Fakkeenyaaf jechootni kanneen akka cal-gochuu, cal-jechuu, hillim-gochuu, gab-gochuufi kkf dhimma kana ibsuu danda'u.

\subsection{Gochima Tahuu/Ta'uu}
Gochimni tahuu\index{gochima ta'uu} karaalee lama ibsama. Tokko eenyummaa ykn maalummaa agarsiisa. Inni lammaffaammoo bakka waan tokko
ykn namni tokko jiru agarsiisa. Gochimni Tahuu karaalee tumsiiwwan dha, ti fi –i eenyummaa ykn maalummaa agarsiisa; hi’eentaan gochima ta’uu karaa miti agarsiifama \cite[p.90]{griefenow2001grammatical}.
Fakkeenya gochima ta’uu tumsii dha, ti fi \{-i\} waliin akka armaan gadiitti ilaalla:\\
\\
a. Ani barsiisaa dha.\\
b. Ati barsiisaa/tuu dha.\\
c. Isheen barsiistuu dha.\\
d. Ani abbaa kee ti.\\
e. Ati abbaa isaa ti.\\
f. Isheen haadha koo ti. \\
g. Kun aannani.\\
h. Kun bishaani.\\
i. Kun ilkaani. \\

Gochimni tahuu hennaa dabrennaa, amsiqaafi duranaas agarsiisa. \\
\\
a. Isheen barsiistuu tahaa jirti (Amsiqa).\\
b. Ani barsiisaan tahaa jira (Amsiqa).\\
c. Isheen barsiistuu turte (Dabrennaa).\\
b. Ani baradhufu Finfinneen taa’a taha (Duranaa). 
Gochimni tahuu/ta’uu kan bakka waliin walqabatee jiraachuu agarsiisa. \\
\\
a. Ani jira.\\
b. Ati jirta.\\
c. Isheen jirti.\\

Hi’eentaan jiraachuu agarsiisuuf hin-ida’uudha.\\
\\
a. Ani hin-jiru;\\
b. Ati hin-jirtu;\\
c. Isheen hin-jirtu;\\

\subsection{Gochima Qabeenyaa\index{gochima qabeenyaa}}

Jechi qabuu jedhu hiika adda addaa qaba. Hiikni tokko waan tokko harkaan ykn meeshaatiin qabuu jechuudha. Hiikni gara biroon ammoo waan tokko qabaachuu ykn qabeenya godhachuu jechuudha.\\
\\
a. Ani mana qab-a.\\
b. Ati lafa qab-da (ta).\\
c. Isheen fira qab-di (ti). 

Hi’eentaan gochima kanaa hin-gochimatti dabaluun agarsiisama. Fkn, Ani mana hin-qab-u; Ati lafa hin-qab-du;
Isheen fira hin-qab-du fi kkf. Fakkeenyonni armaan olii hennaa ammeennaa agarsiisu; hennaa biroos agarsiisuu.\\
\\
a. Ani mana qaba ture (Dabrennaa).\\
b. Ati lafa qabda turte (Dabrennaa).\\
c. Ani mana nan qabaadha (Duranaa). \\

Jechi qaba jedhu gochima birootti ida’amee dirqama agarsiisa. \\
\\
a. Ati barachuu qabda; \\
b. ati yerootiin mana barumsaadeemuu qabda fi kkf.

\subsubsection{Gaaffilee Boqonnichaa }

1. Fufileen ramaddii gochimarratti ida’aman keessaa 9 tuqi.\\
2. Fufilee hennaa aldhumata, ajaja, dhumata, ammeenna, dabreenna, amsiqa, tarsiqa, murannaafi murannaalaf fakkeenya kenni.\\
3. Akaakuuwwan dhalatoo gochimaa tarreessi.\\
4. Gochimni irra deebii maali?\\
5. Gochimni tiishoo maali?\\
6. Gochimni tahuu maali?

\newpage
\subsection{Jechoota Tajaajilaa\index{jechoota tajaajilaa}}
Akkumaan duraan eerretti, jechoonni akaakuuwwan lamatti qoodamu. Isaanis jechoota qabiyyeefi jechoota tajaajilat.
Jechoonni qabiyyee kanneen boqonnaawwan darban keessatti ilaalle, maqaa, maqibsa, gochimafi ibsa gochimaa fa’i.
Jechoonni qabiyyee jijjiiramaaf saaxilamoodha. Adeemsa siyaasaa, dinagdeefi hawaasummaa waliin hidhata guddaa
qabu. Sirni tokko yoo dhufe jechoota haaraatu gara afaanichaatti makama. Sirni tokko yoo bademmoo jechoonni
qabiyyee sirnicha ibsan nimanca’u. 

Jechoonni tajaajila garuu jechoota qabiyyeerra adda. Jechoonni tajaajilaa wayita barreefaman ykn dubbataman sammuu namaa keessatti fakkii hinmul’isani. Jechoonni tajaajilaa seera afaanichaa, caasluga afaanichaa eegu malee jijjiirama sirnaa kamiyyuu waliin hidhata hinqaban. Kanaaf , jechoonni tajaajilaa jijjiiramaaf saaxilamoo miti; umrii dheeraas jiraatu. Jechoonni kanneen akka durduubee, walqabsiistuufi xiyyeeffannoo gara garaa jechoota tajaajilaa jedhamu. Jechoota tajaajilaa kanneenis kutaa kana keessatti gad fageenyaan ilaalla. 

\subsubsection{Durduubee}

AO keessatti durduubee\index{durduubee} akaakuu adda addaatu jiru. Isaanis kanneen jechatti fufaman, kanneen jecha dursanii dhufan,
kanneen jecha booda dhufaniifi kanneen lamatti qoodamanii jecha ofiisaanii gidduu galshanidha. 

Durduubeewwan armaan gadii jechatti fufamanii argamu \cite[p.50]{griefenow2001grammatical}. \\
\\
a. -itti: manaa gad-itti\\
b. -tti: mana-tti\\
c. -rra (irra): mana-rra fi kkf.\\

Durduubeewwan dura jechaa dufanis \cite[p.50]{griefenow2001grammatical}.\\
\\
a. akka: akka miilaan deemu\\
b. gara: gara manaa deemaa \\
c. hanga: hanga sa’atii lamaa turaa\\
d. karaa: karaa mirgaa koottaa \\
e. waa'ee: waa’ee dinagdee dubbadhaa fi kkf.\\
\\
Durduubeewwan lamatti qoodamanii jecha of gidduu galshan\cite[p.53]{griefenow2001grammatical} kanneen armaan gadii fa’i. \\
\\
a. gara ... tti: gara manaatti\\
b. hanga ... tti: hanga ammaatti fi kkf.\\
\\
Durduubeewwan booda jechaa dhufan \cite[p.52]{griefenow2001grammatical} jiru.\\
\\
a. ala: hojiin ala \\
b. bira: mana bira\\
c. booda: sa'a afur booda\\
d. duuba: mana duuba\\
e. dura: nama dura\\
f. gad: manaa gadfi gubbaa, gidduu, jala, keessa, diida, malee, teellaa, waliinfi kkf.

\subsubsection{Walqabsiistota}

Walqabsiistuun\index{walqabsiistota} akaakuu jecha tajaajilaati. Walqabsiistuun himaafi hima ykn gaaleefi gaalee walqabsiisuuf tajaajila. AO keessa walqabsiistota akaakuu sadiitu jiru. Isaanis kanneen of danda’anii dhaabbatan, kanneen hirkataniifi kanneen irra deebi’ani dha \cite[p.55]{griefenow2001grammatical}. 

Walqabsiistotni hima of danda’ee dhaabbate walqabsiisanii tajaajilan hedduudha \cite[p.56]{griefenow2001grammatical}.\\
\\
a. ammoo:  Achi deemeeti ammoo deebi’a.\\
b. garuu:  Re’een garuu baala jaalatti.\\
c. yookaan: Inni yookaan isheen haa dhuftu.\\
d. akkasumas: Borus akkasuma dalagi.\\
e. booda:  Hojii booda wal agarra.\\
f. eegasii: Eegasii ani deebi’uun qaba.\\
g. erga: Erga ani dhufee inni deeme.\\
h. kanaaf: Kanaaf qarshii qusatti.\\
i. malee: Malee inni sooroma ba’eerafi kkf. Walqabsiistonni armaan olii kunneen himoota of danda’anii
dhaabbatan lamaafi lamaa ol walqabsiisuun tajaajilu. \\
\\
AO keessatti walqabsiistonni himoota hirkatoo walqabsiisuun tajaajilanis hedduudha.\\
\\
a. akka: Akka ati dhuftu innuu beeka.\\
b. jechuun: Leenca jechuun bineensa bosonaati.\\
c. osoo: Osoo allaatti taatee nibalaliita.\\
d. akkuma: Akkuma ati yaaddu namnis yaada.\\
e. booda: Dhufaatiishee booda roobni caame.\\
f. dura: Sirbuu dura shaakaluu barbaachisa.\\
g. erga: Erga roobni caamee manaa ba’an.\\
k. kanaaf: Kanaaf jettee nama hinlolliin.\\
l. waan: Waan kana ta’eef tasgabbaa’i.\\
m. otuma: Otuma ta’e wal ooduun sirrii miti.\\
n. ennaa: Ennaa biiftuun seentu horiin gala.\\
o. hoggaa Hoggaa wal argan walitti boo’ufi kkf. Himoota armaan olii keessa himoonni lama lama jiru. Himoota
kanneenis walqobsiistonni walitti firoomsaniiru.\\
\\
Walqabsiistonni unka fufii qabanis jiru. Fakkeenya isaanii armaan gaditti ilaalla. \\
\\
\{-f\} Kanaa-f jettee hojii hindhiisin.\\
\{-(i)s\} Inni-s haadhufu.\\
\{-ti\} Kanaaf jedhee-ti barumsa dhiise.\\
\{-tii\} Isa gaafadhaa-tii adda baafadhaa.\\
\{-ammoo\} Isaani-sammoo dhufuu fedhu.\\
\{-oo\} Inn-oo gara namaa hindhufu.\\
\{-illee\} Ishee-llee argeera.\\
\{-moo\} Isa-mmoo hinargine.\\
\\
Walqabsiistonni unkota adda addaa lama qabanis jiru.\\
\\
a. kanaaf jecha: Kanaaf jecha biyya tajaajilla.
b. kanaaf egaa: Kanaaf egaa gara biyyaa galla.
c. akkam akka: Akkam akka ta’e hinargine.
d. silaa osoo: Silaa osoo dafee dhufe, anis gara hojiin
deema.
e. yommu yeroo: Yommuu yeroon argamu gara biyyaa
deemna.\\
\\
Walqabsiistonni lamatti qoodamanii gidduu isaanii gaalee biroo galchanis jiru \cite[60]{griefenow2001grammatical}.\\
\\
a. akkasuma...s: Akkasuma hoolaa-s bite.\\
b. osoo...dura: Osoo inni hoolaa hinbitiin dura isheen gara manaa deemte.\\
c. hamma...tti: Hamma ammaa-ttii barattoonni as hindhufne.\\
d. erga...booda: Erga waraabessi darbee booda sareen dutte.\\
e. akka...f: Akka hammina keessanii-f ani deebi’ee hindhufun ture.\\
f. akka...tti: Akka ilaalcha keessan-tti yaadni kun sirrii miti.\\
g. waan...if: Waan isin dhuftani-if inni deeme.\\
h. ijaa...if: Ijaa isin dhuftani-if isheen deemte.\\
i. yoo...malee: Yoo inni dhufe malee isheen midhaan hin nyaattu.\\
j. yoo...llee: Yoo inni dhufe-llee isheen gadi hinteessu.\\
k. yoo...yyuu: Yoo inni dhufe-yyuu isheen gadi hinteessu. \\
\\
Walqabsiistonni irra deebi’anii qaama hima walqabsiisanis jiru \cite{griefenow2001grammatical}.\\
\\
a. \{-s...s...s\} Isa-s, ishee-s, Ganamoo-s hinargine.\\
b. \{-llee...llee....llee\} Isa-llee, ishee-llee hinwaamne.\\
c. \{yookaan...yookaan\} Yookaan isa ergi yookaan ofiikee koottu.

\subsubsection{Xiyyeeffannoo}

AO keessa fufiileen gara garaa xiyyeeffannoo\index{xiyyeeffannoo} adda addaa agarsiisu. Isan keessaa kanneen jecha, kutaa himaa, qoddataaf xiyyeeffannoo godhan jiru. Akkasumas gaaffii gaafatameef kanneen xiyyeeffannoo kennan jiru. Akkasumas ciroof kanneen xiyyeeffannoo kennan jiru. Kana malees kan miiraf xiyyeeffannoo kennan jiru. Qabxileen kunneen armaan gaditti gabaabumaan dhiyaatu.

AO keessatti fufileen xiyyeeffannoo matimaafi kutima irratti mul’atu. Fufiin xiyyeeffannoo hiika mataa isaa danda’e
hinqabaatu. Fayidaan isaa tajaajila caaslugaa qofa. Fufiin xiyyeeffannoo matimaa fufii boodaa ta’ee dhufa. Fufiin
xiyyeeffannoo kutimaa garuu gochima irratti fufii duraa ta’ee argama. Fufiin xiyyeeffannnoo matimaa \{-tu\} dha \cite[p.371]{baye1988focus}.\\
\\
a. Eenyu-\textbf{tu} foon hoolaa bite?\\
b. Galgaloo-\textbf{tu} daree keessaa tokkoffaa ba’e.\\
c. Namicha dheeraa sana-\textbf{tu} hoolaa bite. \\
\\
Akka fakkeenyota armaan oliirraa hubachuu dandeenyutti fufiin xiyyeeffannoo matimaa, gaalee maqaa akka matimaatti
tajaajilu irratti karaa boodaa fufamti. 


AO keessatti xiyyeeffannoo kutimaa kan agarsiisuu fufii \{-ni\} dha \cite[p.54]{griefenow2001grammatical}. \\
\\
a. Manni daldalaa bara dhufu \textbf{ni}-banama.\\
b. Koottu \textbf{ni}-taphanna.\\
c. Achi deemnee \textbf{ni}-baranna.\\

Fufiilee hasbarruu, qooddataa, jecha yookiin qaama jechaaf xiyyeeffannoo kennu \cite[pp.61-62]{griefenow2001grammatical}. \\
\\
\{-llee/illee\} Namni tokk-llee hindhufne.\\
\{-mmoo/-ammoo\} Amma-mmoo biyya gargaari.\\
\{-oo\} Ijooleen-oo eessa?\\
\{-uma\} Kan na arge isaan-uma.\\
\{-yyuu/iyyuu\} Ini-yyuu nama gaariidha.\\
\\
Kana malees fufiileen gaaffii xiyyeeffannoo kennan jiru:\\
\\
\{-ree\} Maaliif dhugaatii dhugda-ree?\\
\{laata/laa\} Ka’ee barsiisuu laata/laa?\\
\{mitii\} Kun qabenyaa mitii?\\
\{-moo\} Hojjedhe-moo hinhojjenne?\\
\{-m\} Dhufeeraa-m.\\
\\
Jechoonni ciroo firoomseef xiyyeeffannoo kennan AO keessa jiru. Jechoonni xiyyeeffannoo kunneenis kan, waan, warra,
inni fi isa, isheefi kkf dha \cite[pp.63-64]{griefenow2001grammatical}.\\
\\
a. Nama lafarraa \textbf{kan} balleessu jibbadha.\\
b. Ati \textbf{waan} bituu barbaaddu natti himi. \\
c. Ati \textbf{warra} nama jibbumoo warra nama jaalatu?\\
d. Namichi \textbf{inni} kaleessa dhufe hoolaa bite.\\
e. Dubartiin \textbf{isheen} na barsiifte asiin darbite.\\
\\

Fufiileen kanneen akka \{-yyo, yaa, -kaa\} miiraaf xiyyeeffannoo kennu \cite[pp.63-64]{griefenow2001grammatical}.\\
\\
a. Yoo nabarbaaddee, koottu-\textbf{kaa}.\\
b. Leencatu baroode-\textbf{kaa}.\\
d. \textbf{Yaa} waaqa-\textbf{yyo}, na dhaga’i!

\subsubsection{Gaaffilee Kutichaa }

1. Jechoota tajaajilaa boqonnaa kana jalatti ibsaman kan
boqonnaa 3 keessatti tuqaman waliin tokko moo addadda?
2. Xiyyeeffannoon matimaa kami?
3. Xiyyeeffannoon kutimaa kami?
4. Walqabsiistuuwwan tarreessiitii tajaajila isaanii ibsi.
5. Durduubeewwan gara gara tarreessiiti tajaajila isaanii ibsi.
6. Fufiilee xiyyeeffannoo akaakuu addaddaa tuqi.

\newpage

\section{Gochimibsa}
Gochimibsi\index{gochimibsa} jechoota qabiyyee keessaa tokko akka ta’e boqonnaawwan darban keessatti eerameera. Kanaaf, gochimibsi, akkuma jechoota qabiyyee biroo, akka maqaa, maqibsaafi gochimaa , jijjiiraaf saaxilamoodha. Jijjiirraa
siyaasaa, dinagdeefi hawaassummaa waliin walqabatee gochimibsoonni haaraan kuusaa jechootaa afaanichaatti
dabalamu; kan duraanii ammoo haqamu. 

Hima keessatti, gochimibsi dhimma raawwatamaa jiruuf odeeffannoo dabalataa kenna. Dhimmichis eessatti, yoomfi
akkamitti akka raawwatames agarsiisa. Akaakuuwwan gochimibsa adda addaatu jiru. Baayyeen isaanis garee
jechotaa kanneen akka maqaa, maqibsafaa irraa dhalatu. AO keessatti gochimibsoonni, gochima dursanii dhufu. 

Gochimibsootni yeroo agarsiisu:\\
\\
a. altokko: Altokko Ameerikaa deemi.\\
b. amma: Ati amma baradhu.\\
c. booda: Hojii booda walagarra.\\
d. boru: Boru baargama deemna.\\
e. dura: Dubbachuu dura caqasi.\\
f. dheengadda: Dheengadda eessa turte?\\
g. dhiyoo: Ati dhiyoo jirta.\\
h. gaaf: tokko Aniifi ati gaaf tokko walagarra.\\
i. har’a: Barnootni har’a eegala.\\
j. iftaan: Iftaan eessa deemtu?\\
k. yeroodhaan: Yeroodhaan galuun gaariidha.\\
l. yeroo: baayyee Yeroo baayyee kitaaban dubbisa.\\
\\
Gochimibsoonni bakka agarsiisan jiru:\\
\\
a. achitti: Isa achitti argitee?\\
b. ala: Maaltu ala jira?\\
c. alatti: Alatti saree argitan.\\
d. as: Wareebessi as jira eenyu jedhe?\\
e. asitti: Asitti wal haabeellamnu.\\
f. dhiyoo: Firri keenya dhiyoo jira.\\
g. fagoo: Diinni keenya fagoo jira.\\
h. gama: Tulluu gama lagatu jira.\\
i. gamana: Tulluu gamana magalaatu jira.\\
\\
Gochimibsoonni armaan gadii akkaataa agarsiisu:\\
\\
a. dansaatti: Dansaatti dalagi.\\
b. mannaa: Isa taa’e mannaa isa deeme wayya.\\
c. saffisaan: Saffisaan dubbisuun gaariidha.\\
d. sirriitti: Sirriitti qubeessuun filatama.\\
e. suuta: Suutaa ofuun lubbuu tikisa.\\
f. wayya: Barreessuumoo dubbisuu wayya?\\
\\
Gochimibsoonni armaan gadii madaallii agarsiisu \cite[pp.97-98]{griefenow2001grammatical}:\\
\\
a. baayyee: Isheen baayyee dhukubsatte.\\
b. danuu: Rifeensa danuu qaba.\\
c. hedduu: Inni hedduu nama jaallata.\\
d. qofaa: Ati qofaa kee hindeemiin.\\
e. duwwaa: Qarshii lama kuma lama duwwaa isaaf kenni.\\
f. dafee: Inni dafee dafee baargama deema.\\

\subsubsection{Gaaffilee Boqonnichaa}
1. Gochimibsa armaan gadii hima keessatti fayyadamii agarsiisi:\\
a. Gochimibsa yeroo\\
b. Gochimibsa bakkaa\\
c. Gochimibsa akkaataa\\
d. Gochimibsa madaallii\\

\newpage

\chapter{Caashima\index{caasima}}
\subsubsection{Qabiyyee}
\begin{itemize}
	\item Gaalee Maqaa
	\item Gaalee Maqibsaa
	\item Gaalee Gochimaa
	\item Gaalee Gochimibsaa
	\item Gaalee Durduubee
	\item Hima
\end{itemize}
\subsubsection{Gaaffiilee Ka'umsaa}
\begin{enumerate}
	\item Gaalee jechuun maal jechuudha?
	\item Gaaleen maqaa maali?
	\item Gaaleen maqibsaa maal akka ta'e ibsi.
	\item Ciroon maali?
	\item Gaaleen gochimaafi gochimibsaa maali?
	\item Addaddummaan gaaleefi himaa maali?
\end{enumerate}
	
	\section{Seensa}
	
Boqonnaan kun caasaalee gaaleefi himaa AO qaacceessa. Akkuma beekamutti jechi garee shan qaba. Kunneenis maqaa, maqibsa, gochima, gochimibsa fi durduubee jedhamu. Gareen jechaa shanan kunneen bu’uura caasaa gaaleeti. Kana jechuun gaaleewwan akaakuu shantu jiru jechuudha. Isaanis gaalee maqaa, gaalee maqibsa, gaalee gochimaa, gaalee durduubee fi gaalee gochimibsaati. Gaaleen tokko hangafa ykn durabu’aa qaba. Hangafni gaalee maqaa maqaadha. Hangafni gaalee maqibsa ibsuma maqaati. Haaluma wal fakkaatuun hangafni gaalee gochima, gochima kkf. Mataa ykn hangafa gaalee jechuun jecha gaaleen sun irratti bu’uureffate, kan hafuu ykn haqamuu hinqabne jechuudha. Afaan tokko keessatti hangafni gaalee karaa bitaa ykn karaa mirgaa dhufa. Karaa bitaa ykn karaa mirgaa dhufaatiin hangafa gaalee afaan tokko isa biroorraa adda godha. Egaa walumaa galatti caasaa gaalee tokko ibsuuf mataa ykn hangafa gaalee sanaa baruuniifi karaa mirgaa ykn karaa bitaa dhufaatii isaa adda baasuun qabxii barbaachisaadha.

Caasaa gaalee sirriitiin qacessuuf qabxiin dagatamuu hinqabne inni gara biroon walitti dhufeenya jechoota gaalee keessatti argamanii hubachuudha. Walitti dhufeenya gaalee keessatti argamu karaa lama hubatama. Tokko, walitti dhufeenya hariroo jedhama. Walitti dhufeenya hariiroo kan jedhamu jechi ykn gaaleen tokko jechoota ykn gaalota bitaafi mirga jiran wajjin walittidhufeenya qabu jechuudha. Walitti dhufeeny inni lammaffan ammoo walitti dhufeenya sadarkaa jedhama. Walitti dhufeenya sadarkaa jechuun jechi tokko ykn gaaleen tokko
jecha ykn gaalee biro wajjin ol-aantummaa fi gad-aantummaa inni qabu kan mul’isu jechuudha. Kana jechuunis jechi tokko ykn gaaleen tokko ol-aanaa ykn gad-aanaa gaalee biro ta’uu danda’a jechuudha. Walitti dhufeeny caasaa gaaleetiin agarsiisama. 

\section{Gaalee Maqaa}
Gaaleen maqaa\index{gaalee maqaa} bu'uurri ykn mataan isaa maqaadha. Ka'umsaafi caasaa gaalee maqaa armaan gadii haa ilaallu:\\

\begin{figure}[H]
	\caption{gaalee}
	\centering
	\begin{forest}
		[GM \\ nama dheeraa
			[M\\nama]
			[IM\\dheeraa]
		]
	\end{forest}
\end{figure}
Akka gaalee maqaa armaan olii irraa ilaalamu kana jechi nama (maqaa) jedhu fi jechi dheeraa (maqibsa) jedhu waliif
hariiroodha. Jechoonni lamaanuu walcinaatti argamu. Gaaleen maqaa (GM) nama dheeraa jedhu ammoo ol-aanaa maqaa
nama jedhuu fi maqibsa deeraa jedhuuti. Karaa jecha gara biro ammoo maqaan nama jedhu fi maqibsi dheeraa jedhu gad
-aantota gaalee maqaa nama dheeraa jedhuuti.

Seera gaalee keessatti gad-aanaan tokko ol-aantota lamaa fi lamaa ol qabaachuu hindanda’u. Yookiin ol-aantonni lama gadaanaa tokko qabaachuu hindanda’an. Kana jechuun maqaan sadarkaa gad aanaa irru jiru ol aanaa tokko qofaatti hariiroo uummachuu qaba. Yoo olaantota lamatti hariiroo godhate hiikni gaalee waldha'aa ta'a. Fakkeenyaaf caasaawwan gaalee armaan gadii haa ilaallu:
\begin{figure}[H]										
	\caption{Gaalee maqaa}
	\centering
	\begin{forest}
		[GMG
			[GM1
				[M\\daadhii]
			[	M\\damma]
			]
			[GM1
				[M\\damma]
				[M\\durii]
			]
		]
	\end{forest}
\end{figure}

Akka armaan ol irratti mula’atu kana gaalee maqaa guddaa (GMG)n gad-aantota lama qaba isaanis gaalee maqaa xiqqaa
(GM1) bitaafi mirgaa ti. GM1 (bitaa)n gad-aantota lama qabaa; isaanis daadhii fi damma dha. GM1s (mirga) akka gad
aantotaatti dammaa fi durii qaba; damman ol aantota lama qabdi; kun sirrii miti; kan dogoggora taasises gad-aanaan tokko ol-aantota lama ykn maqaan dammaa jedhu ol-aantota GM1 (bitaa) fi GM1 (mirga) qabaachuudha.

Waan kana ta'eefis hiikni isaa daadhii (dammaa) isa durii jechuu ta'a-xiyyeeffannoon daadhiidha; daadhii damma\footnote{Gad fageenyaaf \cite{baye1986phrase} ilaalaa.} isa durii jechuus nita'a - xiyyeeffannoon daadhiidha. Adda hinbaasu. Waliin dha'a. Kana jechuun maal jechuudha? Kana jechuun hiikni gaalee kanaa ifa miti jechuudha. Daadhii durii jechuudha imoo damma durii jechuudha? Caasaa kan sirreessuuf akka armaan gadiitti fooyyessuu dandeenya. 

\begin{figure}[H]										
	\caption{Gaalee maqaa}
	\centering
	\begin{forest}
		[GMG
			[GM1
				[M\\daadhii]
				[M\\damma]
			]
			[GM1
				[M\\durii]
			]
		]
	\end{forest}
\end{figure}
Hiikni gaalee maqaa armaan olii daadhii dammaa isa durii sana jechuudha. Akka caasaa kana keessatti ilaallutti maqaan \textbf{damma} jedhu ol aanaa tokko qofaatti firoomeera; maqaa kana ol aanaa birootti yoo firoomsinee hiikni caasicha ni jijjiirama. Fakkeenyaaf caasaa armaan haa ilaallu:

\begin{figure}[H]										
	\caption{Gaalee maqaa}
	\centering
	\begin{forest}
		[GMG
			[GM1
				[M\\daadhii]
			]
			[GM1
				[M\\damma]
				[M\\durii]
			]
		]
	\end{forest}
\end{figure}

Gaaleen maqaa, maqaa tokko yookiin maqaa fi jechoota maqaa sana waliin adeeman irraa ijaarama. Kana jechuun maqaan
tokko yookiin maqaa fi jechoonni maqaa waliin adeeman akka gaalee maqaatti fudhatamu. Maqaan tokko bakka gaalee yoo
taa’e akkuma gaalee maqaatti fudhatama. Maqaa fi jechoonni maqaa waliin adeemanis bakkuma gaalee waan qabataniif
akkuma gaalee maqaatti lakkaa’amu. Qabxii kana ibsuuf fakkeenyota armaan gadii haa hubannu: \\
\\
a. Namni dhufe.\\
b. Namni dheeraan dhufe.\\
c. Namni dheeraan sun dhufe.\\

Himoota armaan olii keessaa warri jala sararamanii jiran gaalee maqaa jedhamu. (a) irratti gaaleen maqaa, maqaa
namni jedhu irraa ijaarame; (b) irratti gaaleen maqaa, maqaa namni jedhu fi jecha maqaa waliin adeemuu jechuun maqibsa dheeraa irraa ijaarame; (c) irratti maqaa namni jedhu, maqibsa dheeraa jedhuu fi agarsiistuu sun jedhu irraa ijaarame. Akka fakkeenyota kanneen irraa hubachuu dandeenyutti gaaleewwan namni, namni dheeraan fi namni
dheeraan sun jedhan bakka wal jijjiiranii hima keessatti tajaajila gaalee maqaa kennuu danda’u.

Dhimma gaalee irratti qabxiin sirriitti hubatamuu qabus jira. Qabxiin kun qabxii bu’uura qabu. Gaaleen tokko gaalee ta’uu isaa fi gaalee ta’uu dhiisuu isaa maaliin beekama? Gaaleen tokko, gaalee ta’uuf dhiisuu isaa ulaagaawwan sadiin beekama.
\begin{enumerate}
	\item Gaaleen tokko, gaalee gosa tokkoo waliin bakka waljijjiiree hima keessatti tajaajila kennuu qaba. Qabxii kana fuula itti aanee dhufu keessatti ibsina.
	\item Jechoota walitti aananii dhufanii gaalee ijaaranii jiran 	gidduu qaamni biroon (jechi ykn gaaleen akaakuu biro) seenuu hin qabu, yoo seene dogoggora uuma. Qabxii kana fakkeenya armaan gadiirraa hubachuu dandeenya:\\
	\\
	a. Namni dheeraan hoolaa bite.\\
	b. *Namni hoolaa dheeraan sa’a bite.\\
	\\
	Fakkeenyi (a) sirrii yoo ta’u (b) dogoggora qaba. Hima (b) kana dogoggora kan taasise gaalee namni dheeraan jedhu gidduu maqaan hoolaa jedhu waan seeneef. Gaaleen maqaa namni dheeraan jedhu qaama tokko; sababni isaas namnifi dheeraan walitti aanani dhufanii gaalee maqaa waan ijaaraniif. Qaama gaalee kana gidduu qaamni biroon seenuu hindanda’u. Akka (b) irraa ilaallutti garuu maqaan gara biroon (hoolaa) namni fi dheeraa gidduu seenee adda isaan baaseera. Kanaaf himni (b) fudhatama dhabeera.
	\item Gaaleen tokko osoo addaan hin ba’iin yeroo hunda walfaana adeemuu (sosso’uu) qaba. Qabxii kanas fakkeenya
	armaan gadiitiin ibsuu dandeenya: \\
	\\
	a. Namni dheeraan hoolaa bite.\\
	b. Hoolaa bite namni dheeraan.\\
	c. * Namni hoolaa bite dheeraan. \\
	\\
	Akka (a)fi (b) irratti agarsiifamee jiruutti gaaleen maqaa walfaana sosso’ee gaalee gochimaa booda waan galee jiruuf himni sirrii ta’eera. (c) irratti garuu gaaleen maqaa adda adda ba’eera. Maqibsi qofaa isaa sosso’ee gaalee gochimaa booda galeera; maqaan garuu bakkuma jirutti hafeera. Kanaaf jecha, himichi hima fudhatama hin qabne ta’eera. 
\end{enumerate}
	
As irratti waan tokko ifa ta’uu qaba. “Sadarkaa guddina yookiin bal’ina jechoota qabateen gaalee bakka meeqatti qooduu dandeenya?” gaaffiin jedhu yeroo kana deebi’uu qaba. Akka guddina isaatiin gaaleen bakka lamatti qoodamee ilaalamuu danda’a. Isaanis gaalee guddaa (GMG) fi gaalee xiqqaadha (GM1). Gaalee maqaa xiqqaan (GM1) handhuura gaalee maqaati. Gaalee xiqqaa keessatti hangafni ykn mataan gaalee guuttu argata. Guuttuun kunis maqaadhuma ta’eeti madda ykn uumaa maqaa hangafa gaalee ta’eetu sanaa agarsiisa. Fakkeenyaaf gaalee armaan gadii gaalee maqaa xiqaa jedhama:

\begin{figure}[H]										
	\caption{Gaalee maqaa}
	\centering
	\begin{forest}
		[GMG
			[GM1
				[M\\mana]
				[M\\dhagaa]			
			]		
		]
	\end{forest}
\end{figure}

Maqaan \textbf{mana}jedhu hangafa gaalee maqaati. Maqaan \textbf{dhagaa} jedhanammoo guuttuu hangafa gaalee maqaati; innis manni maal irraa akka ijaarame agarsiisa waan ta'eef. Hangafni gaalee maqaa karaa harka bitaa dhufeera; guuttuun ammoo karaa harka mirgaa dhufeera. Gaalee maqaa guddaan gaalee maqaa xiqqaa gabbisuuf dhufa. Kana jechuun gaalee maqaa guddaa irratti gaaleen maqaa maqibsafi murteessituu dabalata jechuudha. Fakkeenyaaf gaalee maqaa murteessituu (MU) of keessaa qabu kan armaan gadii haa ilaallu: 


\begin{figure}[H]										
	\caption{Gaalee maqaa}
	\centering
	\begin{forest}
		[GMG
			[GMG
				[GM1
					[M [mana]]
					[M [dhagaa]]
				]
				[GIM
					[IM [guddaa]]				
				]
			]
			[MU [sana]]
				]
	\end{forest}
\end{figure}

Akka caasaa armaan olii irratti ilaalamu kana gaaleen maqibsa guddaa jedhuu fi GM1 gaalee maqaa guddaa sadarkaa
gadaanaa GMG ijaarani; GMG irratti murteessituun sana yoo ida’ames GMG tu ijaarame.

Kana malees gaaleen maqaa hima hirkaataatiin gabbachuu danda’a. Himni akka kanaa kun ciroo firoomsee jedhama. Fakkeenyota armaan gadii keessaa kanneen jala muramanii jiran ciroo firoomsee jedhamu:\\
\\
a. Namicha \underline{isa kaleessa dhufe}.\\
b. Fardi \underline{inni Tolaan bite}.\\
Akka caasaa armaan gaditti agarsiifameeru kana \textbf{namicha} n hangafa gaalee maqaati; ciroon firoomsee \textbf{isa kaleessa dhufe } kan jedhu gabbistootuu gaalee maqaati. 

\begin{figure}[H]										
	\caption{Gaalee maqaa}
	\centering
	\begin{forest}
	[GMG
		[GM1
			[M [namicha]]
		]
		[H [isa kaleessa dhufe,roof]]
	]
	\end{forest}
\end{figure}

Gaalee maqaa keessatti jechoonni akka sana, kana, sanneenfi kanneen jedhan murteessitoota jedhamu. Isaan waa agarsiisu. Kanneen gara biroon ammoo qabeenya mul’isu. Fakkeenyaaf kitaabakoo, kitaabakee, kitaabasaa, kitaabakeenya, kitaabakeessan, kitaabashee fi kitaabasaanii keessaa murteessitoonni qabeenya agarsiisan jiru. Murteessitoonni lakkoofsa agarsiisanis jiru. Fakkeenyaaf fardeen lama, hoolota sadii, qarshii kuma , xaafii quunnaa tokko, daadhii birillee sadii, asheeta mataa lamaafi kkf. murteessitoonni kunneen unkaatiin maqibsafi maqaadha.

\begin{figure}[H]										
	\caption{Gaalee maqaa}
	\centering
	\begin{forest}
		[GMG
			[GM
				[M [fardeen]]
			]
			[GIM
				[IM [lama]]
			]
		]
	\end{forest}
\end{figure}

Fakkeenya armaan olitti eerame irraa akka hubachuu dandeenyutti murteessituun unkaa maqibsa qaba; kan inni
argamus dhuma (karaa mirgaa) gaalee irratti.

\subsubsection{Gaaffilee Kutichaa}
1. Gaalee jechuun maal jechuudha?\\
2. Ulaagaawwan gaaleen itti beekamu tarreessiitii tokko tokkoo isaanii fakkeenyaan ibsi.\\
3. Hangafa gaalee maqaa jechuun maal jechuu akka ta'e ibsi.

\section{Gaalee Maqibsa\index{gaalee maqibsaa}}
Akkuma gaalee maqaa gaaleen maqibsas hangafa yookiin dura bu’aa qaba. Hangafni gaalee maqibsaa maqibsa mataasaati.
Akkuma gaalee maqaa, gaaleen maqibsa sadarkaa guddina isaatiin gaalee guddaafi gaalee xiqqaa qaba. Gaalee maqibsa
xiqqaa keessatti hangafa gaalee maqibsafi guuttuu hangafa gaalee maqibsa sanaatu argama. Kana jechuun garuu gaaleen
maqibsaa dirqamaan guuttuu qabaachuu qaba jechuu miti. Hangafni gaalee maqaa guuttuudhaan hordofamuus dhiisuus
nidanda’a. Karaa jecha gara biro gaaleen maqibsa kan guuttuu qabus kan guuttuu hinqabnes jira jechuudha. Kunis gaalee maqaa wajjin wal isa fakkeessa (gaaleen maqaas kan guuttuu qabuufi kan guuttuu hinqabne jira). Fakkeenyaaf maqibsa maqaa armaan gadii haa ilaallu:

\begin{figure}[H]										
	\caption{Gaalee maqibsaa}
	\centering
	\begin{forest}
		[GMG
			[GM1
				[M [dubartii]]
			]
			[GIM
				[IM [dheertuu]]
			]
		]
	\end{forest}
\end{figure}

Gaaleen maqibsa kun guuttuu hin qabu. Garuu gaaleen maqibsa dheertuu dubartii jedhu guuttu qaba.
Guuttuun maqibsa kanas maqaa dubartii jedhudha. Caasaa gaalee maqibsa kanaa ammoo akka armaan gadiitti ilaaluu
dandeenya. 

\begin{figure}[H]										
	\caption{Gaalee maqibsaa}
	\centering
	\begin{forest}
		[GMG
			[GIMG
				[GIM1
					[IM [dheertuu]]
					[M [dubartii]]
				]
			]
		]
	\end{forest}
\end{figure}

Maqibsi guuttuu qofaa osoo hintaane gabbistootallee qaba. Gabbistuun Maqibsa akka ol-aanaa isaatti gaalee maqibsa
gudduu qaba. Kana jechuunis gabbistuun maqibsa gaalee isa guuddaa keessaa ba’ee gaalee maqibsa xinnaa mul’isa
jechuudha. Gabbistuun gaalee maqibsa gaalee durduubeti. Fakkeenyaaf:\\
\\
\underline{hamma abbaa isaa} sorreessa.\\

\begin{figure}[H]										
	\caption{Gaalee maqaa}
	\centering
	\begin{forest}
		[GIMG
			[GIM1
				[GD [hamma abbaa isaa,roof]]
				[IM [sooressa]]
			]
		]
	\end{forest}
\end{figure}

Fakkeeny armaan olitti jala murameeru gabbistuu gaalee maqibsati. Gabbistuun kun galee durduubeeti. Innis wan lama wal bira qabee madaaluun gaalee maqibsa gabbisa.

\section{Gaalee Durduubeefi Gochimibsaa}

Hangafni gaalee durduubee\index{gaalee durduubee} durduubeedha. Gaaleen durduubee xiqqaan guuttuu durduubee malee ijaaramuu hin
danda’u. Kunis gaalee maqaa, gaalee maqibsafi gaalee gochima irraa adda isa godha. Haala guuttuu isaaniitiin durduubonni AO bakkeewwan sadiitti qoodamanii ilaalamu. Isaanis, durduubee gaalee maqaa guuttuu godhatu, durduubee hima guuttuu godhatuufi durduubee gaalee durduubee guuttuu godhatu jedhamu. Fakkeenyota armaan gadii keessatti
durduubeen gaalee maqaa guuttuu godhata:\\
\\
a. Kumsaan \underline{gara mana qorichaa} deeme.\\
b. Dhugumaan\underline{ akka abbaa isaa} gaarii dha.\\

Himoota armaan olii keessatti gaaleewwan jala sararamaniiran gaaleewwan durduubeeti. Gaaleewwan durduubee kunneen
hangafoota qabu. (a) irratti hangafni gaalee durduubee gara dha. (b) irratti ammoo hangafni gaalee durduubee akka dha.Gaaleewwan durduubee kunneen guuttotas qabu. (a) irratti guuttuun durduubee gaalee maqaa mana qorichaa jedhu yoo ta’u (b) irratti ammoo gaalee maqaa abbaa isaa jedhu. Gaaleewwan kanneen caasaatiin akka armaan gadiitti haa
ilaallu: 

\begin{figure}[H]										
	\caption{Gaalee durduubee}
	\centering
	\begin{forest}
		[GDG
			[GD
				[D [gara]]
			]
			[GMG [mana qorichaa,roof]]
		]
	\end{forest}
\end{figure}

\begin{figure}[H]										
	\caption{Gaalee durduubee}
	\centering
	\begin{forest}
		[GDG
			[GD
				[D [akka]]
			]
			[GMG [abbaa isaa,roof]]
		]
	\end{forest}
\end{figure}

Akka armaan olitti ilaalamu kana hangafni gaalee durduubee karaa harka bitaa dhufeera. Gaaleewwan durduubee warri
armaan gaditti mul’atan ammoo hangafa gaalee karaa harka mirgaa fidu: \\
\\
a. Kitilaan \underline{abbaa isaa wajjin }dhufe.\\
b. Cimdeessan \underline{hojii booda} farsoo dhuga.\\
\\

Armaan olitti gaaleewwan jala sararaman gaaleewwan durduubeeti. Durduubonni \textbf{wajjin} fi \textbf{booda} jedhan hangafoota gaaleewwan durduubeeti; isaanis karaa harkaa mirgaa dhufaniiru. Fakkeenyaafcaasaatin haa ilaallu:

\begin{figure}[H]										
	\caption{Gaalee durduubee}
	\centering
	\begin{forest}
		[GDG
			[GMG [abbaa isaa,roof]]
			[GD
				[D [wajjin]]
			]
		]
	\end{forest}
\end{figure}

Akaakuun durduubee lammaffaan durduubee isa hima guuttuu godhatu. Durduubeen gaalee maqaa qofaa osoo hintaane hima
hirkataas guuttuu godhachuudhaan gaaleesaa ijaarrata. Mee fakkeenyota armaan gadii haa ilaallu:\\
\\
a. Siifan \underline{erga isheen deessee} rafte.\\
b. \underline{Yommu Caaltuun dufte} Hundeessaan deeme.\\
c. \underline{Yoo kana jedhe} soba jedhe.\\

Himoota armaan olii irratti agarsiifaman keessa kanneen jala sararaman gaalee durduubee jedhamu. Gaaleewwan kanneen
keessaa (a) irratti isheen deesse, (b) irratti Caaltuun dhuftefi (c) irratti kana jedhe kanneen jedhan himoota; himoonni kunneenis guuttota durduubeewwan erga, yoommufi yoo jedhaniiti, walduraaduubaan. Gaaleewwan kanneen keessaa tokko fudhannee mee caasaatiin haa ilaallu:

\begin{figure}[H]										
	\caption{Gaalee durduubee}
	\centering
	\begin{forest}
		[GDG
			[GD1
				[GD [erga]]
			]
			[H [isheen deessee,roof]]
		]
	\end{forest}
\end{figure}

Akaakuun sadaffaan durduubee isa gaalee durduubee guuttuu godhatu. Gaaleewwan durduubee tokko tokko
durduubeedhuma guuttuu godhatu. Kanas himoota armaan gadii keessaa kanneen jala sararamaniiran hubanna:\\
\\
a. \underline{Hamma manaatti} na gaggeesse.\\
b. Obboleessi kee \underline{akka hangafaatti} na kabaje.\\
c. Inni \underline{gara manaatti} dafee deebi’e.\\

Himoota armaan olii irratti ilaalaman keessatti warri jala sararaman gaaleewwan durduubeeti. (a) irratti \textbf{hamma}, (b) irratti \textbf{akka}, (c) irratti \textbf{gara} hangafoota gaalee durduubeeti. Guuttuuwwan isaanii ammoo (a) irratti \textbf{manatti},(b) irratti \textbf{hangafaattifi} (c) irratti \textbf{manaatti} dha. Caasaa akaakuu gaalee kanaa fakkeenyaan haa ilaallu:

\begin{figure}[H]										
	\caption{Gaalee durduubee}
	\centering
	\begin{forest}
		[GDG
			[GD1
				[D [hamma]]
			]
			[GDG
				[GMG
					 [GM1 [M [manaa]]]
				]
				[GD1
					[D [-tti]]
				]
			]
		]
	\end{forest}
\end{figure}
 
 Gaaleen durduubee guuttuu qofaa osoo hintaane  murteessituus qaba. Galeen durduubee AO keessatti  murteessituu waan tokko hammam akka ta’e agarsiisuufi  murteessituu lakkoofsa agarsiisu qaba. Murteessituuwwan  kanneenis fakkeenyota armaan gadii irraa hubachuu dandeenya:\\
 \\
 a. \underline{Tarkaanfii sadii gara manaa} deemi.
 b. Dabalaan \underline{baayyee rakkoo irra} jira.\\
 
 Himoota armaan olii keessaa kanneen jala muramaniiran  gaalee durduubeeti; isaan keessaas (a) irratti \textbf{tarkaanfii sadii} murteessituu lakkoofsaa (b) irratti \textbf{baay’ee} murteessituu heddumminaati. (a) irratti \textbf{gara manaa} (b) irratti ammoo \textbf{rakkoo irra} kan jedhan gaaleewwan durduubeeti.
  
 \begin{figure}[H]										
 	\caption{Gaalee durduubee}
 	\centering
 	\begin{forest}
 		[GDG
 			[MU [baay'ee]]
 			[GD1
 				[GMG
 					[GM1
 						[M [rakkoo]]
 					]
 				]
 				[D [irra]]
 			]
 		]
 	\end{forest}
 \end{figure}

Gaaleen gochimibsaa\index{gaalee gochimibsaa} caasaatiin akkuma gaalee durduubeetti ilaalama. Gaaleen Gochimibsa, akkuma gaaleewwan biroo, gaalee guddaafi gaalee xinnaa of keessaa qaba. Akkasumas gabbistootafi murteessitoota qabaachuu danda’a. Gochimibsi guuttuu malee, gaalee ijaarrachuu danda’uun isaa gaaleewwan biro irraa adda isa taasisa. Fakkeenyaaf himoota armaan gadii mee haa ilaallu:\\
\\
a. Dammeen \underline{daddaftee} fiigde.
b. Dammeen \underline{Daraartuu caalaa daddaftee} fiigde. 

 \begin{figure}[H]										
	\caption{Gaalee durduubee}
	\centering
	\begin{forest}
		[GIGG
			[GDG
				[GMG
					[GM1
						[M [Daraartuu]]
					]
				]
				[GD1
					[D [caalaa]]
				]
			]
			[GIG1
				[IG [daddaftee]]
			]
		]
	\end{forest}
\end{figure}

Akka fakkeenya armaan olitti tuqame irraa hubachuun danda’amutti \textbf{Daraartuu caalaa} kan jedhu gabbistuu yoo ta’u \textbf{daddaftee} kan jedhu ammoo mataa gaalee gochimibsati.

As irraatti waan hubatamuu qabnu tokko, yeroo baay’ee gaaleen gochimibsaa, unkaa durduubee qabaatee mul’achuu
isaati. Gaaleewwan gochimibsaa kanneen bakka, meeshaa, si’afi kkf agarsiisan unkaa gaalee durduubee gonfatanii dhufu. Kanaaf gaaleen durduubee tokko tajaajilaan gaalee gochimibsa
ta’uu danda’a jechuudha.

\section{Gaalee Gochimaa}

Gochimni hangafa gaalee gochimaati\index{gaalee gochimaa}. Gaaleen gochimaa qaamota adda addaarraa ijaaramuu danda’a. Qaamonni
kunneenis gaalee maqaa, gaalee gochimibsa, gaalee maqibsa, gaalee durduubeefi himadha. Gochimni tokko qaamota kam
kam waliin akka deemuuf akka hindeemne kan murteessu uumamafi amala gochima sanaati. Odeeffannoon uumaafi
amala gochimaa kuusaa jechootaa keessa funaanama. Haala kanaan gochimni bakkeewwan hedduutti qoodama. Ammaaf
garuu gochimoota akaakuu sadii qofaa irratti xiyyeeffanna. Isaanis gochima hafoo, gochima ce'aafi gochima taasisuuti. Fakkeenyaaf gochimni \textbf{rafe} jedhu hafoodaha; \textbf{jaallate} kan jedhu ammoo ce'aadha; \textbf{beeksise} kan jedhu taasisaadha. 

\subsection{Gaalee Gochima Hafoo\index{gaalee gochima hafoo}}
Gochimni hafoo caasaa gaalee salphaa qaba. Fakkeenyaaf, \textbf{ gara manaa deeme} haa ilaallu. Asirratti
gaaleen durduubee (GD) guuttuu gaalee gochimaa waan ta’eef gaalee gochimaa xiqqaa (GG1) keessaa ba’e. Caasaa kana fakkii mukeetiin akka armaan gadiitti mul’isuu dandeenya: 

 \begin{figure}[H]										
	\caption{Gaalee gochimaa}
	\centering
	\begin{forest}
		[GGG
			[GDG
				[GD1
					[D [gara]]
					[GMG
						[GM1
							[M [manaa]]
						]
					]
				]
			]
			[GG1
				[G [deeme]]
			]
		]
	\end{forest}
\end{figure}

\subsection{Gaalee Gochima Ce'aa\index{gaalee gochima ce'aa}}
Akaakuun gochimaa inni lammaffaa gochima ce'aadha. Gochimni ce'aan akaakuu lama qaba. Isaanis gochima guuttu
tokko dirqiidhaan barbaaduufi gochima guuttuuwwan lama dirqiidhaan barbaadudha. Fakkeenyonni armaan gadii gochima
ce'aa guuttuu tokko dirqiidhaan barbaadan agarsiisu. Fakkeenyota kanneen keessaa maqileen \textbf{caccabsaa} fi , \textbf{uccuu adii }njedhan guuttuuwwan. \\
\\
a. Musaan caccabsaa nyaate.\\
b. Uumeen uccuu adii uffatte.\\
\\
Caasaa gaalee gochima ce'aa (b) akka armaan gadiitti kaa'uu dandeenya: 

\begin{figure}[H]										
	\caption{Gaalee gochimaa}
	\centering
	\begin{forest}
		[GGG
			[GG1
				[GMG
					[GM1
						[M [uccuu]]
						[GIMG
							[GIM1
								[IM [adii]]
							]
						]
					]
				]
				[G [uffatte]]
			]
		]
	\end{forest}
\end{figure}

Akaakuun gaalee gochimaa biroon ammoo guuttuu tokko qofaa osoo hintaane guuttuuwwan lama barbaada. Gochimoonni
kanneen akka kenne fi kaa’e jiran as keessatti ramadamu. Fakkeenyaaf himoota armaan gadii haa ilaallu: \\
\\
a. Fayyisaan kitaaba \underline{Tolasaaf kenne}.\\
b. Uumeen \underline{qarshii saanduqa keessa keesse}. \\

(a) fi (b) keessatti qaamonni jala sararaman gaaleewwan gochimaati. (a) irratti xumurri kenne jedhu guuttota lama
qaba; isaanis antima kalatti kitaaba jedhuuf antima al kalatti Tolasaaf jedhu. (b) irrattis xumurri keesse jedhu antima kalatti qarshii jedhuuf antima al kalatti saanduqa jedhu qaba. Fakeenyota lamanuu keessatti gochimoonni antimoota lama lama qabu jechuudha. 

\begin{figure}[H]										
	\caption{Gaalee gochimaa}
	\centering
	\begin{forest}
		[GGG
			[GG1
			[GMG
				[GM1
					[M [qarshii]]
				]
			]
			[GDG
				[GD1
					[GMG
						[GM1
							[M [saanduqa]]
						]
					]
					[D [keessa]]
				]
			]
			[G [keesse]]
			]
		]
	\end{forest}
\end{figure}


\subsection{Gaalee Gochima Taasisuu\index{gaalee gochima taasisuu}}
Akaakuun gochimaa inni sadaffaan gochima taasisuu jedhama. Gochimni taasisuu akaakuu adda addatu jiru. Gochimni
taasisuu inni salphaa jedhamu gochima hafoo gara gochima ce'aatti jijjiira. Fakkeenyaaf, \\
\\
a. Mucaan \underline{rafe}; \\
b. Tolasaan \underline{mucaa raff-is-e}.\\
b. Yaalaan \underline{Tolasaa-tiin mucaa raff-is-iis-e}. \\
\\
Himoota (a), (b) fi (c)keessatti akka hubachuu dandeenyutti gochimni \textbf{rafe} jedhu hafodha; \textbf{raff-is-e} inni jedhu ammoo ce'a/taasisaadha; \textbf{raff-is-iis-e }n taasisadha. Caasaa gaalee gochima taasisuu akka armaan gadiitti agarsiisuu dandeenya:

\begin{figure}[H]										
	\caption{Gaalee gochimaa}
	\centering
	\begin{forest}
		[GGG
			[GG1
				[GMG
					[GM1
						[M [mucaa]]
					]
				]
				[GG1
					[G [raf-]]
					[G [-is-e]]
				]			
			]
		]
	\end{forest}
\end{figure}

\begin{figure}[H]										
	\caption{Gaalee gochimaa}
	\centering
	\begin{forest}
		[GGG
			[GDG
				[GD1
					[GMG
						[GM1
							[M [Tolasaa]]
						]
					]
					[D [-tiin]]
				]
			]
			[GG1
				[GMG
					[GM1
						[M [mucaa]]
					]
				]
				[GG1
					[GG1
						[G [raf-]]
						[G [-is-]]
					]
					[G [-iis-e]]
				]
			]
		]
	\end{forest}
\end{figure}

\subsection{Gochima Hima Guuttu Godhatu\index{gochima hima guuttuu godhatu}}
Gochimooni tokko tokko hima guuttuu gochachuu filatu. Fakkeenyaaf himoota armaan gadii haa ilaallu: \\
\\
a. Kabaan \underline{akka Mootiin deeme} beeka.\\
b. Soorettiin \underline{akka Boontuun dhufte} dhageesse.\\
c. Tolasaan \underline{akka Garbaan fayye} arge.\\
\\

Fakkeenya caasaa (a) haa ilaallu:
\\

\begin{figure}[H]										
	\caption{Gaalee gochimaa}
	\centering
	\begin{forest}
		[GGG
			[GG1
				[H [akka Mootiin deeme,roof]]
				[G [beeka]]
			]
		]
	\end{forest}
\end{figure}

Himoota armaan olii keessatti kanneen jala muraman himoota. Gochimootni kunneen \textbf{beeka}, \textbf{dhageesse} fi \textbf{arge} ti. Hundi isaanii gochimoota ceetuudha. Guuttonni isaanii himoota hirkatoodha. Kana malees gochimoonni \textbf{danda’a}, \textbf{jedhe} fi \textbf{gaafate}n hima guuttuu godhatu. \\
\\
a. Toobarraan \underline{uffata miicuu danda’a}.\\
b. Kumsaan \underline{‘biyyi kee eessa?’ jedhe}.\\
c. Kumsaan \underline{biyyi isaa eessa akka ta’e gaafate}.\\
\\
Himootni jala sararaman guuttuuwwan gochimoota \textbf{danda’a}, \textbf{jedhee} fi \textbf{gaafate} ti.\\

\begin{figure}[H]										
	\caption{Gaalee gochimaa}
	\centering
	\begin{forest}
		[GGG
			[GG1
				[H ['biyyi kee eessa?,roof]]
				[G[jedhe]]
			]
		]
	\end{forest}
\end{figure}


Gaaleen gochimaa gabbistuufi murteessitus qaba. Kana jechuun gochimni gabbistu waliin dhufee gaalee gochimaa akka
ijaara jechuudha. Gabbistoonni gochimaa, gochima waliin yoo dhufani gaalee gochimaa guddaa (GXG) keessaa ba’u.
Gabbistoonni gochimaa, maqaa \textbf{gochimibsa} jedhuun beekamu. 

Gochimibsoonni kanneen akka bakka, yeroo, akkaataa, akeekaa, meeshaafi kkf agarsiisan gochima waliin dhufanii gaalee gochimaa ijaaru. Gochimibsi unkaatiin akka gaalee durduubeetti tajaajilu garuu akka gaalee gochimibsa ta’e dhufuu dnada’a.\\
\\
a. Inni \underline{hiddaan saree hidhe}.\\
b. Inni \underline{haadha warraa isaa waliin daadhii dhuge}.\\
c. Inni \underline{hoolaa bituuf gara Horroo deeme}.\\
d. Inni \underline{daddafee gara Booranaa deeme}.\\
e. Inni \underline{kaleessa na barbaade}.\\

As keessaa fakkeenyaaf caasaa (a) haa ilaallu:

\begin{figure}[H]										
	\caption{Gaalee gochimaa}
	\centering
	\begin{forest}
		[GGG
			[GG1
				[GDG
					[GD1
						[GMG
							[GM1
								[M [hidda]]
							]
						]
						[D [-an]]
					]
				]
				[GG1
					[GMG
						[GM1
							[M [saree]]
						]
					]
					[G [hidhe]]
				]
			]		
		]
	\end{forest}
\end{figure}

Gaaleen gochimaa gabbistuu ykn gochimibsa qofaa osoo hin taane murteessitullee ida’atee ijaarama. Fakkeenaaf himoota kanatti aananii dhufan haa ilaallu: \\
\\
a. Dubartiin sun \underline{baayyee kolfite}.\\
b. Riqituun \underline{yeroo sadii saree ilaale}.\\

(a)irratti jechi baayyee jedhu hammam akka iyyame agarsiisaati. (b) irratti ammoo gaaleen yeroo sadii jedhu
murteessituudha. Bakka lamaanittu murteessitoonni unka gaalee maqa qabu. 

\begin{figure}[H]										
	\caption{Gaalee gochimaa}
	\centering
	\begin{forest}
		[GGG
			[MU [yeroo sadii]]
			[GG1
				[GMG
					[GM1
						[M [saree]]
					]
				]
				[G [ilaale]]
			]
		]
	\end{forest}
\end{figure}

\section{Hima}
Kutaa kana keessatti hima\index{hima} ce’aa, raawwatamaa, taasisuu, hirkataa irratti xiyyeeffachuun caasima isaanii ilaalla. 

\subsection{Hima Ce'aa\index{hima ce'aa}}
Hima jechuun qindoomina gaaleewwanii isa yaada guutuu tokko dabarsuu danda’u jechuudha; kana jechuunis eenyu, gochaa maal akka raawwate siirriitti agarsiisa jechuudha. Fakkeenyaaf hima armaan gadii caasaatiin haa ilaallu:\\
\\
Hima: Namichi dheeraan sun hoolaa adii bite. 

\begin{figure}[H]										
	\caption{Caasima}
	\centering
	\begin{forest}
		[H
			[GMG
				[GM1
					[M [Namichi]]
				]
				[GIMG
					[IM [dheeraan]]
					[MU [sun]]
				]
			]
			[GGG
				[GG1
					[GMG
						[GM1
							[M [hoolaa]]
						]
						[GIMG
							[GIM1
								[IM [adii]]
							]
						]
					]
					[G [bite]]
				]
			]		
		]
	\end{forest}
\end{figure}

Waluumaa galatti himni ce'aan tokko yoo xinnaate gaaleewwan gosa lama irraa ijaaramuu qaba. Gaaleewwan gosa lama ta’anii himaf bu’uura ta’anis gaalee maqaa fi gaalee gochimaati. Gaaleen maqaa eenyu gocha tokko akka raawwate yoo
agarsiisu, gaaleen gochimaammoo gochi eenyurratti ykn maal irratti akka raawwatame ifa godha. Kana jechuun gaalee gochimaa antimaafi gochima of keessaa qaba jechuudha. Hima AO keessatti gaaleen maqaa dura yoo dhufu gaaleen gochimaa ammoo dhuma irra dhufa. Ulaagaa kanaan caasimni armaan olii hima ce'aa jedhama. Himni ce'aan kunis gaaleewwan lamarraa ijaarame. Isaanis gaalee maqaa \textbf{namichi dheeraan sun }jedhuufi gaalee gochimaa \textbf{hoolaa adii bite} jedhudha. 

\subsection{Hima Raawwatamaa\index{hima raawwatamaa}}
Akka maddi caaslugaa\footnote{Kitaaba kana keessatti, gaaleen 'madda caaslugaa' jedhu kun gaalee Afaan Inglizii 'generative grammar' bakka bu'a.} jedhutti himni raawwataman armaan gadii hima ce'a armaan oliirraa ce’umsaan\footnote{Jechi 'ce'umsa' jedhu kun caasluga maddaa keessatti jecha Afaan Inglizii 'transformation' jedhu bakka bu'a.} argame \cite{chomsky1965aspects,chomsky1982some}. \\
\\
Hima: Hoolaa adiin namicha dheeraa sanaan bitame.

Himni armaan olii kun hima raawwatamaa jedhama. Hima raawwatamaa kana keessatti bakka matimaa kan qabatee jiru
antima \textbf{hoolaa adiin} jedhudha. \\
\\
Hima: Hoolaa adiin bitame. \\


Caasima raawwatamaa uumuuf guuttuun gaalee gochimaa GMG (hoolaa adii) kan jedhu bakka antimaa gadhiisee bakka matimaa gala. 

\begin{figure}[H]										
	\caption{Caasima}
	\centering
	\begin{forest}
		[H
		[GMG,name=item f]
		[GGG
		[GG1
		[f,draw]
		{
			\draw[->,double] () to[out=south west,in=south] (item f);
		}
		[G]
		]
		]
		]
	\end{forest}
\end{figure}

\begin{figure}[H]										
	\caption{Caasima}
	\centering
	\begin{forest}
		[H
			[GMG
			[GM1
				[M [Hoolaa]]
			]
			[GIMG
				[GIM1
					[IM [adiin]]
				]
			]
			]
			[GGG
				[GG1
					[G [bitame]]
				]
			]
		]
	\end{forest}
\end{figure}

Kanaaf maayii matimaa dabalate, \textbf{hoolaa adiin} matima hima raawwatamaa ta'e. Yeroodhuma kanatti
ammoo gochimni \{bit-\} fufii raawwatamaa \{-am-\} ida'ate. Yeroo akka kanaatti fufiin kun akkuma gochima tokkotti lakkaa'amti. Sababni isaas waan antima gara matimaatti jijjiirteef.

Walumaa galan hima raawwatamaa keessatti wantota gurguddoo sadiitu mul'ata. Inni tokko antimni matima ta'uu isaati. Inni lammaffaan ammoo fufiin \{-am-\} gochima irratti ida'amuudha. Inni sadaffa ammoo matimni antima al kallattiitti jijjiiramuudha. Qabxii isa sadaffaa kana hima armaan gaddii irraa haa hubannu. Hima: Hoolaa addiin \textbf{namicha dheeraa-tiin} bit-am-e. 
\begin{figure}[H]										
	\caption{Caasima}
	\centering
	\begin{forest}
		[H
			[f,name=item f]
			[GGG
				[GDG
					[GMG,draw]
					{
					\draw[<-,double] () to[out=south west,in=south] (item f);
					}
					[GD1 [D]]
				]
				[GG1
					[G]
				]
			]
		]			
	\end{forest}
\end{figure}

Hima kana wayita xiinxallu qaamni gocha raawwate bakka matimaa gadhiisee haala antima alkalattiitiin ibsameera. Yeroo bakka antimaa qabatus unka gaalee durduubee fakkaateeti. Walumaagalatti, hima raawwatamaa keessatti maayiin waljijjiruun jira. Hima ce'aa keessatti kan matima ta'e hima raawwatamaa keessatti antima ta'a; hima ce'aa keessatti kan antima ta'e ammoo hima raawwatamaa keessatti matima ta'ee mul'ata jechuudha.

\begin{figure}[H]										
	\caption{Caasima}
	\centering
	\begin{forest}		
		[H		
			[GMG [f]]
			[GGG
				[GDG
					[GMG
						[GM1
							[M [namicha]]
						]
						[GIMG
							[GIM1
								[IM [dheeraa]]
							]
						]
					]
					[D [-tiin]]
				]
				[GG1
					[G [bitame]]
				]
			]		
		]		
	\end{forest}
\end{figure}



\subsection{Hima Taasisuu}

Himni taasisuu\index{hima taasisuu} fufii fufii taasisuu gochima irratti dabala. Akkasumas matima gara biroo himatti dabala. Matima duraan ture ammoo gara antimaatti jijjiira.\\
\\
Hima Ce’aa: Namichi hoolaa \textbf{gurgure}.\\
Hima Taasisuu 1: Abdiin namicha hoolaa \textbf{gurgur-siis-e}. \\

\subsubsection{Taasisuu Tokkeessoo\index{hima taasisuu tokkeessoo}}
Hima ce'aa kana irraa hima taasisaa uumuuf gochimarratti fufii \{-siis-\} ida'uu qabna. Fufiin taasisuu wayita ida’amtu matimni gara biraan (Abdiin) himichatti ida’ama. Fufii taasisuu tokko qofa ida'anii matima tokko dabaluun kun hima taasisuu tokkeessoo\footnote{Hima taasisuu tokkeessoon  ogbarruu addaddaa keessatti 'single causative' jedhama.} jedhama. 

\begin{figure}[H]										
	\caption{Caasima}
	\centering
	\begin{forest}		
			[H
				[GMG
					[GM1 [M [Abdiin]]]
				]
				[GGG
					[GMG
						[GM1 [ M [namicha]]]
					]
					[GG1
						[GMG
							[GM1 [M [hoolaa]]]
						]
						[GG1
							[G[gurgur-]]
							[G [-siis-e]]
						]
					]					
				]
			]		
	\end{forest}
\end{figure}

Akka fakkeenya armaan olii irratti ilaallu kana hima ce'aa keessatti kan matima ture amma hima taasisaa keessatti antima ta'ee dhufeera; kanaafi maqaan \textbf{namichi} jedhu unka matimaa hinqabu. Akkasumas fufiin \[-siis-\] gochima irratti ida'ameera. Fufiin kun akkuma gochimaatti lakkaa'ama. Akkuma duraan jenne, hima taasisuu keessatti matimni hima ce'aa gara antimaatti jijjiiramuufi fufiin ida'amuu qofa osoo hintaane matimni haaraa hima taasisuutti dabalama. Kanas fakkeenya armaan oliirraa hubachuun danda'ama; matimni \textbf{Abdiin} jedhu fufii taasisuu waliin hima duraan turetti dabalamniiru. Kanaaf himni kun hima taasisaa tokkeessoti jechuu dandeenya.

\subsubsection{Taasisuu Lammeessoo\index{taasisuu lammeessoo}}
 Himni taasisuu gad fageenya qaba. Fufiin gochima taasisuullee akaakuu adda adda qaba. Caasaan hima taasisuullee akaakuu adda adda qaba. Caasaan hima taasisuullee hedduudha \citeA{tolemariam2009}. AO keessatti himni taasisuu tokkeessoo qofa miti. Himni taasisuu lammeessoollee jira. Himni taasisuu lammeessoo fufiilee taasisuu lamaafi matimoota lama hima ce'aatti ida'a. Fakkeenyaaf,\\
 \\
 Hima Taasisuu 2: \textbf{Uumeen Abdiitiin} namicha hoolaa \textbf{gurgursis-iis-te.}\\
 \\
 \begin{figure}[H]
 	\caption{Caasima}
 	\centering
 	\begin{forest}
 		[H
 			[GMG
 				[GM1[M[Uumeen]]]
 			]
 			[GGG
 				[GDG
 					[GD1
 						[GMG
 							[GM1[M[Abdii]]]
 						]
 						[D[-tiin]]
 					]
 				]
 				[GG1
 					[GMG
 						[GM1[M[hoolaa]]]
 					]
 					[GG1
 						[G[gurgur-]]
 						[GG1
 							[G[-sis-]]
 							[G[-iis-te]]
 						]
 					]
 				]
 			]
 		]
 	\end{forest}
 \end{figure}

Akka armaan olitti ilaallu kana gochima \textbf{gurgur-} jedhu irratti walduraa duubaan fufileen taasisuu ida'amaniiru. Dura \{-sis-\} tu ida'ame; achii ammoo \{-iis-\} tu ida'ame. Yeroo kamuu fufiin
taasisuu gochima irratti ida'amte jechuun matimni haaraan dabalame jechuudha. Akkasumas matimni haaraan dabalame jechuun matimni duraan ture bakka matimaa gadhiisee bakka antimaa qabatee antima ta'a jechuudha. Haaluma kanaan maqaan \textbf{Uumeen} jedhu matima haaraa ida’ame yoo ta’u, maqaan \textbf{Abdii} jedhu ammoo bakka matimaa gadhiisee antima alkalattii ta’eera. Kanaaf himni kun hima taasisuu lammeessooti jenna.

\subsection{Hima Hirkataa\index{hima hirkataa}}

Himoonni hamma amaatti ilaalaa turre himoota ofdanda’anii dhaabbatan. Ammammoo himoota ofisaatiin ofdanda’anii hindhaabbanne ilaalla. Himni ofisaatiin ofdanda’ee hindhaabbanne hima hirkataa jedhama. Himni hirkataan ciroo firoomsee jedhamees waamama. Tajaajilli ciroo firoomsee gaalee maqaa ykn hima gara biroo gabbisuu ykn ibsuudha. Himni hirkataan hima of danda’aatti saagamee mul’ata. Hima
hirkataa bakkeewwan sadiitti qoodnee ilaaluu dandeenya. Isaanis: \textbf{ciroo firoomsee}, \textbf{ciroo guuttuu} fi \textbf{ciroo gochimibsaa}ti. Ciroo firoomsee jechuun hima hirkataa ta’ee gaalee maqaa ibsuuf kan dabalamu jechuudha. Ciroon firoomsee kun gaalee maqaatti kan hidhamu karaa maqdhaalii/bamaqaa firoomseeti. AO keessa maqdhaalii firoomsee adda addaa jiru. Isaan keessaas jechi \textbf{kan} jedhamu bal’inaan beekama. Maqdhaaliin firoomsee kun ciroowwan gosa adda addaa wajjin akka fedhetti galee tajaajiluu danda’a. Kana malees \textbf{isa} fi \textbf{isheen} akka maqadhaalii firoomseetti tajaajilu. Akkasumas \textbf{warra} fi \textbf{waan} haaluma kanaan tajaajilu. \textbf{Isa}n yeroo gaaleen maqaa ibsamu sun kornyaa kormaa agarsiisu, \textbf{isheen} ammoo yeroo gaaleen maqaa ibsamu sun kornyaa dhalaa agarsiisu galu. Karaa gara biroo ammoo \textbf{warran} yeroo gaaleen maqaa ibsamu sun hedduu ta’u dhufa; \textbf{waan} ammoo yeroo gaaleen maqaa ibsamu sun kan lubbuu qabuun ala ta’u gala. Ciroowwan firoomsee kanneen fakkeenyota armaan gadii irraa hubachuu dandeenya: \\
\\
a. Namichi \underline{inni hoolaa bite }obboleessa kooti.\\
b. Ati \underline{loltoota warra biyya kee gargaare} beekta.\\
c. Dubartiin \underline{isheen kaleessa buna danfifte} haadha manaa kooti. \\

Himoota armaan olitti barreeffaman irratti akka hubatamu kanneen jala sararamaniiran ciroowwan firoomseeti. (a) irratti maqaan \textbf{namichi} jedhu matima hima sanaati. Ciroon firoomsee
hima sana keessa jirus matima gabbisa ykn ibsa. (b) irratti garuu ciroon firoomsee gaalee maqaa isa akka antimaatti tajaajilu ibsa ykn gabbisa. (c) irratti gaaleen maqaa ciroo firoomseetiin ibsamu dubartiidha. Gaaleen maqaa kun korniyaa dubartii agarsiisa. Kanaafis ciroon firoomsee karaa
bamaqaa \textbf{isheen} jedhuu himatti hidhameera. Hima hirkataa gosa kanaa fakkeenyaan agarsiisuu dandeenya: \\
\\

\subsubsection{Ciroo Firoomsee\index{ciroo firoomsee}}

Ciroo firoomsee: Namichi \textbf{inni hoolaa bite} Boruu dha.\\
\\
Akka fakkeenya kana irraa hubatamutti himoota lamatu argamu, hima of danda’aafi hima hirkataa. Matimni hima of danda’aa maqaa isa \textbf{namichi} jedhu yoo ta’u, gochimni isaa moo \{-dha\} dha. Matimni hima hirkataa (ciroo firoomsee) bamaqaa isa \textbf{inni} jedhu yoo ta’u gochimni isaa ammoo jecha
bite jedhu.

\begin{figure}[H]
	\caption{Caasima}
	\centering
	\begin{forest}
		[H
			[GMG
				[GM1[M[Namichi]]]
				[H
					[GMG[GM1[M[inni]]]]
					[GGG
						[GMG[GM1[M[hoolaa]]]]
						[GG1[G[bit-e]]]
					]
				]
			]
			[GGG
				[GMG[GM1[M[Boruu]]]]
				[GG1[G[dha]]]
			]
		]
	\end{forest}
\end{figure}

\subsubsection{Ciroo Guuttuu}

Akaakuun hima hirkataa inni lammaffaan ciroo guuttuuti\index{ciroo guuttuu}. Ciroon guuttuu karaa jecha \textbf{akka} jedhuu himatti hidhama. Akaakuu ciroo kanaa hubachuuf fakkeenyota armaan gadii haa ilaallu: \\
\\
a. Tulluun \underline{akka Fayyisaan dhufe} beeka.\\
b. Kabaan \underline{akka Boontuun fayyite} dhaga’e.\\
c. Toleeraan \underline{akka ilmisaa guddate} arge.\\
\\
Fakkeenyota armaan oliirraa akka hubannutti himoonni jala muraman kunneen ciroo firoomsee jedhamu. Ciroowwan kunneen hundi isaaniiyyuu guuttuu gaalee gochimaati. Kanneen armaan olii keessaa fakkeenya tokko caasima mukeetiin haa ilaallu: \\
\\
Ciroo guuttuu: Tulluun\underline{ akka Fayyisaan dhufe} beeka.

\begin{figure}[H]
	\caption{Caasima}
	\centering
	\begin{forest}
		[H
			[GMG[GM1[M[Tulluun]]]]
			[GGG				
				[GDG
					[GD[D[akka]]]
					[H
						[GMG[GM1[M[Fayyisaan]]]]
						[GGG[GG1[G[dhuf-e]]]]
					]
				]
				[GG1[G[beeka]]]
			]
		]
	\end{forest}
\end{figure}

\subsubsection{Ciroo Gochimibsaa}
Inni sadaffaan ciroo gochimibsati. Ciroon gochimibsa\index{ciroo gochimibsaa} akaakuu adda addaa qaba. Isaan keessaa inni tokko ciroo gochimibsa isa yeroo agarsiisu.\\
\\
Ciroo gochimibsaa: Ganamoon \underline{yeroo qorumsa fixu} dhufa.\\
\\
Fakkeenya armaan olii irratti kan jala sararame kun ciroo gochimibsati. Caasaatiin ciroon kun ol-aanaan isaa gaalee gochimaa guddaadha. Jechi \textbf{yeroo} jedhu kun ciroo gochimibsaa
isa yeroo agarsiisu himatti hidhuuf tajaajila. Caasima ciroo firoomsee kana akka armaan gadiitti ilaaluu dandeenya. 

\begin{figure}[H]
	\caption{Caasima}
	\centering
	\begin{forest}
		[H
			[GMG[GM1[M[Ganamoon]]]]
			[GGG
				[GIG
					[GMG[GM1[M[yeroo]]]]
					[H
						[GMG[GM1[M[(Ganamoo)]]]]
						[GGG							
							[GG1
								[GMG[GM1[M[qorumsa]]]]
								[G[fix-u]]
							]
						]
					]
				]
				[GG1[G[dhuf-a]]]
			]
		]
	\end{forest}
\end{figure}

Akaakuun ciroo gochimibsa bironammoo gochimibsa isa bakka agarsiisu. Ciroon gosa kanaa kan himatti hidhamu karaa jechoota \textbf{bakkaa}ti. Fakkeenyaaf himoota armaan gadiitti ilaaluu dandeenya.\\
\\
a. Raffisaan lafa Tolaan ture deeme.\\
b. Tolasheen iddoo abbaan ishee taa’e teesse.\\
c. Rabbirraan bakka barumsi jiru dhaqe. \\

Ciroon biroon hima of danda’aatti suuqamu ciroo gochimibsa isa akkaataa agarsiisu. Ciroon akaakuu kanaa himatti kan suuqamu karaa \textbf{akka...tti} ti. Durduubeen \textbf{akka} jedhu gaalee
maqaa matimaa (hima hirkataa keessatti) dursee dhufa. Fufii \{-ttin\} ammoo gochima ciroo gochimibsatti fufii ta’ee dhufa. Fakkeenyaaf himoota armaan gadii haal ilaallu: \\
\\
a. Tolasaan \underline{akka Fiixaan fiigu} fiiga.\\
b. Caaltuun \underline{akka Boontuun dhugdutti} dhugdi. \\

Himoota armaan olii keessatti warri jala sararaman ciroo gochimibsati. Calaqaba ciroo kanneen irratti jechi \textbf{akka} jedhu jira. Caasaan himoota kanneeniif fakkeenyaan kennuu dandeenya: 

Hima: Ganamoon \textbf{aaka Fiixaan fiigu} fiiga.


\begin{figure}[H]
	\caption{Caasima}
	\centering
	\begin{forest}
		[H
			[GMG [GM1[M[Ganamoo-n]]]]
			[GGG
				[GDG[GD1[D[akka]]]]
				[H
					[GMG[GM1[M[Fiixaa-n]]]]
					[GGG[GG1[G[fiigu]]]]
				]
			]
		]
	\end{forest}
\end{figure}

\subsubsection{Gaaffilee Boqonnichaa }

1. Yaadiddimni caasluga maddaa maali?\\
2. Caacculeen caasluga maddaa maal faati?\\
3. Yaadota armaan gadii ibsa kenniitii caasaatiin agarsiisii:\\
	a. Gaalee maqaa\\
	b. Gaalee maqibsaa\\
	c. Gaalee gochimaa\\
	d. Gaalee durduubee\\
	e. Gaalee Gochimibsa\\
4. Unkalee armaan gadii caasaatiin kaa’i:\\
	a. Galgaloon hoolaa bite.\\
	b. Hoolaan Galgaloon bitame.\\
	c. Galgaloon gurbaa hoolaa gurgursiise.\\
	d. Gurbaan dheeraan sun konkolaataa bite.\\
	e. Gurbaan inni kaleessaa hoolaa bite kitaaba bite.
	
	\newpage
	
\chapter{Hiikcaasima\index{hiikcaasima}}

\subsubsection{Qabiyyee}
\begin{itemize}
	\item Hiikcaasima Hima Hafoo
	\item Hiikcaasima Hima Ta'uu
	\item Hiikcaasima Hima Ce'aa
	\item Hiikcaasima Hima Raawwatamaa
	\item Hiikcaasima Hima Garlamee
	\item Hiikcaasima Hima Giddugalaa
	\item Hiik caasiam Hima Taasisuu
\end{itemize}
\subsubsection{Gaaffilee Ka'umsaa}
\begin{enumerate}
	\item Hiikcaasima hima hafoo ibsi.
	\item Akaakuuwwan caasima raawwatamma ibsi.
	\item Akaakuuwwan hiika giddugalaa ibsi.
	\item Akaakuuwwan hiika taasisaa ibsi.
	\item Akaakuuwwan hiika hima garlamee ibsi.
	\item Akaakuuwwan hima raawwatamaa ibsi. 
	\item Taasisuu qeentee maali?
	\item Taasisuu lammeessoofi sadessoon maali?
\end{enumerate}


\section{Seensa}
Kutaan kun hiikcaasimaa dhiyeessa. Hiikcaasimni waan lama of keessaa qaba. Inni tokko caasaa himaa yoo ta'u innni lammaffaa ammoo hiika. Akkuma beekamu caasaan unkadha. Boqonnaa 5 keessatti unka himaa ykn caasaa himaa ilaalleerra. Boqonnaa kana keessatti ammoo caasaalee himaa gara garaa boqonnaa 5 keessatti ilaalleef hiika kennina. Hiikaafi unka himaa xiinxalla. Hiikaafi caasima himaa kan qo'atu caasluga tajaajilaa \footnote{Boqonnaa kana keessatti, gaalee 'hiikcaasima' jedhu 'functional grammar' bakka buuseen itti gargaarameera.}ti. Kanaaf xiyyeeffannoon boqonnaa kana hiikcaasima. Hiikcaasimni addunyaa irratti yaadiddimoota beekaman keessaa tokko \footnote{Hayyoonni hedduun hiikcaasima irratti barreessaniiru. Hojiiwwan \cite{siewierska1984passive,shibatani1985passives,halliday1994introduction,kulikov1993second,mous2001middle,tolemariam2009} fi kkf dubbisuun beekumsa gadfageenyaa nama gonfachiisa.}. 

Boqonnaa kana keessatti hiikaafi caasimaa himaa qaaccessuuf qabxiiwwan armaan gadii gargaaramna:\\
\begin{itemize}
	\item Inni jalqabaa, jechoota hima keessatti tuqaman garee isaanii adda baasuun isa calqabaati. Fkn maqaa, gochimafi kkf. 
	\item Inni lammeessoon, tajaajila jechoota hima keessatti mul’atan adda baasuudha. Fkn matima, anitima, kutimafi kkf.
	\item Inni sadeessoon, gaaleewwan adda baasanii baruudha. 
	\item Inni afreessoon, tartiiba matimaafi gochimaa baruudha.
	\item Inni shaneessoon, gosafi tajaajila maxxantoota hima keessatti argaman adda baasanii qaacessuudha.
	\item Inni ja'afaan hiika baruudha.
	
\end{itemize}


\section{Hiikcaasima Hima Hafoo\index{hiikcaasima hima hafoo}}
Hima hafoo tokko keessatti matimnifi gochmni bu’uura.\\
\\
a. Kamisee-n dhuf-te.\\
b. Siddiiqee-n fiig-de. \\
\\
Haa ta'u malee, himni hafoo qaama gara biroo ida’achuu hindanda’u jechuu miti.
Yomiyyuu taanan himni hafoo tokko gochimibsa ida’achuu mala. \\
\\
a. Galgaloon-n [gara mana barumsaatti] fiig-e.\\
b. Kamisee-n [Awuroopaatii] dhuf-te.\\
c. Siddiiqee-n [gara konkolaataatti] fiig-de. 

Himoonni armaan olitti eeraman kunneen hafoodha. Himoota kanneen keessatti qaamonni hammatuu keessa galaniiran tajaajilaan gochimibsa jedhamu; akaakuun gaalee isaanii garuu gaalee durduubee jedhama. Sababni kanaasa qaamonni kununneen durduubee irratti waan bu’uureffataniif. (a) keessatti bu’uurri durduubee isa \textbf{gara} jedhu, (b) keessatti \{-tii\} yoo ta’u (c) keessatti \textbf{gara} dha.

As irratti qabxiin hubannoo argachuu malu, himni hafoon tokko matima, gochimibsafi gochima of keessaa qaba yoo ta’e matimni dura, itti aanee gochimibsa dhumarratti gochimatu dhufa. Yeroo mara matimni dura dhufa; gochimni ammoo dhumarra dhufa. Kun seera caasima hima AOti.

Himni hafoo hiika qaba. Akaakuun hiika hima hafoo AO sadii ta’u. Tokka hima hafoo
matima abgocha jedhamu qaba. Inni lammaffaan matima mitiabgocha jedhamu qaba. Inni sadaffaanammoo matimaa muuxataa jedhamu qaba. Matima abgocha jechuun nama itti yaadee ykn ta’e jedhee kaka’umsa mataa isaatiin gocha tokko raawwate agarsiisa. \\
\\
a. Barattuun mukarra utaal-te.\\
b. Gurbaan lafa taa’-e.\\
c. Ijoolleen laga ceet-e.

Himoota armaan olii caasaatiin matima tokko tokkoofi gochima tokko tokko qabu. Karaa hiikaa yoo ilaallu garu akaakuun matimaa maal? jennee gaafachuu qabna. Kana jechuun matimni itti yaadeti gocha tokko raawwatemoo miti? jennee yoo gaafannu, deebii eyyee jedhu arganna. (a) irratti barattuun itti
yaaddee, baaftee buuftee, danda’amoo hindanda’uu jettee, murteessitee mukarra utaaluushee hubachuu dandeenya. Kanaaf gosti matima kanaa abgocha. (b) irrattis matimni abgocha. Maaliif? yoo jenne, namni bakka tokko ta’uuf ykn dhaabachuuf itti yaadee qaamasaa ajajeeti kan sosso’uu danda’u. (c)s haaluma wal fakkaatuun ibsama.

Matimni miti-abgocha ta’e garuu fakkeenyota armaan olitti agarsiifaman irraa adda. Gosti matima kana itti yaadee, ta’e jedhee, ykn baasee buusee miti kan gocha tokko raawwatu. Fakkeenyaaf himoota armaan gadii haa ilaallu:\\
\\
a. Mucaan guddat-e.\\
b. Abdiisaan hamuummat-e.\\
c. Namni du’-e.

Himoonni armaan olii karaa caasimaa matima tokko tokkofi gochima tokko tokko qabu. Haa ta’u malee himoonni (a-c) tti agarsiifamani matimni isaanii miti-abgocha malee abgocha miti. Maaliif? yoo jenne, matammoonni eeraman kunneen itti yaadanii, ta’e jedhanii miti kan gocha tokko raawwatan. (a) irratti mucaan itti yaadee miti kan guddate, uumaman qaamnisaa guddate malee maliif akka guddates, akkamitti akka guddates waan beekus waan himus hinqabu. (b) irratti matimni osoo itti hinyaadni hamuummate, namni itti yaadee hamuummatu hinjiru. (c) irrattis namni osoo itti hinyaadni, yoom akka
du’u osoo hinbeekiin du’a; waan kana ta’eef matammoonni (a-c) irratti eeraman miti abgocha jedhamu.

Matimni muuxataa jedhamu matima abgocharraas mitiabgocharras adda. Matimni muuxataan namni gocha raawwatu tokko jireenya isaa keessatti muuxannoo argachuu isaa agarsiisa. Fakkeenyaaf himoota armaan gadii haa hubannu: \\
\\
a. Hintalli aar-te.\\
b. Shamarran gammad-an.\\
c. Gurbaan yaada’-e.

Himoonni (a-c) tti agarsiifaman caasimaan akkuma himoota duran ilaallee matima tokko tokkoofi gochima tokko tokko qabu. Garuu hiikan matima adda ta’e qabu. Hiikan matimoonni kunneen namoota jireenya keessatti muuxannoo argatan agarsiisu. Gochaawwan kanneen akka \textbf{aaruu}, \textbf{gammaduu}fi \textbf{yaadda’uu} kanneen sammuu namaa keessatti gaggeeffaman yoo ta’u muuxannoo namni tokko keessa darbu mul’isu.

Walumagalaan, himni hafoo matima tokkoofi gochima tokko qaabaachuu qaba. Matimni gochima dursee dhufa. Fufiin maayii matimaa maqaa matima ta’erratti maxxantee argamti. Fufiin waliigalteemmoo gochimatti maxxantee qaama gocha raawwate ramaddiitiin, lakkoofsaa, korniyaafi yerootiin mul’isti. Karaa hiikaa, matimni hima hafoo abgocha, mitiabgocha ykn muuxataa ta’uu danda’a.

\section{Hiikcaasima Hima Ta'uu}

Kutaa kana keessatti caasimaafi hiika hima ta’uu \index{hiikcaasima hima ta'uu} ilaalla. Himni ta’uu hima hafoo waliin walfakkaata. Himni ta’uu gochimoota ta’uu muurasa irratti bu’uureffata. Hima ta’u jechuun, akkuma maqaansaa agarsiisutti, namni tokko ykn waan tokko qaamaan, sammuutiin, sadarkaa jireenyaatiinfi kkf haala tokkorraa jijjiiramee haala biroo ta’uu agarsiisa. Duraan dursinee maalummaafi caasima hima ta’uu
haa ilaallu. Himni ta’uu AO keessatti karaa gochimoota sadii ibsama. Gochimoonni kunneenis \textbf{ta’e}, \textbf{dha} fi \textbf{miti} jedhamu; \textbf{ta’ee} fi \textbf{dha} n eentaa agarsiisu \textbf{miti} n garuu hi'eentaa agarsiisa. \\
\\
Hima: Gaarii-n \underline{barsiisaa ta’-e}. \\
\\

Himni kun hima ta’uu jedhama. Mee caasima hima kanaa haa ilaallu. Himni kun jechoota sadii of
keessaa qaba. Jechoonni kunneenis maqaa dhuunfaa \textbf{Gaarii} jedhu, maqaa waloo \textbf{barsiisaa} jedhufi gochima \textbf{ta’e} jedhu of keessaa qaba. Haa ta’u malee gaaleewwan lama qofaa of keessaa qaba. Gaaleewwan kunneenis gaalee maqaa \textbf{Gaarii-n} jedhuu fi gaalee gochimaa \textbf{barsiisaa ta’e} jedhu.

Maqaan dhuunfaa \textbf{Gaarii} jedhu matima, maqaan \textbf{barsiisaa} jedhu guuttuufi gochima ta’uu tokko qaba. Karaa tartiiba jechoota hima ta’uu yoo ilaalle, matimni dura dhufuu qaba, itti
aanee guuttuu gochimaa (maqaa isa barsiisaa jedhu jechuudha) dhumarratti gochima ta’uutu dhufa. Akkuma himoota hafoo duraan ilaallee, maqaan matima ta’e matima ta’usaa kan agarsiisuu fufii maayii matimaa of irratti maxxanfata. Fufiin kunis \{-n\} dha. Akkasumas fufiin \{-e\} gochima irratti akka fufii
boodaatti dhuftee matimni barsiisaa ta’e ramaddii sadaffaa, lakoofsa qeentee, korniyaa kormaa yoo qabaatu yeroon darbaa ta’uu mul’isti. Hima armaan oliitti eerame kana karaa gochima
\textbf{dha} ibsuu dandeenya: \\
\\
Hima: Gaarii-n barsiisaa \textbf{dha}.\\
\\

Himni kun eenyummaa \textbf{Gaarii} agarsiisa. Gaariin eenyumaansaa hojjaa isaatiin ibsame. Gochima \textbf{ta’e} jedhurraa \textbf{dha} kan adda taasisuu ramddoota qeentee hundaaf unkaa tokkoon yoo
tajaajiluu ramaddoota heddummina agarsiisaniif garuu filannoodha; hafuu danda’a jechuudha. Fakkeenyaaf himoota armaan gadii haa hubannu: \\
\\
a. Ani barataa \textbf{dha}; \\
b. ati barattuu \textbf{dha}; \\
c. isheen barattuu \textbf{dha}; \\
d. inni barataa \textbf{dha}; \\
e. nuti barattoota \textbf{dha}; \\
f. isin barattoota \textbf{dha}; \\
g. isaan barattoota \textbf{dha}.

Gochamni \textbf{ta’e} jedhu garuu ramaddoota hundaaf unkaa adda
addaa qaba; fakkeenyaaf,\\
a. ani barataa ta'e;\\
b. Ati barattuu taate;\\
c. Isheen barattuu taate;\\
d. Inni barataa ta'e;\\
e. Nuti barattoota taane;\\
f. Isin barttoota taatan;\\
g. Isaan barattoota ta'an;

 
Hi'entaan gochima \textbf{ta’e} karaa \textbf{hin} yoo ibsamu hi'eentaan gochima \textbf{dha} karaa \textbf{miti} ibsama. 

Karaa hiikaa himni ta’uu akkuma hima hafoo bakka sadiitti qoodamee ilaalama. Matimni hima ta’u hiikaan \textbf{abgocha}, \textbf{mitiabgocha} ykn \textbf{muuxataa} ta’uu danda’a. Fakkeenyaf himoota
armaan gadii haa ilaallu:  \\
\\
a. Hintalli diimtuu taa’-te;\\
b. Gurbaan gammadaa ta’-e;\\
c. Oboleessikoo ogganaa ta’-e.\\

Himni (a) irratti ilaallu, matima tokko qaba. Matimni kun maqaa \textbf{hintalli} jedhu. Hintalli kun qaamaan jijjiirraa agarsiifteetti; fuullishee kan dur diimaa hin turre amma diimateera. Haa ta’u malee matimni kun diimina qaamaa irratti dhiibaa hinfiddu; jijjiiraan kana waan itti yaaddee hinfidneef. Fakkeeny (b) irratti mul’atemmoo muuxannoo jireenyaa agarsiisa; matimni duraan amala gammaduu hinqabaanne waan muuxannoo gammachuu godhateera. Kanaaf matimni kun muuxataadha jenna. (c) garuu matima abgochaati; sababni kanaas matimni eerame itti yaadee, hojjaa oggantummaaf isa geessisuu dalagee waan ogganaa ta’ef. Walumagalaan, gochimni \textbf{ta’e} jedhu waan tokko haala tokko irraa gara birootti, jireenya sadarkaa tokko irraa gara biroottifi kkf jjijjiiramuu agarsiisa. Gochimni \textbf{dha} n waan
tokko haala tokko keessa jiraachuu isaa agarsiisa. Hi'eentaan gochima \textbf{ta’e} jedhuu \textbf{hintaane} yoo ta’u hi'eentaan \textbf{dha}, jecha miti jedhuudha.

\section{Hiikcaasima Hima Ce'aa}

Himni ce’aan \index{hiikcaasima hima ce'aa} humna qaama tokkorraa isa birootti ce’ee gocha raawwatu agarsiisa. Himni ce’aan caasimaan yoo xinnaate xinnaate matima tokko, antima tokkofi gochima ceetuu tokko qabaachuu mala. Garuu himni ce’aan antimoota lama qabulle jira. Mee duraan dursinee isa matima tokko, antima tokkoofi
gochima tokko qabu haa ilaallu: \\
\\
a. Soorettii-n mana gurgur-te;\\
b. Abdii-n mana bit-e;\\
c. Gaariin-n farda qab-e.\\

Himoonni armaan olii kunneen hundi hima ce’aa jedhamu. Hundi isaaniyyuu matima tokko tokko, antima tokko tokkoofi gochima ce’aa tokko tokko of keessaa qabu. Tartiiba jechatiin yoo ilaalle, matimni dura dhufa, itti aanee antimni dhufa, dhumarratti gochimni ce’an dhufa. Hunda keessatti fufiin maayii matimaa agarsiistu maqaa isa matimaa tajaajileeru irratti akka fufii boodatti maxxantee argamti. Akkasumas fufiin waliigaltee matima ramaddiitiin, lakkoofsaan, korniyaatiinfi yerootiin kam akka ta’e agarsiisti. Fufileen waliigaltee \{-te\} fi \{-en\} akka fufii boodaa ta’anii gochimarratti ida’amaniiru. 


AO keessaa himni ce’aan antimoota lama qabu jira. Fakkeenyaaf himoota armaan gadii haa hubannu:\\
\\
a. Isaan qarshii Soorettii-f kenn-an; \\
b. Gonfaa-n galmee sanduqa keessa kaa’-e; \\
c. Tollaanii-n lafa abbaasheerraa fudhat-te.\\

Himoonni armaan oliitti agarsiifaman matima, antima kallatti, antima alkallattiifi gochima qabu. Hunda keessatti matimni dura dhufeera, itti aanee antima kallatti, itti aanee antima alkallattiifi dhumarratti gochimni dhufeera. (a) irratti fufiin matimaa bamaqaa \textbf{isaan} jedhurratti hinida’amne, bamaqaan kun fufii matimaa hinida’atu (dhuma jecha kanaarra sagaleen [n] waan jirtuuf). (b) fi (c) irratti garuu fufiin matima agarsiiftu matimarratti ida’amteetti. Antimni alkallattii fufii
durduubee ofrraa qaba; (a) irratti fufiin durduubee \{-f\} jettu, (b) irratti durduubeen \textbf{keessa} jedhufi (c) irratti durduubeen \{-rra\} jedhu antimoota alkallattii \textbf{soorettii}, \textbf{galmee}fi \textbf{abbaashee} jedhanrratti fufamaniiru. Antimni kallattii garuu fufii malee dhufee jira. Akkuma beekamu gochimni hima keessatti fufii waliigaltee waan ida’achuu qabuuf fufileen waliigaltee \{-an\}, \{-e\} fi \{-te\}n gochimoota irratti fufilee boodaa ta’anii ida’amuun ramaddii, lakoofsa, korniyaafi yeroo gochi raawwatame agarsiisaniiru.

Karaa gara biroo himni ce’an antima malee mul’atu jira. Kana jechuun antimni keessa beekkanootiin waan jiruuf osoo hindubbatamiin ykn hinbarreessamin argama. Keessumattuu gochimni waa’ee nyaataafi dhugaati odeessu antima malee argamu danda’a. Fakkeenyaaf himoota armaan gadii haa ilaallu:\\
\\
a. Isaan dhug-an; \\
b. Isheen nyaat-te.\\

Himootni kunneen ce’odha; garuu matimaafi gochima malee antima hinqabani. Antimni isaanii keessa beekkanootiin waan hubatamuuf bira darbameera. Kana malees akkuma hima hafoo himni ce’aa matima malee dubbatamuu danda’a. Akkuma matima hafoo matimni ce’as fufilee waliigalteetiin beekamu.\\
\\
a. Hoolaa bit-te; \\
b. Qarshii ishee-f keenn-e; \\
c. Nyaat-te.\\

Himoonni armaan olitti eeraman himootuma duraan ilaalle. Himoonni kunneen fakkeenyota duraan ilaallerraa adda kan ta’an waan matima hinqabaanneef qofa. Fufiin waliigaltee hamma gochimarratti argamtetti matimni hima keessaa hafuu danda’a. Eenyu gocha akka raawwate ilaalchisee odeeffannoo
ga’aa fufii waliigalteerraa argachuu dandeenya.

Karaa hiikaa hima ce’aa  hirmaattotaa irratti hundoofnee qaaccessuu dandeenya. Himni ce’aan inni bal’inaan beekamu hirmaattota lama of keessaa qaba. Hirmaattota kanneen matimafi antima jenneerra. Egaa waan beekamuu qabu matima ykn antima jechuun hima keessatti maqaa isa kamtu dura dhufe isa
kamtummoo ittii anee dhufe jennee tartiiba jechootaa irratti bu’uureffaneeti malee hiika irratti bu’uureffannee miti. Karaa jecha gara biroo maqaan tokko hiika isaa osoo hinilaaliin bakka
matimaa galee fufii matimaa maxxanfateera yoo ta’e matima jenna. Maqaan tokkommoo bakka antimaa galeera yoo ta’e antima jenna hiika isaa osoo hinilaaliin. Caasaadhuma isaa ilaallee maqaa tokko antima kallatti ykn antima alkallattii jenna. Karaa hiikaa yoo ilaalle garuu matimni hima ce’aa tokko
\textbf{abgocha} ykn \textbf{muuxataa} ta’uu danda’a.\\
\\
a. Namich-i muka mur-e;\\
b. Siddisee-n abaaboo fuunfat-te. \\

Himni (a) irratti mul’ate matima \textbf{abgocha} qaba; sababni isaas matimni ta’e jedhe, itti yaadee, meeshaa barbaadee waan gocha raawwateef. Himni (b) irratti agarsiifamemmoo \textbf{muuxataa}dha; \textbf{Siddiseen} muuxannoo jireenyaa waan argachaa jirtuuf. Antima hima ce’aas hiikaan qaaccessuu dandeenya. Yeroo baay’ee antimni kallattii isa gocha tokkoon \textbf{miidhame} agarsiisa. Fakkeenyaaf himni (a) irratti maqaan \textbf{muka} jedhu gochaan miidhameera. Antimni alkallattiammoo \textbf{galma} gocha tokkoo agarsiisa. \\
\\
Hima: Siddiseen qarshii hintala ishee-f kenn-ite.\\
\\
Karaa hiikaa matimni hima armaan olii abgocha. Antimni \textbf{qarshii} jedhu qaama gocha tokkoon miidhame waan ta’eef aanitma kallattii jenna. Antimni alkallattii hiikan galma gocha tokkoo agarsiisa; hima kana keessatti galmi ykn qaamni qarshiin itti kennamu maqaa \textbf{hintala ishee }jedhu yoo ta’u fufii boodaa \{-f\} maxxanfata. 

Dimshaashumatti, himni ce’aan yoo xinnaate, matima tokko, antima tokkofi gochima ce’a tokko qabaachuu qaba. Haa ta’u malee himni ce’aan antimoota lama qabus jira; antimni kallattiifi antima alkallattii. Hima ce’aa keessaa matimni hafuu danda’a. Himoota nyaataafi dhugaatii ibsan keessammoo antimni hafuu danda’a. Karaa hiikaa, matimni abgocha ykn muuxataa ta’uu danda’a. Antimni kallatti qaama gochaan miidhame yoo ta’u antimni alkallattiammoo galma gocha tokkoo agarsiisa.

\section{Hiikcaasima Hima Garlamee/Waliyyoo}

Himni garlamee \index{hiikcaasima hima garlamee/waliyyoo} gocha qaamni lama ykn lama ol waliif deebisan agarsiisa. Gochi raawwatamus ce’aadha. Kutaa kana keessatti caasimaafi hiika hima garlamee ilaalla. 

Karaa caasimaa yoo ilaallu, himni garlamee matima tokkoof antima tokko qaba. Akkasumas iochimibsaa \textbf{wal}- jedhu gochima dursee dhufuu qaba. Fakkeenyaaf himoota armaan gadii haa hubannu: \\
\\
a. Isaan wal dhungat-an; \\
b. Nam-oot-ni wal ilaal-an.\\
c. *Nam-ni wal ilaal-ani (hinjedhamu). \\

Himootni (a) fi (b) irratti eeraman kunneen akaakuu hima garlamee keessatti ramadamu. Matimoonni himoota kanneeni heddummina agarsiisa. (a) irratti matimni \textbf{isaan} jedhu namootni lakkoofsaan tokkoo ol ta’an gocha irratti hirmaachuu isaanii agarsiisa. (b) irrattis matimni maqaa \textbf{namoota} jedhu yoo ta’u lakkoofsaan namootni lamaa ol ta’an gocha irratti hirmaachuu isaanii agarsiisa. Himni (c) irratti eerame garuu hima garlamee miti; himni kunis dogongora. Kana jechuun himni
(c) irratti eerame kun AO keessatti fudhatama hinqabu. Sababni fudhatama dhabuu (c) matimni isaa waan qeentee ta’ef; maqaan lakkoofsa qeentee agrsiisu hima garlameef matima ta’uu hindanda’u. Akkuma duraan ibsametti himni garlamee matima hedduu (maqaa lakkoofsa lama ykn lamaa ol agarsiisu) qaabaachuu qaba. Akkasumas bamaqaan\textbf{ wal-} jedhu gochima dursee dhufuu mala. Kana malees fufiin waliigaltee
kan dhuma gochimaatti maxxanfamu lakkoofsa heddumminaa agarsiisee matima agarsiisuu qaba. Kanaaf fufiin waliigaltee \{-an\} jedhu matima agarsiisa akka (a) fi (b) irratti eerame kana.

Himni garlamee hiika adda addaa qabaachuu danda’a. Hiikota kanneen akka armaan gadiitti ilaalla.

\begin{itemize}
	\item Hiika gocha waliif deebisuu. Fkn, \\
	\\
	a. Isaan wal reeb-an; \\
	b. Ijoolee-n wal dhungat-te.\\
	c. Dargaggoo-n wal dorgom-an.\\
\end{itemize}
(a) irratti akka ilaalamu kana namoonni lakkoofsaan lama ykn lamaa ol ta’an miidhaa wal fakkaatu ykn wal gitu waliif deebisuu isaanii agarsiisa. (b) irrattimoo ijoolleen lakkoofsaan lama ykn lamaa ol taate mallattoo jaalalaa kan ta’e dhunguu waliif deebsuu ishee agarsiisa. (c) irrattis dargaggoonni
lakkoofsaan lama ykn lamaa ol ta’an morkii wal fakkaatu waliif deebisuu isaanii agarsiisa. 
\begin{itemize}
	\item Hiika gochaan waldeeggaruu agarsiisu. Fkn,\\
	\\
	a. Qote bultoo-nni yeroo mara waliin hojjet-u;\\
	b. Namoot-ni bay’een wal gargar-u. \\
\end{itemize}
(a) irratti qotee bultoonni lakkoofsaan lamaa ol ta’an hojjaa wal gargaaraan hojjechuu isaanii agarsiisa. Kana jechuun dandeettiiwwan adda addaa walitti fidani waan kaayyoo tokko
galmaan ga’uuf akka hojjetan agarsiisa malee gocha walii ilaalanii waliif deebisuu jechuu miti. Himni (b) irratti tuqames hiikaan namni tokko isa birootiif gargaarsa gochuu agarsiisa;
namni gargaarsa godhe dirqiidhaan nama duraan gargaaramee hinkoo deebisu ta’uu dhiisuu danda’a. 
\begin{itemize}
	\item Hiika miiltoo waliif ta’uu. Fkn, "Dubartoo-nni waliin deem-an". Hiikni hima kanaa miiltoo wallif ta’uu agarsiisa malee gocha waliif deebisuu agarsiisa miti.
	\item Hiika tartiiba agarsiisu. Fkn, "Barattoo-nni wal hordof-an". Himni kun toora galanii adeemuu agarsiisa; kan dura deemu jira, kan itti aanu jira,… kan booda deemus jira jechuudha.
	\item Hiika aantee agarsiisu. Fkn, "Isaan walitti aanan". Himni armaan ol irratti eerame namni eenyutti akka aanee dhufu agarsiisa. 
	\item Hiika fageenyaafi dhiyeenyaa agarsiisu. Fkn, "Isaan wal irraa faagatani malee walitti hindhiyaatanne." Himoonni kunneen hiikaan fageenyaafi dhiyeenya agarsiisu. 
	\item Hiika tasummaa agarsiisu. Fkn, "Isaan walitti bahani." Himni garlamee armaan ol irratti eerame ammoo hiikan waan tasa mul’ate tokko agarsiisa. Kana jechuun qooddattoonni hima kana keessatti eeramani osoo itti hinyaadiini wal arguu ykn wal quunnamuu isaanii agarsiisa. 
\end{itemize}
	
Walumaa galatti himni garlamee hiikota adda adda 	qaba. Himni garlamee matima tokko qaba, matimni kunis 	maqaa heddummina agarsiisudha. Akaakuun hima kanaa 	gochima tokko qofaas qaba. Gochimni kun ce’aa ykn hafoo 	ta’uu danda’a. Himni garlamee matimaafi antima qofaa osoo
hintaane bamaqaa akka antimatti fayyadus qaba. Bamaqaan kun of danda’ee dhaabbatee ykn maxxantoota maxxanfatee argamuu danda’a. Hima garlameef hiikni bu’uura ta’e gocha waliif deebisu dha. 

\section{Hiikcaasima Hima Raawwatamaa}
Kutaa kana keessatti hima raawwatama \index{hiikcaasima hima raawwatamaa} ilaalla. Himni raawwatamaan akaakuuwwan gurguddaa lamatti qoodamee
ilaalama; isaanis raawwatamaa dhuunfaa\footnote{Raawwatamaa dhuunfaan ogbarruu xiinqooqaa hedduu keessatti "personal passive" jedhama.}fi mitidhuunfaadha\footnote{Raawwatamma mitidhuunfaan "impersonal passive" jedhamee beekama.}. Himini raawwatamaa dhuunfaa isa hima ce’aa
irraa ijaaramu. 
	\subsection{Hiikcaasima Raawwatamaa Dhuunfaa}
Karaa caasimaa raawwatamaan dhuunfaa \index{hiikcaasima raawwatamaa dhuunfaa} matima gochi irratti raawwatame qaba. \\
\\
a. Biiftuu-n hoolaa bit-te;\\
b. Hoolaa-n Burqituu-tiin bit-am-te. \\

(a)n hima ce’aa agarsiisa; (b)n ammoo hima raawwatamaa. (a) irratti matimni hima ce’aa \textbf{Burqituu}dha; matimni kun hima raawwatamaa (b) irratti akka antima alkallattii ta’ee
durduubee \{-tiin\} jettu fufatee dhufeera. Maqaan \textbf{hoolaa} jedhu (a) irratti antimadha. (b) irrattammoo matimaa ta’ee \{-n\} fufateera. Kana malees hima ce’aa (a) irratti gochimni \textbf{bit-te}
jedhu (b) irratti fufii raawwatamaa \{-am-\} ida’ateera. Walumaa galatti hima raawwatamaa keessatti matimni hima ce’aa antima, antimni ammoo matima ta’a.

Garuu qaamni gocha raawwate hima keessaa hafuu danda’a.\\
\\
a. Burqituu-n hoolaa bit-te;\\
b. Hoolaa-n (Burqituu-tiin) bit-am-te;\\
c. Hoolaa-n bit-am-te.\\

(b) irratti matimni haala antima alkallttiitiin dhufeeru (Burqituu-tiin) kan jedhu (c) keessa hinjiru; hima raawwatamaa keessaa antimni alkallattii hafeera. Hima raawwatamaa keessatti fufiin \{-tiin\} antima alkallattii agarsiisti. Hima raawwatamaa tokko keessatti gochamni alkallattiifi meeshaan
yoo jirataan himni fudhatama hinqabne ykn hiikni isaa kan ifa hintaane uumamuu danda’a.\\
\\
a. Gurbaa-n waraabessa eeboo-tiin waran-e;\\
b. ?Waraabessi gurbaa-tiin eeboo-tiin waraan-am-e;\\

a)n hima ce’aa agarsiisa (b)n ammoo hima rawwatamaa kan fudhatama hinqabaatiin agarsiisa. (b)keessatti antimni alkallattii (gurbaa-tiin)fi meeshaan (eeboo-tiin) walitti aananii waan dhufaniif himni fudhatama ga’aa argachuu hindandeenye. Himni akkasii kan sirraa’uu danda’u qaamni –tti fufateeru
tokko hima keessaa yeroo hafu. \\
\\
a. Waraabessi eeboo-tiin waraan-am-e.\\
b. Waraabessi gurbaa-tiin waraan-am-e.\\
\\
Himoonni (a) fi (b) irratti eeraman lamaanuu fudhatama argataniiru; sababni isaas maqaan fufii –tiin fufatee hima keessatti mul’ate tokko qafaa waan ta’eef. Haala kanaan (a) irratti antimni alkallattii (gurbaa-tiin) hafeera; (b) irrattammoo meeshaan (eeboo-tiin) hafeera. Durduubeewwan adda addaa faayidaarra oolaniiru yoo ta’e himni fudhatama qaba. \\
\\
a. Gurbaa-n hoolaa Tolassaa-f bite.\\
b. Hoolaa-n gurbaa-tiin Tolasaa-f bit-am-e. \\

Akka fakkeenya (b) irratti hubannu kana gaaleen gurbaa-tiin jedhufi \textbf{Tolasaa-f} jedhu walitti aananii hima tokko keessa galaniiru; himni kunis dogongora hin qabu. Fufiin \{-tiin\} jedhu meeshaa yoo agarsiisu kan \{-f\} jedhummoo kallattii agarsiisa; lamaanuu tajaajilaan adda adda.

AO keessatti hima raawwatamaa keessatti maqaan akka meeshaatti tajaajilu akka matimaattis tajaajiluu danda’a.\\
\\
a. Gurbaa-n burcuqoo-tiin farsoo dhug-e.\\
b. Burcuqoo-n ittiin-dhug-am-e.\\
c. Burcuqoo-wwan ittiin-dhug-am-an. \\

Gaaleen \textbf{burcuqoo-tiin} jedhu (a) irratti gaalee durduubee ture. Hima (b) irratti agarsiifamerratti garuu maqaan \textbf{burcuqoo-n} jedhu matima. Sababni kanas waan jalqaba himaa irratti dhufeef fufii maayii matimaa agarsiistu \{-n\} waan dabalateef. Akkasumas fufiin waliigalaa \{-e\} kan gochima irratti ida’amte lakkoofsa qeentee waan agarsiiftuuf \textbf{burcuqoon} matima. Yoo
matima \textbf{burcuqoo} jedhu baa’ifnee \textbf{burcuqoo-wwan} jenne fufiin waliigalaa gochima irratti ida’amtus heddummina agarsiifti. Qabxiin kunis (c) irratti mirkanaa’eera. Yeroo guuttuun gaalee
durduubee, jechuun \textbf{burcuqoo-n} matima ta’u matimni hima raawwatamaa inni dhugaan jechuun \textbf{farsoon-n} hima keessaa citee hafeera. Garuu matimni raawwatamaa inni dhugaan (antimni) hima keessatti mul’achuus danda’a. \\
\\
a. Gurbaa-n shinii-tiin haraqee dhug-e.\\
b. Haraqee-n shinii-tiin dhug-am-e.\\
c. Shinii-n ittiin haraqee-n dhug-am-e.\\

(a)n hima ce’aadha; sababni isaas qaamni gocha raawwate jechuun \textbf{gurbaa-n} matima yoo ta’u qaamn gochi irratti raawwatame jechuun antimni \textbf{haraqee}dha. (b)n hima raawwatamaadha. Hima kana keessatti antimni bakka matimaa galeera; gochimni fufii dhalatoo \{-am\} ida’ateera. (c)n hima
raawwatamaa haa ta’u malee amalli isaa adda; sababn isaas 

matima lama waan of keessaa qabuuf. Matimni \textbf{shinii-n} jedhu matima sobaati, shiniin antima waan hin taaneef. Matimni \textbf{haraqee-n} jedhu garuu matima hima raawwatamaa isa sirriidha, waan haraqeen antima ta’eef. AO keessatti himni tokko matima lama qabaachuun darbee darbee nama
quunnama; seeraan garuu himni tokko matima lama qabaachuu hinmalu (Chomsky, 1982). Fufiin waliigaltee gochima irratti ida’ameeru matima sobaa waliin waliigala yoo matimni dhugaa (antimni) citee hafeera ta’e; yoo matimni dhugaan hima keessatti mul’ateera ta’e garuu waliigalteen matima dhugaa waliin wallif gala; matimni dhugaan waliigaltee abbooma. \\
\\
a. Burcuqoo-wwan ittiin dhug-am-an.\\
b. Burcuqoo-wwan ittiin haraqee-n dhug-am-e.\\
\\
(a) irratti fufiin waliigaltee \{-an\} jedhu inni gochima irratti ida’ameeru matima sobaa (antima osoo hin ta’iin meeshaa) tiin abboomameera. (b) irratti garuu fufiin waliigaltee \{-en\} matima
raawwatamaa dhugaatiin abbomaameera. 

AO keessatti himni ce’aan tokko antimoota lama jechuun antima kallattiif antima alkallattii qaba yoo ta’e, hima raawwatamaa keessatti antima kallattii qofaatu matima ta’a. Antimni alkallatti matima yoo ta’e himni raawwatamaa sun fudhatama hinargatu. Fakkeenyaaf himoota armaan gadii haa ilaalluu:\\
\\
a. Lalisee-n kitaaba Lalisaa-f kenn-e.\\
b. Kitaab-ni Lalisaa-f kenn-am-e.\\
c. *Lalisaa-n kitaaba kenn-am-e.\\

Akkuma fakkeenyota (a-c) irratti agarsiifame AO keessatti antima kallatti qofaatu matima ta’uu danda’a hima raawwatamaa keessatti, kunis fakkeenya (b) irratti agarsiifameera; yoo antimni alkallatti teessoo matimaa qabate, himni fudhatama hin qabu, kunis (c) irratti agarsiifameera. 

\subsection{Hiikcaasima Raawwatamaa Mit-Dhuunfaa \index{raawwatamaa mit-dhuunfaa}}
Himni raawwatamaa miti-dhuunfaa\footnote{Raawwatamaa mit-dhuunfaa ogbarruu keessatti 'impersonal passive' jedhama.} gochima hafoo irraa ijaarama. Fakkeeny raawwatamaa miti-dhuunfaa armaan gadii
sosso’ina qaamaa agarsiisu: \\
\\
a. Ijoolleen bakka kanatti fiigde.\\
b. Bakka kanatti fiigame.\\
c. Dubartiin mukarra ni utaalti.\\
d. Mukkarra ni utaalama. \\

Hima (a) irratti maqaan \textbf{Ijoolleen} jedhu matima yoo ta’u jechi \textbf{fiigde} jedhummoo gochimaa hafoodha. Gaaleen \textbf{bakka kanatti} jedhu gochimibsa bakka agarsiisu. Haaluma kanaan (c) yoo ilaalle \textbf{Dubartiin} matima, gochimni \textbf{utaalti} jedhu hafoo yoo ta’u gaaleen \textbf{mukarra} jedhummoo gochimibsa. (b) fi (d) n raawwatamaa hafoo (a) fi (c) irraa ijaaraman. (b) fi (d) keessa matimni hinjiru; akkasumas fufiin raawwatamaa \{-am-\} jedhu gochima irratti fufameera. Kana caalaa, fufileen waliigaltee kaneen gochima \textbf{fiigame} fi \textbf{utaalama} jedhan irra jiran jechuun \{-e\} fi \{-a\} korniyaa kormaa lakkoofsa qeentee yeroo dabraa agarsiisu.

Akkuma fakkeenyota armaan olitti ilaaman irratti hubachuu dandeenyutti gochimibsaa bakka agarsiisuu jechuun kan akka \textbf{bakka kana} fi \textbf{mukarra} jedhan raawwatamaa miti-dhuunfaa
keessatti fayyadu; haa ta’u malee kun dirqama miti; gochimibsa bakka agarsiisu hafuu danda’a.

Raawwatamtoonni miti-dhuunfaa armaan gadimmoogochimoota sagalee maddisiisuu irratti bu’uureffattu:\\
\\
a. Gabaan ni \textbf{wacti} (wacci);\\
b. Ni \textbf{wacama}.\\
c. Namni \textbf{iyye}; \\
d. Fagootii \textbf{iyyame}.\\

Walumaagalatti, hima raawwatamaa keessatti matimni antima ta’a; antimni ammoo matima ta’a. Yeroo kana fufiin \{-am\} gochima irratti ida’amti. Raawwatamaa dhuunfaan hima ce’aarraa uumama; raawwatamaa miti-dhuunfaan garuu gochima hafoo irraa uumamu. 

\section{Hiikcaasima Hima Giddugalaa \index{hima giddugalaa}}

Himni giddugalaan hima ce’aafi hima raawwatamaa giddu jira \cite{kemmer1993middle,tolemariam2009}. Kanaaf himni giddugalaan amala hima ce’aafi amala hima raawwatamaa qaba. Akka hima ce’aa matmni isaa gocha tokko raawwata; akka hima raawwatamaammoo gocha raawwatame sanaan matmni miidhama. Kana jechuun hima giddugalaa keessatti hirmaattonni lamaanuu jechuun matimniifi antimni tokko jechuudha. Kana jechuun qaamni miidhuufi qaamni miidhamu tokko. Fakkeenyaaf hima \textbf{gurbaan harka} \textbf{dhiqate} jedhu yoo ilaalle matmni (gurbaan)fi antimni (harka) tokko; maaliif yoo jenne harki qaama gurbaa waan ta’eef.
Matimniifi antimni tokko ta’uun beekamtii hima giddugalaati. 

Haa ta’u malee amalli gara biroon hinmul’atu jechuu miti. Fakkeenyaaf himni \textbf{gurbaan mufate} jedhu giddugalaadha. Hima kana keessa qaamni gurbaa mufachiise gurbaa hinfakkaatu. Himni \textbf{gurbaan guddate }jedhus hima giddugalaati. As keessattis waan gurbaan akka guddatu godhe
qaama gara biraa fakkaata. Fufileen \{-at/-aat, -om/oom, -ah/aah\} gochima giddugalaa agarsiisu.

Karaa caasimaa himni giddugalaa akaakuu afur qaba. Inni
tokko, lakkoofsa hirmaattota hima keessa jiranii hindabalus,
hinhirri’sus. Fakkeenyaf hima armaan gadii haa ilaallu: \\
\\
a. Galatoo-n harka dhiq-at-e.\\
b. Galatoo-n harkee dhiq-e.\\

(a) irratti akka ilaalamutti maqaan \textbf{Galatoo-n} jedhu matima yoo ta’u maqaan \textbf{harka} jedhummoo antima. Hima kana keessatti mallattoo hima giddugalaa kan taate fufiin \{-at-\} gochima irra
jirti. (b)n hima ce’aadha. Haa ta’u malee lakkoofsi qooddattootaa lamuma akka hima giddugalaa (a) irratti ilaallee. Kana jechuun maxxanuun \{-at-\} ida’amuuniif dhiisuun lakkoofsa
qooddattootaa irratti dhiibaa hingeessisne jechuudha.

Akaakuun caasimaa giddugalaa inni lammaffaan isa lakkoofsa qoddattootaa hir’isu. Kana jechuun wayita fufiin \{-at-\} gochima irratti ida’amu matimni ni hir’ata akkuma hima raawwatamaa
jechuudha. Gosti hima giddugalaa kanaa heddumminaan AO keessatti hinmul’atu. Fakkeenyaaf hima armaan gadii haa ilaallu: \\
\\
a. Nam-ni mana gub-e; \\
b. Man-ni gub-at-e.\\

Himni (2a) irratti mul’ate kun hima ce’aa jedhama. Himni (2b) irratti agarsiifame kunammoo hima giddugalaa jedhama. Himni (2a) irratti agarsiifame qooddattoota lama qaba; isaanis
matimaaf antima. Himni (2b) irratti agarsiifame garuu qooddataa tokko qofaa qaba, qooddataan kunis matima qofaadha. Sababni kanaas fufiin giddugalaa \{-at-\} waan gochima irratti ida’amteef jechuudha. As irratti fufiin \{-at-\} akkuma fufii raawwatamaa, \{-am-\} ta’ee tajaajile jechuudha (fufiin
raawwatamaa lakkoofsa qooddattootaa hir’isuutiin akka beekamtu haa yaadannu).

Akaakuun caasima hima giddugalaa inni sadaffaan isa lakkoofsaa qooddattootaa ida’u. Akaakuun kun AO keessa heddumminaan hinargamu. Fakkeenyaaf himoota armaan gadii haa ilaallu: \\
\\
a. Gurbaan iyy-e.\\
b. Gurbaa-n miidhaa-saa mootummaa-tti iyy-at-e.\\
\\
Himni (a) irratti agarsiifame hima hafoodha; gochimni fufii –athinida’anne. Himni (b) irratti agarsiifameeru garuu hima giddugalaadha. Gochimni hima kana keessatti argamu fufii
giddugalaa \{-at-\} ida’ateera. Sababa \{-at-\} ida’ameefis antimoonni miidhaafi mootummaa jedhan hima keessa seenaniiru. Kana jechuun fufii \{-at-\} lakkoofsa qooddattootaa ida’eera jechuudha.

Akaakuun caasimaa hima giddugalaa inni arfaffaan isa Maqibsa gara gochimaatti jijjiiree hima hafoo uumudha. Fakkeenyaaf ibsitoota maqaa kanneen akka diimaa, guddaa, furdaafi kkf irratti fufii \{-at-\} ida’uun hima hafoo uumuun nidanda’ama.\\
\\
a. Bunni diim-at-e.\\
b. Mucaan gudd-at-e.\\
c. Dubartii-n furd-at-te. \\
\\
Akka (a) irraa ilaallutti gochimni \textbf{diim-at-e} jedhu Maqibsa \textbf{diimaa} jedhu irraa uumame; gochimni kunis matima tokko qofaa fudhata. (b) irrattis gochimni \textbf{gud-at-e} jedhu Maqibsa \textbf{guddaa} jedhu irraa uumame. (c) irrattis gochimni \textbf{furd-at-e}
jedhu Maqibsa \textbf{guddaa} jedhu irraa uumame. Gochimoonni kunneen hundi matima tokko qofaa fudhatu; kanaaf himoota hafoodha.

Karaa hiikaa himni giddugalaa bakkeewwan gurguddoo afuritti qoodamu danda’a. Isaanis kanneen qaama giddugala godhatan, kanneen adeemsa yaadaa himan, kanneen gocha tasa raawwate odeessanfi kanneen gocha ofiif raawwatame mul’isani. Himni giddugalaa inni qaama irratti xiyyeeffatu gosa adda addaa
qaba. Gosoonni kunneenis kuunuunsa qaamaa, sochii dhaabaa qaamaa, osoo hinbeekiin qaama sochoosuufi bakka tokkoo bakka birootti sochii gochuudha.

AO keessatti himni giddugalaa hiikaan qaamaa kunuunsuu agarsiisuu hedduutu jira. Fakkeenyaaf hiika himoota armaan gadii haa ilaallu: \\
\\
a. Siddiiqee-n ilkaan tum-at-te.\\
b. Tobarraa-n qeensa qor-at-e.\\
c. Isaan miila dhiq-at-ani.\\
d. Bulloo-n gaabii uff-at-e.\\

Fakkeenyota armaan olii hunda yoo hubanne hiikni isaanii qaama kunuunsuu irratti xiyyeeffata. Kana malees himoonni giddugalaa kanneen biroon ammoo hiikni isaanii sochii dhaaba qaamaa agarsiisa. Fakkeenyaaf himoota armaan gadii haa ilaallu:\\
\\
a. Inni quxuuxx-at-e. \\
b. Isheen dhaab-at-te.\\
c. Isaan muka-tti hirk-at-ani.\\
d. Gurbaa-n diriirf-at-e.\\
e. Uumee-n harka ciqilf-at-te.\\

Himoonni armaan olitti agarsiifaman kunneen gosti isaani hima giddugalaa jedhamu; sababni isaas hundi isaaniiyyuu fufii giddugalaa \{-at-\} waan ofirraa qabaniif. Himoonni kunneen hiikni isaanii wal fakkaata; maaliif ? yoo jenne, waan sochii dhaaba qaamaa agarsiisaniif jechuu dandeenya.

Gosti hiikaa hima giddugalaa inni biroon, osoo hinbeekiin qaama sochoosuudha. Kana jechuun sochii qaamaa gosa kanaa irratti fedhii abbaan ala sochiin qaamaa jira; yookiin gosa sochii kanaa abboomuu hin danda’u. Fakkeenyaaf himoota armaan gadii haa xiinxallu:\\
\\
a. Garaa-n ishee gub-at-e.\\
b. Mataa-n isa bobaaf-at-e.\\
c. Ij-i isa liphis-at-e.\\
d. Qub-ni isa kottonf-at-e.\\
e. Qaam-ni ishee holl-at-e. \\
\\
Himni giddugalaa inni qaama irratti bu’uureffatu hiiknisaa sochii qaamaa bakkaa fi yeroo waliirraa siqillee agarsiisuu danda’a. Kanaafis himoonni armaan gadii akka fakkeenyaatti eeramuu danda’u: \\
\\
a. Ijoollee-n tulluu yaabb-at-te.\\
b. Dhagaa-n tulluu irraa gadi konkol-aat-e.\\
c. Gudeesh-i gangal-at-e.\\
\\

Fakkeenyotti (a-c) irratti eeraman kunneen qaamni tokko yeroo fudhatee bakka tokkoo ka’ee bakka biraa ga’uu isaa agarsiisu. Hiikni hima giddugalaa akaakuun biroo isa gocha sammuu keessatti raawwatu agarsiisu. Akaakuu kana bakkeewwan sadiitti qoodnee ilaaluu dandeenya. Isaanis, giddugala miira agarsiisu, giddugala adeemsa yaadaa agarsiisuufi giddugala beekumsa qeensuu agarsiisu jedhamu.

Himni giddugalaa hiikaan gocha sammuu keessatti raawwatu agarsiisan keessaa tokko isa miira agarsiisudha. Himoonni kanneen miira namaa agarsiisan hedduutu jiru. Himoonni tokko
tokko amantii waliin wal qabatani walitti dhufeenya namaafi ayyaanaa ykn waaqaa agarsiisu. Kanaafis himoonni armaan gadii fakkeenya ta’uu danda’u:\\
\\
a. Ayyaanni dabe abbaatti hamm-aat-a.\\
b. Rakkina keessaa ba’uuf waaqa kadh-at-e.\\
c. Nam-ni malkaati irreeff-at-a. \\
\\
Fakkeenyota armaan olii irraa akka hubanutti gochimoonni hammaata, kadhatefi irraanfate jedhan fufii giddugalaa of keessaa qabu; hiikni isaaniammoo amantii agarsiisaa. Kanneen malees gochimoonni armaan gadii miira waahummaa agarsiisu. Isaanis, rif-at-e, him-at-e, dhaad-at-e, kak-at-e, lag-at-e, goom-ate, muf-at-e fi kkf. Akkasumsa gochimoonni guddugalaa kanneen akka jaall-at-e, abd-at-e, sossob-at-e, haw-at-e fi taph-at-e jirani ammoo eentaa miiraa agarsiisu. Himoonni armaan gadii ammoo hiikaan adeemsa yaadaa agarsiisu: \\
\\
a. Giiloon bilbilasaa manatti irraanf-at-e.\\
b. Siifan waan irraanf-at-te amma yaad-at-te.\\
c. c. Ijoolleen waan haaraa qalbis-at-ti.\\
d. Gurbaan waan barate yaadatti qab-at-a.\\
e. Hintalli herreega bar-at-te.\\
\\
Himoonni armaan olii kunneen fufii -at- gochima irratti ida'ataniiru. Hundi isaaniiyyuu adeemsa yaadaa agarsiisu. Waan sammuu nama tokkoo keessatti ta'aa jiru mul'isu. 

Giddugala gocha sammuu keessatti raawwatu agarsiisan keessaa akaakuu biroon kanneen miirota shanan agarsiisan. Miironni shanan kunneen odeeffannoo naannoo irraa argachuuf nama gargaaru. Fakkeenyaaf, himoota armaan gadii haa ilaallu:\\
\\
a. Hintalli daraaraa fuunf-at-te.\\
b. Gurbaan nyaata dhadham-at-e.\\
c. Inni harka qaqqab-at-e.\\
d. Isheen walaloo dhaggeeff-at-te.\\
e. Dargaggeessi durba ilaall-at-a.\\
\\
Akka himoota armaan oliirraa hubachuu dandeenyutti, fufiin \{-at-\} gochimoota himootaarratti argama. Himoonni kunneen gidugalaa, hiikni isaanii odeeffannoo naannoo irraa argachuu irratti bu'uureffata; kanaaf walfakkaatu. 

Akaakuun hiika hima giddugalaa inni biroon isa gocha tasa raawwatame agarsiisu. Himni giddugalaa gocha tasa raawwatame agarsiisu AO keessatti beekamaadha. Fakkeenyaaf, himoota armaan gadii haa ilaallu:\\
\\
a. Mukni gudd-at-e.\\
b. Baalli babal'-at-e.\\
c. Saanii gabb-at-te.\\
d. Bunni diim-at-e.\\
\\
Himoonni gidugalaa armaan olii kun gocha ofiin tasa ta'e agarsiisu. 

Akaakuun hiikaa AO keessatti heddumminaan beekamu isa faayidaa mataa ofiif jecha gocha raawwatamu agarsiisu. AO keessatti fufiin giddugalaa \{-at-\} gochima ce'aa irratti ida'amee fayidaa ofii jecha gocha raawwachuu agarsiisa. Fakkeenyaaf himoota armaan gadii haa ilaallu:\\
\\
a. Namihci hoolaa gurgur-at-e.\\
b. Oggantuun sun konkolaataa bit-at-te.\\
c. Qotee bulaan saree guddis-at-e.\\
\\
Himoonni armaan olii kunneen himoota giddugalaa jedhamu. Hunda isaanii yoo ilaalle, fufiin giddugalaa \{-at-\} gochima ce'aa irratti ida'amteetti. Karaa hiikaa yoo ilaalle namni tokko faayidaa mataa isaatiif jecha gocha tokko raawwachuu agarsiisu.

\section{Hiikcaasima Hima Taasisuu}

Himni taasisuu \index{hima taasisuu} yoo xinnaate himoota lama of keessatti hammata (qabata). Himootni lamaan kunneen hima gocha raawwachiisee fi hima gocha raawwate jedhamu. Himoonni kunneen garuu karawwan lama agarsiifamu (barreeffamu ykn dubbatamu). Tokko, hima qaacesun barreeffamu; kana jechuun himni raawwachisee akka hima tokkootti, himni raawwates akka hima tokkootti adda adda ba’anii hima tokko keessatti hammatamanii barreeffamu. Lama, karaa gochima dhamjecha taasisuun, gochi raawwachisees gochi raawwates walitti qindaa’anii hima tokkoo ta’anii barreeffamu. 

AO keessatti maxxantoota gochima taasisuu uuman keessaa \{-is-\} bal’inaan beekamti. Fufiin kun firdhamjecha tokko qabdi. Isheenis \{-s-\} dha. Fufiin \{-is-/ -s-\} gochimoota hafoo tokko
tokkotti maxxananii taasisuu uumu. Fakkeenyaaf \textbf{raff-is-e} fi \textbf{dammaq-s-e} ilaaluu dandeenya. Akkasumas fufiin kun hundee Maqibsa yookiin maqaa irratti maxxntee gochima taasisuu uumuudhan beekamtuudha. Fakkeenyaaf gochimoota taasisuu \textbf{gudd-is-e} fi \textbf{furd-is-e} yoo ilaalle Maqibsa irraa uumamani. Gochima \textbf{dubb-is-e} jedhu yoo ilaallemmoo maqaa irraa uumame. 

Himni taasisuu karaa caasimaa bakkeewwan gurguddoo lamatti qoodamee ilaalama. Isaanis hima taasisuu addeessaa/ qaacessaa fi hima taasisuu makaa (gochimaafi fufii taasisuu walitti maku). 

\subsubsection{Taasisuu Qaaccessaa \index{taasisuu qaaccessaa}}

Gosni hima taasisuu inni qaacessu hima gocha tokko gaggeessisuufi hima gocha gaggeessu of keessatti hammata. Himni gosa kanaa bakkeewwan gurgoddoo lamatti qoodamee ilaalama. Gosni tokko hima taasisuu isa gochima godhe jedhutti fayyadamu yoo ta’u inni biroonimoo isa gochima taasise jedhutti fayyadamu. Duraan dursine mee hima taasisuu isa gochima godhe jedhutti fayyadamu fakkeenyaan haa ilaallu: \\
\\
a. Garbaan [akka barataan fiigu] godhe.\\
b. Garbaan [akka barataan muka muru] godhe. \\
\\
Himoonni (1a) fi (1b) keessatti mul’atan himoota lama lama of keessatti qabu; isaanis hima buu’uuraa fi hima hirkataadha.Himoota lamaan keessattuu himni bu’uuraa matima \textbf{Garbaa} jedhamefi gochima \textbf{godhe} jedhu qabu; matimni kunis waa gochuu isaa mul’isu. Himooni hirkatoommoo hammattuu
keessatti agarssifamaniiru. (1a) irratti himni hirkataa matima barataa jedhuu fi gochima fiigu jedhu of keessaa qaba; (1b) keessatti garuu matima barataa jedhu, antima muka jedhuufi gochima ce’aa muru jedhu of keessa qaba. Himni hirkaataan kun matimni hima kanaa akka gocha tokko raawwatu taasifamuu isaa mul’isa. Mee ammammoo hima taasisuu isa gochima taasisuutti fayyadamu haa ilaallu. Akka mariif toluttis himoota armaan gadii haa xinxallu: \\
\\
a. Eebbaan [akka mucaan rafu] taasise; \\
b. Eebbaan [akka gurbaan hoolaa bitu] taasise.\\
\\
Himoota (a) fi (b) yoo ilaallu gochima gosa tokko qabu; innis isa taasise jedhu . Himootni lamaanuu matima walfakkaatu, \textbf{Eebbaa}, fi gochima gosa tokko qabu jechuuda. Akkasumas himni (a) irratti agarsiifame hima hafoo ta’e akka hima hirkataatti yoo qabaatu himni (b) irratti agarsiifamemmoo
hima ce’aa akka hima hirkaataatti qaba. Hima (a) irratti kan gocha raawwate mucaa yoo ta’u kan raawwachiseemmoo \textbf{Eebbaadha}; hima (b) keessatti kan gocha kalattiitiin raawwate gurbaa yoo ta’u kan gocha raawwachise \textbf{Eebbaa}dha. Haa ta’u malee bakka gochima taasisuu kana gochima godhe jedhu yoo bakka buufnee fayyadamne himni taasisuu fudhatama qaba. Fakkeenyaaf, \\
\\
a. Eebbaan [akka mucaan rafu] godhe; \\
b. Eebbaan [akka gurbaan hoolaa bitu] godhe.\\
\\
Dimshaashumatti himni taasisuu inni qaacessuun barreeffamu xumurtoota adda addaa lamaan ibsama. Xumurtoonni kunneenis godhe fi taasise dha. Haa ta’u malee xumurtoonni lamaanuu bakka wal jijjiiruun tajaajilarra ooluu danda’u jijjiirraa hiikaa osoo hinfidiin. Akkuma waliigalaatti, AO keessa
himni taasisuu qaacceessan gosa lamatu jira. Inni tokko isa gochima \textbf{godhe} jedhu irratti hundaa’u yoo ta’u inni lammaffaammo isa gochima \textbf{taasise} jedhu irratti hundaa’u.
Gosa hima taasisuu lamaanuu keessatti himoota lama lamatu jiru. Himoonni lamaanuu matimaa mataa isaanii qabu. Himni inni dura dhufu isa gocha raawwachise yoo ta’u inni itti aanee dhufummoo isa gocha raawwatedha. Gochimni \textbf{taasise} gochima tahe jedhurraa dhufe. Hundeen \textbf{tah}- (yookiin \textbf{ta’}-) jedhu wayita fufii taasisuu \{-sis-\} jedhu ida’atu hundeen xumurraa \textbf{taasis}- jedhu uumama (sagaleen [h] ykn hudhaan akka haqamufi sagaleen [a] akka dheeratu hubadhaa). 

\subsubsection{Taasisuu Makaa}
Taasisuu makaa \index{taasisuu makaa} bakkeewwan afurittti qoqqoodnee xiinxaluu dandeenya. Isaansi taasisuu qeentee, taasisuu lammeessoo/sadeessoo, taasisuu matima dhokataafi taasisuu hafoodha. 

\paragraph{Taasisuu Qeentee}

Tasisuu qeenteen \index{taasisuu qeentee} gochima hafoo ykn gochima ce’aa irraa uumamuu dandanda’a. Kana jechuun fufiin taasisuu tokko qofaan wayita gochima hafoo ykn gochima ce’aa irratti maxxantee gochima taasisuu ijaartu mula’ata. Fakkeenyaf himoota armaan gadii haa hubannu:\\
\\
a. gurbaa-n fiig-e.\\
b. Nam-ičč-i gurbaa fiig-s-e.\\
\\
Himni (a) irratti agarsiifame kun hima hafoo jedhama. Himni kun matima tokkoofi gochima tokko qaba. Fufiin \{-n\} matima irratti maxxantee maqaan gurbaa jedhu matima ta’uu isaa agarsiifiti. 

Fufiin \{-e\} jettummoo gochima irratti maxxantee matimni gocha gaggeesse Korniyaa kormaa, ramaddii sadaffaa, qeentee ta’uu isaa mul’isuun matimichi gurbaa akka ta’e mirkaneessiti. Himni
(b) irratti mul’atu hima (a) irraa adda wal fakkeenyas qaba. Himni (a) hima hafoo yoo ta’u himni (b)n ammoo hima ce’aadhas hima taasisaadhas. Hima ce’aa kan jennu gochi matima irraa gara hojetamaatti waan ce’eruufi. Hima taasisaa kan jennu ammoo fufiin taasisuu \{-s-\}n gochima fiig- jedhu irratti
ida’amtee gochima taasisuu waan uumteettufidha. Haa ta’u malee himoonni kunneen lamaan walis fakkaatu. Himoota lamaanu keessa maqaan gurbaa jedhuu fi xumurri fiig- jiru. Garuu maqaan gurbaa jedhu hima (a) keessatti matima ture (b) keessatti hojjetamaa ta’eera. Hima (a) keessa kan hin jiraanne
matimni namichi jedhu (b) keessatti ida’ameera; akkasumas (a) keessatti kan hin mul’atiin fufiin taasisuu \{-s-\} (b) keessa gochima irratti maxxaneera. Egaa dhimshaashumatti gochima
hafoo irratti fufiin taasisuu kamiyyuu yoo ida’ame matimni hima hafoo gara antimaatti jijjirama; wayita kanas matimni gara biroon ni ida’ama akkuma (b) irratti agarsiifamee jiru kana. 

Himni taasisuu gochima ce’aa irraa wayita uumamus kan duraan matima ture antima ta’a, matimni biroon ni ida’ama; akkasumas gochima ce’aa irratti fufiin taasisuu ni ida’ama. Yaada kana ifa gochuuf fakkeenyota armaan gadii haa hubannu: \\
\\
a. gurbaa-n hoolaa gurgur-e.\\
b. Namich-i gurbaa holaa gurgur-siis-e.\\
\\
(a) n hima ce’aa jedhama; (b)n ammoo hima taasisuu jedhama. Hima ce’aa (a) keessatti maqaan gurbaa jedhu matima; garuu hima taasisuu (b) keessatti antima; akkasumas hima ce’a keessa kan hin jiraatiin maqaan \textbf{namichi} jedhu hima taasisuu keessa galeera. Jijjiramoonni caasaa kunneen sababa fufiin
taasisuu \{-siis-\} gochima \textbf{gurgur}- jedhu irratti ida’amteefi.

\paragraph{Taasisuu Lammeessoofi Sadeessoo \index{taasisuu lammeessoofi sadeessoo}}
Taasisuun lammeessoon maxxantoota taasisuu lama irratti hundaa’u. Fakkeenyyaf, \\
\\
a. nam-ich-i gurbaa fiig-s-e.\\
b. Dubartii-n nam-ich-aan gurbaa fiig-s-is-te.\\
\\
(b)n caasaa taasisuu lammeessotii; sababni isaas fufii taasisuu lama jechuun \{-s-, -is-\} fi hundee gochimaa irratti waan ida’ameef. Taasisuu qeenteen gara taasisuu lammeessotti wayita ce’u jijjiiramni caasaas ni mul’ata. Matimni taasisuu qeentee taasisuu lammeessoo keessatti gara antimaatti
jijjiiramee maayii meeshaa ida’atee ibsama. Haaluma kanaan (a) keessatti maqaan namichi jedhu matima ture; garuu (b) keessatti gara antimaatti jijjiiramee fufii maayii meeshaa agarsiistu ida’achuun namich-aan jedhame ibsameera. Matimni dubartii jedhummoo (b) keessatti ida’ameera.
Jijjiiraan caasaa kun kan sababa godhate fufiin taasisuu –isjettu fufii taasisuu duran turte \{-s-\} irratti ida’amuusheeti.

Caasaan taasisuu lammeessoo gochima ce’aarrattis hundaa’ee uumamuu ni danda’a akkasuma fakkeenya armaan gadiirraa arginutti: \\
\\
a. Namich-i gurbaa hoolaa gurgur-siis-e.\\
b. Dubartii-n namich-aan gurbaa hoolaa gurgursis-iis-te.\\
\\
Fakkeenyonni (a) fi (b)n gochima ce’aa gurgur- jedhu irratti bu’uureffatu. Akkuma (a) fi (b) irratti ilaaletti matimni namichi jedhu taasisuu lammeessoo keessatti gara antimaatti jijjiirame
namichaan jedhameera; matimni dubartii jedhummoo (b) keessatti ida’ameera; sababnisaas fufiin taasisuu \{-iis\} jettu fufii taasisuu duraan turte \{-sis-\} jettu irratti maxxantee waan taasisuu lammeessoo uumteef. AO taasisuu sadeessoollee qabaachu ni danda’a. Kana jechuun maxxantoonni taasisuu sadii walitti aananii dhufuun hima keessatti mul’achuu danda’u. Fakkeenyaaf himoota armaan gadii haa ilaallu: \\
\\
a. Xuwwee-n cab-e.\\
b. Gurbaa-n xuwwee cab-s-e.\\
c. Dubartii-n gurbaa xuwwee cab-s-iis-te.\\
d. Namich-i dubartii-tiin gurbaa xuwwee cab-sis-iis-e. \\
\\
Hima (a) irratti himni argamu matima tokkoofi gochima tokko qofaa qaba; xumurri kun fufii taasisuu hin qabu. Himni (b) irratti mul’atummoo matima tokko, antima tokkoo fi gochima tokko qaba; inni antima ta’e (a) irratti matima ture. Xumurri (b) irratti mul’atu fufii taasisuu tokko waan qabuuf taasisuu
qeentee jechuu dandeenya. (c)n taasisuu lammeessooti, xumurri waan maxxantoota taasisuu lama maxxanfateeruuf. Fufii taasisuu tokkotu ida’ame ykn lamatu ida’ame jechuuf matima fi antima hima sana keessa jiran waliin wal biraqabnee hubachuu qabna. Ida’amuun fufii taasisuu lammaffaa matima
duraan ture gara antimaatti jijjiiree matima gara biroommoo dabala; (c) irrattis kan mul’ate kanuma. (d)n ammoo taasisuu sadeessoodha; sababni isaas fufiin taasisuu al sadaffaa ida’amteetti; matimni (c) keessa ture gara antima alkalattiitti jijjiramee matimni haaraan namichi jedhu ida’ameera. Akkuma
kanaa taasisuu sadeessoon AO keessa akka jiru nibeekna. Haa ta’u malee, gochima ce’aa irraa taasisuu sadeessoo uumuun hiika adda hin taane ykn gocha tokko eenyu akka raawwates akka raawwachiises adda baasuun ni ulfaata. Fakkeenyaaf himoota armaan gadii mee haa ilaallu: \\
\\
Hima: Tolasaan gurbaa-tiin dubartii-tiin hoolaa bich-isis-iis-e.\\
\\

Fakkeenya hima kanaa keessatti matimni Tolasaa jedhu gocha raawwachiisaa akka ta’e beekama. Garuu Tolasaan gurbaa ajajeeti gurbaanammoo dubartii ajajemoo dubartiitu gurbaa ajaje? Lamaanuu ta’uu mala. Kanaaf hiikaan ifa miti himni taasisaa armaan ol irra jiru; taasisaa sadeessoo keessatti haalli
akkasii nama quunnamuu mala. Waan kana ta’eefis, yeroo baay’ee dubbii keessatti antimni tokko irra utaalamee dubbatama. Fakkeenyaaf, \\
\\
a. Tolasaan hoolaa bich-is-is-iis-e.\\
b. Tolasaan gurbaa-tiin hoolaa bich-is-is-iis-e.\\
c. Tolasaan dubartii-tiin hoolaa bich-is-is-iis-e.\\
\\
(a) irratti antimoonni lamaanuu irra utaalamaniiru. Matimni gocha raawwachiise \textbf{Tolasaa} ta’uunsaa haa beekamu malee enyuun ajajee akka gocha raawwachiisee waan beekamu hin jiru, yoo keessa beekkannootiin ta’e malee. (b) irratti namni gocha hoolaa bituu raawwate hinbeekamu, garuu ajajaan
lammaffaa antima alkallattii isa \textbf{gurbaa-tiin} jedhu akka ta’e ni beekama. (c)n fakkeenyuma (b) waliin wal fakkaatu. AO keessaa taasisuun afureessoonillee jiraachuu mala. Kana jechuun maxxantoota taasisuu sadii olitti walitti aansinee yoo fidne seerri nu dhoowwu hin jiru. Karaa jecha gara biroo
maxxantoota taasisuu gochima irratti maxxanuu qaban seerri daangessu hin jiru. Fakkeenyaaf hima armaan gadii haa ilaallu:\\
\\
Hima: Tolasaan hoolaa bich-is-is-is-iis-e. \\
\\
Gochi hima armaan olii keessatti taa’e maal agarsiisa? \textbf{Tolasaan} nama tokko ajaje, inni ajajamemmoo nama lammaffaa ajaje, inni lammaffaanammoo nama sadaffaa ajaje, inni sadaffaanammoo nama afraffaa ajaje, inni afraffaan ammoo nama isa hoolaa bite ajaje jechuudha. 

\paragraph{Taasisuu Matima Dhokataa \index{taasisuu matima dhokataa}}
AO keessa himni taasisuu matima addaan ba’ee hin beekamne hedduu qaba.Xumurtoonni hawwii waan tokkoo agarsiisan kanneen akka hawwe, kajeele, barbaadee fi kkf fufii taasisuu qeentee ida’atanii hima ijaaruu danda’u. Garuu himoonni haala kanaan ijaaraman tokko tokko matima addaan ba’ee beekamu
hin qaban. Fufiin waliigaltee gochima irratti ida’amu Korniyaa sadaffaa lakkoofsa qeentee haa ta’u malee matimni gocha taasisuu raawwachiise enyummaan isaa hin beekamu. Fakkeenyaaf himoota armaan gadii haa ilaallu:\\
\\
a. ani mana jireenya barbaad-e.\\
b. Man-ni jireenyaa na barbaach-is-a.\\
c. Ani dhugaatii haww-e.\\
d. Dhugaatii na haww-isiis-a.\\
e. Ani nyaata kajeel-e.\\
f. Nyaata na kajeel-sis-a.\\
\\
Himoonni (b), (d) fi (f) irratti agarsiifaman kunneen karaa ittiin wal fakkaatan qabu. Hundi isaanii matima hin qaban. Garuu (a, c fi e) matima adda ba’ee beekamu qaba. Matimni ani jedhu (a), (c) fi (e) irratti gocha sammuu keessatti mul’atu akka agarsiise ifa. Kana jechuun (a), (c) fi (e) irratti kan barbaade, kan hawwe fi kan kajeele matima ani jedhudha; xumurtoota himoota kanneenii irrattis fufiin waliigaltee \{-e\} jettu ida’amteetti. Fufiin waliigaltee kunis matimni gocha raawwate ramaddii tokkooffaa lakkoofsa qeentee jechuun ani ta’uu isaa agarsiifti. Walbira qabnee ilaaluuf, fufiin waliigaltee kan (b), (f) fi (f) irratti mul’atu \{-a\} kan jedhu. Fufiin kun ramaddii sadaffaa, Korniyaa kormaa fi lakkoofsa qeentee (inni) haa agarsiisu malee matimni gocha raawwachiise adda ba’ee kan beekamu miti.

\paragraph{Taasisuu Hafoo}
AO keessa himni taasisuu hafoo \index{taasisuu hafoo} jira. Irra keessa yoo ilaalan,himni gosa kanaa hibboo namatti ta’a. Sababni kanaas fufiin taasisuu wayita gochima irratti ida’amtu matima gara biroo ida’uun matima duraan ture gara antimaatti jijjiiruun waan beekamtuuf. Waan kana ta’eef yeroo baay’ee xumurri hafoo
fufii taasisuu ida’achuun gara gochima ce’aatti jijjiiramti. Waan kana ibsuuf fakkeenya armaan gadii haa ilaallu: \\
\\
a. sibiilli dab-e.\\
b. Inni sibiila dab-s-e.\\
c. Gurbaan fiig-e.\\
d. Inni gurbaa fiig-is-e\\
\\
Himootni (a) fi (c) keessatti argaman hafoo jedhamu; sababni isaas matima tokko tokkoo fi gochima tokko tokko waan qabaataniif gochi qaama tokko iraa gara qaama birootti dabreeru waan hinjirreefi. Himoonni kunneen (b)fi (d) irratti gara ce’aatti jijjiiramaniiru. Sababni isaas (b) irratti fufiin
taasisuu \{-s-\}n (d) irrattammoo fufiin taasisuu \{-is-\} gochima irratti ida’amanii matimni inni jedhu akka dabalamani matimni duraan turemmoo gara antimaatti akka jijjiiramu waan taasisaniifidha. Haalli kun adeemsa beekamaadha waan amala fufii taasisuu agarsiisufidha. Taasisuu hafoo jechuun hima
hafoo gara hima ce’aatti jijjiiruu jechuu miti. Taasisuu hafoo keessatti fufiin taasisuu jecha irratti ida’ame matima gara biroo hinida’u; gochimnis ce’a hin ta’u. Fakkeenyaaf himoota armaan gadii haa ilaallu:\\
\\
a. Allaattiin barr-is-e.\\
b. Roobni jand-is-e.\\
c. Qorophisni qoroph-is-e.\\
d. Namichi qaxx-is-e.\\
e. Qilleensi bubb-is-e. \\
\\
Fakkeenyota armaan olii keessatti fufiin taasisuu yoo jiraatteyyuu himni hafoodhuma. Fufiin \{-is-\} jettu fufii ta’uu ishee baruuf himoota kanneen karaa gara biroo ibsinee haa ilaallu: \\
\\
a. Allaatiin bar jedhe.\\
b. Roobni jandoo buuse.\\
c. Qorophisni qoroph jedhe.\\
\\
Himoota armaan oliirraa akka hubannutti fufiin taasisuu –ishundee jechootaa bar, jandoofi qorop jedhan irraa adda baateetti. Kanaaf fufiin taasisuu altokko tokko hundee jechaa irratti ida’mtee gochimuma hafoo uumti jechuu dandeenya. Fakkeenyaaf, maqaa jandoo jedhurratti fufii taasisuu \{-is-\} yoo
idaane jandise ta’a, xumurri kunis hafoodha.

\subsubsection{Hiika Hima Taasisuu \index{hiika hima taasisuu}}

Himni taasisuu hiikota adda addaa qaba. Hiikonni kunneenis kallatti, alkallattii, hawwaasee (walta’insaa), walabee, tasee, ta’e jedhee,to’atee, kakaasee fi uumee jedhamu. Taasisuu
kalatti kan jedhamu matimni tokko qaamaan ykn kalattiin raawwatama qabee ykn tuqee yoo gocha gaggeessedha. Fakkeenyaaf himoota armaan gadii haa ilaallu: \\
\\
a. Isheen mucaa raf-is-te.\\
b. Gurbaan namicha kuff-is-e.\\
\\
Hima (a) irratti matimni ishee yoo ta’u gochamni mucaa dha. Himni kun dubartiin tokko qaamaan mucaa qabdee ykn hammattee akka mucaan raftu gochuu ishee nu agarsiisa. Waan kana ta’eefis taasisuu kalattii jedhama. (b) irrattis matimni gurbaa jedhame qaamaan antima namicha jedhame waliin wal qunnamee akka gochamni kufu godheera; kanaafu taasisuu kallatti jedhama. Taasisuun alkallattii taasisuu kallattii irraa adda. Taasisuu alkallattii keessatti matimaa fi antima giddu qaamni biroon jira. Matima hima taasisuu taasisaa yoo jenne qaama ajaja matima irraa fudhatee gocha tokko raawwatuun ammoo taasisamaa haa jennu. Egaa hima taasisuu alkallattii keessa taasisaa fi antima gidduu taasifaman jira jechuudha.
Kana jechuunis taasisaa fi gochi raawwatamu kallattiin wal hin quunnaman; kanaaf gosti taasisaa akkasi taasisaa alkallatti jedhama jechuudha. Fakkeenyaaf himoota armaan gadii haa hubannu: \\
\\
a. Bultumaan Garbiitiin sangaa gurgur-siis-e.\\
b. Mucaan dubartii xuwwee cab-s-iis-te.\\
c. Namichi gurbaatiin sibiila dab-s-iis-e.\\
\\
Himoonni armaan olitti eeraman kunneen taasisuu alkallatti jedhamu. Fakkeenyaaf (a) haa ilaallu. Himni kun taasisaa Bultumaa jedhamu, taasisamaa Garbii jedhamuufi antima sangaa jedhamu qaba. Butumaan kallattiitiin deemee sangaa hin gurgurre. 

Namni kallattiin sangaa gurgure Garbiidha. Bultumaa fi sangaa gurguruu gidduu Garbiin jira. Bultumaan sanga kan gurgursiise Garbiitiin gorsa kennee ykn ajaja kenneeti. Kanaaf matimni gocha kana raawwate alkallatiinidha. Waan kana ta’eef taasisuun kun alkallattii jedhama. Fakkeenyonni (b) fi (c)
haaluma wal fakkaatuun qaacessamuu danda’u. Hima (b) keessatti gochi raawwatame xuwween cabuudha. Matimni akka gochi kun raawwatamuuf sababa ta’e matima mucaa jedhu. Qaamni kallattiin gocha kana raawwate taasifamaa dubartii jedhu. Matimni alkallattiin waan gocha raawwateef taasisuu alkallattii jedhama. Hiikni garri biroon hima taasisuu keessatti beekamu hawaasee jedhama. Hiikni kun hirmaattonni hima taasisuu keessatti eraman walta’insa, walgargaarsa, waliin jireenya fi gochaawwan kanneen fakkaataan agarsiisu. Fakkeenyonni armaan gaditti kennamaniiran taasisuu hawaasee jedhamu: \\
\\
a. Hiriyaankoo laaqana na nyaach-is-e.\\
b. Obboleessikoo bishaan na obaas-e.\\
c. Namichi farda ishee yaab-sis-e.\\
d. Gaariin uffannaa na bit-ach-iis-e.\\
e. Gaariin uffannaa na bich-isiis-e.\\
\\

(a) afeerraa agarsiisa; (b) nama dheebote dheebuu baasuu agarsiisa; (c) nama muuxannoo ykn human hinqabneef gargaarsa gochuu mul’isa; (d) fi (e) human waliif ta’u agarsiisu. Bakka tokko tokkotti, fkn, baha Wallaggaatti (d) ni tajaajila. Bakka birootti, fkn, naannoo Hararitti (e) tu tajaajila hiikuma wal fakkaatu ibsuufidha. Fakkeenyonni kunneen hundinuu walta’anii hawaasaa keessa wal gargaaraa waliin jiraachuu waan agarsiisaniif taasisuu hawaasee jedhamu.

Hiikni gara biroon ilaaluu qabnu walabee jedhama. Taasisuun walbee taasisaan taasisamaaf hayyama kennuun gocha tokko akka raawwatuuf walaba godha. Fakkeenyaaf himoota taasisuu armaan gadii haa ilaallu:\\
\\

a. Barreessituun gara oggaaniččaatti na dabar-s-ite.\\
b. Hintala isaa gara mana barumsaatti lakk-is-e.\\
\\

(a) keessatti barreessituun nama tokkof hayyama kennuu ishee argina. Haalum wal fakkaatuun (b) keessattis namni tokko hintalasaa walaba gochuu isaa argina. Himoonni lamaanu taasisuu walabee jedhamanii beekamu. Gosni gara biroon taasisuu tasee jedhama. Gosti hiika hima taasisuu kanaa waan
tokko osoo itti hin yaadamiin tasa yoo raawwatamu.Taasisuu tasee hubachuuf himoota armaan gadii haa ilaallu:\\
\\

a. Inni manaa gadi nawaamee qorra narukkuch-iis-e.\\
b. Inni oduutiin na qabee dhagaa na rukuch-iis-e.\\
\\

Himoonni kunneen lamaan fakkeenya tase ta’uu danda’u. (a) irratti namni tokko nama biroo ta’e jedhee qorra rukuchiisuuf kaayyoo hin qabu. Namni biroon sun tasa qorraan rukutame, waan mana gadi waamameef garuu sababa biroo qaba. (b) haaluma wal fakkaatuun ibsama. Namni tokko isa biroo miidhuuf kaayyoo hin qabu; namni tokko osoo oduu odeessaa deemuu tasa dhagatiin rukutamuu isaa hubanna hima kanarraa. Matimni himoota kanneen maqdhaalii inni jedhu waan ta’eef dhala namaa akka ta’e ni beekama. Garuu hiikni tase matima uumama lubbuu hin qabnellee ta’uu danda’a; fakkeenyaaf himoota armaan gadii haa ilaallu: \\
\\
a. aduun damma baq-s-e.\\
b. Bishaan ibidda dhaam-s-e.\\
c. Ibiddi dhadhaa baq-s-e.\\
\\
Himoonni taasisuu armaan olitti
eeraman matima, antima fi gochima qabu. Garuu matimoonni himoota kanneen keessatti mul’atan uumama
lubbuu hin qabne. Waan kana ta’eefis matimoonni kunneen itti yaadanii gocha tokko hinraawwanne. Gochaawwan eeraman kunneen akka tasaa raawwataman. Hiikni tasee waliin wal morkatu ta’e jedhee dha. Gosni hiika kanaa namni tokko ta’e jedhee, itti yaadee wayita raawwatudha.\\
\\
a. Gaariin Bullootiin qotiyoo bitach-iis-e.\\
b. Barsiistuun barattoota herreega bar-siis-te.\\
\\
Himoonni armaan olitti eeraman lamaanuu hiikni isaanii ta’e jedhee dha. (a) irratti Gaariin ta’e jedhee Bullootiin qotiyoo akka bitu godheera. Gochi qotiyoo bituu gocha itti yaadame
raawwatame malee kan tasa raawwatame miti. (b) keessattis Barsiistuun ta’e jette, itti yaaddee, itti
qophooftee barattoota herreega barsiistee jirti. Gochi kunis gocha itti yaadamee raawwatame.\\
\\
a. Gurbaan horii gaaratti deeb-is-e.\\
b. Isheen ganama ganama hoolota manaa gadi yaa-s-ti.\\
\\
Hima (a) keessatti horii bakka tokkoo gara bakka birootti qajeelchuun to’annaa godha. (b) keessatti dubartiin tokko hoolota mana gadi yaafitee bobbaasuun to’annaa akka gaggeessitu hubachaa
jirra. Waan kana ta’eef himoonni kunneen hiika to’atee jedhu qabu jechuu dandeenya. Hiikni gara biroon kakaasee jedhama. Akkeesseen namni tokko hojjaa ykn gocha nama biroo ilaalee haaluma wal fakkaatuun yoo hojjetu agarsiisa. Fakkeenyaaf himoota armaan gadii haa ilaallu: \\
\\
a. Dubbatee na dubbach-iis-e.\\
b. Fiigee na fiig-sis-e.\\
c. Nyaatee na nyaach-is-e.\\
d. Hojjetee na hojjech-iis-e.\\
\\
Himootni armaan olitti eeraman hiika wal fakkaatu qabu. Hunda keessatti gochi tokko gocha nama gara birootiin kaka’uutiin raawwatame. (a) irratti dubbiin nama tokkoo namni gara biroon akka dubbatu kakaase, (b) irratti fiigichi nama tokkoo nama gara birootiif kaka’umsa ta’e, (c) irratti namni
tokko isa biroo ilaalee nyaate, (d) irratti hojiin nama tokko isa birootiif kaka’umsa ta’era. 

Hiikni gara biroon isa uumee jedhamu. Uumeen namni tokko gochi tokko osoo hin raawwatamiin ykn osoo hin mul’atiin akka raawwatamee ykn akka mul’ate fakkeessee yoo dubbatu.\\
\\
a. Isheen kan dhiyoo fag-eessi-tee ilaalti.\\
b. Isheen soba dhugaa fakk-eessi-tee dubbatti.\\

Himni (a) irratti dubartiin tokko waan dhiyoo jiru fagoo akka jiru fakkeessitee ilaalti sammuu ishee keessatti. Kana jechuun haala qabatamaan hinjirre tokko akka jirutti uumtee yaaddi jechuudha. Kanaaf himni kun hiika uumee jedhu qaba. (b) keessattis dubartiin tokko waan soba ta’e akka dhugaa ta’eetti
uumtee sammuu ishee keessatti ilaalti jechuudha. Waliigalaan, akkuma kutaa kana keessatti ilaaluu yaalletti himni taasisuu hiikota adda addaa qaba. Hiikota kanneenis kallattii, alkallattii,
hawwaasee (walta’insaa), walabee, tasee, ta’ejedhee, to’atee, kakaasee fi uumee jedhamu jennee tarreesinee tokko tokkoo isaaniif fakkeenya kenninee jirra.

Haa ta’u malee, hiikota tarreeffaman kanneen jidduu daangaan hin jiru. Kana jechuunis, himni taasisuu tokko hiikota gosa adda addaa lama ykn isaa ol qabaachuu mala. Himni taasisuu hiikaan alkallattii ta’e tokko hiika ta’e jedhees qabaachuu mala. Himni hiika kallattii qabu tokkos hiika dabalataa tasee ykn
kakaasee qabaachuu mala. Kanaaf caasaan hima taasisuu tokko hima adda addaa qabaachuun waanuma jiru ta’usaa hubachuun barbaachisaadha.

Yaadni gara biroon hubatamuu qabu caasaan hima taasisuu tokko hiika tokko qaba ykn hiika gara biroo qaba jechuuf yeroo, bakka fi haala himni sun keessatti dubbatamu baruu qabna. Yoomiyyuu taanan hiika hima tokkoo kan murteessuu haala himni sun keessatti faayidaa irra oolu ta’usaa beekuu qabna.



\newpage

\bibliographystyle{apacite}
\bibliography{tol}


\printindex 

\end{document}
